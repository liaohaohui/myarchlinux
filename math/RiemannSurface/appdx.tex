\documentclass[a4paper,11pt]{article}
\usepackage{amsmath}
\usepackage{amsfonts}
\usepackage{verbatim}
\usepackage{enumerate}
\usepackage{url}
\usepackage{framed}
\usepackage{epsfig}
\usepackage{textcomp} %\textquotesingle
\usepackage[
%sorting=nyt,
firstinits=true, % render first and middle names as initials
useprefix=true,
maxcitenames=3,
maxbibnames=99,
style=authoryear,
dashed=false, % re-print recurring author names in bibliography
natbib=true,
url=false
]{biblatex}
%%% http://tex.stackexchange.com/questions/12254/biblatex-how-to-remove-the-parentheses-around-the-year-in-authoryear-style
\usepackage{xpatch}
\addbibresource{complex.bib} % run: biber topic1
\usepackage{color}

\usepackage{xypic}
\usepackage{tikz}
\usetikzlibrary{lindenmayersystems}
\usetikzlibrary[shadings]


%%% Page Layout
\oddsidemargin=0truecm
\evensidemargin=0truecm
\textwidth=160truemm
\textheight=260truemm
\leftmargin=0truemm
\rightmargin=0truemm
\voffset=-23truemm
\topmargin=0truemm

\newif\iflecturer
%\lecturerfalse
\lecturertrue

\iflecturer
\usepackage{marginnote} % \marginpar
%\usepackage[color]{showkeys}
\definecolor{refkey}{rgb}{1,0,0}
\definecolor{labelkey}{rgb}{1,0,0}
\else
\def\marginpar#1{}
\fi

\iflecturer
\newcommand{\Answer}[1]{\dotfill\underline{\mbox{\hspace{1em}\color{blue}#1}}}
\newcommand{\BoxAns}[1]{\fbox{\color{blue}#1}}
\newcommand{\Reason}[1]{{\par\color{blue}\par{}Reason: #1}}
\else
\newcommand{\Answer}[1]{\dotfill\underline{\mbox{\hspace{1em}\color{white}#1}}}
\newcommand{\BoxAns}[1]{\fbox{\color{white}#1}}
\newcommand{\Reason}[1]{{\par\color{white}\par{}Reason: #1}}
\fi

\newif\ifamsstyle
\amsstylefalse
\newcounter{topic}
\setcounter{topic}{9}
\input common
\renewcommand\theequation{\arabic{equation}}
\renewcommand\thesection{\arabic{section}}
\input symbol
\makeatletter
\newcommand{\Arctan}{{\mathop{\operator@font Arctan}\nolimits}}
\makeatother

%\parindent=0pt
\parskip=1pt
\linespread{1}


\begin{document}

\title{{\sc Rudiments of Riemann Surfaces\\
    Appendix}}
\author{Author: B. Frank Jones, Jr. (Rice Univ. 1971)\\
Seminar: Dr Liew How Hui (\url{liewhh@utar.edu.my})}
\date{}

\maketitle

%%% page 278
\section{Final Examination}

\begin{enumerate}
\item Let $a$ and $b$ be relatively prime positive integers.  Analyse
  the Riemann surface of the polynomial
  $$
  A(z,w) = w^{2a} - 2z^b w^a + 1.
  $$
  Do the same for the polynomial
  $$
  B(z,w) = z^{2a} - 2w^b z^a + 1.
  $$
  Be sure to compute the genus in each case and check that they are
  equal.

\item Let $A(z,w)$ be an irreducible polynomial of degree at least 2
  in $w$.  Prove that there does not exist a rational function $f$
  such that
  $$
  A(z, f(z)) \equiv 0.
  $$

\item If $A(z,w)$ is an irreducible polynomial and $S$ is the Riemann
  surface of $A$, prove that $S$ cannot have exactly one branch point
  (of possibly high order).

\item The Riemann surface of the polynomial $w^3 + z^3 - 1$ is easily
  seen to have genus 1.  Thus, it is homeomorphic to a torus and by
  our general theorem is analytically equivalent to the Riemann
  surface of a polynomial of the form
  $$
  w^2 - 4(z-e_1)(z-e_2)(z-e_3).
  $$
  Find such a polynomial explicitly. \textbf{Hint:} Use algebra only

  %%% page 279
\item Prove that the sum of two algebraic functions is algebraic.
  Compute explicitly a polynomial $A(z,w)$ such that
  $$
  A(z, z^{1/2} + z^{1/3}) = 0.
  $$

\item Are the following (noncompact) Riemann surfaces parabolic or
  hyperbolic?
  \begin{enumerate}
  \item A compact Riemann surface minus a point.
  \item A Riemann surface minus the closure of an analytic disk.
  \item A Riemann surface on which a Green's function exists.
  \end{enumerate}
\end{enumerate}



\section{Solutions to Problems 5.15 and 5.16}

\underline{Problem 5.15} % page 139+8
Analysis of $A(z,w) = w^3 - 3zw + z^3$.

\begin{mdframed}
  \textbf{Irreducible}: If not $A$ must have a linear factor, so
  $$
  A = (w + \alpha) (w^2 + \beta w + \gamma),
  $$
  where $\alpha, \beta, \gamma$ are polynomials in $z$ which must
  satisfy
  $$
  \alpha + \beta = 0,\quad
  \alpha \beta + \gamma = -3z,\quad
  \alpha \gamma = z^3.
  $$
  The third relation shows that $\alpha = cz^k$, where $c \ne 0$ and
  $k \in \{0,1,2,3\}$; solving for $\alpha$ and $\beta$ using the
  second relation show that
  %%% page 280
  $$
  -c^2 z^{2k} + c^{-1} z^{3-k} \equiv -3z,
  $$
  an impossible identity.

  \textbf{Critical Points}:  By definition, $z = \infty$ is critical.
  Since $A(0,w) = w^3$, $z = 0$ is critical.  For other $z$, we look
  for solutions of the pair of equations $A(z,w) = 0$ and
  $$
  \frac{\p A}{\p w} = 3w^2 - 3z = 0.
  \Rightarrow
  \left\{
  \begin{aligned}
    w^3 - 3zw + z^3 &= 0,\\
    w^2 &= z.
  \end{aligned}
  \right.
  $$
  Thus, $w^3 - 3w^3 + w^6 = 0$, so $w^6 = 2w^3$ and since $w \ne 0$ we
  have $w = 2^{1/3} w^k$, where $2^{1/3} > 0$ and $w = e^{2\pi i/3}$,
  $k = 0,1,2$.  Thus,
  $$
  z = 2^{2/3} w^{2k}, \quad k = 0,1,2.
  $$

  \textbf{Puiseaux expansions}:

  \underline{$z=0$}.  First, we argue heuristically.  If $w_1$, $w_2$,
  $w_3$ are the zeros of $A$, then
  $$
  \left\{
    \begin{aligned}
      w_1 + w_2 + w_3 &= 0,\\
      w_1 w_2 + w_2 w_3 + w_3 w_1 &= -3z,
      w_1 w_2 w_3 &= -z^3.
    \end{aligned}
  \right.
  \eqno{(*)}
  $$
  If there s no branching, these are all holomorphic near $z = 0$ and
  $|w_k| \le C|z|$, contradicting the second line of $(*)$.  If the
  branching is of order 2, then each $|w_k|$ is asymptotic to constant
  $|z|^{\ell/3}$ for some integer $\ell$.  The third line of $(*)$
  shows $\ell = 3$, again contradicting the second line.  The only
  other possibility is a branch point of order 1 and a holomorphic
  solution $e(t,tQ)$.  For this solution we have
  $$
  t^3 Q^3 - 3t^2 Q + t^3 = 0.
  $$
  Thus,
  $$
  tQ^3 - 3Q + t = 0.
  $$
  Thus, $Q(0) = 0$, so we let $Q = tQ_1$ and find
  $$
  t^4 Q_1^3 - 3t Q_1 + t = 0.
  $$
  Thus,
  $$
  t^3 Q_1^3 - 3Q_1 + 1 = 0.
  $$
  The derivative of this polynomial with respect to $Q_1$ equals $-3$
  at $t = 0$, so the implicit function theorem implies $Q_1$ exists
  with $Q_1(0) = 1/3$.  Thus, the Riemann surface has an element
  $$
  e(t, t^2/3 + \cdots).
  $$

  The branched element we represent as $e(t^2, tQ)$ and find
  $$
  t^3 Q^3 - 3t^3 Q + t^6 = 0.
  $$
  Thus,
  %%% page 282
  $$
  Q^3 - 3Q + t^3 = 0.
  $$
  At $t=0$ there is a solution $Q(0) = \sqrt{3}$ and the implicit
  function theorem again can be applied to provide an element
  $$
  e(t^2, \sqrt{3} t + \cdots).
  $$

  \underline{$z= \infty$}.
  The heuristic analysis is similar. Now we try $e(\frac{1}{t},
  \frac{Q}{t})$ and obtain
  $$
  \begin{aligned}
    t^{-3}Q^3 - 3t^{-2}Q + t^{-3} &= 0;\\
    Q^3 - 3tQ + 1 = 0.
  \end{aligned}
  $$
  At $t = 0$ we obtain 3 \emph{distinct} solutions $Q(0) = -w^j$, so
  we find 3 unbranched solutions
  $$
  e(\frac{1}{t},\frac{-1}{t}+\cdots),\quad
  e(\frac{1}{t},\frac{-w}{t}+\cdots),\quad
  e(\frac{1}{t},\frac{-w^2}{t}+\cdots).
  $$

  \underline{$z = 2^{2/3} w^{2k}$}. $(k=0,1,2)$
  Again we omit the heuristics, except to note that $w = 2^{1/3} w^k$
  is \emph{exactly} a double root, so there is at least one unbranched
  solution, and the corresponding root is $-2 \cdot 2^{1/3} w^k$.
  Thus, there is an element
  $$
  e(2^{2/3} w^{2k} + t,\ -2 \cdot 2^{1/3} w^k + \cdots).
  $$
  Now we see whether the other two solutions are branched.  If not,
  then there is an element
  %%% page 283
  $$
  e(2^{2/3} w^{2k} + t,\ 2^{1/3} w^k + \cdots).
  $$

  \textbf{Observation}: The total branching order is $V = 4$ (first
  order branch points at 0, $2^{2/3} w^{2k}$) and $n = 3$, so the
  genus is 0 (recall $V = 2(n+g-1)$).
\end{mdframed}


\begin{exmp}[Another example of an algebraic function]
  Let
  $
  A(z,w) = 2zw^5 - 5w^2 + 3z^2.
  $
  Then
  $\ds
  \frac{\p A}{\p w} = 10zw^4 - 10w.
  $
  Now the critical points are $z = 0$, $z=\infty$, and for the others
  we obtain
  %%% page 284
  $$
  \frac{\p A}{\p w} = 0 = 10w(zw^3-1).
  $$
  If $w = 0$, then $A=0 \Rightarrow z = 0$, which we are not now
  considering.  Thus,
  $$
  zw^3 = 1\quad\text{and}\quad A = 0 = 2w^2 - 5w^2 + 3z^2,
  $$
  so
  $$
  z^2 = w^2. \quad \text{Therefore, } z^2 w^6 = 1 = w^8.
  $$
  Thus, if $\omega = e^{2\pi i/8}$, $w = \omega^k$, $0 \le k \le
  7$,and $z = \omega^{-3k}$.  We still must check $z^2 = w^2$:
  $\omega^{-6k} = \omega^{2k}$, which is valid.  So we have found all
  the critical points, and we now analyse them.

  \underline{$w = \omega^{-3k}$}.  The only possible multiple value
  for $w$ is $\omega^k$, and at these points
  $$
  \frac{\p^2 A}{\p w^2} = 40 z w^3 - 10 = 30 \ne 0.
  $$
  We guess a branch point occurs, so we try for an element
  $e(\omega^{-3k} + t^2, \omega^k + tQ)$.  Then
  $$
  2(\omega^{-3k} + t^2)(\omega^k + tQ)^5 - 5(\omega^k + tQ)^2
  + 3(\omega^{-3k} + t^2)^2 \equiv 0.
  $$
  Expanding,
  $$
  \begin{aligned}
    2(\omega^{-3k} + t^2)(\omega^{5k} + 5\omega^{4k} tQ +
    10\omega^{2k} t^2 Q^2 + \cdots) - 5(\omega^{2k} + 2\omega^k tQ +
    t^2 Q^2) + 3(\omega^{-6k} + 2\omega^{-3k} t^2 + t^4) &\equiv 0;\\
    10\omega^k tQ + 20 t^2 Q^2 + 2\omega^{5k} t^2 + 10 \omega^{4k} t^3
    Q + \cdots - 10\omega^k tQ - 5t^2 Q^2 + 6\omega^{-3k} t^2 + 3t^4
    &\equiv 0;
  \end{aligned}
  $$
  dividing by $t^2$,
  $$
  15Q^2 + 8\omega^{5k} + 10\omega^{4k} tQ + \cdots + 3t^2 \equiv 0,
  $$
  %%% page 285
  where the omitted terms vanish at $t=0$.  At $t=0$ we can let $\ds
  Q(0) = \sqrt{-\frac{8w^{5k}}{15}}$ (either determination) and note
  that at $t=0$ and for this value of $Q(0)$ the above expression has
  its derivative with respect to $Q$ equal to
  $$
  30Q \ne 0.
  $$
  Thus, the implicit function theorem is in force and we obtain branch
  points
  $$
  e(\omega^{-3k} + t^2,\ \omega^k + \sqrt{-\frac{8\omega^{5k}}{15}} t
  + \cdots),\quad 0 \le k \le 7.
  $$

  In order to treat the critical points 0 and $\infty$ we look for
  meromorphic elements of the form
  $$
  e(t^m,\ t^\ell Q),
  $$
  where $m$ and $\ell$ are integers ($m\ne 0$) and $Q$ is holomorphic
  near 0, $Q(0) \ne 0$.  Then
  $$
  2t^{m+5\ell} Q^5 - 5t^{2\ell} Q^2 + 3t^{2m} \equiv 0.
  $$
  We now try to juggle $m$ and $\ell$ to obtain some definite
  information as $t \to 0$.  Thus, we would like to have at least two
  exponents of $t$ in this equation coincide and to correspond to the
  dominant terms near $t=0$.  Obviously it is impossible to have all
  three exponents coincide.  Thus, the various possibilities in this
  case are 
  %%% page 286
  \begin{enumerate}[(a)]
  \item $m + 5\ell = 2\ell < 2m$,
  \item $2\ell = 2m < m + 5\ell$,
  \item $m + 5\ell = 2m < 2\ell$.
  \end{enumerate}

  In \underline{case (a)} we have $m = -3\ell > \ell$, o $\ell < 0$.
  Thus, we must have $\ell = -1$, $m = 3$, and the equation for $Q$
  becomes
  $$
  2Q^5 - 5Q^2 + 3t^8 \equiv 0.
  $$
  Thus, $2Q(0)^3 = 5$ and the derivative with respect to $Q$ is $10Q^4
  - 10Q = 15Q \ne 0$ for $Q(0)$.  Thus, the implicit function theorem
  shows we obtain the branch point
  $$
  e(t^3,\ (\frac{5}{2})^{1/3} \frac{1}{t} + \cdots).
  $$
  In \underline{case (b)} we have $m = \ell < 3\ell$ so $\ell> 0$.
  Choosing $m = \ell = 1$ gives
  $$
  2t^4 Q^5 - 5Q^2 + 3 \equiv 0.
  $$
  Again we obtain solutions corresponding to either choice of $Q(0)$
  and we get two \emph{regular} elements
  $$
  e(t,\ (\frac{3}{5})^{1/2} t + \cdots),\quad
  e(t,\ -(\frac{3}{5})^{1/2} t + \cdots).
  $$
  In \underline{case (c)} we have $m = 5\ell < \ell$ o $\ell < 0$.
  Thus, we must $\ell = -1$, $m = -5$, and we obtain
  $$
  2Q^5 - 5t^8 Q^2 + 3 \equiv 0.
  $$
  Again we obtain the branch point
  $$
  e(t^{-5},\ -(\frac{3}{2})^{1/5} t^{-1} + \cdots).
  $$

  %%% page 287
  This completes the analysis of this example except for the
  observation that the branch point corresponding to $z = \infty$ is
  of order 4 and thus all five ``sheets'' of the Riemann surface are
  branched at $\infty$ in a single cycle.  This proves that $A$ is
  \emph{irreducible}.

  Notice the total branching order here is $V = 8 + 2 + 4 = 14$, so
  the genus $g$ satisfies
  $$
  7 = n + g - 1 = 4 + g,
  $$
  or $g = 3$.
\end{exmp}

\underline{Problem 5.16} % page 139+8
Analysis of $A(z,w) = z w^3 - 3w + 2z^a$, $a$ any integer.  now
$\frac{\p A}{\p w} = 3zw^2 - 3$, so critical points other than $z = 0,
\infty$, come from solving
$$
\left\{
\begin{aligned}
  zw^2 &= 1,\\
  -2w + 2z^a &= 0.
\end{aligned}
\right.
$$
So $w = z^a$ and thus $z^{2a+1} = 1$.  Let $b = 2a+1$ and
$$
\omega = e^{2\pi i/b}.
$$
Then we have the critical points $z = \omega^k$, $0 \le k \le |2a+1| -
1$; the corresponding double value of $w$ is $\omega^{ak}$.  Here is a
good place to present a criterion for branching: suppose $z_0 \ne
\infty$ is a critical point for a polynomial $A$ and that $A(z_0, w_0)
= \frac{\p A}{\p w}(z_0, w_0) = 0$ but $\frac{\p A}{\p z}(z_0, w_0)
\ne 0$.  Then any element $e(z_0 + t^m, Q(t))$ in the Riemann surface
for $A$ must be a branch point if $Q(0) = w_0$.  That is, $m \ge 2$.
For suppose $m =1$.  Then for $t$ near 0
%%% page 288
$$
A(z_0 + t, Q(t)) \equiv 0.
$$
Differentiate this identity with respect to $t$ and set $t = 0$ to
obtain
$$
0 = \frac{\p A}{\p z}(z_0, w_0) + \frac{\p A}{\p w}(z_0, w_0) Q'(0)
= \frac{\p A}{\p z}(z_0, w_0) \ne 0,
$$
a contradiction.

In the present case we have $z_0 = \omega^k$, $w_0 = \omega^{ak}$, and
$$
\frac{\p A}{\p z}(z_0, w_0) = w_0^3 + 2az_0^{a-1}
= \omega^{3ak} + 2a\omega^{(a-1)k} = 0.
$$
Here we also have more information.  Namely, $\ds \frac{\p^2 A}{\p
  w^2}(z_0, w_0) = 6z_0 w_0 \ne 0$, so $w_0$ is \emph{exactly} a
double root.  Thus, the meromorphic element in this case has $m = 2$,
and we can express it as
$$
e(\omega^k + t^2,\ \omega^{ak} + \cdots),\quad
0 \le k \le |2a+1|-1.
$$

To examine the critical points $z = 0$ and $\infty$ consider elements
$$
e(t^m, t^\ell Q),\ Q(0) \ne 0.
$$
Then
$$
t^{m + 3\ell} Q^3 - 3t^\ell Q + 2t^{am} \equiv 0.
$$
For these exponents to the equal we require $m = -2\ell = -2am$, so
$bm = 0$, and thus $m = 0$, which is not allowed.

\begin{mdframed}
\underline{Case (a). $m+3\ell = \ell < am$}

Here $m = -2\ell$ so we must have $\ell = 1$, $m = -2$ and $1 + 2a <
0$, or $\ell = -1$, $m = 2$, and $1 + 2a > 0$.  We obtain in either
case
%%% page 289
$$
Q^3 - 3Q + 2t^{|b|} = 0,
$$
and we have the branch point
$$
e(t^{\pm 2}, \sqrt{3} t^{\mp 1} + \cdots)
\begin{cases}
  \text{top signs if } a > 0,\\
  \text{bottom signs if } a < 0.
\end{cases}
$$

\underline{Case (b). $m+3\ell = am < \ell$}

Here $3 \ell = (a-1)m$ and $m + 2\ell < 0$.  The equation is
$$
Q^3 - 3t^{-m - 2\ell} Q + 2 = 0,
$$
so
$$
Q(0)^3 = -2.
$$
We note that $0 > 3m + 6\ell = 3m + (2a-2)m = bm$.  We can take $m =
\pm 1$ if and only if $a \equiv 1 \pmod{3}$, and we thus obtain smooth
solutions only:
$$
e(t^{\mp 1}, -2^{1/3} t^{\mp 1} + \cdots)
\begin{cases}
  \text{top signs if } a > 0,\\
  \text{bottom signs if } a < 0,
\end{cases}
$$
where we use all three determinations of $2^{1/3}$.  If $a \not\equiv
1 \pmod{3}$, we must choose $m = \mp 3$ and we have a branch point of
order 2.

\underline{Case (c) $\ell = am < m + 3\ell$}

Thus, $am < m + 3am$, or $bm > 0$.  We can take $m = \pm 1$, $\ell =
\mp a$ and obtain the smooth solution
$$
e(t^{\pm 1}, \frac{2}{3}t^{\pm a} + \cdots).
$$
%%% page 290

\textbf{Summary}.
\begin{itemize}
\item If $a \ge 0$,
  \begin{itemize}
  \item $z = 0$ corresponds to a first order branch point,
  \item $z = \infty$ corresponds to a second order branch point if $a
    \not\equiv 1 \pmod{3}$, to no branch point if $a \equiv 1
    \pmod{3}$.
  \end{itemize}
\item If $a < 0$,
  \begin{itemize}
  \item $z = \infty$ corresponds to a first order branch point,
  \item $z = 0$ corresponds to a second order branch point if $a
    \not\equiv 1 \pmod{3}$, to no branch point if $a \equiv 1
    \pmod{3}$.
  \end{itemize}
\end{itemize}
Thus, $\ds V = |2a+1| + 1 +
\begin{cases}
  2\text{ if } a \not\equiv 1 \pmod{3},\\
  0\text{ if } a \equiv 1 \pmod{3},
\end{cases}
$

\noindent
$$
g = \frac{V}{2} - 2 =
\begin{cases}
  a \text{ if } a \not\equiv 1 \pmod{3},\ a \ge 0,\\
  a - 1 \text{ if } a \equiv 1 \pmod{3},\ a \ge 0,\\
  |a| - 1 \text{ if } a \not\equiv 1 \pmod{3},\ a < 0,\\
  |a| - 2 \text{ if } a \equiv 1 \pmod{3},\ a < 0.
\end{cases}
$$
\end{mdframed}


\section{Solutions to Final Exam Problems}

\begin{enumerate}
\item $A(z,w) = w^{2a} - 2z^b w^a + 1$
  $$
  \left\{
    \begin{aligned}
      \frac{\p A}{\p w} &= 2aw^{2a-1} - 2az^b w^{a-1}\\
      \frac{\p^2 A}{\p w^2} &= 2a(2a-1)w^{2a-2} - 2a(a-1)z^b w^{a-2}\\
      \frac{\p A}{\p z} &= -2bz^{b-1} w^a
    \end{aligned}
  \right.
  $$
  \begin{itemize}
  \item Critical points:
    \begin{itemize}
    \item $z = \infty$ by definition is critical.

      Suppose $\frac{\p A}{\p w} = 0$ and $A = 0$.  Then $w^a = z^b$
      so
      $$
      A = z^{2b} - 2z^{2b} + 1 = -z^{2b} + 1.
      $$
      Let $\omega = e^{\pi i/b}$.  Then $z = \omega^k$, $0 \le k \le
      2b - 1$, and
      $$
      w^a = z^b = \omega^{bk} = (-1)^k.
      $$
      Thus, for each $z = \omega^k$ there are distinct solutions of
      $A(z,w) = 0$.  Since $\frac{\p^2 A}{\p w^2} = 2a^2 w^{2a-2} \ne
      0$, we have no more than double roots.  And since $\frac{\p
        A}{\p z} = -2bz^{2b-1} \ne 0$, we have a branch point of order
      1 associated with each solutions $w$ of $A(\omega^k, w) = 0$.
      Thus, there are a branch points of order 1 lying over each $z =
      \omega^k$, so the branching associated with these critical
      points is $2ab$.

      Now consider $z = \infty$.  Look for elements of $\overline{M}$
      of the form
      $$
      e(t^{-m}, t^\ell Q),
      $$
      where $m > 0$ and $Q(0) \ne 0$.  Substituting,
      $$
      t^{2a\ell} Q^{2a} - 2t^{a\ell-bm} Q^2 + 1 \equiv 0.
      $$
      We have 3 cases:
      \begin{enumerate}[(a)]
      \item \underline{$2a\ell = a\ell - bm < 0$}

        Here $\ell < 0$ and $a \ell = -bm$.  If we choose $m = a$ and
        $\ell = -b$, we obtain
        $$
        Q^{2a} - 2Q^a + t^{2ab} \equiv 0
        $$
        and thus
        $$
        Q(0)^a = 2.
        $$
        Then we obtain
        $$
        e(t^{-a}, 2^{1/a} t^{-b} + \cdots).
        $$
        Since $a$ and $b$ are relatively prime, this is an element of
        $\overline{M}$.

      \item \underline{$a\ell - bm = 0 < 2a\ell$}

        Here $\ell > 0$ and $a \ell = bm$.  Choose $m=a$ and $\ell =
        b$, obtaining 
        $$
        t^{2ab} Q^{2a} - 2Q^a + 1 \equiv 0,
        $$
        so that $Q(0)^a = \frac{1}{2}$.  Then we obtain
        $$
        e(t^{-a}, 2^{1/a} t^b + \cdots).
        $$

      \item \underline{$2a\ell = 0 < a\ell - bm$}

        Here $\ell = 0$ and $bm < 0$, which is impossible.
        Summarising, at $z = \infty$ we have two branching points,
        each of order $a-1$.  Thus, $V = 2ab + 2(a-1)$, 
        %%% page 293
        so
        $$
        ab + a - 1 = 2a + g - 1
        $$
        or
        $$
        g = ab - a.
        $$
      \end{enumerate}

    \item \underline{Second part}. $B = z^{2a} - 2w^b z^a + 1$.

      The equation $B = 0$ is
      $$
      w^b = \frac{z^{2a} + 1}{2z^a} = \frac{z^a + z^{-a}}{2}.
      $$
      Thus, we simply obtain branch points at $z = 0$, $z = \infty$,
      and where $w = 0$, which is $z^{2a} + 1 = 0$.  Since
      $$
      w = \left(\frac{z^a + z^{-a}}{2}\right)^{1/b},
      $$
      the branch points at finite $z$ are of order $b-1$ and at $z =
      0$ or $\infty$ are of order $b-1$ as well, since $a$ and $b$
      are relatively prime.  Thus, $V = 2a(b-1) + 2(b-1)$, so
      $$
      a(b+1) + b - 1 = b + g - 1
      \Rightarrow g = a(b-1) = ab-a.
      $$

      \underline{Alternative solution}: Solution for $w^a$:
      $$
      w^a = z^b + \sqrt{z^{2b} - 1}\quad\text{(either determination)}.
      $$
      By inspection there are branch points over the roots of
      $z^{2b}  1$, each of order 1.  This gives $2ba$ to the total
      branching.  Then near $z = \infty$ we have 
      $$
      w^a \approx
      \begin{cases}
        z^b \pm z^b = 2z^b &\text{on half the sheets and}\\
        z^b - z^b(1 - \frac{1}{2}z^{-2b} + \cdots) 
        = \frac{1}{2z^b} + \cdots &\text{ on the other half}.
      \end{cases}
      $$
      %%% page 294
      Thus, $\ds w \approx 2^{1/a} z^{b/a}$ gives a branch point of
      order $a-1$ and $w \approx 2^{1/a} z^{b/a}$ of order $a-1$.
      So
      $$
      V = 2ba + 2(a-1),
      $$
      and
      $$
      ab + a-1 = 2a + g-1 \Rightarrow g = ab-a.
      $$
    \end{itemize}
  \end{itemize}

\item Let $n$ be the degree of $A$ with respect to $w$. Let $S_A$ be
  the Riemann surface of $A$.  Let $S$ be the component of
  $\overline{M}$ which contains all the germs $[f]_a = e(a+t,
  f(a+t))$, assuming $f$ is rational.  By hypothesis, $S \subset
  S_A$.  Clearly, $\pi : S \to \widehat{\CCC}$ is an analytic
  equivalence, so $S$ is compact.  As $S_A$ is connected, $S = S_A$.
  Thus, $\pi : S_A \to \widehat{\CCC}$ is an analytic equivalence.
  But $\pi$ restricted to $S_A$ takes every value $n$ times.  Thus, $n
  = 1$.

  \underline{Alternative solution}:
  Suppose $f$ is rational: $f(z) = \frac{P(z)}{Q(z)}$ in lowest
  terms.  Let
  $$
  B(z, w) = Q(z) w - P(z).
  $$
  Then $A(z, f(z)) = B(z, f(z)) = 0$ for all $z$.  Thus, Lemma 3 on
  p. 121 ????? implies $A$ and $B$ have a common factor.  Since $A$ is
  irreducible and $B$ is linear in $w$, this implies that $A = $
  constant $B$, which shows $A$ has degree 1.

\item We prove something more general.  We have $\pi : S \to
  \widehat{\CCC}$, 
  %%% page 294
  taking every value $n$ times.  Here $n > 1$, since otherwise $\pi$
  is an analytic equivalence and there are \textbf{no} branch points.
  Suppose $S$ has branch points $e_1, \cdots, e_m$ and that $\pi(e_1)
  = \cdots = \pi(e_m) = z_0$.  Since $\widehat{\CCC} - \{z_0\}$ is
  simply connected, the corollary on p. 119 ????? implies there exists
  a meromorphic function $f$ on $\widehat{\CCC} - \{z_0\}$ such that
  $A(z, f(z)) = 0$.  (Actually, the corollary is stated for regions in
  $\CCC$ and holomorphic $f$, but the generalisation to this case is
  easy.)  By familiar estimates, $f$ grows at most like a power of
  $z-z_0$ as $z \to z_0$.  Thus, $f$ is rational and the previous
  point implies $n = 1$, a contradiction.

\item Let $S$ be the Riemann surface of $w^3 + z^3 - 1$.  Then $\pi$
  and $v$ restricted to $S$ satisfies $v^3 + \pi^3 = 1$.  Let
  $$
  v = \frac{1+g}{f},\quad \pi = \frac{1-g}{f},
  $$
  so that $f = \frac{2}{v+\pi}$ and $g = \frac{v-\pi}{v+\pi}$ are
  meromorphic on $S$.  Then
  $$
  1 = (\frac{1+g}{f})^3 + (\frac{1-g}{f})^3 = \frac{2+6g^2}{f^3},
  $$
  so that
  $$
  g^2 = \frac{1}{6}(f^3 - 2).
  $$
  Now let $g_1 = 2\sqrt{6}g$, obtaining
  %%% page 296
  $$
  g_1^2 = 4(f^3 - 2).
  $$
  Now apply the argument which appears on p. 256 and p. 262 ???????,
  concluding that $S$ is analyticall equivalent to the Riemann surface
  of the polynomial $w^2 - 4(z^3 - 2)$.  A little care seems to be
  needed at this step.  Namely, we need to know that $f$ take every
  value 2 times in order to be able to apply item 4 of Theorem 6.5 in
  Topic 6.  We do this by checking that $f$ takes the value 0 two
  times, or that $v + \pi$ takes the value $\infty$ two times.  This
  is esy.  The surface $S$ has 3 smooth sheets over $\infty$, and if
  $\omega = e^{\pi i/3}$ (cube root of $-1$), then on these three
  sheet we have respectively
  $$
    V = \omega \pi (1 - \pi^{-3})^{1/3},\quad
    V = \omega^2 \pi (1 - \pi^{-3})^{1/3},\quad
    V = \omega^3 \pi (1 - \pi^{-3})^{1/3},
  $$
  where $(1-\pi^{-3})^{1/3}$ is the principal determination near $\pi
  = \infty$.  Thus, in the first two case, we have
  $$
  v + \pi = \pi[1 + \omega + \cdots]
  $$
  and
  $$
  v + \pi = \pi[1 + \omega^2 + \cdots]
  $$
  and thus $v + \pi$ takes the value $\infty$ one time on each sheet.
  On the third sheet
  $$
  v + \pi = \pi[1 - (1-\pi^{-3})^{1/3}]
  = \pi[1 - (1-\frac{1}{3}\pi^{-3} + \cdots)]
  = \frac{1}{3\pi^2} + \cdots,
  $$
  %%% page 297
  and thus $v + \pi$ takes the value 0 on this sheet (at the point
  lying over $z = \infty$).  Thus, $v + \pi$ takes the value $\infty$
  exactly two times.

  \medskip
  \underline{Alternative solution}:  Define
  $$
  F = a\frac{1+\pi}{1-\pi}\quad\text{on}\quad S,
  $$
  so also $F$ takes every value 3 times.  Then
  $$
  F - F\pi = a + a\pi
  \Rightarrow \pi = \frac{F-a}{F+a}.
  $$
  Thus,
  $$
  v^3 + \frac{(F-a)^3}{(F+a)^3} \equiv 1
  \Rightarrow ((F+a)v)^3 \equiv 2(3F^2a + a^3).
  $$
  Let 
  $$
  G = \frac{(F+a)v}{(24a)^{1/3}}.
  $$
  Then
  $$
  24aG^3 \equiv 6aF^2 + 2a^3
  \Rightarrow F^2 = 4G^3 - \frac{a^2}{3}.
  $$

\item We write the hypothesis in the following way.  On a disk $\Delta
  \subset \CCC$ are given meromorphic functions $f$ and $g$ such that
  for certain polynomials $A(z,w)$ and $B(z,w)$,
  %%% page 298
  $$
  A(z, f(z)) = B(z,g(z)) = 0, \quad z \in \Delta.
  $$
  We can assume $A$ and $B$ are irreducible.  Let $z_1, \cdots, z_N$
  be the critical points of either $A$ or $B$.  Then define for $z \ne
  z_j$
  $$
  C(z, w) = \prod_{\substack{A(z,\alpha)=0\\B(z,\beta)=0}} (w - \alpha
  - \beta).
  $$
  By the usual symmetry argument, $C$ is a polynomial in $w$ with
  coefficients which are holomorphic functions of $z \in
  \widehat{\CCC} - \{z_1, \cdots, z_N\}$.  By the usual estimates,
  these coefficients have polynomial growth at these exceptional
  points, and thus are rational functions of $z$.  Obviously, $C(z,
  f(z) + g(z)) \equiv 0$ for $z \in \Delta$.

  To do the second part we use the above formula.  The required
  polynomial is therefore
  $$
  \begin{aligned}
    C(z,w) &= (w-z^{1/2}-z^{1/3})(w+z^{1/2}-z^{1/3})(w-z^{1/2}-\omega
    z^{1/3}) (w+z^{1/2}-\omega z^{1/3})\\
    &\qquad (w-z^{1/2}-\omega^2 z^{1/3}) (w+z^{1/2}-\omega z^{1/3}),
  \end{aligned}
  $$
  where $\omega = e^{2\pi i/3}$ and $z^{1/2}$ and $z^{1/3}$ are any
  values of the roots.  After multiplying all these terms together we
  are bound to get a polynomial.  Here is the arithmetic: take the
  terms \#1,3,5 together and likewise \#2,4,6 to obtain
  $$
  \begin{aligned}
    C(z,w) &= [(w-z^{1/2})^3 - z][(w+z^{1/2})^3 - z]\\
    &= [w^3 - 3z^{1/2} w^2 + 3zw - z^{3/2} - z]
    [w^3 + 3z^{1/2} w^2 + 3zw + z^{3/2} - z]\\
    &= (w^3 + 3zw - z)^2 - (3z^{1/2} w^2 + z^{3/2})^2\\
    &= w^6 + (6z-9z)w^4 + (-2z) w^3 + (9z^2-6z^2)w^2
    - 6z^2 w + z^2 - z^3\\
    &= w^6 - 3zw^4 - 2zw^3 + 3z^2 w^2 - 6z^2 w + z^2 - z^3.
  \end{aligned}
  $$
  %%% page 299
\item 
  \begin{enumerate}[a)]
  \item \underline{Parabolic}.  We check item 2 of Proposition 6.41 in
    Topic 6.  If $S$ is the compact Riemann surface and $p \in S$, let
    $D$ be an analytic disk in $S - \{p\}$, and let $u$ be a bounded
    continuous nonnegative function in $S - \{p\} -D$ which is
    harmonic in $S - \{p\} - D^-$ and $\equiv 0$ on $\p D$.  Since $u$
    is bounded near $p$, $u$ has a unique extension to a harmonic
    function in $S - D^-$  As $S - D$ is compact, the maximum
    principle holds and implies that $\sup_{S-D} u = \sup_{\p D} u =
    0$, so $u \le 0$.  Thus, $u \equiv 0$.
  \item \underline{Hyperbolic}.  Choose a nonconstant function $f$
    which is continuous and real-valued on the boundary of the
    analytic disk in question.  By Proposition 6.35, there exists a
    harmonic function $u$ in the Riemann surface minus the closed
    disk, continuous up to the boundary, where it equals $f$.
    Moreover, $u$ is bounded.  Since $f$ is not constant, $u$ is not
    constant.  Thus, $u$ is a bounded nonconstant subharmonic
    function, and we apply Proposition 6.41 item 1.
  \item \underline{Hyperbolic}.  By definition, there is a point $p
    \in S$ and a function $g$ on $S - \{p\}$ satisfying the conditions
    of Definition~6.48 of Topic 6.  Since 
    %%% page 300
    $g \to \infty$ as one approaches $p$, if $A$ is a sufficiently
    large constant the function $u = \min(g,A)$ is superharmonic and
    not constant.  In fact, $u$ is superharmonic on $S$ since $u
    \equiv A$ near $p$.  As $0 < u \le A$, $u$ is also bounded.  Apply
    Proposition 6.41 item 1 of Topic 6.
  \end{enumerate}
\end{enumerate}


\section{References}

\begin{enumerate}
\itemsep=0pt
\parskip=0pt
\item \cite{ahlfors60:_rieman_surfac}
\item Behnke, H., and Sommer, F., Theorie der analytischen Funktionen
  einer komplexnen Veranderlichen, 2nd ed., Springer, Berlin, 1962.
\item \cite{bers58:_rieman_surfac}
\item Chevalley, C., Introduction to the Theory of Algebraic Functions
  of One Variable, Amer. Math. Soc., Providence, R. I., 1951.
\item Cohn, H., Conformal Mapping on Riemann Surfaces, McGraw-Hill,
  1967.
\item Favard, J. Cours d'analyse de l'{\'E}cole Polytechnique, Vol. 3,
  Gauthier-Villars, Paris, 1962.
\item \cite{gunning66:_lectur_rieman_surfac}
\item Heins, M., Complex Function Theory, Academic Press, N. Y., 1968.
\item Hurwitz, A., and Courant, R., Vorlesungen \"uber allgemeine
  Funktionentheorie und elliptische Funktionen.  Geometrische
  Funktionentheorie, 4th ed., Springer, Berlin, 1964.
\item \cite{knopp47:_theor_funct}
\item Mackey, G., Lectures on the Theory of Functions of a Complex
  Variable, Van Nostrand, Princeton, N. J., 1967.
\item \cite{markushevich67:_theor_funct_compl_variab3:da92}
\item Narashimhan, M., et al, Riemann Surfaces, Tata Inst. of
  Fundamental Research, Bombay, 1963.
\item Osgood, W. F., Lehrbuch der Funktionentheorie, Vol. 1, 5th ed.,
  Teubner, Leipzig, 1928.
\item Pfluger, A. Theorie der Riemannschen Fl{\"a}chen, Springer,
  Berlin, 1957
\item Riemann, B., Collected Works, Dover, N. Y., 1953.
\item \cite{springer57:_introd_rieman_surfac:da78}
\item Sto{\"\i}low, S., Le\c{c}ons sur les Principes Topologieques de
  la Th\'eorie des Fonctions Analytiques, Gauthier-Villars, Paris,
  1938.
\item Veech, W., A Second Course in Complex Analysis, Benjamin, N. Y.,
  1967.
\item \cite{weyl64:_concep_rieman_surfac:da111}
\end{enumerate}

%%% BibLaTeX
\xpatchbibmacro{date+extrayear}{%
  \printtext[parens]%
}{%
  \setunit{\addperiod\space}%
  \printtext%
}{}{}
\printbibliography



\end{document}

%%% Local Variables: 
%%% mode: latex
%%% TeX-master: t
%%% End: 
