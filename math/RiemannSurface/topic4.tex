\documentclass[a4paper,11pt]{article}
\usepackage{amsmath}
\usepackage{amsfonts}
\usepackage{verbatim}
\usepackage{url}
\usepackage{framed}
\usepackage{epsfig}
\usepackage{xypic}
\usepackage{enumerate}
\usepackage{textcomp} %\textquotesingle
\usepackage[
%sorting=nyt,
firstinits=true, % render first and middle names as initials
useprefix=true,
maxcitenames=3,
maxbibnames=99,
style=authoryear,
dashed=false, % re-print recurring author names in bibliography
natbib=true,
url=false
]{biblatex}
%%% http://tex.stackexchange.com/questions/12254/biblatex-how-to-remove-the-parentheses-around-the-year-in-authoryear-style
\usepackage{xpatch}
\addbibresource{complex.bib} % run: biber topic1
\usepackage{color}
\usepackage{listings}
\definecolor{gray}{gray}{0.5} 
\definecolor{key}{rgb}{0,0.5,0} 
\lstset{ 
  language=[90]Fortran,
  basicstyle=\ttfamily\small, 
  keywordstyle=\color{blue}, 
  stringstyle=\color{red}, 
  showstringspaces=false, 
  emphstyle=\color{black}\bfseries, 
  emph={[2]True, False, None, self}, 
  emphstyle=[2]\color{key}, 
  emph={[3]from, import, as},
  emphstyle=[3]\color{blue}, 
  upquote=true, 
  morecomment=[s]{"""}{"""}, 
  commentstyle=\color{gray}\slshape, 
  %framexleftmargin=1mm, framextopmargin=1mm, frame=shadowbox, 
  rulesepcolor=\color{blue},
  numbers=left,
  stepnumber=1,
}
\usepackage{enumerate}
\usepackage{tikz}
\usetikzlibrary{lindenmayersystems}
\usetikzlibrary[shadings]


%%% Page Layout
\oddsidemargin=0truecm
\evensidemargin=0truecm
\textwidth=160truemm
\textheight=260truemm
\leftmargin=0truemm
\rightmargin=0truemm
\voffset=-23truemm
\topmargin=0truemm

\newif\iflecturer
%\lecturerfalse
\lecturertrue

\iflecturer
\usepackage{marginnote} % \marginpar
%\usepackage[color]{showkeys}
\definecolor{refkey}{rgb}{1,0,0}
\definecolor{labelkey}{rgb}{1,0,0}
\else
\def\marginpar#1{}
\fi

\iflecturer
\newcommand{\Answer}[1]{\dotfill\underline{\mbox{\hspace{1em}\color{blue}#1}}}
\newcommand{\BoxAns}[1]{\fbox{\color{blue}#1}}
\newcommand{\Reason}[1]{{\par\color{blue}\par{}Reason: #1}}
\else
\newcommand{\Answer}[1]{\dotfill\underline{\mbox{\hspace{1em}\color{white}#1}}}
\newcommand{\BoxAns}[1]{\fbox{\color{white}#1}}
\newcommand{\Reason}[1]{{\par\color{white}\par{}Reason: #1}}
\fi

\newif\ifamsstyle
\amsstylefalse
\newcounter{topic}
\input common
\input symbol
\makeatletter
\newcommand{\Arctan}{{\mathop{\operator@font Arctan}\nolimits}}
\makeatother


%\parindent=0pt
\parskip=1pt
\linespread{1}


\setcounter{topic}{4}

\begin{document}

\title{{\sc Rudiments of Riemann Surfaces\\
    Topic \thetopic{}: Branch Points and Analytic Configurations}}
\author{Author: B. Frank Jones, Jr. (Rice Univ. 1971)\\
Seminar: Dr Liew How Hui (\url{liewhh@utar.edu.my})}
\date{}

\maketitle

%%% E.page 85
Before going to the definitions we give some motivating thoughts.  The
basic thing we want to do is give up the special role played by the
independent variable.  So consider $[f]_a$.  This germ of course is
determined by a meromorphic function $f$ defined near $a$, the
correspondence being written $z\mapsto f(z)$.  We could also consider
$z$ as depending on a complex parameter $t$ and write for example $a +
t \mapsto f(a+t)$ as the correspondence, where now $t$ i near zero.
But also we could write $a + \sin t \mapsto f(a+sin t)$, or $a +
e^{3t}-1 \mapsto f(a+e^{3t}-1)$, etc.  All these would be legitimate
representations of $f$ because the correspondence $t = z$
indicated in each case is a conformal equivalence of a neighbourhood
of $t = 0$ onto a neighbourhood of $z = a$.  Thus, in general we could
consider a \emph{pair} of meromorphic functions
$$
P(t) = a + \rho(t),\quad Q(t) = f(a + \rho(t)),
$$
where $\rho$ is a conformal equivalence of a neighbourhood of 0 onto a
neighbourhood of 0.  Thus each small parameter value $t$ corresponds
uniquely to a value of $z$ ($= P(t)$) near $a$ and the corresponding
value $Q(t)$ of $f$.  We would not like to allow a representation of
the form
%%% E.page 86
$$
P(t) = a + t^2,\quad Q(t) = f(a + t^2),
$$
however.  The reason is basically because two difference values of $t$
can give the same value of $P$.  However, the thing that is
\emph{really} wrong here is that two difference values of $t$ can give
the same value both of $P$ and of $Q$.  This will be an important
observation in our preparation for the definition.

Now consider the Riemann surface in $M$ for the function $z^{1/m}$.
This consists of germs $[f]_a$, $a\ne 0$, such that $f$ is some
determination of $z^{1/m}$ near $a$.  So we have a representation
$$
P(t) = a + t,\quad
Q(t) = (a+t)^{1/m}\quad \text{(some determination)},
$$
for $t$ near 0.  Suppose $Q(0) = \alpha$ so that $\alpha$ is one of
the $m$\textsuperscript{th} roots of $a$.  The we can introduce a new
parameter $\tau$ by the equation
$$
a + t = (\alpha + \tau)^m.
$$
%%% page 87
In fact, $\frac{dt}{d\tau}(\tau=0) = m\alpha^{m-1} \ne 0$.  Thus, we
can also represent $[f]_a$ by the pair of functions
$$
P_1(\tau) = (\alpha + \tau)^m,\quad
Q_1(\tau) = \alpha + \tau.
$$
In our desire to obtain a representation near the branch point, we
would like to use a pair $P(t) = t$, $Q(t) = t^{1/m}$.  Of course,
this is not allowed, but the answer to the dilemma is obtained by just
formally setting $\alpha = 0$ in the above formulas to obtain the pair
$$
P_1(\tau) = \tau^m,\quad
Q_1(\tau) = \tau.
$$
Note how useful such a pair is.  We obtain all the values of $z^{1/m}$
just by using the $m$ different solutions of $\tau^m = z$.  These
yield the same value of $P_1$ (regarded as the independent variable)
for the $m$ different corresponding values of $Q_1$.  Thus in a very
real sense we have introduced a point corresponding to the branch
point 0, and it fits in very well with the regular points near 0.  
%%% page 88
Of course, we again would allow parameter changes as before, so that
the pair
$$
P(t) = \rho(t)^m,\quad
Q(t) = \rho(t),
$$
is regarded as equivalent to the pair $P_1$, $Q_1$ if $\rho$ is a
conformal equivalence, with $\rho(0) = 0$.  And as before we do not
allow a pair such as
$$
P(t) = t^{2m},\quad
Q(t) = t^2,
$$
because different values of $t$ can yield the same values for both $P$
and $Q$.

Finally, we exhibit pairs which we want to imagine as germs at
$\infty$.  If $f$ is meromorphic in a neighbourhood of $\infty$, then
we use the parameter $t$ near 0 and let  the independent variable be
$z = \frac{1}{t}$.  Thus we have
$$
P(t) = t^{-1},\quad Q(t) = f(t^{-1}),
$$
defined for $t$ near 0.  More generally, we can also
%%% page 89
consider $\infty$ as a branch point, yielding for the Riemann surface
for $z^{1/m}$ the pair of functions
$$
P(t) = t^{-m},\quad
Q(t) = t^{-1}.
$$

Now we are ready for the formal development.

\begin{defn}
  \label{def:1}
  A \emph{parameter change} is a function $\rho$ holomorphic in a
  neighbourhood of 0 such that
  $$
  \rho(0) = 0,\quad
  \rho'(0) \ne 0.
  $$
  Equivalently, we would say that $\rho(0) = 0$ and $\rho$ is
  one-to-one in a neighbourhood of 0.
\end{defn}

\begin{defn}
  \label{def:2}
  A \emph{pair} is an ordered couple of functions $P$, $Q$ meromorphic
  in a neighbourhood of 0 such that in a sufficiently small
  neighbourhood of 0
  \begin{enumerate}
  \item $P$ is not constant,
  \item the mapping $t \mapsto (P(t), Q(t))$ is one-to-one.
  \end{enumerate}
\end{defn}

%%% page 90
Examples of pairs have already been given.  Here are some other
examples.  First, $(\sin t,\sin t)$ is a pair, although the points $t
= 0$ and $t = \pi$ give the same value to both $P$ and $Q$.  Second,
$(t^m, t^n)$ is a pair if and only if $m \ne 0$, and either $n=0$ and
$m = \pm 1$ or $n\ne 0$ and $m$ and $n$ are relatively prime.  The
only thing which really needs checking here is that if $m$ and $n$ are
relatively prime, then $(t^m, t^n)$ is a pair.  This follows because
the Euclidean algorithm (see Topic 5) shows there exist integers
$m'$ and $n'$ such that $mm' + nn' = 1$.  Now suppose
$$
(t_1^m, t_1^n) = (t_2^m, t_2^n).
$$
Then $t_1^{mm'} = t_2^{mm'}$ and $t_1^{nn} = t_2^{nn'}$.  Multiplying,
we obtain
$$
t_1^{mm'+nn'} = t_2^{mm'+nn'};\quad
t_1 = t_2.
$$

\begin{defn}
  \label{def:3}
  Let $(P,Q)$ and $(P_1,Q_1)$ be pairs.  Then $(P,Q)$ is
  \emph{equivalent} to $(P_1,Q_1)$ if there exists a parameter change
  $\rho$ such that the equations
  $$
  P_1 = P\circ \rho,\quad
  Q_1 = Q\circ \rho,
  $$
  %%% page 91
  are valid in a neighbourhood of 0.  If $(P,Q)$ is equivalent to
  $(P_1, Q_1)$, this will be written $(P,Q) \sim (P_1,Q_1)$.
\end{defn}

\begin{ques}
  \label{problem:3}
  Suppose $(P,Q)$ and $(P_1,Q_1)$ are pairs.  Prove that if there
  exits a function $\rho$ holomorphic in a neighbourhood of 0 such
  that $\rho(0) = 0$ and $P_1 = P\circ \rho$, $Q_1 = Q \circ \rho$
  near 0, then $\rho$ must be a parameter change.  Also, $\rho$ is
  uniquely determined (near 0) by these equations.
\end{ques}

\begin{lem}
  \label{lem:1}
  $\sim$ is an equivalence relation.
\end{lem}

\begin{mdframed}
  \textbf{Proof}:
  \begin{description}
  \item[Reflexive:] $(P,Q) \sim (P,Q)$ since $\rho(t) = t$ works.
  \item[Symmetric:] If $(P,Q) \sim (P_1,Q_1)$ and $\rho$ satisfies
    $$
    P_1 = P \circ \rho,\quad Q_1 = Q \circ \rho,
    $$
    then also
    $$
    P = P_1 \circ \rho^{-1},\quad Q = Q_1 \circ \rho^{-1},
    $$
    near 0 and $\rho^{-1}$ is holomorphic, proving $(P_1,Q_1) \sim
    (P,Q)$.
    %%% page 92
  \item[Transitive:] If $(P,Q) \sim (P_1,Q_1)$ and $(P_1,Q_1) \sim
    (P_2,Q_2)$ and we have parameter changes $\rho$ and $\rho_1$
    satisfying $P_1 = P\circ \rho$, $P_2 = P_1 \circ \rho_1$, likewise
    for the $Q$'s, then
    $$
    P_2 = P_1 \circ \rho_1 = P\circ \rho \circ \rho_1,\quad
    Q_2 = Q_1 \circ \rho_1 = Q\circ \rho \circ \rho_1,
    $$
    and $\rho\circ \rho_1$ is also a parameter change, showing that
    $$
    (P,Q) \sim (P_2,Q_2).
    $$
  \end{description}
\end{mdframed}

\begin{defn}
  \label{def:4}
  An equivalence class of pairs is a \emph{meromorphic element}.  The
  meromorphic element containing a pair $(P,Q)$ is designated
  $e(P,Q)$.  Thus,
  $$
  e(P,Q) = \{(P_1,Q_1) ~:~ (P_1,Q_1)\text{ is a pair and }(P,Q) \sim
  (P_1, Q_1)\}.
  $$
  Define $\overline{M}$ to be the collection of all meromorphic
  elements.
\end{defn}

%%% page 93
\begin{defn}
  \label{def:5}
  The two functions $\pi : \overline{M} \to \widehat{\CCC}$, $v :
  \overline{M} \to \widehat{\CCC}$, are give by the formulas
  $$
  \pi(e(P,Q)) = P(0),\quad
  v(e(P,Q)) = Q(0).
  $$
\end{defn}
We simply remark that $\pi$ and $v$ are well defined since if $(P,Q)
\sim (P_1,Q_1)$, then clearly $P(0) = P_1(0)$ and $Q(0) = Q_1(0)$.
The number $\pi(e(P,Q))$ is sometimes called the \emph{centre} of
$e(P,Q)$, and $v(e(P,Q))$ is called the \emph{value} of $e(P,Q)$.

Another remark which is simple but useful is that if $(P,Q)$ is a pair
and $\rho$ is a parameter change, then $(P\circ \rho, Q\circ \rho)$ is
also a pair and is therefore equivalent to $(P,Q)$.

As has been indicated in the motivation for $\overline{M}$, we
definitely wish to consider $M \subset \overline{M}$ in a natural
manner.  Of course, the way we do this is to define a function on $M$
with values in $\overline{M}$ and prove this function is one-to-one.
This means that each element of $M$ is identified with an element of
$\overline{M}$ in a one-to-one fashion, and the identification is
this: to a germ $[f]_a$ we associate the meromorphic element $e(a+t,
f(a+t))$.  Now we prove this is a one-to-one function.  Suppose
$[g]_b$ is another germ and that $e(a+t, f(a+t)) = e(b+t, g(b+t))$.
This means that there exists a parameter change $\rho$ such that
%%% page 94
for $t$ near 0
$$
a + t = b + \rho(t),\quad
f(a+t) = g(b+\rho(t)).
$$
The first equation implies $a = b$ and $\rho(t) = t$, and then the
second equation implies $f(a+t) = g(a+t)$ for $t$ near 0.  Thus,
$[f]_a = [g]_b$.

We now begin to topologise $\overline{M}$, then make $\overline{M}$ a
surface, then a Riemann surface.  We remakr that as \textbf{sets} the
inclusion $M \subset \overline{M}$ is an isomorphism of $M$ onto its
image in $\overline{M}$ (this we have just proved), and we will
eventually see that as Riemann surfaces this is still true; the
mapping of $M$ onto its image in $\overline{M}$ will be seen to be an
analytic equivalence.

Before beginning this program we wish to spell out a notational
convenience.  Frequently we shall write
$$
e(P,Q) = e(P(t),Q(t))
$$
to designate a meromorphic element.  We have already used this type of
notation in the discussion of $M \subset \overline{M}$, where we wrote
$e(a + t, f(a+t))$.  Of course, this means $e(P,Q)$, where $P(t) =
a+t$, $Q(t) = f(a+t)$, but it would seem pedantic to be so strict
with the notation and certainly would be confusing.
%%% page 95
We couldn't use notation such as $e(t^2,t)$.  In order to attempt to
be consistent we shall try to use $t$ or the dummy variable in the
expression such as the above.  Thus, for example,
$$
e(P(t_0 + t), Q(t_0 + t))
$$
stands for the meromorphic element $e(P_1,Q_1)$, where for small $t$
$$
P_1(t) = P(t_0 + t),\quad Q_1(t) = Q(t_0 + t).
$$

\begin{defn}
  \label{def:6}
  Let $(P,Q)$ be a pair and assume $P$ and $Q$ are both meromorphic on
  a disk $\Delta$.  Then let $U(P,Q,\Delta)$ be the collection of
  meromorphic elements according to the formula
  $$
  U(P,Q,\Delta) = \{e(P(t_0 + t), Q(t_0 + t)) ~:~ t_0 \in \Delta\}.
  $$
  We have assumed $\Delta$ sufficiently small that the mapping $t
  \mapsto (P(t), Q(t))$ is one-to-one on $\Delta$
  (cf. Definition~\ref{def:2}).
\end{defn}

Note that this latter assumption each couple $(P(t_0+t),Q(t_0+t))$ for
$t_0 \in \Delta$ is indeed a pair, and thus $U(P,Q,\Delta)$ makes
sense.

%%% page 96
The sets $U(P,Q,\Delta)$ will form a neighbourhood basis of $e(P,Q)$
when $\Delta$ is allowed to vary over all sufficiently small disks
centred at 0.  Since $U(P,Q,\Delta)$ is not defined in terms of the
equivalence class $e(P,Q)$ but rather in terms of the particular pair
$(P, Q) \in e(P,Q)$, we shall need a lemma comparing two
neighbourhoods constructed with different but equivalent pairs.

\begin{lem}
  \label{lem:2}
  Suppose $(P,Q) \sim (P_1,Q_1)$.  If $U(P,Q,\Delta)$ is defined, then
  there exists a disk $\Delta_1$ centred at 0 such that
  $$
  U(P_1,Q_1,\Delta_1) \subset U(P,Q,\Delta).
  $$
\end{lem}

\begin{mdframed}
  \textbf{Proof}: By definition there exists a parameter change $\rho$
  such that $P_1 = P \circ \rho$, $Q_1 = Q\circ \rho$ in a disk
  $\Delta_1$ centred at 0.  We choose $\Delta_1$ sufficiently small
  that $\rho'$ never vanishes in $\Delta_1$ and $\rho(\Delta_1)
  \subset \Delta$, and also that $U(P_1,Q_1, \Delta_1)$ is defined.
  Now let $e \in U(P_1,Q_1,\Delta_1)$.  Then $e = e(P_1(t_0 + t),
  Q_1(t_0 + t))$ for some $t_0 \in \Delta_1$.  Now
  $$
  P_1(t_0 + t) = P(\rho(t_0 + t)) = P(\rho(t_0) + \rho_1(t)),
  $$
  where
  %%% page 97
  $$
  \rho_1(t) = \rho(t_0+t) - \rho(t_0).
  $$
  Note that $\rho_1(0) = 0$ and $\rho_1'(0) = \rho'(t_0) \ne 0$ and
  thus $\rho_1$ is a parameter change.  Since also
  $$
  Q_1(t_0 + t) = Q(\rho(t_0) + \rho_1(t)),
  $$
  we conclude that $(P_1(t_0 + t), Q_1(t_0 + t)) \sim
  (P(\rho(t_0) + t), Q(\rho(t_0) + t))$.  Thus,
  $$
  e = e(P(t\rho(t_0) + t), Q(t\rho(t_0) + t)) \in U(P,Q,\Delta)
  $$
  and this proves the lemma.
\end{mdframed}

\begin{propn}
  \label{propn:1}
  The collection of sets $U(P,Q,\Delta)$ is a system of basic
  neighbourhoods for a topology on $\overline{M}$.
\end{propn}

\begin{mdframed}
  \textbf{Proof}:  Clearly any point $e(P,Q)$ in $\overline{M}$
  belongs to $U(P,Q,\Delta)$ and we have two things to check:
  \begin{enumerate}
  \item Suppose there are given $U(P,Q,\Delta)$ and $U(P_1, Q_1,
    \Delta_1)$, basic sets defined in terms of pairs $(P,Q) \sim (P_1,
    Q_1)$.  By Lemma~\ref{lem:2} there exists a disk $\Delta_2$
    centred at 0 such that $\Delta_2 \subset \Delta_1$ and $U(P_1,
    Q_1, \Delta_2) \subset U(P,Q,\Delta)$.  Thus,
    %%% page 98
    $$
    U(P,Q,\Delta) \cap U(P_1, Q_1, \Delta_1) \supset 
    U(P_1, Q_1, \Delta_2).
    $$
  \item Suppose $e(\widetilde{P},\widetilde{Q}) \in U(P,Q,\Delta)$.
    Then for a point $t_0 \in \Delta$, $(\widetilde{P},\widetilde{Q})
    \sim (P(t_0 + t), Q(t_0 + t))$.  If $\Delta'$ is the disk centred
    at 0 whose radius is the radius of $\Delta$ minus $|t_0|$, then it
    is clear that
    $$
    U(P(t_0+t), Q(t_0 + t), \Delta') \subset U(P,Q,\Delta).
    $$
    By Lemma 2 there exists a disk $\widetilde{\Delta}$ centred at 0
    such that $U(\widetilde{P},\widetilde{Q},\widetilde{\Delta})
    \subset U(P(t_0 + t), Q(t_0 + t), \Delta')$.  Thus
    $$
    U(\widetilde{P},\widetilde{Q},\widetilde{\Delta})
    \subset U(P,Q,\Delta).
    $$
  \end{enumerate}
\end{mdframed}

Before proving that $\overline{M}$ is a Hausdorff space, we introduce
some \textbf{normal} representations for meromorphic elements.
Suppose we consider an element $e(P,Q)$.  Since a particularly simple
form $P$ has in terms of a judiciously chosen chart for the Riemann
surface, in the present framework we conclude that there exists a
parameter change $\rho$ such that near $t=0$
$$
P(\rho(t)) = 
\begin{cases}
  P(0) + t^m, & \text{if }P(0) \ne \infty,\\
  t^{-m}, & \text{if }P(0) = \infty.
\end{cases}
$$
Thus, if $Q_1 = Q \circ \rho$, we see that
%%% page 99
$$
e(P,Q) = 
\begin{cases}
  e(P(0) + t^m, Q_1), & \text{if }P(0) \ne \infty,\\
  e(t^{-m}, Q_1), & \text{if }P(0) = \infty.
\end{cases}
$$
Note that the integer $m$ is well defined, being the multiplicity of
$P$.  For if $P_1$ is derived from $P$ by means of any parameter
change, then $P_1$ has the same multiplicity $m$.

A point of $\overline{M}$ of the form $e(a+t^m, Q)$ or $e(t^{-m}, Q)$
is called a \emph{branch point of order $m-1$}.

It should be remarked that the normal form is not unique if $m > 1$.
In fact, if $w$ is any root of $w^m = 1$, then for example
$$
(a + t^m, Q(t)) \sim (a + t^m, Q(wt))
$$
as the parameter change $t \to wt$ shows.  Thus, e.g. $e(t^2, t) =
e(t^2, -t)$.  This the only possible type of ambiguity.

\begin{propn}
  \label{propn:2}
  $\overline{M}$ is a Hausdorff space.
\end{propn}

\begin{mdframed}
  \textbf{Proof}:  The fact that $M$ is Hausdorff is an obvious and
  immediate consequence of the uniqueness of analytic continuation.
  The present proof is surprisingly more involved.  Suppose that
  $e(P,Q)$ and $e(P_1,Q_1)$ are not contained in disjoint
  neighbourhoods.  We can assume both these elements to be in normal
  representation, so that
  %%% page 100
  $$
  P(t) = a + t^m \quad \text{or}\quad t^{-m}
  $$
  and
  $$
  P_1(t) = b + t^n \quad \text{or}\quad t^{-n}.
  $$
  Let $\Delta_k$ be the disk centred at 0 with radius $k^{-1}$.  Then
  for any sufficiently large $k$ the neighbourhoods $U(P,Q,\Delta_k)$
  and $U(P_1,Q_1,\Delta_k)$ have a common point, say
  $$
  e(P(s_k + t), Q(s_k + t)) = e(P_1(t_k + t), Q_1(t_k + t)),
  $$
  where $s_k, t-k \in \Delta_k$.  In particularly $\pi$ and $v$ have
  the same value at these two points, so
  $$
  P(s_k) = P_1(t_k),\quad Q(s_k) = Q_k(t_k).
  $$
  If ever $s_k = t_k = 0$, then $e(P,Q) = e(P_1, Q_1)$, which i swhat
  we are trying to prove.  Thus, we can assume $s_k$ or $t_k \ne 0$.
  Now letting $k \to \infty$ implies first $P(0) = P_1(0)$, so we have
  either
  $$
  P(t) = a + t^m \quad\text{and}\quad P_1(t) = a + t^n,
  $$
  %%% page 101
  or
  $$
  P(t) = t^{-m} \quad\text{and}\quad P_1(t) = t^{-n}.
  $$
  We shall eventually prove that $m=n$, so then $P=P_1$.  Also, we see
  immediately that in either case $s_k^m = t_k^n$.

  Choose arbitrary $n$\textsuperscript{th} roots of $s_k$, say
  $$
  \sigma_k^n = s_k.
  $$
  Then
  $$
  \left(\frac{\sigma_k^m}{t_k}\right)^n
  = \frac{s_k^m}{t_k^n}  = 1.
  $$
  Since there are only $n$ choices for each number $\frac{\sigma_k^m}{
    t_k}$, we can choose a subsequence of $k$'s such that these
  numbers are all equal to a common $n$\textsuperscript{th} root of 1,
  say $w^{-1}$.  Renaming this subsequence, it follows that we can
  assume 
  $$
  w \sigma_k^m = t_k.
  $$
  Then
  $$
  Q(\sigma_k^n) = Q_1(w\sigma_k^m).
  $$
  Since this equation is valid for $\sigma_k \to 0$, $\sigma_k \ne 0$,
  we can now apply the uniqueness of analytic continuation to conclude
  $$
  Q(\sigma^n) = Q_1(w\sigma^m),\quad s\text{ small}.
  $$
  Thus, note that
  $$
  (P(s^n), Q(\sigma^n)) \equiv 
  (P_1(w\sigma^m), Q_1(w\sigma^m)),\quad s\text{ small}.
  $$
  Since the mapping $t \mapsto (P(t), Q(t))$ is one-to-one (small
  $t$), then the mapping $s \mapsto (P(s^n), Q(s^n))$ is exactly
  $n$-to-one (small $s$).  Likewise, the mapping $s \mapsto (P_1(w
  s^m), Q_1(ws^m))$ is exactly $m$-to-one.  Since these mappings are
  identical, we must have $m = n$!

  Thus, $Q(s^m) \equiv Q_1(ws^m)$ and we conclude
  $$
  Q(t) \equiv Q_1(wt).
  $$
  Since $P(t) \equiv P_1(wt)$ (as $w^m = 1$), we have
  $$
  (P, Q) \sim (P_1, Q_1),
  $$
  the parameter change being just $\rho(t) = wt$.  Thus,
  $$
  e(P, Q) = e(P_1, Q_1).
  $$
\end{mdframed}


Now we have certain obvious charts for $\overline{M}$.  Namely, we
define the mapping
$$
\varphi : U(P,Q,\Delta) \to \Delta
$$
by the formula
$$
\varphi(e(P(t_0 + t), Q(t_0 + t))) = t_0,
\quad\text{or}\quad
\varphi^{-1}(t_0) = e(P(t_0 + t), Q(t_0 + t)).
$$
The definition of $\varphi^{-1}$ is of course clear enough, but for
$\varphi$ to be well defined something must be checked.  Namely, if
two points in $U(P,Q,\Delta)$ are the same, then they correspond to
the same $t_0$.  Another way of saying this is that $\varphi^{-1}$ is
one-to-one.  But if $\varphi^{-1}(t_0) = \varphi^{-1}(t_0')$, then
$\pi(\varphi^{-1}(t_0)) = \pi(\varphi^{-1}(t_0'))$ and
$v(\varphi^{-1}(t_0)) = v(\varphi^{-1}(t_0'))$, so hat $P(t_0) =
P(t_0')$ and $Q(t_0) = Q(t_0')$.  Since the mapping $t \mapsto (P(t),
Q(t))$ is one-to-one for $t \in Delta$, this implies $t_0 = t_0'$.
This shows that at least $\varphi$ maps $U(P,Q,\Delta)$ to $\Delta$ in
a bijective fashion (one-to-one and onto).

It is now easy to see that $\varphi$ is a homeomorphism. In fact, if
$e_0 = e(P(t_0 + t), Q(t_0 + t))$ is any point in $U(P,Q,\Delta)$,
then a neighbourhood basis for $e_0$ consists
%%% page 104
of sets $U(P(t_0 + t), Q(t_0 + t), \Delta') = \{e(P(t_0 + t_1 + t), 
Q(t_0 + t_1 + t)) ~:~ t_1 \in \Delta'\}$, where $\Delta'$ is a
sufficiently small disk centred at 0.  The image of a set like this
under the mapping $\varphi$ is precisely $\{t_0 + t_1 ~:~ t_1 \in
\Delta'\}$, and these sets form a neighbourhood basis for the point
$t_0 \in \Delta$.  Thus, $\varphi$ induces a one-to-one correspondence
between a neighbourhood basis for $e_0$ and a neighbourhood basis for
$\varphi(e_0)$.  Thus, $\varphi$ is a homeomorphism.  Thus,
$\overline{M}$ is a surface.

\begin{propn}
  \label{propn:3}
  The given charts form an analytic atlas for $\overline{M}$.  Thus,
  $\overline{M}$ is a Riemann surface.
\end{propn}

\begin{mdframed}
  \textbf{Proof}: Suppose two neighbourhoods $U(P,Q,\Delta)$ and
  $U(P_1, Q_1, \Delta_1)$ meet.  Let $\varphi$ and $\varphi_1$ be the
  respective charts.  Let $e_0$ be a common point in these
  neighbourhoods and $\varphi(e_0) = t_0$, $\varphi_1(e_0) = t_1$.  We
  need to check the analyticity of $\varphi_1 \circ \varphi^{-1}$ in a
  neighbourhood of $t_0$.  Now by definition
  $$
  e_0 = e(P(t_0 + t), Q(t_0 + t)) 
  = e(P_1(t_1 + t), Q_1(t_1 + t)).
  $$
  Therefore there exists a parameter change $\rho$ such that for $t$
  near 0
  $$
  P(t_0 + t) = P_1(t_1 + \rho(t)),\quad
  Q(t_0 + t) = Q_1(t_1 + \rho(t)).
  $$
  For $z$ near $t_0$ we have
  $$
  \varphi^{-1}(z) = e(P(z+t), Q(z+t)).
  $$
  We need to express this in terms of $P_1$ and $Q_1$, rather than $P$
  and $Q$.  So we compute as follows:
  $$
  P(z + t) = P(t_0 + (z-t_0 +t))
  = P_1(t_1 + \rho(z-t_0 +t))
  = P_1(t_1 + \rho(z-t_0) + [\rho(z-t_0+t)-\rho(z-t_0)])
  $$
  The function
  $$
  \rho_1(t) = \rho(z-t_0+t)-\rho(z-t_0)
  $$
  satisfies $\rho_1(0) = 0$ and $\rho_1'(0) = \rho'(z-t_0)$.  Thus,
  $\rho_1'(0) \ne 0$ if $z-t_0$ is sufficiently small, so $\rho_1$ is
  also a parameter change.  Since the same computation is valid for
  $Q$ and $Q_1$, we obtain
  $$
  \varphi^{-1}(z) = e(P_1(t_1 + \rho(z-t_0) + t),
  Q_1(t_1 + \rho(z-t_0) + t)).
  $$
  Therefore,
  $$
  \varphi_1 \circ \varphi^{-1}(z) = t_1 + \rho(z-t_0),
  $$
  which holds for all $z$ sufficiently near $t_0$. Since
  %%% page 106
  $\rho$ is holomorphic, we have now proved that $\varphi_1 \circ
  \varphi^{-1}$ is holomorphic near $t_0$.
\end{mdframed}

Now we list various properties of $\overline{M}$.

\begin{propn}
  \label{propn:4}
  $\overline{M} - M$ is a discrete set.  That is, each point of
  $\overline{M}$ has a neighbourhood consisting only of itself and
  points of $M$.
\end{propn}

\begin{mdframed}
  \textbf{Proof}: Suppose $e(P,Q) \in \overline{M}$ and that $\Delta$
  is a disk centred at 0 such that $U(P,Q,\Delta)$ is defined and for
  $t \in \Delta - \{0\}$, $P'(t) \ne 0$, $P(t) \ne \infty$.  Then
  $U(P,Q,\Delta)$ is a neighbourhood of $e(P,Q_)$ having the required
  properties. Indeed, if $e_0 \in U(P,Q,\Delta)$ but $e_0 \ne e(P,Q)$,
  then for some $t_0 \in \Delta - \{0\}$
  $$
  e_0 = e(P(t_0 + t), Q(t_0 + t)).
  $$
  Now $t \to P(t_0 + t)$ is holomorphic and one-to-one near $t = 0$,
  so
  $$
  \rho(t) = P(t_0 + t) - P(t_0)
  $$
  is a coordinate change.  Thus, if $a = P(t_0)$
  $$
  e_0 = e(P(t_0) + t, Q(t_0 + \rho^{-1}(t)))
  = [Q(t_0 + \rho^{-1}(z-a)]_a,
  $$
  %%% page 107
  that is, $e_0$ is the germ of the meromorphic function $z \mapsto
  Q(t_0 + \rho^{-1}(z-a))$ at $a$.  Thus, $e_0 \in M$.
\end{mdframed}

\begin{ques}
  \label{problem:4}
  Prove that $\pi$ and $v$ are meromorphic functions on
  $\overline{M}$.  Prove that the mapping which identifies $M$ as a
  subset of $\overline{M}$ is an analytic equivalence of $M$ onto its
  image in $\overline{M}$.
\end{ques}

The second half of this problem completely justifies regarding $M$ as
a subset of $\overline{M}$.  Of course, we previously could only
consider $M \subset \overline{M}$ as sets, but now also as Riemann
surfaces.  Also, $M$ is open in $\overline{M}$, as is implied by
Proposition~\ref{propn:4}.

\begin{propn}
  \label{propn:5}
  If $e \in \overline{M}$, then $e$ is a branch point of order $m-1$
  if and only if $m_\pi(e) = m$.
\end{propn}

\begin{mdframed}
  \textbf{Proof}: Suppose $e = e(a+t^m, Q)$, and $\varphi : U(a+t^m,
  Q, \Delta) \to \Delta$ is a related chart.  Then
  $$
  \pi \circ \varphi^{-1}(t_0) = a + t_0^m,
  $$
  so that the multiplicity of $\pi \circ \varphi^{-1}$ at 0 is $m$.  A
  similar computation applied if $e = e(t^{-m}, Q)$.
\end{mdframed}

%%% page 108
\begin{propn}
  \label{propn:6}
  Any two points in the same component of $\overline{M}$ can be joined
  by a path in $\overline{M}$ every point of which except the initial
  and terminal points lies in $M$.
\end{propn}

\begin{mdframed}
  \textbf{Proof}: 
  This a topological consequence of Proposition~\ref{propn:4}.
  Suppose $\gamma$ is a path.  Since $\gamma(I)$ is compact and
  $\overline{M} - M$ is discrete and closed, $\gamma(I) \cap
  (\overline{M} - M)$ is finite.  Let $e_0$ be a point in this set
  which is not an initial or terminal point of $\gamma$.  Let $t_0$ be
  the smallest and $t_1$ the largest numbers $t$ in $(0,1)$ such that
  $\gamma(t) = e_0$.  Choose a neighbourhood $U$ of $e_0$ and a chart
  $\varphi : U \to \Delta$, where $\Delta \subset \CCC$ is a disk.

  \begin{center}
  \begin{minipage}{0.4\textwidth}
    ~
    \vspace*{4cm}
  \end{minipage}%
  \begin{minipage}{0.5\textwidth}
    Choose $0 < t_0' < t_0 \le t_1 < t_1' < 1$ such that $\gamma(t_0')
    \in U$, $\gamma(t_1') \in U$.  Choose a path $\delta$ in $\Delta$
    joining $\varphi \circ \gamma(t_0')$ and $\varphi \circ
    \gamma(t_1')$ and missing $\varphi(e_0)$.  Then let
    $$
    \gamma_1(t) =
    \begin{cases}
      \gamma(t), & 0 \le t \le t_0',\\
      \varphi^{-1} \circ \delta(\frac{t-t_0'}{t_1'-t_0'}),
      & t_0' \le t \le t_1',\\
      \gamma(t), & t_1' \le t \le 1.
    \end{cases}
    $$
  \end{minipage}
  \end{center}
  %%% page 109
  Then $\gamma_1$ is a path in $\overline{M}$ having the same end
  points as $\gamma$, but $\gamma_1$ does not pass through $e_0$.
  Since we need to remove only finitely many points like $e_0$, the
  theorem follows.
\end{mdframed}

\begin{propn}
  \label{propn:7}
  There is a natural one-to-one correspondence between components of
  $M$ and components of $\overline{M}$.  Namely, if $s$ is a component
  of $M$, $S$ is contained in a unique component of $\overline{M}$,
  which is the closure of $S$ in $\overline{M}$; conversely, if
  $\overline{S}$ is a component of $\overline{M}$, then $\overline{S}$
  contains a unique component of $M$, which is $\overline{S} \cap M$.
\end{propn}

\begin{myproof}
  Let $S$ be a component of $M$.  Certainly $s$ is contained in a
  unique component $\overline{S}$ of $\overline{M}$.  Since components
  are closed, $\overline{S}$ contains the closure of $S$.  But also
  any $e \in \overline{S}$ can be joined to a fixed $e_0 \in S$ by a
  path $\gamma$ such that $\gamma(0) = e_0$, $\gamma(1) = e$,
  $\gamma([0,1)) \subset M$, by Proposition~\ref{propn:6}.  Since
  $\gamma([0,1))$ is connected and $\gamma(0) \in S$, it follows that
  $\gamma([0,1)) \subset S$ (since $S$ is a component).  Thus $e$ is a
  limit point of $S$ and thus belongs to the closure of $S$, showing
  $\overline{S}$ is contained in the closure of $S$.

  Conversely, let $\overline{S}$ be a component of $\overline{M}$.  If
  a component of $M$ is contained in $\overline{S}$, this component is
  also contained in $\overline{S} \cap M$.  Thus, it suffices to show 
  %%% page 110
  $\overline{S} \cap M$ is a component of $M$.
  Proposition~\ref{propn:6} shows that $\overline{S} \cap M$ is
  connected, for if $e$ and $e$ (????) are in $\overline{S} \cap M$,
  they can be joined by a path in $M$.  Since the end points are in
  $\overline{S}$, the entire path is in $\overline{S}$ ($\overline{S}$
  is a \textbf{component}) and thus is in $\overline{S} \cap M$.  And
  if a point of $M$ can be joined by a path in $M$ to a point in
  $\overline{S} \cap M$, that point must be in $\overline{S}$ and thus
  in $\overline{S} \cap M$.  Thus $\overline{S} \cap M$ is a component.
\end{myproof}

\begin{defn}
  \label{def:7}
  A component of $\overline{M}$ is called an \emph{analytic
    configuration}.  This is a translation of the term ``analytische
  Gebilde'' used by Weyl.  Another term is \emph{analytic entity}.
\end{defn}

\begin{defn}
  \label{def:8}
  Let $f$ be a meromorphic function in a neighbourhood of a point $a
  \in \CCC$.  The \emph{Riemann surface of $f$} is the analytic
  configuration containing $[f]_a$.
\end{defn}

This definition is finally the complete idea which was begun in
Definition 3.17 of Topic 3.  We have now included the branch points in
the surface and nothing else needs to be added.

It is important to observe that the nice analytic continuation or
lifting properties of $M$ do not hold in $\overline{M}$.  For example,
Proposition~\ref{propn:2}, the unique lifting theorem, of Topic 3
would be false if phrased for $\overline{M}$.  Just consider a
neighbourhood of a branch point to see this.  For example, let
%%% page 111
$$
\widetilde{\gamma}_1(s) = e((s+t)^2, s+t), \quad
\widetilde{\gamma}_2(s) = e((|s|+t)^2, |s|+t), \quad
-1 \le s \le 1.
$$
Then $\pi \circ \widetilde{\gamma}_1(s) = \pi \circ
\widetilde{\gamma}_2(s) = s^2$, but
$$
\begin{aligned}
\widetilde{\gamma}_1(s) &= \widetilde{\gamma}_2(s) \quad \text{for } 0
\le s \le 1,\\
\widetilde{\gamma}_1(s) &\ne \widetilde{\gamma}_2(s) \quad \text{for } -1
\le s < 0.
\end{aligned}
$$
We thus are led to a pictorial idea of branch point: two liftings of a
given path in $\CCC$ which begin at a common point in $\overline{M}$
must be the same until a branch point is reached.  But then the
liftings can branch into several different paths in $\overline{M}$.

If $f$ is a meromorphic function which is not one-to-one, it of course
has no inverse.  But we can easily consider the Riemann surface
inverse to its Riemann surface, as follows.

\begin{propn}
  \label{propn:8}
  Let $S$ be an open connected subset of $\overline{M}$ such that $v$
  is not constant on $S$.  Then the mapping
  %%% page 112
  $$
  i : S \to \overline{M}\qquad (i\text{ for ``inverse''})
  $$
  defined by
  $$
  i(e(P,Q)) = e(Q,P)
  $$
  is an analytic equivalence of $S$ with $i(S)$.
\end{propn}

\begin{mdframed}
  \textbf{Proof}: First, note that $i$ is well defined.  This depends
  on the obvious fact that if $(P,Q) \sim (P_1, Q_1)$, then $(Q,P)
  \sim (Q_1, P_1)$.  Now we prove $i$ is analytic.  To do this we
  introduce charts in the canonical way:
  $$
  \varphi : U(P,Q,\Delta) \to \Delta,\quad
  \psi : U(Q,P,\Delta) \to \Delta,
  $$
  where
  $$
  \varphi^{-1}(t_0) = e(P(t_0 + t), Q(t_0 + t)),\quad
  \psi^{-1}(t_0) = e(Q(t_0 + t), P(t_0 + t)).
  $$
  Then
  $$
  \psi \circ i \circ \varphi^{-1}(t_0)
  = \psi(e(Q(t_0 + t), P(t_0 + t))) = t_0,
  $$
  so that trivially $\psi \circ i \circ \varphi^{-1}$ is analytic.
  Thus $i$ is analytic.

  %%% page 113
  Since $i$ is analytic, $i(S)$ is an open connected subset of
  $\overline{M}$, and $\pi$ is not constant on $i(S)$ since $v$ is not
  constant on $S$.  Furthermore, $i : i(S) \to S$ is analytic by what
  we have already proved and $i \circ i = $identity.  Thus, the
  inverse of $i$ is analytic.
\end{mdframed}

We now give an interesting and rather surprising application of some
these ideas.

\begin{defn}
  \label{def:9}
  Let $f$ be meromorphic on an open set $D \subset \CCC$ and let $w
  \in \widehat{\CCC}$.  Then $w$ is an \emph{asymptotic value} of $f$
  is there exists a path of the form
  $$
  \gamma : [0, \sigma) \to D\qquad (\text{where }0 < \sigma \le \infty)
  $$
  such that
  $$
  \lim_{s \to \sigma} f(\gamma(s)) = w
  $$
  and $\gamma \to \p D$, meaning that for every compact set $K \subset
  D$, there exists $s_0$ such that $\gamma((s_0, \sigma)) \subset D - K$.
\end{defn}

\begin{thm}
  \label{thm:1}
  If $f$ is holomorphic on an open set $D \subset \CCC$, then there
  exists an asymptotic value of $f$.
\end{thm}

%%% page 114
\begin{myproof}
  Clearly it suffices to treat the case in which $D$ is connected and
  $f$ is not constant on $D$.  Now define
  $$
  S = \{ [f]_a ~:~ a \in D\}.
  $$
  Clearly, $S$ is an open subset of the sheaf of germs $M$, and the
  mapping $\pi : S \to D$ is an analytic equivalence.  Now we
  complicate the situation by regarding $S \subset \overline{M}$ and
  letting $T = i(S)$ in the sense of Proposition~\ref{propn:8}.  Then
  $v : T \to D$ is an analytic equivalence since $v = \pi \circ
  i^{-1}$.  According to the definition of $M \subset \overline{M}$,
  we have 
  $$
  T = \{e(f(a+t), a+t) ~:~ a \in D\}.
  $$
  Thus, $f \circ v = \pi$ on $T$.  That is, we have a commutative
  diagram
  $$
  \xymatrix{
    & T\ar[d]^{\pi} \ar[dl]_{v} \\
    D \ar[r]^f & C
    }
  $$
  %%% page 115
  and $v$ represents the multiple-valued inverse of $f$.  Note that a
  point $e(f(a+t), a+t)$ is a branch point (order at least 1) iff
  $f'(a) = 0$, and this holds only for a discrete and thus countable
  set of points $a \in D$.  Let $E = \{a \in D : f'(a) = 0\}$ and note
  that $f(E)$ is a countable subset of $\CCC$.  Choose arbitrarily
  $a_0 \in D - E$.  Since $f(E)$ is countable and there are
  uncountably many rays from $f(a_0)$ to $\infty$, it follows that
  there exists a ray from $f(a_0)$ to $\infty$ contains no points of
  $f(E)$.  Let this ray be represented by a path $\alpha : [0,\infty)
  \to \CCC$, so that $\alpha(0) = f(a_0)$, $\alpha(s) \notin f(E)$,
  $\lim_{s\to \infty} \alpha(s) = \infty$.

  Now we consider the process of lifting $\alpha$ to $T$ by the
  mapping $\pi$.  Note that if $e \in T$ and $\pi (e) \notin f(E)$,
  then $v(e) \notin E$ and thus $e \in M$.  Thus, lifting $\alpha$ is
  a problem of lifting to $M$, not merely $\overline{M}$, and the
  unique lifting theorem obtains.  Let $s_0$ be the supremum of all
  numbers $s_1$ such that there exists a lifting $\widetilde{\alpha}$
  on $[0, s_1)$ with $\widetilde{\alpha}(0) = e(f(a_0 + t), a_0 +
  t)$.  Then there is a unique path $\widetilde{\alpha}$ corresponding
  to the maximal $s_0$,
  $$
  \widetilde{\alpha} : [0, s_0) \to T
  $$
  such that $\pi \circ \widetilde{\alpha} = \alpha$ and
  $\widetilde{\alpha}(0) = e(f(a_0 + t), a_0 + t)$.   Then define
  $\gamma = v \circ \widetilde{\alpha}$, so that $\gamma$ is a path in
  $D$ such that
  %%% path 116
  $$
  \lim_{s \to s_0} f(\gamma(s))
  = \lim_{s \to s_0} f \circ v \circ \widetilde{\alpha}(s)
  = \lim_{s \to s_0} \pi \circ \widetilde{\alpha}(s)
  = \lim_{s \to s_0} \alpha(s)
  = \alpha(s_0).
  $$
  Here $\alpha(s_0) = \infty$ if $s_0 = \infty$.  Thus, the theorem
  follows if we know that $\gamma$ leaves every compact set in $D$.
  If $s_0 = \infty$ this is perfectly clear since $\lim_{s \to \infty}
  f(\gamma(s)) = \infty$.  Suppose $s_0 < \infty$, and suppose that
  for some compact $K \subset D$, $\gamma$ does not eventually leave
  $K$.  By the Bolzano-Weierstrass Theorem, there exists a point $z_0
  \in K$ and a sequence $s_1 < s_2 < \cdots$, $s_n \to s_0$, such that
  $\gamma(s_n) \to z_0$.  Since $v : T \to D$ is a homeomorphism,
  $\widetilde{\alpha}(s_n) \to v^{-1}(z_0) = e(f(z_0+t), z_0+t)$.
  Since $\pi$ is continuous, $f(z_0) = \lim \pi \circ
  \widetilde{\alpha}(s_n) = \lim \alpha(s_n) = \alpha(s_0) \notin
  f(E)$, so $e_0 = e(f(z_0 + t), z_0 + t) \in T \cap M$.  This
  contradicts the maximality of $s_0$, for the topology of $M$ implies
  that a neighbourhood $U$ of $e_0$ is homeomorphic by a homeomorphism
  $\varphi$ to a disk $\Delta$ centred at $f(z_0)$ and $\varphi$ is
  just the restriction of $\pi$ to $U$.  Therefore, if $n$ is chosen
  such that $\alpha(s_n) \in \Delta$, then for sufficiently small
  $\epsilon > 0$, we can define
  $$
  \widetilde{\alpha}(s) = \varphi^{-1} \circ \alpha(s),\quad
  s_n \le s < s_0 + \epsilon
  $$
  and we obtain a lifting of the required sort past the 
  %%%
  supposedly maximal $s_0$.  This contradiction shows that $\gamma$
  leaves every compact set in $D$.
\end{myproof}

A comprehensive reference to questions of this sort can be found in
MacLane, G. R., ``Asymptotic values of holomorphic functions,'' Rice
University Studies 49 (No. 1) 1963, pp. 1--83.  The example we have
just treated can be found on page 7 of MacLane's monograph.

\end{document}


%%% Local Variables: 
%%% mode: latex
%%% TeX-master: t
%%% End: 
