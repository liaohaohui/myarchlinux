\documentclass[a4paper,11pt]{article}
\usepackage{amsmath}
\usepackage{amsfonts}
\usepackage{verbatim}
\usepackage{url}
\usepackage{framed}
\usepackage{epsfig}
\usepackage{textcomp} %\textquotesingle
\usepackage[
%sorting=nyt,
firstinits=true, % render first and middle names as initials
useprefix=true,
maxcitenames=3,
maxbibnames=99,
style=authoryear,
dashed=false, % re-print recurring author names in bibliography
natbib=true,
url=false
]{biblatex}
%%% http://tex.stackexchange.com/questions/12254/biblatex-how-to-remove-the-parentheses-around-the-year-in-authoryear-style
\usepackage{xpatch}
\addbibresource{complex.bib} % run: biber topic1
\usepackage{color}
\usepackage{listings}
\definecolor{gray}{gray}{0.5} 
\definecolor{key}{rgb}{0,0.5,0} 
\lstset{ 
  language=[90]Fortran,
  basicstyle=\ttfamily\small, 
  keywordstyle=\color{blue}, 
  stringstyle=\color{red}, 
  showstringspaces=false, 
  emphstyle=\color{black}\bfseries, 
  emph={[2]True, False, None, self}, 
  emphstyle=[2]\color{key}, 
  emph={[3]from, import, as},
  emphstyle=[3]\color{blue}, 
  upquote=true, 
  morecomment=[s]{"""}{"""}, 
  commentstyle=\color{gray}\slshape, 
  %framexleftmargin=1mm, framextopmargin=1mm, frame=shadowbox, 
  rulesepcolor=\color{blue},
  numbers=left,
  stepnumber=1,
}
\usepackage{xypic}
\usepackage{tikz}
\usetikzlibrary{lindenmayersystems}
\usetikzlibrary[shadings]


%%% Page Layout
\oddsidemargin=0truecm
\evensidemargin=0truecm
\textwidth=160truemm
\textheight=260truemm
\leftmargin=0truemm
\rightmargin=0truemm
\voffset=-23truemm
\topmargin=0truemm

\newif\iflecturer
%\lecturerfalse
\lecturertrue

\iflecturer
\usepackage{marginnote} % \marginpar
%\usepackage[color]{showkeys}
\definecolor{refkey}{rgb}{1,0,0}
\definecolor{labelkey}{rgb}{1,0,0}
\else
\def\marginpar#1{}
\fi

\iflecturer
\newcommand{\Answer}[1]{\dotfill\underline{\mbox{\hspace{1em}\color{blue}#1}}}
\newcommand{\BoxAns}[1]{\fbox{\color{blue}#1}}
\newcommand{\Reason}[1]{{\par\color{blue}\par{}Reason: #1}}
\else
\newcommand{\Answer}[1]{\dotfill\underline{\mbox{\hspace{1em}\color{white}#1}}}
\newcommand{\BoxAns}[1]{\fbox{\color{white}#1}}
\newcommand{\Reason}[1]{{\par\color{white}\par{}Reason: #1}}
\fi

\newif\ifamsstyle
\amsstylefalse
\newcounter{topic}
\input common
\input symbol
\makeatletter
\newcommand{\Arctan}{{\mathop{\operator@font Arctan}\nolimits}}
\makeatother

%\parindent=0pt
\parskip=1pt
\linespread{1}


\setcounter{topic}{8}

\begin{document}

\title{{\sc Rudiments of Riemann Surfaces\\
    Topic \thetopic{}: The Torus}}
\author{Author: B. Frank Jones, Jr. (Rice Univ. 1971)\\
Seminar: Dr Liew How Hui (\url{liewhh@utar.edu.my})}
\date{}

\maketitle

%%% page 249
Our use of the classification theorem in proving Theorem~7.4 of the
previous topic is rather disappointing.  For we have applied the
classification theorem in the compact case only, and the proof of this
case occupies half page, whereas the proof of the other two cases
requires eleven more pages.  Essentially all that has been used in
Theorem~6.12 of Topic 6 and its corollary.  In this topic we shall
get to use the full force of the classification theorem in discovering
what all the ``analytic'' tori are.  I.e. we shall ``classify'' the
analytic tori.

At first glance, it perhaps seems that the classification theorem, 
which is addressed to simply connected surfaces, could not be used on
tori, which are manifestly not simply connected.  In any case, the
utility of the classification theorem would be minute if it had no
application to anything but simply connected surfaces.  Indeed, the
theorem states essentially that simply connected Riemann surfaces are
trivial in a certain sense.

%%% page 250
One of the primary applications of the classification theorem is to
the \emph{universal covering surface} of an arbitrary connected
Riemann surface.  The universal covering surface is a connected
Hausdorff space, and can be made into a Riemann surface in a natural
way, as we shall see in Lemma~\ref{lem:1}.  Also, it is simply
connected, so the classification theorem applies.  Once we know that
the universal covering surface is analytically equivalent to the
sphere, plane, or disk, then standard topological methods can be
invoked to obtain analytic information about the original surface.
Actually, in the case of a torus the universal covering surface is
obviously the plane, topologically; the ``covering map'' is also
rather obvious; and as a result in this topic not even the definitions
of the concepts mentioned in this paragraph will be given.  But the
topologically alert reader will know the general setting of what
follows.

\begin{defn}
  \label{def:1}
  If $T$ and $S$ are topological spaces and $f : T \to S$, then $f$ is
  a \emph{local homeomorphism} if for every point $p \in T$ there
  exist a neighbourhood $U$ of $p$ and a neighbourhood $V$ of $f(p)$
  such that $f$ is a homeomorphism of $U$ onto $V$.
\end{defn}

\begin{lem}
  \label{lem:1}
  Let $T$ be a Hausdorff topological space and $S$ a Riemann surface.
  Let $f : T \to S$ be a local homeomorphism.  Then there exists a
  unique complete analytic atlas on $T$ such that $f$ is an analytic
  function from the Riemann surface $T$ to $S$.
\end{lem}

\begin{mdframed}
  \textbf{Proof}:
  First we prove uniqueness, so we suppose first that $T$ is a Riemann
  surface.  If $p \in T$, there exist a neighbourhood $U$ of $p$ and
  a neighbourhood $V$ of $f(p)$ such that $f : U \to V$ is a
  homeomorphism and $V$ is the domain of an analytic chart $\varphi :
  V \to \varphi(V)$ on $S$.  Let $f_1$ be the restriction of $f$ to
  $U$.  Then since $f$ is analytic, $f_1$ is an analytic equivalence
  of $U$ onto $V$, so $\varphi \circ f_1$ must be an analytic chart on
  $U$.  Knowing an analytic chart in a neighbourhood of each point of
  $T$ implies that we know the complete analytic atlas for $T$, so the
  uniqueness follows.

  Conversely, we use the above procedure to \emph{define} charts
  $\varphi \circ f_1$ on $T$.  We now show these charts form an
  analytic atlas.  If we have another choice, $\widetilde{U},
  \widetilde{V}, \widetilde{\varphi}$, and $\widetilde{f}_1$ (the
  restriction of $f$ to $\widetilde{U}$), then where the composition
  is defined we have
  $$
  \widetilde{\varphi} \circ \widetilde{f}_1 \circ
  (\varphi \circ f_1)^{-1}
  = \widetilde{\varphi} \circ \widetilde{f}_1 \circ
  f_1^{-1} \circ \varphi^{-1} 
  = \widetilde{\varphi} \circ \varphi^{-1}
  $$
  since $\widetilde{f}_1 \circ f_1^{-1}$ = identity (we might have to
  decrease the size of everything to achieve this).  Since $S$ is a
  Riemann surface, $\widetilde{\varphi} \circ \varphi^{-1}$ is
  holomorphic, and thus we have an analytic atlas for $T$.  We have to
  show that $f$ is now analytic, but this is clear.  For, \textbf{on
    $U$} we have
  $$
  f = f_1 = \varphi^{-1} \circ (\varphi \circ f_1),
  $$
  %%% page 252
  and this is a composition of two analytic functions.  Thus, $f$ is
  analytic in a neighbourhood of any point of $T$.
\end{mdframed}

\begin{lem}
  \label{lem:2}
  Let $T$ and $S$ be Riemann surfaces and $f : T \to S$ an analytic
  local homeomorphism.  Let $T_1$ be a Riemann surface and $g : T_1
  \to T$ a continuous function such that $f \circ g$ is analytic.
  Then $g$ is analytic.
  $$
  \xymatrix{
    T_1 \ar[r]^g \ar[dr]_{f\circ g} & T \ar[d]^f\\
    & S
    }
  $$
\end{lem}

\begin{mdframed}
  \textbf{Proof}:
  Given $p \in T_1$, there exist neighbourhoods $U_1$ of $p$, $U$ of
  $g(p)$, and  $V$ of $f(g(p))$ such that $g : U_1 \to U$ and $f : U
  \to V$ is an analytic equivalence.  Then \textbf{on $U_1$} we have
  $$
  g = f_1^{-1} \circ (f \circ g),
  $$
  where $f_1$ is the restriction of $f$ to $U$.  Thus, $g$ is
  analytic.
\end{mdframed}

\medskip
Now that the preliminaries are finished, we are ready to discuss
tori.  The situation is this: $S$ is a Riemann surface which is
homeomorphism to a \emph{torus}.
%%% page 253
The problem is to discover what kind of analytic atlas $S$ can have.
What we shall do is prove that $S$ is analytically equivalent to $\CCC
/ \Omega$ mentioned in Topic 2.  The problem is essentially to find
the complex numbers $w_1$ and $w_2$ such that $\Omega = \{ n_1 w_1 +
n_2 w_2 ~:~ n_1, n_2 \in \ZZ\}$.

To start with, it is convenient to choose a \emph{topological}
representation of $S$ as $\CCC / \Omega$ for some $\Omega$ which we
can pick arbitrarily.  Thus, choose arbitrary complex $\rho_1$ and
$\rho_2$ whose ratio is not real.  Then we suppose that $S$ is the set
of all cosets $[z] = \{z + n_1 \rho_1 + n_2 \rho_2 ~:~ n_1, n_2 \in
\ZZ\}$, and $S$ is made into a Hausdorff space in the way described in
Topic 2.  We thus have a concrete representation of $S$ as a
topological space, but the analytic atlas for $S$ is unknown.  In
particular, it is probably not the analytic atlas described there,
unless we happened to choose $\rho_1$ and $\rho_2$ correctly.  We
reiterate that we are going to prove it \textbf{is} such an analytic
atlas with the proper choice of $\rho_1$ and $\rho_2$.

Now we are ready to apply Lemma~\ref{lem:1} and \ref{lem:2}.  First,
let $\CCC$ be the complex plane as a topological space \emph{without}
the usual complete analytic atlas and let
$$
\pi : \CCC \to S
$$
be the natural mapping defined by $\pi(z) = [z]$.  Clearly, $\pi$ is a
local homeomorphism, so Lemma~\ref{lem:1} shows there is a
%%% page 254
unique way to make $\CCC$ a Riemann surface such that $\pi$ is
analytic.  Let $\CCC^*$ denote this Riemann surface; $\CCC^*$ is
homeomorphic but not necessarily analytically equivalent to $\CCC$ as
we usually consider it as a Riemann surface.

There are obvious translations on $\CCC^*$.  For example, let
$$
t_1 ~:~ \CCC^* \to \CCC^*
$$
be defined by
$$
t_1(z) = z + \rho_1.
$$
Then since $\pi(z + \rho_1) = [z + \rho_1] = [z] = \pi(z)$, we have
$$
\pi \circ t_1 = \pi.
$$
Or, we have a commutative diagram.
$$
\xymatrix{
  \CCC^* \ar[r]^{t_1} \ar[dr]^\pi & \CCC^* \ar[d]^\pi\\
  & S
}
$$
By Lemma~\ref{lem:2}, $t_1$ is analytic.  Likewise, $t_2$ is analytic,
where $t_2(z) = z + \rho_2$.  Two obvious facts about these
translations are that $t_1$ and $t_2$ commute:
$$
t_1 \circ t_2 = t_2 \circ t_1,
$$
and that $t_1$ and $t_2$ generate an Abelian group: if $n_1$ and $n_2$
are any integers, the mapping
%%% page 255
$$
t_1^{n_1} \circ t_2^{n_2}
$$
which stands for $n_1$-fold composition of $t_1$ composed with
$n_2$-fold composition of $t_2$, is just the mapping
$$
z \mapsto z + n_1 \rho_1 + n_2 \rho_2.
$$

Now we apply the classification theorem to $\CCC^*$, which is simply
connected, connected and \textbf{not} compact.  Thus, $\CCC^*$ is
analytically equivalent to $\CCC$ or the dis unit disk $\Delta$.  In
spite of appearances, it does not seem obvious that $\CCC^*$ is
equivalent to $\CCC$ and not $\Delta$, which is indeed the case.  It
is clear that this question must be faced.  Let $? = \CCC$ or $\Delta$
as the case may be, and let $f$ be the analytic equivalence:
$$
f : ? \to \CCC^*
$$

Using $t_1$ and $t_2$, we now define corresponding mappings of ? to
itself:
$$
\begin{aligned}
  A_1 &= f^{-1} \circ t_1 \circ f,\\
  A_2 &= f^{-1} \circ t_2 \circ f.
\end{aligned}
$$
Then the properties of $t_1$ and $t_2$ are obviously reflected in
$A_1$ and $A_2$: $A_1$ and $A_2$ are analytic maps of ? onto ?, they
are both one-to-one, they commute, for
%%% page 256
$$
\begin{aligned}
  A_1 \circ A_2 
  &= (f^{-1} \circ t_1 \circ f) \circ (f^{-1} \circ t_2 \circ f)\\
  &= f^{-1} \circ t_1 \circ t_2 \circ f\\
  &= f^{-1} \circ t_2 \circ t_1 \circ f  = A_2 \circ A_1,
\end{aligned}
$$
and they generate an Abelian group, with the formula
$$
A_1^{n_1} \circ A_2^{n_2} = f^{-1} \circ t_1^{n_1} \circ t_2^{n_2} \circ f.
$$
Now we remark that the only analytic equivalences of $\Delta$ on
$\Delta$ or of $\CCC$ onto $\CCC$ are \emph{M\"obius}
transformations.  Thus $A_1$ and $A_2$ are both M\"obius
transformations.

It turns out the thing relevant to our discussion is the \emph{fixed
  point} structure of $A_1$ and $A_2$.  Suppose now that $A$ is a
M\"obius transformation of the form
$$
A(z) = \frac{az+b}{cz+d},\quad ad-bc \ne 0.
$$
A point $z \in \widehat{\CCC}$ is a fixed point of $A$ if $A(z) = z$.
That is,
$$
\frac{az+b}{cz+d} = z.
$$
Observe that $\infty$ is a fixed point if and only if $c = 0$.  If
$c\ne 0$, the above equation can be written
$$
az + b = cz^2 + dz,
$$
a quadratic equation for $z$, which has either two roots or one root.
Thus, every M\"obius $A$ has one or two fixed points in
$\widehat{\CCC}$ (note that if $c = 0$, we can write $A(z) = az + b$,
and the $A$ has two fixed points if and only if $a \ne 1$).

%%% page 257
Now if $A$ is M\"obius and an equivalence of $\Delta$ onto $\Delta$,
and if $z$ is a fixed point of $A$, then the \emph{conjugate} of $z$
with respect to $\p \Delta$ is also a fixed point of $A$.  For suppose
$w$ is the conjugate of $z$ (that is, $w = 1/\overline{z}$).  Then a
property of $A(\p \Delta)$.  But $A(z) = z$ and $A(\p \Delta) = \p
\Delta$, so we see that $z$ and $A(w)$ are conjugate with respect to
$\p \Delta$.  Thus, $A(w) = w$. 

It is obvious that $t_1$ has no fixed points in $\CCC^*$.  Thus,
\emph{$A_1$ has no fixed points in ?}.  If $? = \CCC$, we must have
therefore
$$
A_1(z) = z + w_1,
$$
and likewise
$$
A_2(z) = z + w_2.
$$
Here $w_1$ and $w_2$ are nonzero complex numbers.

If $? = \Delta$, then if $A_1$ has only one fixed point, it must be on
$\p \Delta$.  This follows since $A_1$ has \emph{no} fixed points
\emph{in $\Delta$} and since the conjugate with respect to $\p \Delta$
of a fixed point of $A_1$ is also a fixed point of $A_1$.  Likewise,
if $A_1$ has two fixed points, they both lie on $\p \Delta$.

\textbf{Proof that $? = \CCC$}.  Suppose the contrary, that $? =
\Delta$.  There are two cases to consider.  Suppose first that $A_1$
has two fixed points, $\alpha$ and $\beta$.  Then
$$
A_1(A_2(\alpha)) = A_2(A_1(\alpha)) = A_2(\alpha),
$$
so $A_2(\alpha)$ is a fixed point of $A_1$.  Likewise, $A_2(\beta)$ is
a fixed point of $A_1$.  So either $A_2(\alpha) = \alpha$ and
$A_2(\beta) = \beta$, or $A_2(\alpha) = \beta$ and $A_2(\beta) =
\alpha$.  Now define the M\"obius transformation
$$
m(z) = \frac{z-\alpha}{z-\beta}.
$$
Then the transformation
$$
m \circ A_1 \circ m^{-1}
$$
maps 0 to 0 and $\infty$ to $\infty$, and thus is multiplication by a
complex number $a_1$.  Since $m(\p \Delta)$ is a straight line through
0, it follows that $m(\Delta)$ is a half plane bounded by a straight
line through 0.  And the mapping $z \mapsto a_1 z$ maps this half
plane onto itself.  Thus, $a_1$ is a positive real number.  That is,
$$
m \circ A_1 \circ m^{-1}(z) = a_1 z, \quad 0 < a_1 < \infty.
$$

If we have $A_2(\alpha) = \alpha$ and $A_2(\beta) = \beta$, the also
$$
m \circ A_2 \circ m^{-1}(z) = a_2 z, \quad 0 < a_2 < \infty.
$$
On the other hand, if $A_2(\alpha) = \beta$ and $A_2(\beta) = \alpha$,
then $m \circ A_2 \circ m^{-1}$ maps 0 to $\infty$ and $\infty$ to 0.
Thus, for some nonzero complex $b$,
$$
m \circ A_2 \circ m^{-1}(z) = \frac{b}{z}.
$$
Then it follows that
$$
(m \circ A_2 \circ m^{-1}) \circ (m \circ A_2 \circ m^{-1})(z)
= \frac{b}{b/z} = z,
$$
so that also
$$
A_2 \circ A_2 = \text{identity}.
$$
But then also $t_2 \circ t_2$ = identity, a contradiction since $t_2
\circ t_2$ is translation by $2\rho_2$.  Therefore, we conclude that
$$
m \circ A_j \circ m^{-1}(z) = a_j z, \quad j = 1,2.
$$
Now we need a lemma.

\begin{lem}
  \label{lem:3}
  Let $x,y \in \RR$.  Then there exist integers $m_k$, $n_k$ such that
  for each $k$, $m_k$ and $n_k$ are not both zero, and
  $$
  \lim_{k\to \infty} (m_k x + n_k y) = 0.
  $$
\end{lem}

\begin{myproof}
  We can obviously assume $x$ and $y$ are not both zero and that
  $\frac{x}{y} = \xi$ is irrational.  Let $N$ be any positive
  integer.  For $1 \le j \le N+1$ there exists a unique integer
  $\ell_j$ such that
  $$
  0 < j\xi - \ell_j < 1.
  $$
  Among the $N$ intervals $(0, \frac{1}{N})$, $(\frac{1}{N},
  \frac{2}{N})$, $\cdots$, $(\frac{N-1}{N}, 1)$ there must be one
  which contains two of the numbers $j \xi = \ell_j$, say for $j_1$
  and $j_2$.  Then
  $$
  |(j_1 \xi - \ell_{j_1}) - (j_2 \xi - \ell_{j_2})| < \frac{1}{N}.
  $$
\end{myproof}

Now we apply this lemma to the real numbers $\log(a_1)$ and
$\log(a_2)$ to obtain $m_k \log(a_1) + m_k\log(a_2) \to 0$.
Exponentiating, $a_1^{m_k} a_2^{n_k} \to 1$.  Thus, for each $z$ we
have
$$
m \circ A_1^{m_k} \circ A_2^{n_k} \circ m^{-1} (z) \to z.
$$
Therefore, since $m$ and $m^{-1}$ are continuous,
$$
A_1^{m_k} \circ A_2^{n_k}(z) \to z
$$
for each $z$, and thus
$$
t_1^{m_k} \circ t_2^{n_k}(z) \to z
$$
for each $z$.  This says $z + m_k\rho_1 + n_k \rho_2 \to z$, and thus
$m_k \rho_1 + n_k \rho_2 \to 0$.  This contradict the fact that
$\rho_1$ and $\rho_2$ have a nonreal ratio.

%%% page 261
The only other case is the case in which $A_1$ and $A_2$ each have
only one fixed point.  Suppose $A_1(\alpha) = \alpha$. $A_2(\alpha)$
is a fixed point of $A_1$, so $A_2(\alpha) = \alpha$.  Let $m$ be the
M\"obius transformation
$$
m(z) = \frac{e^{i\theta}}{z-\alpha}.
$$
The $m(\alpha) = \infty$ and $m(\p \Delta)$ is a straight line.  We
choose $\theta$ to force this straight line to be parallel to the real
axis.  Then $m \circ A_1 \circ m^{-1}$ maps $\infty$ to $\infty$ and
has no other fixed point, so
$$
m \circ A_1 \circ m^{-1}(z) = z + a_1.
$$
Since $m \circ A \circ m^{-1}$ maps the associated half plane onto
itself, $a_1 \in \RR$.  Likewise,
$$
m \circ A_2 \circ m^{-1}(z) = z + a_2, \quad a_2 \in \RR.
$$
By Lemma~\ref{lem:3}, there exist integers $m_k$ and $n_k$ such that
$m_k a_1 + n_k a_2 \to 0$.  Therefore, as in the argument above we
obtain $m_k \rho_1 + n_k \rho_2 \to 0$, a contradiction.

Thus, we have now completely contradicted the assumption that $? =
\Delta$.  The only other possibility must hold.  Thus, $\boxed{? =
  \CCC}$.

Now we know that $A_j(z) = z + w_j$, $j=1,2$. Exactly as in the above
discussion, it follows that $w_1$ and $w_2$ have a nonreal ratio.
Thus, if we define $\Omega = \{n_1 w_1 + n_2 w_2 ~:~ n_1, n_2 \in
\ZZ\}$, we have a Riemann surface $\CCC/\Omega$ (as defined in Topic
2).

%%% page 262
Consider the diagram
$$
\xymatrix{
 \CCC^* \ar[d]_\pi & \CCC \ar[l]_f \ar[d]^{\pi_1} \\
 S & \CCC/\Omega \ar[l]_F
}
$$
where the map $\pi_1$ is $z \mapsto z + \Omega$.  What we want to do
is obtain an analytic function $F$ from $\CCC/\Omega$ to $S$.  First,
we can \textbf{define} a function $F$ by
$$
F(z + \Omega) = \pi \circ f(z).
$$
This makes sense, for if $z + \Omega = z' + \Omega$, then $z = z' +
n_1 w_1 + n_2 w_2$ for some integers $n_1$ and $n_2$, and thus
$$
\begin{aligned}
  \pi \circ f(z) &= \pi \circ f(z' + n_1 w_1 + n_2 w_2)\\
  &= \pi \circ f(A_1^{n_1} \circ A_2^{n_2}(z'))\\
  &= \pi \circ t_1^{n_1} \circ t_2^{n_2} (f(z')) = \pi \circ f(z').
\end{aligned}
$$
Thus, $F$ is uniquely defined such that $F \circ \pi_1 = \pi \circ
f$.  Since $\pi_1$ is locally an analytic equivalence, we have $F =
\pi \circ f \circ \pi_1^{-1}$ \emph{locally} and thus $F$ is
analytic.  Since $\pi$ and $f$ surjections, so is $F \circ \pi_1$ and
thus so is $F$.  Finally, $F$ is one-to-one.  For, suppose $F(z +
\Omega) = F(z' + \Omega)$.  Then $\pi \circ f(z) = \pi \circ f(z')$,
so that there exist integers $n_1$ and $n_2$ such that
$$
f(z) = f(z') + n_1 \rho_1 + n_2 \rho_2 = t_1^{n_1} \circ t_2^{n_2}
\circ f(z').
$$
%%% page 263
Thus
$$
z = f^{-1} \circ t_1^{n_1} \circ t_2^{n_2} \circ f(z')
= A_1^{n_1} \circ A_2^{n_2} (z')
= z' + n_1 w_1 + n_2 w_2.
$$
Thus, $z + \Omega = z' + \Omega$, proving $F$ is one-to-one.

We shall now formally state a theorem which includes the above
discussion.  We need to recall Definition~5.10 of Topic 5.

\begin{thm}
  \label{thm:1}
  Let $S$ be a compact, connected Riemann surface.  Then the following
  conditions are equivalent.
  \begin{enumerate}
  \item $S$ is analytically equivalent to the Riemann surface of a
    polynomial
    $$
    w^2 - 4(z-e_1)(z-e_2)(z-e_3),
    $$
    where $e_1, e_2, e_3$ are distinct complex numbers.
  \item $S$ is analytically equivalent to the Riemann surface of a
    polynomial
    $$
    w^2 - (z-a_1)(z-a_2)(z-a_3)(z-a_4),
    $$
    where $a_1, a_2, a_3, a_4$ are distinct complex numbers.
  \item $S$ is homeomorphic to a torus.
  \item $S$ is analytically equivalent to a torus of the form $\CCC /
    \Omega$.
  \end{enumerate}
\end{thm}

%%% page 264
\begin{mdframed}
  \textbf{Proof}: \underline{$1 \Rightarrow 2$}: This is simple
  algebra.  We can assume $S \subset \overline{M}$ \textbf{is} the
  Riemann surface of the polynomial 1.  Let $\alpha \in \CCC$, $\alpha
  \ne e_1$ or $e_2$ or $e_3$.  Define
  $$
  f = \frac{1}{\pi - \alpha},
  $$
  where $\pi$ and $v$ are the functions on $\overline{M}$ discussed in
  Topic 4, restricted to $S$.  Thus,
  $$
  \begin{aligned}
    v^2 &= 4(\pi-e_1)(\pi-e_2)(\pi-e_3);\\
    f^4 v^2 &= 4f(f(\pi - \alpha) + (\alpha + e_1)f)
    (f(\pi - \alpha) + (\alpha + e_2)f)
    (f(\pi - \alpha) + (\alpha + e_3)f)\\
    &= 4f(1 + (\alpha-e_1)f)(1 + (\alpha-e_2)f)(1 + (\alpha-e_3)f)\\
    &= 4(\alpha-e_1)(\alpha-e_2)(\alpha-e_3)
    f(f-\frac{1}{e_1-\alpha})(f-\frac{1}{e_2-\alpha})
    (f-\frac{1}{e_3-\alpha}).
  \end{aligned}
  $$
  Choose a complex number $\beta = \sqrt{4(\alpha-e_1)(\alpha-e_2)
    (\alpha-e_3)}$ and let
  $$
  g = \frac{f^2 v}{\beta}.
  $$
  Thus,
  $$
  g^2 = f(f-\frac{1}{e_1-\alpha})(f-\frac{1}{e_2-\alpha})
  (f-\frac{1}{e_3-\alpha}).
  $$
  Now $f$ and $g$ are meromorphic on $S$ and $f$ takes every value 2
  times, since the same is true of $\pi$.  Now consider the diagram
  from Topic 6
  %%% page 265
  $$
  \xymatrix {
    S \ar[rr]^{\Phi} \ar[dr]_{f,g} && \overline{M} \ar[dl]^{\pi, v}\\
    & \widehat{\CCC} \times \widehat{\CCC}
  }
  $$
  The polynomial equation satisfied by $f$ and $g$ is easily seen to
  be irreducible and it is of degree 2 in $g$.  Therefore, Theorem~6.5
  of Topic 6 implies $\Phi$ is an analytic equivalence.  This
  proves $1 \Rightarrow 2$.


  \underline{$2 \Rightarrow 3$}:
  This follows trivially from the cutting and gluing process described
  in Topic 1.  Also, it follows from the Riemann-Hurwitz formula.  In
  this case $V = 4$ and $n = 2$, so that $g = 1$.

  \underline{$3 \Rightarrow 4$}:
  This is the content of the discussion preceding this problem.

  \underline{$4 \Rightarrow 1$}:
  Now we have to do some work.  In fact, we need to introduce some
  rather classical and famous concepts of the theory of elliptic
  functions.  Suppose that $w_1, w_2 \in \CCC$ have nonreal ration and
  define as usual $\Omega = \{n_1 w_1 + n_2 w_2 ~:~ n_1, n_2 \in
  \ZZ\}$.   Define
  $$
  \wp(z) = \frac{1}{z^2} + \sum_{\substack{\zeta \in \Omega\\ \zeta
      \ne 0}}(\frac{1}{(z-\zeta)^2} - \frac{1}{\zeta^2}).
  $$
  We must first prove this series converges.  If $K$ is a compact
  subset of $\CCC - \Omega$, then for any $z \in K$
  %%% page 266
  $$
  \left| \frac{1}{(z-\zeta)^2} - \frac{1}{\zeta^2} \right|
    = \frac{|-z^2 + 2z\zeta|}{|(z-\zeta)^2\zeta^2|}
    \le c |\zeta|^{-3}.
  $$
  We now check that the series of constants
  $$
  \sum_{\substack{\zeta \in \Omega\\ \zeta \ne 0}} |\zeta|^{-3}
  $$
  converges.  Since $w_1$ and $w_2$ have nonreal ratio, it follows
  that for any $\theta \in [0, 2\pi]$, $w_1 \cos \theta + w_2 \sin
  \theta \ne 0$.  Since this is a continuous function of $\theta$,
  there exists a positive constant $\delta$ such that
  $$
  |w_1 \cos \theta + w_2 \sin \theta|
  \ge \delta, \quad 0 \le \theta \le 2\pi.
  $$
  Multiplying both sides of this inequality by a positive number, we
  obtain
  $$
  |x w_1 + y w_2| \ge \delta\sqrt{x^2 + y^2},\quad x,y\in \RR.
  $$
  Therefore, summing on squares in $\RR^2$, we obtain
  $$
  \begin{aligned}
    \sum_{\substack{\zeta \in \Omega\\ \zeta \ne 0}} |\zeta|^{-3} 
    &\le \delta^{-3} \sum_{\substack{n_1,n_2\text{ not }\\\text{both
          zero}}} (n_1^2 + n_2^2)^{-3/2}\\
    &= \delta^{-3} \sum_{k=1}^{\infty} \sum_{\max(|n_1|,|n_2|)=k}
    (n_1^2 + n_2^2)^{-3/2}\\
    &\le \delta^{-3} \sum_{k=1}^{\infty} 4(2k+1)k^{-3} < \infty.
  \end{aligned}
  $$
  %%% page 267
  Therefore, the series defining $\wp$ converges uniformly on any
  compact subset of $\CCC - \Omega$, and one sees likewise that for
  any lattice point $\zeta_0$, the series for $\wp(z) -
ze  \frac{1}{(z-\zeta_0)^2}$ converges uniformly on any compact subset
  of $\CCC - (\Omega - \{\zeta_0\})$.  Therefore, $\wp$ is meromorphic
  on $\CCC$ and has poles of order 2 at each $\zeta \in \Omega$.  The
  function $\wp$ is called the \emph{Weierstrass pe-function}.

  The first remark to be made is that $\wp$ is an \emph{even}
  function.  For,
  $$
  \wp(-z) = \frac{1}{z^2} + \sum_{\substack{\zeta \in \Omega\\ \zeta
      \ne 0}} (\frac{1}{(-z-\zeta)^2} - \frac{1}{\zeta^2}).
  $$
  Replacing the ``dummy'' $\zeta$ by $-\zeta$, we therefore obtain
  $$
  \wp(-z) = \frac{1}{z^2} + \sum_{\substack{\zeta \in \Omega\\ \zeta
      \ne 0}} (\frac{1}{(z-\zeta)^2} - \frac{1}{\zeta^2}) = \wp(z).
  $$
  Next we compute $\wp'$; since the series for $\wp$ converges
  uniformly locally, this can be done formally:
  $$
  \wp'(z) = \frac{-2}{z^3}\sum_{\substack{\zeta \in \Omega\\ \zeta
      \ne 0}} \frac{-2}{(z-\zeta)^3}
  = -2 \sum_{\zeta \in \Omega} \frac{1}{(z-\zeta)^3}.
  $$
  Thus, if $\zeta_0 \in \Omega$,
  $$
  \wp'(z + \zeta_0) = - \sum_{\zeta \in \Omega} \frac{1}{(z+\zeta_0 -
    \zeta)^3} = -2 \sum_{\zeta\in \Omega} \frac{1}{(z-\zeta)^3} =
  \wp'(z),
  $$
  %%% page 268
  where we have replaced the ``dummy'' $\zeta$ by $\zeta + \zeta_0$.
  This implies that
  $$
  \wp(z + w_1) - \wp(z) \equiv \text{constant}.
  $$
  We evaluate this constant by setting $\ds z = -\frac{w_1}{2}$.
  Since $\wp$ is even,
  $$
  \wp(\frac{w_1}{2}) - \wp(-\frac{w_1}{2}) = 0,
  $$
  and thus the constant is zero.  Therefore, using the same argument
  for $w_2$,
  $$
  \wp(z + w_1) = \wp(z),\quad
  \wp(z + w_2) = \wp(z).
  $$

  These relations imply that we can regard $\wp$ as a meromorphic
  function on $\CCC/\Omega$, whose value at $z + \Omega$ is just
  $\wp(z)$.  To keep track of the notation, let $\wp_0$ be this
  function:
  $$
  \wp_0(z + \Omega) = \wp(z).
  $$
  Then since $\wp$ has double poles at the points in $\Omega$ and no
  other points, we see that $\wp_0$ has a double pole at $0 + \Omega$
  and no other pole.  Since $\CCC / \Omega$ is a compact Riemann 
  %%% page 269
  surface, $\wp_0$ takes every value 2 times.

  Likewise, if we define
  $$
  \wp_0'(z + \Omega) = \wp'(z),
  $$
  then \emph{$\wp_0'$ takes every value 3 times}, since $\wp'$ has
  triple poles at the points of $\Omega$.  We shall be interested in
  particular in the zeros of $\wp'$.  If $\zeta_0 \in \Omega$ but
  $\frac{1}{2} \zeta_0 \notin \Omega$, then since $\wp'$ is
  \emph{odd}, being the derivative of an even function,
  $$
  \wp'(\frac{1}{2}\zeta_0) 
  = \wp'(\frac{1}{2}\zeta_0 - \zeta_0)
  = \wp'(-\frac{1}{2}\zeta_0)
  = - \wp'(\frac{1}{2}\zeta_0).
  $$
  Since $\frac{1}{2}\zeta_0 \notin \Omega$, $\wp'(\frac{1}{2}\zeta_0)
  \notin \infty$, and thus $\wp'(\frac{1}{2}\zeta_0) = 0$.  Thus,
  $$
  \wp_0'(\frac{1}{2}w_1 + \Omega) = 0,
  \wp_0'(\frac{1}{2}w_2 + \Omega) = 0,
  \wp_0'(\frac{1}{2}w_1 + \frac{1}{2}w_2 + \Omega) = 0.
  $$
  Now define
  $$
  \wp_0(\frac{1}{2}w_1 + \Omega) = e_1,
  \wp_0(\frac{1}{2}w_2 + \Omega) = e_2,
  \wp_0(\frac{1}{2}w_1 + \frac{1}{2}w_2 + \Omega) = e_3.
  $$

  Since $\wp_0'$ takes every value 3 times, we have found all of its
  zeros.  And these must therefore also be simple zeros of $\wp_0'$.
  Also, we see that $\wp_0$ takes the value $e_1$ 2 times at
  $\frac{1}{2}w_1 + \Omega$ (a zero of $\wp_0'$) and likewise for
  $e_2$ and $e_3$.  In particular, $e_1, e_2, e_3$ are distinct.

  Now consider the meromorphic function
  $$
  \frac{\wp_0'^2}{(\wp_0-e_1)(\wp_0-e_2)(\wp_0-e_3)}
  $$
  on $\CCC/\Omega$.  The numerator and denominator have poles only at
  $0 + \Omega$, and near there we have the Laurent development
  $$
  \frac{(-\frac{2}{z^3} + \cdots)^2}{(\frac{1}{z^2} + \cdots)^3}
  = \frac{\frac{4}{z^6} + \cdots}{\frac{1}{z^6} + \cdots}
  = 4 + \cdots.
  $$
  Thus, the function has no pole at $0 + \Omega$, and in fact is equal
  to 4 at $0 + \Omega$.  The only other possibilities for poles are at
  $\frac{w_1}{2} + \Omega$, $\frac{w_2}{2} + \Omega$, and
  $\frac{w_1}{2} + \frac{w_2}{2} + \Omega$.  But at these points the
  numerator has zeros of order 2.  Thus, the function has no pole at
  all, and is thus constant.  Therefore,
  $$
  \wp_0'^2 = 4(\wp_0 - e_1)(\wp_0 - e_2)(\wp_0 - e_3).
  $$
  This is the classical differential equation for $\wp$.

  Consider the diagram
  $$
  \xymatrix {
    S \ar[rr]^{\Phi} \ar[dr]_{\wp,\wp'} && T \ar[dl]^{\pi, v}\\
    & \widehat{\CCC} \times \widehat{\CCC}
  }
  $$
  %%% page 271
  Since $\wp_0$ takes every value 2 times and the algebraic equation
  relating $\wp_0$ and $\wp_0'$ has degree 2 in $\wp_0'$ and is
  irreducible, Theorem~6.5 of Topic 6 implies $\Phi$ is an
  analytic equivalence.  And $T$ is the Riemann surface of $w^2 -
  4(z-e_1)(z-e_2)(z-e_3)$.
\end{mdframed}

Even among the tori $\CCC/\Omega$ there are lots of equivalences.  We
now treat this problem.

\begin{thm}
  \label{thm:2}
  Two tori $\CCC / \Omega$ and $\CCC / \widetilde{\Omega}$ are
  analytically equivalent if and only if there exists a nonzero
  complex $a$ such that
  $$
  \widetilde{\Omega} = a \Omega.
  $$
\end{thm}

\begin{mdframed}
  \textbf{Proof}: If $\widetilde{\Omega} = a\Omega$, then we can
  define a mapping $\CCC / \Omega \to \CCC / \widetilde{\Omega}$ by
  the formula $z + \Omega \mapsto az + \widetilde{\Omega}$, and this is
  easily seen to be an analytic equivalence.

  Conversely, suppose $F : \CCC / \Omega \to \CCC /
  \widetilde{\Omega}$ is an analytic equivalence.  If $F(0 + \Omega) =
  \alpha + \widetilde{\Omega}$, then let
  $$
  t : \CCC / \widetilde{\Omega} \to \CCC / \widetilde{\Omega}
  $$
  be defined by
  $$
  t(z + \widetilde{\Omega}) = z - \alpha + \widetilde{\Omega}.
  $$
  Then $t$ is an analytic equivalence and
  $$
  t \circ F(0 + \Omega) = t(\alpha + \widetilde{\Omega}) 
  = 0 + \widetilde{\Omega}.
  $$
  %%% page 272
  By considering $t \circ F$ instead of $F$, we see that there is no
  loss of generality in assuming
  $$
  F(0 + \Omega) = 0 + \widetilde{\Omega}.
  $$

  Given $z_0 \in \CCC$ choose arbitrarily $w_0 \in \CCC$ such that
  $$
  F(z_0 + \Omega) = w_0 + \widetilde{\Omega}.
  $$
  Let $\pi : \CCC \to \CCC / \Omega$ and $\widetilde{\pi} : \CCC
  \to \CCC / \widetilde{\Omega}$ be the canonical mappings and choose
  a neighbourhood $\widetilde{U}$ of $w_0$ such that $\widetilde{\pi}$
  is one-to-one on $\widetilde{U}$.  Then for $z$ near $z_0$ consider
  the mapping
  $$
  g_{w_0}(z) = \widetilde{\pi}^{-1}(F(z + \Omega)).
  $$
  This is a holomorphic function in a neighbourhood of $z$ and if we
  choose a different $w_0'$ such that $F(z_0 + \Omega) = w_0' +
  \widetilde{\Omega}$, then $w_0' = w_0 + \widetilde{\zeta}_0$, where
  $\widetilde{\zeta}_0 \in \widetilde{\Omega}$, and the associated
  $\widetilde{\pi}^{-1}$ is thus equal to the original
  $\widetilde{\pi}^{-1} + \widetilde{\zeta}_0$.  Thus,
  $$
  g_{w_0'}(z) = g_{w_0}(z) + \widetilde{\zeta}_0.
  $$
  It follows that
  $$
  \frac{d}{dz}g_{w_0}(z)
  $$
  is well defined near $z_0$ in the sense that it is independent of
  the choice $w_0$.  Thus, we can define a function $h$ on $\CCC$ by
  the formula
  $$
  h(z) = \frac{d}{dz} g_{w_0}(z),\quad z \text{ near } z_0.
  $$
  %%% page 273
  Then $h$ is holomorphic on $\CCC$ and since for $z_0' = z_0 +
  \zeta_0$, $\zeta_0 \in \Omega$, we can take $w_0$ to be the same and
  thus for $z$ near $z_0'$
  $$
  g_{w_0}(z) = \widetilde{\pi}^{-1}(F(z + \Omega))
  = \widetilde{\pi}^{-1}(F(z - \zeta_0 \Omega))
  = g_{w_0}(z - \zeta_0),
  $$
  it follows that $h(z) = h(z - \zeta_0)$ for $z$ near $z_0'$.  Hence,
  $h(z_0') = h(z_0' - \zeta_0) = h(z_0)$, so $h$ represents a
  meromorphic function on $\CCC / \Omega$.  But $h$ is holomorphic and
  thus constant, say $h \equiv a$.

  Therefore, following the above notation we have
  $$
  g_{w_0}(z) = az + b
  $$
  for $z$ near $z_0$.  Applying $\widetilde{\pi}$ to both sides we
  obtain
  $$
  F(z + \Omega) = az + b + \widetilde{\Omega} \text{ for } z \text{
    near } z_0.
  $$
  Here $b$ is a constant which can depend on $z_0$.  By a connectivity
  argument it is easy to see that the constants $b$ which can appear
  here must differ from each other only by elements of
  $\widetilde{\Omega}$.  Since $F(0 + \Omega) = 0 +
  \widetilde{\Omega}$, the $b$ associated with $z_0 = 0$ must itself
  belong to $\widetilde{\Omega}$.  Therefore, we have proved
  $$
  F(z + \Omega) = az + \widetilde{\Omega}.
  $$
  %%% page 274
  This much has been done assuming only that $F$ is analytic and not
  necessarily one-to-one.

  If we assume that $F : \CCC / \Omega \to \CCC / \widetilde{\Omega}$
  is one-to-one and onto, then since $F(0 + \Omega) = 0 +
  \widetilde{\Omega}$, we have
  $$
  z + \Omega = 0 + \Omega
  \Leftrightarrow F(z + \Omega) = F(0 + \Omega)
  \Leftrightarrow az + \widetilde{\Omega} = 0 + \widetilde{\Omega}.
  $$
  That is,
  $$
  z \in \Omega \Leftrightarrow az \in \widetilde{\Omega}.
  $$
  But this means that $\widetilde{\Omega} = a\Omega$.
\end{mdframed}

Of course, we can describe the relation $\widetilde{\Omega} = a\Omega$
algebraically rather than geometrically.  If
$$
\Omega = \{n_1 w_1 + n_2 w_2 ~:~ n_1, n_2 \in \ZZ\},\quad
\widetilde{\Omega} = \{n_1 \widetilde{w}_1 + n_2 \widetilde{w}_2 
~:~ n_1, n_2 \in \ZZ\},
$$
then we have
$$
aw_1 = n_{11} \widetilde{w}_1 + n_{12} \widetilde{w}_2,\quad
aw_2 = n_{21} \widetilde{w}_1 + n_{22} \widetilde{w}_2,
$$
for certain integers $n_{jk}$.  Also,
$$
\widetilde{w}_1 = m_{11} a\widetilde{w}_1 + m_{12} a\widetilde{w}_2,\quad
\widetilde{w}_2 = m_{21} a\widetilde{w}_1 + m_{22} a\widetilde{w}_2,
$$
%%% page 275
for integers $m_{jk}$.  Therefore, we have the product of matrices
$$
\begin{pmatrix}
  n_{11} & n_{12} \\ n_{21} & n_{22}
\end{pmatrix}
\begin{pmatrix}
  m_{11} & m_{12} \\ m_{21} & m_{22}
\end{pmatrix}
=
\begin{pmatrix}
  1 & 0 \\ 0 & 1
\end{pmatrix}.
$$
Therefore, the product of the determinants is 1, or
$$
n_{11} n_{22} - n_{21} n_{12} = \pm 1.
$$
Conversely, if this equation holds, then the relations expressing
$aw_1$ and $aw_2$ in terms of $\widetilde{w}_1$ and $\widetilde{w}_2$
can be inverted, and thus $\widetilde{\Omega} = a \Omega$.

Almost as an afterthought we mention that if $S$ is a compact,
connected Riemann surface, then a necessary and sufficient condition
that there exist a meromorphic function on $S$ which takes every value
2 times is that $S$ be analytically equivalent to the Riemann surface
of a polynomial
$$
w^2 - (z-a_1)(z-a_2) \cdots (z-a_\ell),
$$
where $a_1, a_2, \cdots, a_\ell$ are distinct.

The proof is almost trivial.  If $S$ is the Riemann surface of the
above polynomial, then the function $\pi$ takes every value 2 times
since the polynomial has degree 2 in $w$.  Conversely, suppose $f$ is
a meromorphic function on $S$ which takes every value 2 times.  By the
proof of the Corollary 6.7, there exists a meromorphic function $g$ on
%%% page 276
$S$ such that $f$ and $g$ satisfy a polynomial equation which is
irreducible and has degree 2 in $g$.  Thus, for certain rational
functions $a$ and $b$,
$$
g^2 - 2a(f)g + b(f) = 0.
$$
Completing the square,
$$
(g - a(f))^2 = a(f)^2 - b(f).
$$
Let $g_1 = g - a(f)$ and $f_1 = a(f)^2 - b(f)$.  Then
$$
g_1^2 = f_1.
$$
Since $f_1$ is a rational function of $f$, we can write
$$
g_1^2 = \alpha^2 \frac{\prod_{k=1}^m (f - \alpha_k)^2}{
  \prod_{k=1}^{m'} (f - \alpha_k')^2}
  \times \frac{\prod_{j=1}^n (f - \beta_j)}{
    \prod_{j=1}^{n'} (f - \beta_j')},
$$
where the $\alpha$'s and $\beta$'s are complex constants and the
numbers $\beta_j$ and $\beta_j'$ are distinct.  Let
$$
g_2 = \frac{1}{\alpha} \frac{\prod_{k=1}^{m'} (f - \alpha_k')}{
  \prod_{k=1}^{m} (f - \alpha_k)}
  \prod_{j=1}^{n'} (f - \beta_j') g_1.
$$
Thus, we have produced distinct complex numbers $\alpha_1, \cdots,
\alpha_\ell$ ($\ell = n + n'$) and a meromorphic function $g_2$ such
that
$$
g_2^2 = \prod_{k=1}^\ell (f - \alpha_k).
$$
This type of Riemann surface is called \emph{hyperelliptic}.

\end{document}

%%% Local Variables: 
%%% mode: latex
%%% TeX-master: t
%%% End: 
