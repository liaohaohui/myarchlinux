\documentclass[a4paper,11pt]{article}
\usepackage{amsmath}
\usepackage{amsfonts}
\usepackage{verbatim}
\usepackage{url}
\usepackage{framed}
\usepackage{epsfig}
\usepackage{enumerate}
\usepackage{textcomp} %\textquotesingle
\usepackage[
%sorting=nyt,
firstinits=true, % render first and middle names as initials
useprefix=true,
maxcitenames=3,
maxbibnames=99,
style=authoryear,
dashed=false, % re-print recurring author names in bibliography
natbib=true,
url=false
]{biblatex}
%%% http://tex.stackexchange.com/questions/12254/biblatex-how-to-remove-the-parentheses-around-the-year-in-authoryear-style
\usepackage{xpatch}
\addbibresource{complex.bib} % run: biber topic1
\usepackage{color}
\usepackage{listings}
\definecolor{gray}{gray}{0.5} 
\definecolor{key}{rgb}{0,0.5,0} 
\lstset{ 
  language=[90]Fortran,
  basicstyle=\ttfamily\small, 
  keywordstyle=\color{blue}, 
  stringstyle=\color{red}, 
  showstringspaces=false, 
  emphstyle=\color{black}\bfseries, 
  emph={[2]True, False, None, self}, 
  emphstyle=[2]\color{key}, 
  emph={[3]from, import, as},
  emphstyle=[3]\color{blue}, 
  upquote=true, 
  morecomment=[s]{"""}{"""}, 
  commentstyle=\color{gray}\slshape, 
  %framexleftmargin=1mm, framextopmargin=1mm, frame=shadowbox, 
  rulesepcolor=\color{blue},
  numbers=left,
  stepnumber=1,
}
\usepackage{enumerate}
\usepackage{tikz}
\usetikzlibrary{lindenmayersystems}
\usetikzlibrary[shadings]


%%% Page Layout
\oddsidemargin=0truecm
\evensidemargin=0truecm
\textwidth=160truemm
\textheight=260truemm
\leftmargin=0truemm
\rightmargin=0truemm
\voffset=-23truemm
\topmargin=0truemm

\newif\iflecturer
%\lecturerfalse
\lecturertrue

\iflecturer
\usepackage{marginnote} % \marginpar
%\usepackage[color]{showkeys}
\definecolor{refkey}{rgb}{1,0,0}
\definecolor{labelkey}{rgb}{1,0,0}
\else
\def\marginpar#1{}
\fi

\iflecturer
\newcommand{\Answer}[1]{\dotfill\underline{\mbox{\hspace{1em}\color{blue}#1}}}
\newcommand{\BoxAns}[1]{\fbox{\color{blue}#1}}
\newcommand{\Reason}[1]{{\par\color{blue}\par{}Reason: #1}}
\else
\newcommand{\Answer}[1]{\dotfill\underline{\mbox{\hspace{1em}\color{white}#1}}}
\newcommand{\BoxAns}[1]{\fbox{\color{white}#1}}
\newcommand{\Reason}[1]{{\par\color{white}\par{}Reason: #1}}
\fi

\newif\ifamsstyle
\amsstylefalse
\newcounter{topic}
\input common
\input symbol
\makeatletter
\newcommand{\Arctan}{{\mathop{\operator@font Arctan}\nolimits}}
\makeatother

%\parindent=0pt
\parskip=1pt
\linespread{1}


\setcounter{topic}{7}

\begin{document}

\title{{\sc Rudiments of Riemann Surfaces\\
    Topic \thetopic{}: Classification of Simply Connected Riemann Surfaces}}
\author{Author: B. Frank Jones, Jr. (Rice Univ. 1971)\\
Seminar: Dr Liew How Hui (\url{liewhh@utar.edu.my})}
\date{}

\maketitle

%%% page 229
As an application of the results of the previous topic, we are going
to prove that every simply connected Riemann surface is analytically
equivalent to the sphere $\widehat{\CCC}$, the complex plane $\CCC$,
or the unit disk $\Delta$.  These cases are exclusive, of course,
since the compactness of the sphere shows it is not even homeomorphic
to the plane or disk; and the plane and disk, though homeomorphic, are
not analytically equivalent (Liouville's Theorem).

We shall require some slight generalisations of some of the basic
results in Topic 3.  Namely, we shall require a \emph{permanence of
  functional relations} and a \emph{monodromy theorem}.  In addition,
we shall require a generalisation of Lemma~5.4 from Topic 5 which
deals with unrestricted analytic continuation.

The framework for this discussion has been mentioned -- the analytic
continuation of meromorphic functions defined on arbitrary Riemann
surfaces, rather than $\CCC$.  Given a Riemann surface $S$, we can
form definitions as the beginning of Topic 3 and speak of $M_S$, the
\emph{sheaf of germs of meromorphic functions on $S$}.  The
applications we have in mind are given in the next two lemmas.

%%% page 230
\begin{lem}
  \label{lem:1}
  Let $p \in S$, a simply connected Riemann surface.  Let $u$ be
  harmonic on $S - \{p\}$ such that if $\varphi$ is an analytic chart
  near $p$ with $\varphi(p) = 0$, then
  $$
  u \circ \varphi^{-1}(z) = \Re(\sum_{k=N}^{\infty} a_k z^k),\quad
  z\text{ near }0,
  $$
  where $a_N \ne 0$.  Here $-\infty < N < \infty$.  Then there
  exists a meromorphic function $f$ on $S$ such that
  \begin{equation}
    \label{e:1}
    \Re(f) \equiv u.
  \end{equation}
\end{lem}

\begin{myproof}
  It is obvious that we may define $f$ near $p$ by setting
  $$
  f \circ \varphi^{-1}(z) = \sum_{k=N}^{\infty} a_k z^k,\quad
  z \text{ near } 0.
  $$
  It is now a question of continuing $f$ analytically to all of $S$.
  The generalised principle of the permanence of functional relations
  implies that the analytic continuation will always satisfy the
  identity \eqref{e:1}.  Briefly, the reason is that if $f$ is
  meromorphic in an analytic disk $D$ and $\Re(f) = u$ in a
  neighbourhood of some point of $D$, then $\Re(f) = u$ holds
  throughout $D$ (see Topic 6 Proposition 6.11).

  The second point is that analytic continuation is possible along
  every path in $S$ with initial point $p$.  The reason is that
  item 1 of Topic 6 Proposition 6.9 shows that \eqref{e:1} holds
  locally; this and the permanence of functional relations combine as
  in the proof of Lemma 5.4 (Topic 5) to show that the process of
  analytic continuation never ``stops''.

  %%% page 231
  Now we have the hypothesis needed to apply the monodromy theorem,
  and the lemma is proved.
\end{myproof}

\begin{lem}
  \label{lem:2}
  Let $p \in S$, a simply connected Riemann surface.  Let $u$ be
  harmonic on $S - \{p\}$ such that if $\varphi$ is an analytic chart
  near $p$ with $\varphi(p) = 0$, then
  $$
  u \circ \varphi^{-1}(z) = \log|z| + \Re(\sum_{k=0}^{\infty} a_k
  z^k),
  \quad z\text{ near } 0.
  $$
  Then there exists a holomorphic function $f$ on $S$ such that
  \begin{equation}
    \label{e:2}
    |f| \equiv e^u.
  \end{equation}
\end{lem}

\begin{myproof}
  The outline of the proof is the same as in the previous lemma.
  First, we prove that $f$ exists near $p$.  Using $e^{\log|z|} =
  |z|$, we naturally choose
  $$
  f \circ \varphi^{-1}(z) = z \exp(\sum_{k=0}^{\infty} a_k z^k),
  \quad z\text{ near } 0.
  $$
  Then \eqref{e:2} obviously holds near $p$.  Second, we apply the
  permanence of functional relations to show that \eqref{e:2} remains
  valid under analytic continuation of $f$.  The point to be checked
  %%% page 232
  is that if $f$ is holomorphic in an analytic disk and \eqref{e:2}
  holds in a neighbourhood of some point, then \eqref{e:2} holds
  throughout the disk.  This follows as before since $\log|f|$ is
  harmonic.  One might think there is trouble here at \emph{zeros} of
  $f$; by \eqref{e:2}, however, if $f$ has a zero along some path of
  analytic continuation, then \eqref{e:2} will have been violated
  before the zero is reached.

  Third, \eqref{e:2} holds locally at least.  For locally we can write
  $u = \Re F$, $F$ holomorphic (we are not now treating neighbourhoods
  of the exceptional point $p$).  Then we set $f = e^F$, implying
  \eqref{e:2}.  Therefore, as before, analytic continuation is
  possible along every path from $p$.  We are also using in this step
  the fact that \eqref{e:2} determines $f$ locally essentially
  uniquely.  That is, any other choice of $f$ is just $f$ multiplied
  by a constant of modulus 1, since holomorphic functions with
  constant modulus must be constant.  (A similar fact about
  \eqref{e:1} was used implicitly in the proof of Lemma~\ref{lem:1};
  in that case functions satisfying \eqref{e:1} constant
  \emph{differences}.)

  Fourth, the monodromy theorem finishes the proof.
\end{myproof}

Next, a technicality.
%%% page 233
\begin{lem}
  \label{lem:3}
  Let $E$ be any bounded nonempty set in $\CCC$.  Then there exist
  complex numbers $\alpha$ and $\beta$, $\alpha \ne 0$, such that if
  $$
  \widetilde{E} = \{\alpha z + \beta ~:~ z \in E\},
  $$
  then
  $$
  \sup \{|w| ~:~ w \in \tilde{E}\} = 1,\quad
  \inf \{|w| ~:~ w \in \tilde{E}\} = \frac{1}{2}.
  $$
\end{lem}

\begin{myproof}
  Let $a = \inf \{ \Re(z) : z \in E\}$ and choose $b$ such that $a +
  ib \in E^-$ (using the boundedness of $E$).  Let $E_1 = \{z - a - ib
  ~:~ z \in E\}$, so that $\inf \{\Re(z) : z \in E_1 \} = 0$, $0 \in
  E_1^-$.  Define for $t \ge 0$
  $$
  m(t) = \inf\{|z+t| ~:~ z \in E_1\},\quad
  M(t) = \sup\{|z+t| ~:~ z \in E_1\}.
  $$
  Then $m$ and $M$ are continuous increasing functions, $m(0) = 0$,
  and the boundedness of $E_1$ implies $\frac{m}{M} \to 1$ as $t \to
  \infty$.  Choose $t$ such that $\frac{m(t)}{M(t)} =
  \frac{1}{2}$. Let $c = M(t)$ and
  $$
  \tilde{E} = \{\frac{z+t}{c} ~:~ z \in E_1\}.
  $$
  Then $\tilde{E}$ satisfies the conditions of the lemma, and
  $$
  \alpha = \frac{1}{c},\quad
  \beta = \frac{t-a-ib}{c}.
  $$
\end{myproof}

%%% page 234
\begin{thm}[The Famous Classification Theorem: Uniformisation Theorem]
  Any connected, simply connected Riemann surface is analytically
  equivalent to the Riemann sphere, the complex plane, or the unit
  disk.
\end{thm}

\begin{mdframed}[skipbelow=1ex]
  \textbf{Proof}:
  Let $S$ be the connected, simply connected Riemann surface.  We have
  three cases to consider.

  \textbf{$S$ is compact}:  By the Corollary~6.47 to Theorem~6.12, if
  $p \in S$, then there exists a harmonic function $u$ on $S - \{p\}$
  such that in terms of a given analytic chart $\varphi$ near $p$ near
  $\varphi(p) = 0$,
  $$
  u \circ \varphi^{-1}(z) = \Re(\frac{\alpha}{z}) + \Re(F(z)),\quad
  z \text{ near } 0,\ \alpha \ne 0,
  $$
  where $F$ is holomorphic near 0.  By Lemma~\ref{lem:1} there exists
  a meromorphic function $f$ on $S$ such that
  $$
  \Re(f) \equiv u.
  $$
  Now the only pole of $f$ is the point $p$, and this is a pole of
  order 1.  Thus, $f$ takes the value $\infty$ exactly one time.  By
  item 1 of Proposition 2.24, $f$ takes every value in
  $\widehat{\CCC}$ exactly one time.  That is, $f : S \to
  \widehat{\CCC}$ is an analytic equivalence between $S$ and
  $\widehat{\CCC}$, proving the result in this case. 

  \textbf{$S$ is parabolic}: If $p \in S$, and $\varphi$ is an
  analytic chart in a neighbourhood of $p$, then by the Corollary~6.47
  to Theorem~6.12, there exists a harmonic function $u$ on $S - \{p\}$
  such that
  $$
  u \circ \varphi^{-1}(z) = \Re(\frac{\alpha}{z}) + \Re(F(z)),\quad
  z \text{ near } 0,\ \alpha \ne 0,
  $$
  %%% page 235
  where $F$ is holomorphic near 0.  We are assuming $\varphi(p) = 0$.
  The construction of $u$ shows that $u$ is \emph{bounded} outside any
  neighbourhood of $p$.  For this, checkout Topic 6, where $0 \le v
  \le 1$ in $S-D_1$; $u = v$ in $S-D_1$; and our function $u$ is a
  linear combination of two functions bounded in $S-D_1$.  Note that
  we are tacitly assuming that $\varphi$ is rescaled if necessary to
  guarantee the existence of $D_2$, the analytic disk given by
  $|\varphi| < 2$.

  By Lemma~\ref{lem:1} there exists a meromorphic function $f$ on $S$
  such that
  $$
  \Re(f) \equiv u.
  $$
  Note that $f$ has a pole only at $p$, that $p$ is a simple pole, and
  that $\Re(f)$ is bounded outside any neighbourhood of $p$.  We wish
  to obtain another function with the stronger property that $|f|$ is
  bounded outside any neighbourhood of $p$.

  To do this let $a_1 < a_2 < a_3 < \cdots$ be a sequence of positive
  integers.  Since these are real numbers tending to $\infty$ and the
  real part of $f$ is one-to-one in a neighbourhood of $p$, it follows
  that for sufficiently large $n$ there exists a unique $p_n \in S$
  such that $f(p_n) = a_n$.  By eliminating the first few terms in the
  sequence, we can assume this holds for all $n$.  Also, it is clear
  that $p_n \to p$.  Since $\Re(f)$ is bounded outside any
  neighbourhood $U$ of $p$, we obtain for sufficiently large $n$
  %%% page 236
  $$
  |f-a_n| \ge a_n - \Re(f) \ge \frac{1}{2}a_n
  \text{ outside } U,
  $$
  and therefore since we can assume $f$ is one-to-one in $U$,
  $\frac{1}{f-a_n}$ has a simple pole exactly at $p_n$ and is bounded
  outside any neighbourhood of $p_n$.  By Lemma~\ref{lem:3} there
  exist constants $\alpha_n$ and $\beta_n$ such that if
  $$
  f_n = \frac{\alpha_n}{f-a_n} + \beta_n,
  $$
  then
  $$
  \sup_{D_2^- - D_1} |f_n| = 1,\quad
  \inf_{D_2^- - D_1} |f_n| = \frac{1}{2}.
  $$
  Since $S$ is parabolic, item 4 of Proposition~6.41 in Topic 6
  implies
  $$
  \sup_{S-D_1} |f_n| = 1
  $$
  ($|f_n|$ is subharmonic).

  By Proposition~6.17, there exists a subsequence $n_1 < n_2 < \cdots$
  such that $\lim_{k\to\infty} f_{n_k}$ exists uniformly on compact
  subsets of $D_2 - D_1^-$, and then item 4 of Proposition 6.41 again
  implies $\lim_{k\to\infty} f_{n_k}$ exists uniformly on $S -
  D_{3/2}$.  Let
  $$
  h = \lim_{k\to\infty} f_{n_k} \text{ in } S - D_1^-.
  $$
  %%% page 237
  Then $h$ is holomorphic on $S-D_1^-$ and $|h| \le 1$.  By renaming
  all the sequences, we can assume $n_k \equiv k$.

  Now we consider $f_n$ in $D_2$, where $f_n$ has a simple pole at
  $p_n$ and no other pole.  Thus, we may write
  $$
  f_n = \frac{g_n}{\varphi - \varphi(p_n)}, \text{ in } D_2,
  $$
  where $g_n$ is holomorphic in $D_2$.  Note that in $D_2 - D_1^-$
  $$
  \begin{aligned}
    |g_n - g_m|
    &= |[\varphi-\varphi(p_n)]f_n - [\varphi-\varphi(p_m)]f_m|\\
    &\le |\varphi-\varphi(p_n)||f_n-f_m| 
    + |\varphi(p_n)-\varphi(p_m)||f_m|\\
    &\le 4|f_n - f_m| + |\varphi(p_n) - \varphi(p_m)|,
  \end{aligned}
  $$
  and therefore the sequence $g_n$ converges uniformly on, say, $\p
  D_{3/2}$ (since $p_n \to p$, $\varphi(p_n) \to 0$).  By the maximum
  principle, $g_n$ converges uniformly in $D_{3/2}$, say
  $$
  \lim_{n\to \infty} g_n = g, \text{ holomorphic  in } D_{3/2}.
  $$
  Now define
  $$
  f_p =
  \begin{cases}
    h &\text{in }S-D_1^-,\\
    \frac{g}{\varphi} &\text{in }D_{3/2}.
  \end{cases}
  $$
  Then we see that $f_p$ is well defined, is meromorphic on $S$, and
  has at most a pole of first order at $p$ and no other poles.  Since
  $g_n \to g$ uniformly in $D_{3/2}$ and $f_n \to h$ uniformly in $S -
  D_{3/2}$, we obtain the result that $f_n \to f_p$ uniformly in
  $D_2^- - D_1$.  Therefore,
  %%% page 238
  $$
  \sup_{D_2^- - D_1} |f_p| = 1,\quad
  \inf_{D_2^- - D_1} |f_p| = \frac{1}{2},
  $$
  proving that $f_p$ is not constant.  Since $S$ is parabolic, the
  nonconstant function $|f_p|$ cannot be bounded, and since $|f_p| \le
  1$ in $S-D_1^-$, it follows that $f_p$ really does have a pole at
  $p$.

  Summarising the construction thus far, we have shown that for every
  $p \in S$ there exists a meromorphic function $f_p$ on $S$ such that
  $f_p$ has a pole of order 1 at $p$ and $f_p$ is bounded outside
  every neighbourhood of $p$.

  These conditions essentially uniquely determine $f_p$.  For if
  $\tilde{f}_p$ has the same properties, then there is a unique
  complex number $\alpha \ne 0$ such that $\tilde{f}_p - \alpha f_p$
  has no pole at $p$, and is therefore a bounded meromorphic function
  on all of $S$.  Since $S$ is parabolic, $\tilde{f}_p - \alpha f_p$
  is constant, and thus
  $$
  \tilde{f}_p = \alpha f_p + \beta.
  $$
  Conversely, for any constants $\alpha$ and $\beta$, $\alpha \ne 0$,
  the function $\alpha f_p + \beta$ has properties similar to those of
  $f_p$.

  Also, for a given fixed $p$, the function $\frac{1}{f_p - f_p(q)}$
  has a simple pole at $q$ if $q$ is in a sufficiently small
  neighbourhood of $p$, and $\frac{1}{f_p - f_p(q)}$ has no other
  poles.
  %%% page 239
  Furthermore, this function is bounded outside any neighbourhood of
  $q$, if $q$ is sufficiently near $p$.  Thus, by the remark above,
  $$
  \frac{1}{f_p - f_p(q)} = \alpha f_q + \beta,
  $$
  where $\alpha$ and $\beta$ are constants depending only on $q$.
  Thus, for $q$ sufficiently near $p$ there exists a M\"obius
  transformation $T_q$ such that
  $$
  f_q = T_q \circ f_p.
  $$

  Now let $p_0$ be a fixed point in $S$ and let
  $$
  A = \{ p \in S ~:~ \exists \text{ M\"obius transformation } T
  \text{ such that } f_p = T \circ f_{p_0} \}.
  $$
  Then $p_0 \in A$, and the argument just given shows that $A$ is
  \textbf{open}.  The same argument shows that $A$ is
  \textbf{closed}.  In both cases we rely on the fact that the
  M\"obius transformations form a \textbf{group} under composition.
  Since $S$ is connected $A = S$.

  Now we prove that $f_{p_0}$ is \emph{one-to-one}.  Suppose
  $$
  f_{p_0}(p) = f_{p_0}(q).
  $$
  Then there exists a M\"obius transformation $T$ such that
  $$
  f_p = T \circ f_{p_0}.
  $$
  %%% page 240
  Therefore,
  $$
  \infty = f_p(p) = T(f_{p_0}(p)) = T(f_{p_0}(q)) = f_p(q).
  $$
  Since $f_p$ has pole at $p$ only, $q = p$.

  Now we prove that $\widehat{\CCC} - f_{p_0}(S)$ cannot have more
  than one point.  Otherwise, there are two complex numbers $\alpha,
  \beta \notin f_{p_0}(S)$ --- note that definitely $\infty \in
  f_{p_0}(S)$.  Since $f_{p_0}(S)$ is simply connected, the monodromy
  theorem implies there exists a holomorphic determination of 
  $$
  \sqrt{\frac{w-\alpha}{w-\beta}}
  $$
  for $w \in f_{p_0}(S)$; choose that determination which is 1 at $w =
  \infty$.  Define
  $$
  F = \sqrt{\frac{f_{p_0}-\alpha}{f_{p_0}-\beta}}.
  $$
  Then $F$ is holomorphic on $S$ and $F(p_0) = 1$.  Furthermore, $F$
  never takes the value zero and it is impossible for $F(p) =
  -F(p')$.  For if this holds, then $F(p)^2 = F(p')^2$, which implies
  $f_{p_0}(p) = f_{p_0}(p')$ and $p = p'$, since $f_{p_0}$ is
  one-to-one.  Since $F$ is not constant and takes the value 1, $F$
  takes every value $z$ for $|z-1| < \epsilon$, some $\epsilon > 0$.
  Therefore,
  $$
  |F(p) + 1| \ge \epsilon
  \text{ for all } p \in S.
  $$
  Thus, $\frac{1}{F+1}$ is a bounded holomorphic function on $S$ and
  is therefore constant since $S$ is parabolic.  This is a
  contradiction.

  %%% page 241
  Now we cannot have $f_{p_0}(S) = \widehat{\CCC}$ by item 2 of
  Proposition 2.24 (Topic 2) since $S$ is \textbf{not} compact.
  Therefore, there is a unique complex $\gamma$ such that $f_{p_0}(S)
  = \widehat{\CCC} - \{\gamma\}$.  Therefore
  $$
  \frac{1}{f_{p_0} - \gamma}
  $$
  is a one-to-one analytic mapping of $S$ onto $\CCC$, and thus forms
  the desired analytic equivalence between $S$ and $\CCC$.


  \textbf{$S$ is hyperbolic}: Here we use Proposition~6.49 (Topic 6)
  and let $g_p$ be the unique Green's function on $S - \{p\}$.  By
  Lemma~\ref{lem:2} there exists a holomorphic function $f_p$ on $S$
  such that
  $$
  |f_p| \equiv e^{-g_p}.
  $$
  Then
  \begin{enumerate}
  \item $f_p(p) = 0$,
  \item $|f_p| < 1$,
  \item $f_p$ is holomorphic on $S$,
  \item $f_p$ does not vanish on $S-\{p\}$,
  \item if $h$ is a function on $S$ satisfying 1., 2., 3. then $|h|
    \le |f_p|$.
  \end{enumerate}
  %%% page 242
  We have to prove the last statement.  By 1., if $\varphi$ is an
  analytic chart with $\varphi(p) = 0$, then near $p$
  $$
  h = \alpha \varphi^n (1 + \beta \varphi + \cdots),
  $$
  where we can assume $\alpha \ne 0$ and $n \ge 1$.  Thus, near $p$ we
  have
  $$
  \log|h| = n\log|\varphi| + \log|\alpha| + \log|1 + \beta \varphi +
  \cdots|,
  $$
  showing that
  $$
  \frac{-\log|h|}{n} + \log|\varphi|
  $$
  is harmonic near $p$.  Also, $\frac{-\log|h|}{n} > 0$ by item 2. and is
  harmonic away from zeros of $h$.  Let
  $$
  \widetilde{h} = \min(g_p, \frac{-\log|h|}{n}).
  $$
  Then $\widetilde{h}$ is superharmonic on $S - \{p\}$, $\widetilde{h}
  > 0$, and near $p$
  $$
  \widetilde{h} + \log|\varphi| 
  = \min(g_p + \log|\varphi|, \frac{-\log|h|}{n} + \log|\varphi|)
  $$
  is superharmonic, being the mininum of two harmonic functions near
  $p$.  By the minimal property of the Green's function,
  $$
  g_p \le \widetilde{h}.
  $$
  Thus,
  $$
  g_p \le \frac{-\log|h|}{n},
  $$
  so
  $$
  |f_p| = e^{-g} \ge e^{\log|h|/n} = |h|^{1/n},
  $$
  %%% page 243
  and thus
  $$
  |h| \le |f_p|^n \le |f_p|.
  $$
  This proves property 5.

  Now let $p,q \in S$ and set
  $$
  h - \frac{f_p - f_p(q)}{1 - \overline{f_p(q)}f_p}.
  $$
  Then since all numbers involved have modulus less than 1, we see
  that $h(q) = 0$, $|h| < 1$, $h$ is holomorphic on $S$.  Therefore,
  property 5 above implies
  $$
  |h| \le |f_q|.
  $$
  Since $h(p) = -f_p(q)$, we obtain in particular
  $$
  |f_p(q)| \le |f_q(p)|.
  $$
  By symmetry we conclude
  $$
  |f_p(q)| = |f_q(p)| \text{ for all } p, q \in S.
  $$
  (In terms of the Green's functions, this relation states
  $$
  g_p(q) = g_q(p).)
  $$

  Thus, we conclude that the holomorphic function $\frac{h}{f_q}$
  satisfies
  %%% page 244
  $$
  \left|\frac{h}{f_q}\right| \le 1 \text{ on } S,\quad
  \left|\frac{h(p)}{f_q(p)}\right| = 1.
  $$
  By the strong maximum principle, it follows that $\frac{h}{f_q}$ is
  constant, and in particular
  \begin{equation}
    \label{e:3}
    \left|\frac{h}{f_q}\right| \equiv 1 \text{ on } S.
  \end{equation}
  Now we prove that $f_p$ is \emph{one-to-one}.  Suppose $f_p(q) =
  f_p(q')$.  Then $h(q') = 0$ and by \eqref{e:3} $f_q(q') = 0$.  By
  property 4. of $f_q$, we conclude that $q' = q$.

  Therefore, for any $p \in S$, $f_p$ is a one-to-one analytic mapping
  of $S$ into the unit disk $\Delta$.  We prove that $f_p(S) =
  \Delta$.  If this is not the case, then a simple topological
  argument shows that there exists
  $$
  \alpha \in \p f_p(S),\quad |\alpha| < 1.
  $$
  Since $f_p(S)$ is open, $\alpha \notin f_p(S)$.  Choose $p_1, p_2,
  p_3, \cdots$ in $S$ such that
  $$
  f_p(p_n) \to \alpha.
  $$
  Since $f_p(S)$ is simply connected and $\alpha \notin f_p(S)$, the
  monodromy theorem implies there exists an analytic determination of
  $\log(w - \alpha)$ for $w \in f_p(S)$.  Note that
  $$
  \Re(\log(f_p - \alpha)) = \log|f_p - \alpha| < \log 2.
  $$
  Let $T$ be a M\"obius transformation mapping
  $$
  \{z ~:~ \Re(z) < \log2\}
  $$
  onto $\Delta$ and such that $T(\log(-\alpha)) = 0$.  Consider the
  function
  $$
  F = T \circ \log(f_p - \alpha).
  $$
  Then $F$ is holomorphic on $S$ and $|F| < 1$, $F(p) =
  T(\log(-\alpha)) = 0$.  By property 5. of $f_p$,
  $$
  |F| \le |f_p|.
  $$
  We therefore conclude successively that
  $$
  \begin{aligned}
    f_p(p_n) - \alpha &\to 0,\\
    \log(f_p(p_n) - \alpha) &\to \infty,\\
    T\circ \log(f_p(p_n) - \alpha) &\to \p \Delta,
  \end{aligned}
  $$
  i.e. $|F(p_n)| \to 1$, and thus $|f_p(p_n)| \to 1$.
  Thus, $|\alpha| = 1$, a contradiction.
\end{mdframed}

\begin{cor}[The Riemann Mapping Theorem]
  Let $S$ be a connected, simply connected open subset of $\CCC$ with
  $S \ne \CCC$.  Then there exists an analytic equivalence of $S$ and
  the unit disk.
\end{cor}

%%% page 246
\begin{mdframed}[skipbelow=1ex]
  \textbf{Proof}: 
  We have only show that the Riemann surface $S$ is hyperbolic.  If
  $\alpha \in \CCC - S$, then there exists a holomorphic determination
  of $\sqrt{w - \alpha}$ for $w \in S$.  Define $F(w) = \sqrt{w -
    \alpha}$.  Then one shows that $F(w) = -F(w')$ implies $w = w'$ by
  squaring both sides, so that it is impossible that $F(w) = -F(w')$.
  Suppose $w_0 \in S$.  Since $F$ is an open mapping, there exists
  $\epsilon > 0$ such that $F(S)$ includes the set $\{ z ~:~ |z -
  F(w_0)| < \epsilon\}$.  Therefore, $F(S)$ is disjoint form the set
  $\{ z ~:~ |z + F(w_0)| < \epsilon \}$, or, $|F(w) + F(w_0)| \ge
  \epsilon$ for all $w \in S$.  Thus $\frac{1}{F + F(w_0)}$ is a
  bounded, nonconstant holomorphic function on $S$, proving that $S$
  is hyperbolic.
\end{mdframed}

Now we want to indicate some applications of the classification
theorem.  The first of these is a trivial application, but answers the
question of which Riemann surfaces are homeomorphic to a sphere.

\begin{thm}
  \label{thm:1}
  Let $S$ be a connected compact Riemann surface.  Then the following
  conditions are equivalent.
  \begin{enumerate}
  \parskip=0pt
  \itemsep=0pt
  \item $S$ is analytically equivalent to $\widehat{\CCC}$.
  \item $S$ is homeomorphic to $\widehat{\CCC}$.
  \item $S$ is simply connected.
    %%% page 247
  \item There exists a meromorphic function $f$ on $S$ such that every
    meromorphic function on $S$ is a rational function of $f$.
  \item There exists a meromorphic function $f$ on $S$ having a simple
    pole at some point and no other pole.
  \end{enumerate}
\end{thm}

\begin{mdframed}[skipbelow=1ex]
  \textbf{Proof}:
  \begin{description}
  \parskip=0pt
  \itemsep=0pt
  \item[1 $\Rightarrow$ 2:] Trivial.
  \item[2 $\Rightarrow$ 3:] Trivial, since a sphere is simply
    connected.
  \item[3 $\Rightarrow$ 1:] Follows from the classification theorem.
  \item[1 $\Rightarrow$ 4:] We can assume $S = \widehat{\CCC}$ and we
    then take $f(z) = z$.  The result is immediate.
  \item[4 $\Rightarrow$ 5:] We prove that the function $f$ of 4. must
    be one-to-one.  Suppose $p, q \in S$, $p \ne q$.  By Theorem~6.6
    of Topic 6, there exists a meromorphic function $g$ on $S$
    such that $g(p) \ne g(q)$.  By condition 4, there exists a
    rational function $A$ such that $g = A \circ f$.  Thus, $A(f(p))
    \ne A(f(q))$, which implies $f(p) \ne f(q)$.  By item 2 of
    Proposition 2.24 of Topic 2, $f$ takes every value the same number
    (one) of times, of $f$ takes the value $\infty$ one time.
  \item[5 $\Rightarrow$ 1:] By item 2 of Proposition 2.24 of Topic 2,
    $f$ maps $S$ onto $\widehat{\CCC}$ in a one-to-one fashion.  Thus,
    $f$ is an analytic equivalence of $S$ onto $\widehat{\CCC}$.
  \end{description}
\end{mdframed}

We are now going to discuss the next easiest case.
Theorem~\ref{thm:1} is concerned with a compact surface of genus 0.
We shall next discuss the compact surfaces of genus 1.  This case is
already so involved that we shall devote a separate topic to it.

\end{document}


%%% Local Variables: 
%%% mode: latex
%%% TeX-master: t
%%% End: 
