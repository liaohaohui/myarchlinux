\documentclass[a4paper,11pt]{article}
\usepackage{amsmath}
\usepackage{amsfonts}
\usepackage{verbatim}
\usepackage{url}
\usepackage{framed}
\usepackage{epsfig}
\usepackage{enumerate}
\usepackage{textcomp} %\textquotesingle
\usepackage[
%sorting=nyt,
firstinits=true, % render first and middle names as initials
useprefix=true,
maxcitenames=3,
maxbibnames=99,
style=authoryear,
dashed=false, % re-print recurring author names in bibliography
natbib=true,
url=false
]{biblatex}
%%% http://tex.stackexchange.com/questions/12254/biblatex-how-to-remove-the-parentheses-around-the-year-in-authoryear-style
\usepackage{xpatch}
\addbibresource{complex.bib} % run: biber topic1
\usepackage{color}
\usepackage{listings}
\definecolor{gray}{gray}{0.5} 
\definecolor{key}{rgb}{0,0.5,0} 
\lstset{ 
  language=[90]Fortran,
  basicstyle=\ttfamily\small, 
  keywordstyle=\color{blue}, 
  stringstyle=\color{red}, 
  showstringspaces=false, 
  emphstyle=\color{black}\bfseries, 
  emph={[2]True, False, None, self}, 
  emphstyle=[2]\color{key}, 
  emph={[3]from, import, as},
  emphstyle=[3]\color{blue}, 
  upquote=true, 
  morecomment=[s]{"""}{"""}, 
  commentstyle=\color{gray}\slshape, 
  %framexleftmargin=1mm, framextopmargin=1mm, frame=shadowbox, 
  rulesepcolor=\color{blue},
  numbers=left,
  stepnumber=1,
}
\usepackage{enumerate}
\usepackage{tikz}
\usetikzlibrary{lindenmayersystems}
\usetikzlibrary[shadings]


%%% Page Layout
\oddsidemargin=0truecm
\evensidemargin=0truecm
\textwidth=160truemm
\textheight=260truemm
\leftmargin=0truemm
\rightmargin=0truemm
\voffset=-23truemm
\topmargin=0truemm

\newif\iflecturer
\lecturerfalse
%\lecturertrue

\iflecturer
\usepackage{marginnote} % \marginpar
%\usepackage[color]{showkeys}
\definecolor{refkey}{rgb}{1,0,0}
\definecolor{labelkey}{rgb}{1,0,0}
\else
\def\marginpar#1{}
\fi

\iflecturer
\newcommand{\Answer}[1]{\dotfill\underline{\mbox{\hspace{1em}\color{blue}#1}}}
\newcommand{\BoxAns}[1]{\fbox{\color{blue}#1}}
\newcommand{\Reason}[1]{{\par\color{blue}\par{}Reason: #1}}
\else
\newcommand{\Answer}[1]{\dotfill\underline{\mbox{\hspace{1em}\color{white}#1}}}
\newcommand{\BoxAns}[1]{\fbox{\color{white}#1}}
\newcommand{\Reason}[1]{{\par\color{white}\par{}Reason: #1}}
\fi

\newif\ifamsstyle
\amsstylefalse
\newcounter{topic}
\input common
\input symbol
\makeatletter
\newcommand{\Arctan}{{\mathop{\operator@font Arctan}\nolimits}}
\makeatother

%\parindent=0pt
\parskip=1pt
\linespread{1}


\setcounter{topic}{2}

\begin{document}

\title{{\sc Rudiments of Riemann Surfaces\\
    Topic \thetopic{}: Abstract Riemann Surfaces}}
\author{Author: B. Frank Jones, Jr. (Rice Univ. 1971)\\
Seminar: Dr Liew How Hui (\url{liewhh@utar.edu.my})}
\date{}

\maketitle


%%% page 24
In the introduction (Topic 1) we have considered on method of
constructing Riemann surfaces and have pointed out various properties.
In the rest of the course several other methods will be given,
especially the extremely important \emph{sheaf of germs of meromorphic
  functions} in Topic 3 and its generalisation, the \emph{analytic
  configuration}, in Topic 4.  Other examples will be considered in
the present topic.  All of these Riemann surfaces have one feature
that cries out for attention, so before coming to the concrete
examples we shall define this characteristic feature and call any
object which possess it a Riemann surface.

\section{Definition}

\begin{defn}
  A \emph{surface} is a Hausdorff space $S$ such that $\forall p \in
  S$ there is an open neighbourhood $U$ of $p$ and an open set $W
  \subset \CCC$ and a homeomorphism $\varphi : U \to W$. Such a
  mapping $\varphi$ is called a \emph{chart} or a \emph{coordinate
    mapping}.
\end{defn}

\begin{defn}
  Let $S$ be a surface.  An \emph{atlas} for $S$ is a collection of
  charts $\{\varphi_\alpha\}$, where $\alpha$ runs through some index
  set, such that every point of $S$ belongs to the domain of some
  $\varphi_\alpha$.  If $\varphi_\alpha : U_\alpha \to W_\alpha$, then
  we are saying that
  $$
  S = \bigcup_\alpha U_\alpha
  $$
  %%% page 25
  Note that if $U_\alpha$ and $U_\beta$ meet, then both
  $\varphi_\alpha$ and $\varphi_\beta$ are defined on the intersection
  $U_\alpha \cap U_\beta$ and these mappings provide homeomorphisms
  between this intersection and the open sets $\varphi_\alpha(U_\alpha
  \cap U_\beta)$ and $\varphi_\beta(U_\alpha \cap U_\beta)$ in $\CCC$,
  respectively.   Therefore, there is defined 
  $$
  \varphi_\alpha \circ \varphi_\beta^{-1} ~:~ \varphi_\beta(U_\alpha
  \cap U_\beta) \to \varphi_\alpha(U_\alpha \cap U_\beta).
  $$
\end{defn}


\begin{mdframed}
  \vspace{4cm}
\end{mdframed}

For brevity we shall frequently speak of $\varphi_\alpha \circ
\varphi_\beta^{-1}$ without mentioning that it is defined only on
$\varphi_\beta(U_\alpha \cap U_\beta)$.  The functions $\varphi_\alpha
\circ \varphi_\beta^{-1}$ are called \emph{coordinate transition
  functions} of the atlas, because if $\varphi_\alpha$ and
$\varphi_\beta$ are thought of as defining coordinates on $U_\alpha
\cap U_\beta$, the mapping $\varphi_\alpha \circ \varphi_\beta^{-1}$
determines how to change from one coordinate system to another.

%%% page 26
\begin{defn}
  An atlas $\{\varphi_\alpha\}$ is \emph{analytic} if each coordinate
  transition function $\varphi_\alpha \circ \varphi_\beta^{-1}$ is
  analytic.
\end{defn}

Just note that this definition makes good sense, as $\varphi_\alpha
\circ \varphi_\beta^{-1}$ is a complex-valued function on an open set
in $\CCC$ and thus the usual meaning of analytic function is what is
meant.

\begin{defn}
  Two charts $\varphi_1$ and $\varphi_2$ on a surface $S$ are
  \emph{compatible} if the functions $\varphi_1 \circ \varphi_2^{-1}$
  and $\varphi_2\circ \varphi_1^{-1}$ are analytic.  A chart $\varphi$
  is \emph{compatible with an analytic atlas} $\{\varphi_\alpha\}$ if
  $\varphi$ and $\varphi_\alpha$ are compatible for all $\alpha$.
\end{defn}

\begin{defn}
  \label{def:5}
  An analytic atlas is \emph{complete} if it contains every chart
  compatible with it.
\end{defn}

We are now almost ready to define a Riemann surface as a surface
together with an analytic atlas.  But there is a slight technical
problem which must be overcome.  Namely, there is almost never a
convenient canonical atlas, and we therefore either need to define
some sort of canonical atlas or need to define an equivalence relation
between analytic atlases.  Since these approaches are really the same,
we arbitrarily pick the former possibility.  This is the reason for
Definition~\ref{def:5}.  Now we give a lemma which actually relates
these concepts.

%%% page 27
\begin{lem}
  \label{lem:1}
  For any analytic atlas $\{\varphi_\alpha\}$ on a surface $S$, there
  exists exactly one complete analytic atlas containing it.  This
  complete analytic atlas is the collection of all charts compatible
  with $\{\varphi_\alpha\}$.
\end{lem}
\begin{myproof}
  Let $\sA$ be the set of all charts compatible with
  $\{\varphi_\alpha\}$.  We first prove that $\sA$ is an
  atlas, then that it is complete.  Suppose then that
  $\varphi,\varphi' \in \sA$. Thus, $\varphi : U\to W$ and
  $\varphi' : U' \to W'$ are homeomorphisms from open sets in $S$ to
  open sets in $\CCC$. 
  Suppose $U$ and $U'$ meet and let $p_0 \in U\cap U'$.  Since
  $\{\varphi_\alpha\}$ is an atlas, there exists $\varphi_\alpha :
  U_\alpha \to W_\alpha$ such that $p_0 \in U_\alpha$.  Then
  $$
  \varphi' \circ \varphi^{-1} = (\varphi' \circ \varphi_\alpha^{-1})
  \circ (\varphi_\alpha \circ \varphi^{-1})
  $$
  and $\varphi' \circ \varphi_\alpha^{-1}$ and $\varphi_\alpha \circ
  \varphi^{-1}$ are both holomorphic since $\varphi, \varphi'$ are
  compatible with $\varphi_\alpha$.  Thus, $\varphi' \circ
  \varphi^{-1}$ is holomorphic.  Thus $\sA$ is an analytic
  atlas.

  To prove that $\sA$ is complete, suppose $\psi$ is compatible with
  $\sA$.  Since $\sA$ contains $\{\varphi_\alpha\}$ (since
  $\{\varphi_\alpha\}$ is itself an analytic atlas), $\psi$ is
  compatible with $\{\varphi_\alpha\}$.  That is, $\psi \in \sA$.
  Thus, $\sC$ is complete.

  Finally, to prove that $\sA$ is unique, suppose $\sB$ is a complete
  analytic atlas containing $\{\varphi_\alpha\}$.  If $\varphi \in
  \sB$, then $\varphi$ is compatible with $\{\varphi_\alpha\}$, and
  thus $\varphi \in \sA$.  This proves $\sB\subset \sA$.  Now suppose
  $\varphi \in \sA$.  Let $\psi \in \sB$.  Arguing as above, we find
  $$
  \begin{aligned}
    \varphi \circ \psi^{-1} &= (\varphi \circ \varphi_\alpha^{-1})
    \circ (\varphi_\alpha \circ \psi^{-1}),\\
    \psi \circ \varphi^{-1} &= (\psi \circ \varphi_\alpha^{-1})
    \circ (\varphi_\alpha \circ \varphi^{-1}),
  \end{aligned}
  $$
  and thus $\varphi$ and $\psi$ are compatible.  Thus $\varphi$ is
  compatible with $\sB$.  As $\sB$ is complete, $\varphi \in \sB$.
  Thus, $\sA\subset \sB$.  Hence $\sA = \sB$.
\end{myproof}


%%% page 28
As a result of this lemma, we see that two analytic atlases are
contained in the same complete analytic atlas if and only if each
chart from one atlas is compatible with each chart from the other
atlas, or if and only if the union of the two atlases is itself an
analytic atlas.

\begin{defn}
  \label{def:6}
  A \emph{Riemann surface} is a surface together with a complete
  analytic atlas.
\end{defn}

Thus to specify an abstract Riemann surface, we must specify a surface
\textbf{and} a complete analytic atlas.

%%% page 29
The effective purpose of Lemma~\ref{lem:1} is to enable us to forget
about the rather cumbersome completeness assumption.  So when we wish
to construct a Riemann surface, we will be satisfied to exhibit
\textbf{one} analytic atlas, keeping in the back of our minds that
Lemma~\ref{lem:1} implies the existence of a unique larger complete
analytic atlas.  This is quite helpful, as it will usually be more or
less obvious what can be chosen to be an analytic atlas.

It is most important for beginners in this subject no to beguiled by
Definition~\ref{def:6}.   The crux of the theory of Riemann surfaces
is \textbf{not} this definition.  This definition just gives a
convenient term in a book-keeping sense to keep track of the structure
implied in the definition of complete analytic atlas.  Thus, this
topic has been called ``\emph{abstract} Riemann surfaces''.  It will
be up to us to verify for the many concrete Riemann surfaces we find
that the above definition obtains.

\section{Examples}

Now we pass to some examples

\begin{exmp}
  \begin{enumerate}
  \item This is by far the most trivial example.  Let $S$ be any open
    subset of $\CCC$; the atlas consists of the single chart $\varphi$
    which is the identity mapping on $S$.  In this case, $\varphi$ is
    obviously a homeomorphism and the only transition function is
    $\varphi \circ \varphi^{-1}$ = identity on $S$.

    %%% page 30
  \item A most important example is the \emph{Riemann sphere}.  We
    take this to be the topological space $\widehat{\CCC} = \CCC \cup
    \{\infty\}$, where points in $\CCC$ have their usual
    neighbourhoods and a neighbourhood basis of $\infty$ consists of
    the sets $\{z ~:~ |z| > a\} \cup \{\infty\}$ for $0 < a <
    \infty$.  This is clearly a topological space and stereographic
    projection is a homeomorphism of $\widehat{\CCC}$ onto the unit
    Euclidean sphere in $\RR^3$.  The atlas we pick will consist of
    two charts.  Let $U_1 = W_1 = \CCC$ and $\varphi_1 : U_1 \to W_1$
    be the identity.  Let $U_2 = \widehat{\CCC} - \{0\}$, $W_2 = \CCC$,
    and $\varphi_2 : U_2 \to W_2$ be given by $\varphi_2(z) = z^{-1}$,
    $\varphi_2(\infty) = 0$.  These are clearly charts, and $\varphi_2
    \circ \varphi_1^{-1}(z) = \varphi_2(z) = z^{-1}$, $\varphi_1
    \circ \varphi_2^{-1}(z) = \varphi_2(z) = z^{-1}$, which shows the
    coordinate transition functions are holomorphic.

  \item As we mentioned above, $\widehat{\CCC}$ is homeomorphic to the
    unit sphere in $\RR^3$.  It is a fact that any topological space
    homeomorphic to a Riemann surfaces can itself be made into a
    Riemann surface.  To see this, suppose $S$ is a Riemann surfacce
    with analytic atlas $\{\varphi_\alpha\}$ and that $T$ is a
    topological space and $\Phi : T \to S$ a homeomorphism.  Then the
    maps $\{\varphi_\alpha \circ \Phi\}$ form an analytic atlas for
    $T$ with transition functions
    $$
    (\varphi_\alpha \Phi) \circ (\varphi_\beta \circ \Phi)^{-1}
    = \varphi_\alpha \circ \varphi_\beta^{-1}.
    $$

    %%% page 31
  \item All the surfaces constructed in Topic 1 are Riemann
    surfaces.

  \item Any open subset of a Riemann surface can be made into a
    Riemann surface in a natural way:  If $T$ is an open set in the
    Riemann surface $S$, then for a chart $\varphi_\alpha : U_\alpha
    \to W_\alpha$ on $S$ let the mapping $\psi_\alpha$ be the
    restriction of $\varphi_\alpha$ to $U_\alpha \cap T$.  Then an
    analytic atlas $\{\varphi_\alpha\}$ on $S$ gives rise to an
    analytic atlas $\{\psi_\alpha\}$ on $T$.

  \item The torus. Of course, the examples mentioned in 4 include a
    Riemann surface homeomorphic to a torus.  Here is another way to
    make a torus into a Riemann surface.

    \textbf{Problem 1}: Let $\omega_1$ and $\omega_2$ be nonzero
    complex numbers whose ratio is not real.  Let $\Omega = \{n_1
    \omega_1 + n_2 \omega_2 ~:~ n_1,n_2 \in \ZZ\}$, and for any $z \in
    \CCC$ let $[z] = z + \Omega$.  Prove that there is $\delta > 0$
    such that $|n_1 \omega_1 + n_2 \omega_2 | \ge \delta$ if $n_1,
    n_2$ are integers which are not both zero.  Let $\CCC/\Omega$ be
    the set of all $[z]$ for $z \in \CCC$, noting that $[z] = [z']
    \Leftrightarrow z - z' \in \Omega$.  For any $[z]$ define a
    neighbourhood basis of $[z]$ to consist of all sets
    $$
    U_\epsilon([z]) = \{[z] ~:~ |z-w| < \epsilon\}
    $$
    for $\epsilon > 0$.  Prove that $\CCC/\Omega$ becomes a
    Hausdorff space.  For $\epsilon \le \delta/2$ let $\varphi :
    U_\epsilon([z]) \to \Delta_\epsilon(0)$ be defined by
    $\varphi([z]) = w-z$.  Prove that these form charts in an
    analytic atlas for $\CCC/\Omega$.

    The relation to a torus is that $\CCC/\Omega$ is homeomorphic to a
    torus in a natural way.  This can perhaps best be seen by
    considering the set $A = \{t_1 \omega_1 + t_2 \omega_2 ~:~ 0 \le
    t_1 < 1,\ 0 \le t_2 < 1\} \subset \CCC$, which is obviously in
    one-to-one correspondence with $\CCC/\Omega$.

    Th topology in $A$ is determined in a natural fashion: a
    neighbourhood basis of a point $t_1 \omega_1 + t_2 \omega_2$ with
    $0 < t_1 < 1$, $0 < t_2 < 1$, can be taken to be sufficiently
    small disks centred at that point.  For a point $p$ as indicated
    on the ``boundary'' of $A$, a neighbourhood basis can be taken to
    be sets 
    $$
    \{z \in A ~:~ |z-p| < \epsilon\} \cup \{z\in A ~:~
    |z-p-\omega_1| < \epsilon\}
    $$
    for all sufficiently small $\epsilon$.  And a neighbourhood basis
    of 0 can be described in a similar fashion, corresponding to the
    four smaller sectors in the figure.  Of course, this topology just
    corresponds to a gluing in the sense of Topic 1 and one easily
    sees that now $A$ is homeomorphic to $\CCC/\Omega$, the
    homeomorphism being the mapping $A \to \CCC/\Omega$ which sends
    $z$ to $[z]$.  Finally, if one imagines this gluing carried out
    with a strip of paper the shape of $A$, it becomes clear that $A$
    is homeomorphic to a torus.

    %%% page 34
  \item The sheaf of germs of meromorphic functions to be discussed at
    length in Topic 3 will be a Riemann surface in a natural way.
  \end{enumerate}
\end{exmp}



\section{Connectedness}

\begin{defn}
  A \emph{path} in a topology space $S$ is a continuous function
  $\gamma : I\to S$ where $I=[0,1]$.  The \emph{initial point} of
  $\gamma$ is $\gamma(0)$ and the \emph{terminal point} of $\gamma$ is
  $\gamma(1)$.  And $\gamma$ is said to be a \emph{path from
    $\gamma(0)$ to $\gamma(1)$}.
\end{defn}

\begin{defn}
  \label{def:8}
  A topological space $S$ is \emph{disconnected} if there are open
  sets $A, B \subset S$ such that $S = A \cup B$, $A$ and $B$ are
  disjoint, and neither $A$ nor $B$ is empty.  A topological space $S$
  is \emph{connected} if it is not disconnected.
\end{defn}

\begin{propn}
  \label{propn:1}
  A Riemann surface $S$ is connected if and only if for any points
  $p_0$ and $p_1$ in $S$ there exists a path in $S$ from $p_0$ to
  $p_1$.
\end{propn}
\begin{proof}
  Suppose $S$ is disconnected, and let $A$ and $B$ be the
  corresponding sets of Definition~\ref{def:8}. Let $p_0 \in A$ and
  $p_1 \in B$.  If there is a path $\gamma$ in $S$ from $p_0$ to $p_1$,
  then $\gamma(I)$ is connected (it is a general result that a
  continuous image of a connected space is connected).  However, the
  sets $A_1 = \gamma(I) \cap A$ and $B_1 = \gamma(I) \cap B$ show that
  in the sense of Definition~\ref{def:8} $\gamma(I)$ is disconnected.

  %%% page 35
  Conversely, suppose $S$ is connected and let $p_0, p_1 \in S$.  Let
  $A = \{p \in S ~:~$ there is a path in $S$ from $p_0$ to $p\}$.
  Then $A$ contains $p_0$ and is thus not empty.  Also, $A$ is open:
  if $p \in A$ then using an open neighbourhood $U$ of $p$ and a chart
  $\varphi : U \to \Delta$ from $U$ onto a disk $\Delta$, then $U
  \subset A$.  For if $p' \in U$ and if $\gamma$ is the part from
  $p_0$ to $p$, then a path $\gamma_1$ from $p_0$ to $p'$ is
  $$
  \gamma_1(t) =
  \begin{cases}
    \gamma(2t), & 0 \le t \le \frac{1}{2},\\
    \varphi^{-1}((2-2t)\varphi(p) + (2t-1)\varphi(p')),
    & \frac{1}{2} \le t \le 1.
  \end{cases}
  $$

  \begin{mdframed}
    \vspace{4cm}
  \end{mdframed}

  %%% page 36
  Thus, $A$ is open.  A similar proof shows that $A$ is closed: if
  $p'$ is a limit point of $A$, then we can use the same picture as
  above, except that $U$ is now picked to be a neighbourhood of $p'$
  homeomorphic to a disk $\Delta$.  Since $p'$ is a limit point of
  $A$, there is a point $p \in U \cap A$.  Then the same construction
  as above  shows that there is a path in $S$ from $p_0$ to $p'$;
  i.e., $p' \in A$.  Thus $A$ contains all its limit points and is
  therefore a closed set.  Since $A$ is open and closed and nonempty,
  and $S$ is connected, we have $A = S$.  Thus $p_1 \in A$.
\end{proof}

\begin{rem}
  Note that the above proof is entirely topological.  In general
  topology this theorem states that a connected, locally arcwise
  connected space is arcwise connected.
\end{rem}



\section{Analytic Functions}

No we turn to the important concept of analytic functions.

\begin{defn}
  \label{def:9}
  Let $S_1$ and $S_2$ be Riemann surfaces, $U$ an open subset of
  $S_1$, and $f$ a continuous function from  $U$ to $S_2$. Then $f$ is
  \emph{analytic} if for every chart $\varphi_1 : U_1 \to W_1$ on
  $S_1$ and every chart $\varphi_2 : U_2 \to W_2$ on $S_2$, the
  function $\varphi_2 \circ f \circ \varphi_1^{-1}$ is holomorphic.
  (Here and elsewhere when we use a phrase like ``every chart
  $\varphi_1$'' we mean every chart $\varphi_1$ in the complete
  analytic atlas for $S_1$.)
\end{defn}

%%% page 37
\begin{rem}
  Since the coordinate transition functions are holomorphic, to check
  the analyticity of $f$ is a neighbourhood of a point $p_0 \in U$ it
  is sufficient to check the analyticity of $\varphi_2 \circ f \circ
  \varphi_1^{-1}$ for some chart $\varphi_1$ in a neighbourhood of
  $p_0$ and some chart $\varphi_2$ in a neighbourhood of $f(p_0)$.
\end{rem}

This remark also immediately leads to 

\begin{propn}
  \label{propn:2}
  In the notation of Definition~\ref{def:9} $f$ is analytic on $U$ if
  and only if $f$ is analytic in some neighbourhood of each point of
  $U$.
\end{propn}

\begin{propn}
  If $f : S_1 \to S_2$ is analytic and $g : S_2 \to S_3$ is analytic,
  then $g \circ f : S_1 \to S_3$ is analytic.
\end{propn}

\begin{myproof}
  Let $p_0 \in S_1$.  Choose a chart $\varphi_3 : U_3 \to W_3$ in a
  neighbourhood of $g\circ f(p_0)$.  Choose a chart $\varphi_2 : U_2
  \to W_2$ in a neighbourhood of $f(p_0)$ such that $g(U_2) \subset
  U_3$.  Choose a chart $\varphi_1 : U_1 \to W_1$ in a neighbourhood
  of $p_0$ such that $f(U_1) \subset U_2$.  Then
  $$
  \varphi_3 \circ g \circ f \circ \varphi_1^{-1}
  = (\varphi_3 \circ g \circ \varphi_2^{-1}) \circ
  (\varphi_2 \circ f \circ \varphi_1^{-1})
  $$
  is a composition of holomorphic functions and is thus holomorphic.
  Thus, $g \circ f$ is analytic in a neighbourhood of $p_0$ and
  Proposition~\ref{propn:2} shows this suffices.
\end{myproof}

%%% page 38
\begin{exmp}
  \label{eg:22}
  \begin{enumerate}
  \item If $S_1$ is an open subset of the Riemann surface $\CCC$ and
    $S_2 = \CCC$, then $f : S_1 \to \CCC$ is analytic according to
    Definition~\ref{def:9} iff $f$ is analytic in the usual sense
    (satisfies the Cauchy-Riemann equation).
  \item If $S_1 = \widehat{\CCC}$ and $f$ is continuous from a
    neighbourhood of $\infty$ into $S_2$, then $f$ is analytic in a
    neighbourhood of $\infty$  iff the function $z \mapsto f(z^{-1})$ 
    is analytic in a neighbourhood of 0.  This follows because a chart
    near $\infty$ on $\widehat{\CCC}$ is the mapping $\varphi(z) =
    z^{-1}$.
  \item Likewise, if $S_2 = \widehat{\CCC}$ and $f : S_1 \to
    \widehat{\CCC}$ is continuous in a neighbourhood of $p_0$ and
    $f(p_0) = \infty$, then $f$ is analytic in a neighbourhood of
    $p_0$ iff $\frac{1}{f}$ is analytic from a neighbourhood of $p_0$
    to $\CCC$.

    %%% page 39
  \item An analytic function from a Riemann surface to $\CCC$ is said
    to be \emph{holomorphic}; an analytic function from a Riemann
    surface to $\widehat{\CCC}$ is said to be \emph{meromorphic}.
  \item Any chart in the complete analytic atlas of a Riemann surface
    is holomorphic.
  \item Consider the torus $\CCC / \Omega$.  Let $\pi : \CCC \to
    \CCC/\Omega$ be the canonical mapping $\pi(z) = [z]$.  Then $\pi$
    is analytic.  To see this consider $\varphi : U_\epsilon([z]) \to
    \Delta_\epsilon$.   Then in a neighbourhood of the fixed point $z$
    we have $\varphi \circ \pi(w) = \varphi([w]) = w-z$, a holomorphic
    function of $w$.
  \item Again for the torus $\CCC/\Omega$, we show that if $S$ is a
    Riemann surface and $f : \CCC/\Omega \to S$, then $f$ is analytic
    iff there is a function $F : \CCC \to S$ analytic such that
    $$
    F = f\circ \pi.
    $$
    First, if $f$ is analytic and $F$ is defined in this way, then $F$
    is a composition of analytic functions and is thus analytic.

    %%% page 40
    Now suppose $F$ is analytic and $F = f\circ \pi$.  We shall then
    prove that $f$ is analytic in a neighbourhood of any point $[z]
    \in \CCC/\Omega$.  Take $\varphi : U_\epsilon([z]) \to
    \Delta_\epsilon$.  Then for $p \in U_\epsilon([z])$ we can write
    $p = [z]$, where $|w-z| < \epsilon$ and $\varphi(p) = w-z$.  Thus
    $$
    f(p) = f(\pi(w)) = F(w) = F(z + \varphi(p)),
    $$
    and we have exhibited $f$ as a composition of analytic functions,
    so that $f$ is analytic near $p_0 = [z]$.

    This example really indicates the importance of the notion of
    analytic functions, since we see that there is a natural
    identification of analytic functions on $\CCC / \Omega$ with
    analytic functions $F$ on $\CCC$ which are \emph{doubly periodic},
    i.e., which satisfy
    $$
    F(z+w_1) = F(z),\quad F(z+w_2) = F(z).
    $$
    When $S = \widehat{\CCC}$ these are the \emph{elliptic} functions.

    %%% page 41
  \item For a Riemann surface constructed in the introduction (Topic
    1) there are corresponding analytic functions.   For example,
    consider the Riemann surface $S$ for $\log(z)$ and the  function
    $f$ on $S$ corresponding to $\log(z)$.  Then $f$ is holomorphic on
    $S$.  Likewise, consider the Riemann surface $\widetilde{T}$ for
    $z^{1/m}$ and the corresponding function $f$.  Then $f$ is
    meromorphic on $\widetilde{T}$.  This really follows from item 5
    above since near the branch point 0 the function $f$ is a chart
    and likewise near the branch point $\infty$, and away from the
    branch points the verification is obvious.
  \item The analytic functions from $\widehat{\CCC}$ to
    $\widehat{\CCC}$ are the rational functions.
  \item The analytic functions from $\widehat{\CCC}$ to $\CCC$ are the
    constant functions (Liouville's thereom).
  \end{enumerate}
\end{exmp}


\section{Properties of Analytic Functions}

Now we shall develop some general properties of analytic functions.
The main thing to note is the fact that local properties of analytic
functions of a complex variable usually go over to corresponding
properties in the general case in an obvious and trivial fashion.  For
example, we have

%%% page 42
\begin{propn}
  \label{propn:4}
  An analytic function $f : S_1 \to S_2$ which is not constant on
  any neighbourhood is an open mapping.
\end{propn}
\begin{myproof}
  We must show that if $p_0 \in S_1$ and $U_1$ is a neighbourhood of
  $p_0$, then $f(U_1)$ contains a neighbourhood of $f(p_0)$.  We can
  assume $\varphi_1 : U_1 \to W_1$ is a chart for $S_1$ and $\varphi_2
  : U_2 \to W_2$  a chart for $S_2$ and $f(U_1) \subset U_2$.  Then
  $\varphi_2 \circ f \circ \varphi_1^{-1}$ is a nonconstant
  holomorphic function on $W_1$ and by the known property that a
  holomorphic function of a complex variable is open if not constant
  we see that $\varphi_2 \circ f \circ \varphi_1^{-1}(w_1)$ contains a
  neighbourhood $G$ of $\varphi_2 \circ f(p_0)$.  As $\varphi_2$ is a
  homeomorphism, this implies $f(U_1)$ contains a neighbourhood
  $\varphi_2^{-1}(G)$ of $f(p_0)$.
\end{myproof}

Also, global topological properties of Riemann surfaces can be
combined with local properties of analytic functions in a decisive
manner.

\begin{propn}
  \label{propn:5}
  If $S_1$ is a connected Riemann surface and if
  $$
  f : S_1 \to S_2,\quad g : S_1 \to S_2,
  $$
  are analytic functions such that $f$ and $g$ coincide on some set
  which has a limit point in $S_1$, then $f = g$.
\end{propn}
%%% page 43
\begin{myproof}
  Let $A = \{ p \in S_1 ~:~ f\text{ and }g\text{ coincide in a
    neighbourhood of }p\}$.  Clearly, $A$ is open by its very
  definition.  Also, $A \ne \emptyset$, for if $f(p_n) = g(p_n)$ with
  $p_n \to p_0$ ($p_n \ne p_0$), then $p_0 \in A$; to see this let
  $\varphi_i : U_i \to W_i$ be charts for $S_i$, $p_0 \in U_1$,
  $f(p_0) = g(p_0) \in U_2$.  Then $\varphi_2 \circ f \circ
  \varphi_1^{-1}$ and $\varphi_2 \circ g \circ \varphi_1^{-1}$ are
  holomorphic in $W_1$ and agree on a sequence in $W_1$ tending to
  $\varphi_1(p_0) \in W_1$, and thus by the known property for
  holomorphic functions of a complex variable, $\varphi_2 \circ f
  \circ \varphi_1^{-1}$ and $\varphi_2 \circ g \circ \varphi_1^{-1}$
  coincide in a neighbourhood of $\varphi_1(p_0)$, and we see that
  $p_0 \in A$.  A similar proof shows that $A$ is closed; just use the
  previous argument with $p_0$ taking the role of a limit point of
  $A$.  As $S_1$ is connected, $A = S_1$.
\end{myproof}

\begin{propn}
  \label{propn:6}
  If $S$ is a connected Riemann surface and if $f : S \to \CCC$ is
  holomorphic, then $|f|$ has no relative maximum in $S$ unless $f$ is
  constant.
\end{propn}
%%% page 44
\begin{myproof}
  Suppose $|f|$ has a relative maximum at $p_0 ~:~ |f(p)| \le
  |f(p_0)|$ for $p$ near $p_0$.  Then the maximum principle for
  holomorphic functions of a complex variable implies $f$ is constant
  in a neighbourhood of $p_0$.  Proposition~\ref{propn:5} implies $f$
  is constant on $S$.
\end{myproof}

\begin{propn}
  \label{propn:7}
  If $f$ is a holomorphic function on a Riemann surface minus a point,
  $S - \{p_0\}$, and if $f$ is bounded in a neighbourhood of $p_0$,
  then $f$ has a unique extension to a holomorphic function on $S$.
\end{propn}
\begin{myproof}
  Apply the usual theorem on removable singularities to show that if
  $\varphi : U \to W$ is a chart in a neighbourhood of $p_0$, then
  there is a holomorphic function $g$ on $W$ such that $f \circ
  \varphi^{-1} = g$ on $W - \{\varphi(p_0)\}$.  The extension of $f$
  near $p_0$ is then $g \circ \varphi$.
\end{myproof}

\begin{propn}
  \label{propn:8}
  If $S$ is a compact connected Riemann surface, the only holomorphic
  functions on $S$ are constants.
\end{propn}
%%% page 45
\begin{myproof}
  Suppose $f : S \to \CCC$ is analytic.  Since $S$ is compact, the
  continuous function $|f|$ assumes its maximum at some point of $S$.
  Since $S$ is connected, Proposition~\ref{propn:6} implies $f$ is
  constant.
\end{myproof}

Now let us examine in some detail the local properties of meromorphic
functions.  Let $f$ be meromorphic in a neighbourhood of $p_0$ in a
Riemann surface $S$.  If $\varphi : U \to W$ is a chart in the
complete analytic atlas for $S$ and $U$ is a neighbourhood of $p_0$,
then a transition of the set $W$  in $\CCC$ allows us to assume
$\varphi(p_0) = 0$.  Thus $f\circ \varphi^{-1}$ is meromorphic in a
neighbourhood of 0 in $\CCC$.  Thus, $f\circ \varphi^{-1}$ has a
Laurent expansion
$$
f \circ \varphi^{-1}(z) = \sum_{k=N}^{\infty} a_k z^k,\quad a_N \ne 0.
$$
If $\psi : U_1 \to W_1$ is another chart in the complete analytic
atlas for $S$, $U_1$ a neighbourhood of $p_0$, $\psi(p_0) = 0$, then
$\varphi \circ \psi^{-1}$ and its inverse are holomorphic and map 0 to
0, and thus near $w = 0$
$$
\varphi \circ \psi^{-1}(w) = \sum_{k=1}^{\infty} c_k w^k,\quad c_1 \ne 0.
$$
Therefore,
%%% page 46
$$
f \circ \psi^{-1}(w) = f\circ \varphi^{-1} \circ \varphi \circ
\psi^{-1}(w) = a_N c_1^N w^N + \cdots,
$$
where the additional terms involve higher powers of $w$.  Therefore,
$$
f \circ \psi^{-1}(w) = \sum_{k=N}^{\infty} b_kw^k,\quad b_N \ne 0.
$$
Thus, the number $N$ does not depend on the particular chart used, but
depends only on the function $f$.  It is called the \emph{divisor of
  $f$ at $p_0$} and is written
$$
N = \p_f(p_0).
$$

There is another integer associated with $f$ which is perhaps more
important.  Suppose $f$ is not constant near $p_0$.  If the divisor
$N$ of $f$ at $p_0$ is negative, then the \emph{multiplicity} of $f$
at $p_0$ is said to be $-N$.  Now suppose $\p_f(p_0) \ge 0$.  Then the
\emph{multiplicity} of $f$ at $p_0$ is the divisor of $f-f(p_0)$ at
$p_0$.  Thus, we have for $m_f(p_0)$, the \emph{multiplicity of $f$ at
$p_0$}, the formula
%%% page 47
$$
m_f(p_0) =
\begin{cases}
  -\p_f(p_0) & \text{if }\p_f(p_0) < 0,\\
  \p_{f-f(p_0)}(p_0) & \text{if }\p_f(p_0) \ge 0.
\end{cases}
$$
Thus, $m_f(p_0)$ is a positive integer which is completely determined
by $f$.

In terms of $m_f(p_0)$ we can obtain a simple representation for $f$
by choosing an appropriate chart near $p_0$.  Thus, let $m = m_f(p_0)$
and consider two cases:

\underline{$\p_f(p_0) \ge 0$}.  In this case the Laurent expansion
appears in the form
$$
f \circ \varphi^{-1}(z) = f(p_0) + a_m z^m + \cdots, \quad a_m \ne 0.
$$
Let $\alpha$ be one of the $m$\textsuperscript{th} roots of $a_m$ and
note that
$$
a_m z^m + a_{m+1} z^{m+1} + \cdots = \alpha^m z^m (1 +
\sum_{k=1}^{\infty} \frac{a_k}{a_m} z^k).
$$
Let $h(z)$ be the principal $m$\textsuperscript{th} root of $1 +
\sum_{k=1}^{\infty} \frac{a_k}{a_m} z^k$ near $z=0$, so that
$$
f\circ \varphi^{-1}(z) = f(p_0) + (\alpha z h(z))^m.
$$
%%% page 48
Now define a new chart near $p_0$ by the equation
$$
\psi(p) = \alpha\varphi(p) h(\varphi(p)),\quad p \text{ near }p_0.
$$
Then $\psi$ is a chart in the complete analytic atlas for $S$ since
the mapping $z \mapsto \alpha z h(z)$ is a conformal equivalence near
0; and
$$
\begin{aligned}
  f\circ \psi^{-1}(w) 
  &= f\circ \varphi^{-1} \circ \varphi \circ \psi^{-1}(w) \\
  &= f(p_0) + (\alpha \varphi \circ \psi^{-1}(w) h(\varphi\circ
  \psi^{-1}(w)))^m\\
  &= f(p_0) + (\psi(\psi^{-1}(w))^m
  = f(p_0) + w^m.  
\end{aligned}
$$

\underline{$\p_f(p_0) < 0$}.  Now the Laurent expansion is
$$
f\circ \varphi^{-1}(z) 
= a_{-m}z^{-m} + a_{1-m}z^{1-m} + \cdots
= a_{-m}z^{-m}( 1 + \frac{a_{1-m}}{a_{-m}} z + \cdots),
\quad a_{-m} \ne 0.
$$
In this case choose $\alpha$ such that $\alpha^{-m} = a_{-m}$ and $h$
holomorphic near 0 with $h(0) = 1$, $h(z)^{-m} = 1 +
\frac{a_{1-m}}{a_{-m}} z + \cdots$.  Then
$$
f \circ \varphi^{-1}(z) = (\alpha z h(z))^{-m},
$$
so a similar argument shows that there is a chart $\psi$ at $p_0$ such
that
$$
f \circ \psi^{-1}(w) = w^{-m}.
$$

Summarising, if $m = m_f(p_0)$, then there is a chart $\psi$ in a
neighbourhood of $p_0$ such that
$$
f \circ \psi^{-1}(w)
=
\begin{cases}
  f(p_0) + w^m, & \text{if }\p_f(p_0) \ge 0,\\
  w^{-m}, & \text{if }\p_f(p_0) < 0.
\end{cases}
$$

We note that it is easy to prove that
$$
m_{g\circ f}(p_0) = m_g(f(p_0)) m_f(p_0).
$$

\begin{defn}
  \label{def:10}
  Two Riemann surfaces $S_1$ and $S_2$ are \emph{equivalent} if there
  are analytic functions $f : S_1 \to S_2$ and $g : S_2 \to S_1$ such
  that $f \circ g = id_{S_2}$ and $g \circ f = id_{S_1}$.  Thus, each
  mapping $f$ and $g$ is bijective, analytic, and has analytic
  inverse.
\end{defn}

It is routine to check that we have defined a equivalence relation.
Note that equivalence Riemann surfaces are homeomorphic.  The converse
is not valid.  We shall see that among the tori $\CCC/\Omega$, there
are infinitely many nonequivalent Riemann surfaces.  However, if a
Riemann surfaces is homeomorphic to $\widehat{\CCC}$, then it is
equivalence to $\widehat{\CCC}$ with its usual complete analytic
atlas.  This will be proved in Topic 7.

%%% page 50
Here is perhaps the simplest example of two homeomorphic nonequivalent
Riemann surfaces.  Let $S_1$ be $\CCC$ with the usual analytic atlas.
Let $\Delta = \{ z ~:~ |z| < 1\}$ and define a homeomorphism 
$$
\varphi : \CCC \to \CCC,\quad
z \mapsto \frac{z}{\sqrt{1 + |z|^2}}.
$$
Let $\Delta$ have the usual analytic atlas and define $S_2$ to be the
Riemann surface induced on $\CCC$ by the homeomorphism $\varphi$, as
in item 3 of Example~\ref{eg:22}. In other words, $S_2$ has an
analytic atlas consisting of the single chart $\varphi$.  Then $S_1$
and $S_2$ are not equivalent.  For suppose $f : S_1 \to S_2$ is
analytic.  Then by definition $\varphi \circ f$ is a holomorphic
function from $\CCC$ to $\Delta$ and is therefore constant by
Liouville's theorem.  Thus, $f$ is constant.

%%% page 51
We wish to consider a final feature of analytic functions.  Suppose
$S$ and $T$ are Riemann surfaces and that $f : S \to T$ is analytic.
If $q \in T$ we say that $f$ \emph{takes the value $q$ $n$ times} if
$f^{-1}(\{q\}) = \{p_1, \cdots, p_\ell\}$ is finite and
$\ds
\sum_{k=1}^{\ell} m_f(p_k) = n
$
(thus we are counting ``according to multiplicity'').
If $f^{-1}(\{q\}) = \emptyset$ we have $n = 0$.  If this situation
occurs, then there are charts $\varphi_k : U_k \to W_k$ in the
complete analytic atlas for $S$ with $p_k \in U_k$ and a chart
$\varphi : U \to W$ in the complete analytic atlas for $T$ such that
the collection of sets $\{U_k\}$ is disjoint and $\varphi \circ f
\circ \varphi_k^{-1}(z) = z^{m_f(p_k)}$ for $z \in W_k$.  By
diminishing the sizes of the $U_k$ (if necessary) we can also assume
that each $U_k$ is contained in a compact set in $S$.  Also, the
explicit form for $\varphi \circ f \circ \varphi_k^{-1}$ given above
shows that there exists a neighbourhood $V$ of $q$ such that the
restriction of $f$ to $U_k$ takes each value in $V$ exactly $m_f(p_k)$
times.  Therefore, the restriction of $f$ to $\bigcup_{k=1}^\ell U_k$
takes each value in $V$ exactly $n$ times.  Using this background
information, we can prove

%%% page 52
\begin{propn}
  \label{propn:9}
  Let $S$ and $T$ be Riemann surfaces and $f : S \to T$ an analytic
  function which is not constant on any neighbourhood.
  \begin{enumerate}
  \item If $S$ is compact and $T$ is connected, then $f$ takes every
    value in $T$ the same number of times.  Also, it follows that $T$
    is compact.
  \item If $f$ takes every value in $T$ the same (finite) number of
    times, then $f$ is proper.  I.e., $f^{-1}$ of any compact set is
    compact.  In particular, if also $T$ is compact, then $S$ is
    compact.
  \end{enumerate}
\end{propn}

\begin{myproof}
  The rest of the proof is just topology.  Assume the hypothesis of
  1.  Since $f(S)$ is a continuous image of a compact set, $f(S)$ is
  compact.  Since $T$ is Hausdorff, $f(S)$ is closed.  Proposition
  \ref{propn:4} implies $f(S)$ is open.  Since $T$ is connected, 
  $f(S) = T$.  Thus, $T$ is itself compact.  If $f$ takes any value 
  infinitely often, then since $S$ is compact there is a limit point 
  in $S$ of the set where $f$ takes this value, and 
  Proposition~\ref{propn:5} implies $f$ is constant in a neighbourhood 
  of this limit point. Thus,
  %%% page 53
  $f$ takes every value $q$ in $T$ a finite number $N(q)$ times.  Now
  we use the argument just preceding this proposition with no change
  in notation.  Since $f$ does not have the value $q$ on the compact
  set $S - \bigcup_{k=1}^\ell U_k$, there exists a neighbourhood $G$
  of $q$ such that the compact and thus closed set $f(S -
  \bigcup_{k=1}^\ell U_k)$ is disjoint from $G$.  Thus, the
  restriction of $f$ to $\bigcup_{k=1}^\ell U_k$ takes each value in
  $V \cap G$ exactly $N(q)$ times, and outside $\bigcup_{k=1}^\ell
  U_k$ $f$ takes no value in $V \cap G$.  Thus, $N(q') \equiv N(q)$
  for $q' \in V\cap G$.  Thus, the integer-valued function $q \mapsto
  N(q)$ is continuous on $T$.  Since $T$ is connected, $N$ is constant
  and part 1 is proved.

  \medskip
  Now we prove 2.  Let $f$ takes every value in $T$ $n$ times.  If $q
  \in T$, the analysis preceding the proposition again shows that $f$
  takes each value in $V$ exactly $n$ times in $\bigcup_{k=1}^\ell
  U_k$.  Therefore, by hypothesis $f^{-1}(V) \subset
  \bigcup_{k=1}^\ell U_k$.  Therefore, $f^{-1}(V)$ is contained in a
  compact set in $S$ since the $U_k$'s are contained in compact sets
  in $S$.  If $F \subset T$ is compact, then $F \subset
  \bigcup_{j=1}^m V_j$,where $f^{-1}(V_j)$ is contained in a compact
  set in $S$.  Thus, $f^{-1}(F)$ is contained in a compact set $S$,
  and since $f^{-1}(F)$ is closed, it is compact.
\end{myproof}

\end{document}


%%% Local Variables: 
%%% mode: latex
%%% TeX-master: t
%%% End: 
