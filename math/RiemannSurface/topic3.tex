\documentclass[a4paper,11pt]{article}
\usepackage{amsmath}
\usepackage{amsfonts}
\usepackage{verbatim}
\usepackage{url}
\usepackage{framed}
\usepackage{showkeys}
\usepackage{epsfig}
\usepackage{enumerate}
\usepackage{textcomp} %\textquotesingle
\usepackage[
%sorting=nyt,
firstinits=true, % render first and middle names as initials
useprefix=true,
maxcitenames=3,
maxbibnames=99,
style=authoryear,
dashed=false, % re-print recurring author names in bibliography
natbib=true,
url=false
]{biblatex}
%%% http://tex.stackexchange.com/questions/12254/biblatex-how-to-remove-the-parentheses-around-the-year-in-authoryear-style
\usepackage{xpatch}
\addbibresource{complex.bib} % run: biber topic1
\usepackage{color}
\usepackage{listings}
\definecolor{gray}{gray}{0.5} 
\definecolor{key}{rgb}{0,0.5,0} 
\lstset{ 
  language=[90]Fortran,
  basicstyle=\ttfamily\small, 
  keywordstyle=\color{blue}, 
  stringstyle=\color{red}, 
  showstringspaces=false, 
  emphstyle=\color{black}\bfseries, 
  emph={[2]True, False, None, self}, 
  emphstyle=[2]\color{key}, 
  emph={[3]from, import, as},
  emphstyle=[3]\color{blue}, 
  upquote=true, 
  morecomment=[s]{"""}{"""}, 
  commentstyle=\color{gray}\slshape, 
  %framexleftmargin=1mm, framextopmargin=1mm, frame=shadowbox, 
  rulesepcolor=\color{blue},
  numbers=left,
  stepnumber=1,
}
\usepackage{enumerate}
\usepackage{tikz}
\usetikzlibrary{lindenmayersystems}
\usetikzlibrary[shadings]


%%% Page Layout
\oddsidemargin=0truecm
\evensidemargin=0truecm
\textwidth=160truemm
\textheight=260truemm
\leftmargin=0truemm
\rightmargin=0truemm
\voffset=-23truemm
\topmargin=0truemm

\newif\iflecturer
%\lecturerfalse
\lecturertrue

\iflecturer
\usepackage{marginnote} % \marginpar
%\usepackage[color]{showkeys}
\definecolor{refkey}{rgb}{1,0,0}
\definecolor{labelkey}{rgb}{1,0,0}
\else
\def\marginpar#1{}
\fi

\iflecturer
\newcommand{\Answer}[1]{\dotfill\underline{\mbox{\hspace{1em}\color{blue}#1}}}
\newcommand{\BoxAns}[1]{\fbox{\color{blue}#1}}
\newcommand{\Reason}[1]{{\par\color{blue}\par{}Reason: #1}}
\else
\newcommand{\Answer}[1]{\dotfill\underline{\mbox{\hspace{1em}\color{white}#1}}}
\newcommand{\BoxAns}[1]{\fbox{\color{white}#1}}
\newcommand{\Reason}[1]{{\par\color{white}\par{}Reason: #1}}
\fi

\newif\ifamsstyle
\amsstylefalse
\newcounter{topic}
\input common
\input symbol
\makeatletter
\newcommand{\Arctan}{{\mathop{\operator@font Arctan}\nolimits}}
\makeatother

%\parindent=0pt
\parskip=1pt
\linespread{1}


\setcounter{topic}{3}

\begin{document}

\title{{\sc Rudiments of Riemann Surfaces\\
    Topic \thetopic{}: The Weierstrass Concept of a Riemann Surface}}
\author{Author: B. Frank Jones, Jr. (Rice Univ. 1971)\\
Seminar: Dr Liew How Hui (\url{liewhh@utar.edu.my})}
\date{}

\maketitle

%%% E.page 54
In this topic we shall consider the process of analytic continuation
and obtain the Weierstrass definition of analytic function.  As we
shall see, a concise language can be given to this process which
brings us to construct a Riemann surface as a replacement for
(``multiple-valued'') analytic function.

The basic idea and the basic difficulty have been indicated in Topic
1.  Given a holomorphic or even a meromorphic function defined on an
open set, we want to extend it as far as possible and somehow take
care of the multiple-valuedness that arises.

\section{Analytic Continuation Along Paths}

What we shall basically consider is analytic continuation along
paths.  First, we shall describe the classical concept and then we
shall fix everything up with lots of notation so that we end up with a
Riemann surface and so that the problems of analytic continuation go
over into statements about topological and other properties of Riemann
surfaces.  Analytic continuation is classically considered in the
following way: suppose $f$ is meromorphic in a disk $\Delta$ and has a
Laurent expansion about the centre $a$ of $\Delta$ converging in
$\Delta$:
%%% page 55
If $b \in \Delta$, it then makes sense to consider the Laurent
expansion of $f$ about $b$.  This can be done in at least two ways; we
can write
$$
(z-a)^k = (z-b + b-a)^k
= (b-a)^k (1 + \frac{z-b}{b-a})^k
= (b-a)^k \sum_{n=0}^{\infty} \binom{k}{n} \frac{(z-b)^n}{(b-a)^n}
$$
by the binomial theorem, and then we insert this into the formula for
$f$, rearrange terms (permissible if $z$ is near $b$), and thus obtain
for $b \ne a$ a Taylor series expansion for $f$ in a neighbourhood of
$b$.  The other procedure would be simply to write
$$
f(z) = \sum_{n=0}^{\infty} \frac{f^{(n)}(b)}{n!}(z-b)^n,\quad
z \text{ near } b.
$$
If it so happens that the new series has radius of convergence larger
than the distance from $b$ to the boundary of $\Delta$, we then have
what is termed a \emph{direct} or an \emph{immediate} analytic
continuation of $f$. Taking the new function in the new disk, we can
again apply this process, etc.  We can thus arrive at a sequence
$\Delta_1$, $\Delta_2$, $\cdots$ of disks an corresponding meromorphic
functions $f_1, f_2, \cdots$ such that
%%% page 56
\begin{quote}
  $f_k$ is meromorphic in $\Delta_k$, the centre of $\Delta_k$ belongs
  to $\Delta_{k-1}$, $f_k$ is a direct analytic continuation of
  $f_{k-1}$.
\end{quote}
Our first adjustment of this process will be to ignore the definite
procedure for direct analytic continuation.  Thus, instead of
considering $f_k$ to be constructed from $f_{k-1}$ by a definite
process, we shall just require $f_{k-1} \equiv f_k$ in $\Delta_{k-1}
\cap \Delta_k$.  Also, there is then no reason to require the centre
of $\Delta_k$ to be in $\Delta_{k-1}$.

\begin{defn}
  \label{def:1}
  Let $\gamma :[0,1] \to \CCC$ be a path.  An \emph{analytic
    continuation along $\gamma$} is a collection of disks $\Delta_1,
  \Delta_2, \cdots, \Delta_n$ and meromorphic functions $f_1, f_2,
  \cdots, f_n$ such that
  \begin{quote}
    $f_k$ is meromorphic in $\Delta_k$, $f_{k-1} \equiv f_k$ in
    $\Delta_{k-1} \cap \Delta_k$,
  \end{quote}
  and such that there exists $0 = t_0 < t_1 < \cdots < t_n = 1$ with
  $$
  \gamma([t_{k-1},t_k]) \subset \Delta_k,\quad k = 1,2,\cdots, n.
  $$
\end{defn}

We clearly wish to consider all possible analytic continuations along
paths, usually starting with a given meromorphic function $f_1$ in a
given disk $\Delta_1$.  This will define a meromorphic function $f_n$
in $\Delta_n$, but the value $f_n(\gamma(1))$ is not in general
independent of $\gamma$, so that we cannot in general define a
meromorphic function in $\CCC$ which extends $f_1$.
%%% page 57
For example, if $f_1$ is the principal determination of $z^{1/2}$ in
$\Delta_1(1)$:
$$
f_1(z) = (1 + (z-1))^{1/2} = \sum_{n=0}^{\infty} \binom{1/2}{n}(z-1)^n,
$$
then analytic continuations along paths from 1 to $-1$ definitely
depend on the path.  Consider the figure:

\begin{mdframed}
  \begin{minipage}{7cm}
  \vspace{3cm}    
  \end{minipage}%
  \begin{minipage}{7cm}
    analytic continuation along $\gamma_1$ yields a holomorphic
    function near $-1$ whose value at $-1$ is $i$; but analytic
    continuation along $\gamma_2$ yields the value $-i$.
  \end{minipage}
\end{mdframed}

These statements are trivial to justify since we can write $z =
re^{i\theta}$ with $-\pi < \theta < \pi$ (except on the negative real
axis) and then $z^{1/2} = r^{1/2} e^{i\theta/2}$.  Along $\gamma_1$
$\theta$ increases to $\pi$ and along $\gamma_2$ $\theta$ decreases to
$-\pi$, and thus the two different values at $-1$ result.

Therefore, if we wish to define some meromorphic function which is a
largest possible analytic continuation of $f_1$, or which is derived
from $f_1$ by analytic continuation on all paths for which
continuation is possible, we shall have to have something other than
$\CCC$ on which to define the extended function.  So we now begin to
introduce the Riemann surface on which these continuations will be
defined.

%% page 58
By the principle of the uniqueness of analytic continuation, it
suffices to know the original function in an arbitrarily small open
neighbourhood.  Such a ``germ'' of a function uniquely will determine
the function everywhere.  So we make the following definitions.

\begin{defn}
  \label{def:2}
  Let $a\in \CCC$ and suppose $f$ and $g$ are functions which are
  meromorphic in neighbourhoods of $a$.  Then $f$ is \emph{equivalent}
  to $g$, written $f \sim g$, if $f=g$ in some neighbourhood of $a$.
\end{defn}

This is an equivalence relation and we then make the following
\begin{defn}
  Let $a \in \CCC$.  Then $M_a$ is a collection of equivalence classes
  of functions meromorphic in a neighbourhood of $a$.  Any element of
  $M_a$ is called a \emph{germ of a meromorphic function}.  If $f$ is
  a meromorphic function in a neighbourhood of $a$, then $[f]_a$ is
  the germ to which $f$ belongs.  We say $[f]_a$ is the \emph{germ of
    $f$ at $a$}.
\end{defn}

By definition, $[f]_a = [g]_a$ if and only if $f \equiv g$ near $a$.

%%% page 59
\begin{defn}
  $M = \bigcup_{a\in\CCC} M_a$.  We also define the obvious mapping
  $\pi : M \to \CCC$ by $\pi([f]_a) = a$.
\end{defn}

Below we shall make $M$ into a topological space in a natural way and
then $M$ will be called the \emph{sheaf of germs of meromorphic
  functions}, and $M_a$ the \emph{stalk} over the point $a$.

We shall define a topology on $M$ by exhibiting a neighbourhood basis
for each point in $M$.  Simple considerations then show that if we
define a set in $M$ to be \emph{open} if it contains one of these
special neighbourhoods of each of its points, then the class of open
sets in $M$ forms a topology for which each point has the given
neighbourhood basis as a basis of open neighbourhoods in this topology
if the given neighbourhood bases satisfy the following conditions:
\begin{enumerate}
\item any two neighbourhoods of a point contain a third neighbourhood
  of that point;
\item any neighbourhood contains a neighbourhood of each of its
  points.
\end{enumerate}
Furthermore, the topology of $M$ is \emph{Hausdorff} if also
\begin{enumerate}
\setcounter{enumi}{2}
\item any two distinct points of $M$ are contained in disjoint
  neighbourhoods.
\end{enumerate}

\begin{mdframed}[skipabove=1ex,skipbelow=1ex]
  \centerline{\bf The Topology of $M$.}
\end{mdframed}

Suppose $[f]_a \in M$.  Then there exists a disk $\Delta$ centred at
$a$ such that $f$ is meromorphic on $\Delta$.  Define
$$
U(a,f,\Delta) = \{[f]_b ~:~ b\in \Delta\}.
$$
A neighbourhood basis of $[f]_a$ is defined to be all sets
$U(a,f,\Delta)$ such that $f$ is meromorphic on $\Delta$.  Although
the definition is quite simple, we have already incorporated into it
the notion of direct analytic continuation, for the definition states
that the germs ``close'' to $[f]_a$ are just the germs of the function
$f$ itself at points close to $a$.  Now we check the various
requirements for neighbourhood bases:
\begin{enumerate}
\item $U(a,f,\Delta_1) \cap U(a,f,\Delta_2) \supset U(a,f,\Delta_3)$
  if $\Delta_3 \subset \Delta_1 \cap \Delta_1$;
\item suppose $[f]_b \in U(a,f,\Delta)$.  Then let $\Delta'$ be a disk
  centred at $b$ such that $\Delta' \subset \Delta$, and note that
  $$
  U(b,f,\Delta') \subset U(a,f,\Delta);
  $$
  \begin{center}
    \begin{tikzpicture}[font={\small}]
      \draw (0,0) circle [radius=1.4] node [below left]{$a$}
      (-1.5,-0.2) node [below] {$\Delta$};
      \draw (0.5,0.5) circle [radius=0.45] node [above] {$b$}
      (0.5,-0.2) node {$\Delta'$};
      \filldraw (0,0) circle [radius=2pt]
      (0.5,0.5) circle [radius=2pt];
    \end{tikzpicture}
  \end{center}

%%% page 61
\item now we check that $M$ is Hausdorff.  Suppose that $[f]_a$ and
  $[g]_{a'}$ are distinct points in $M$.  If $a\ne a'$, then take
  $\Delta$ and $\Delta'$ to be disjoint disks centred at $a$ and $a'$,
  respectively, and note that $U(a,f,\Delta)$ and $U(a',g,\Delta')$
  are obviously disjoint.  If $a=a'$, then choose a disk $\Delta$ such
  that $f$ and $g$ meromorphic in $\Delta$.  Then $U(a,f,\Delta)$ and
  $U(a, g, \Delta)$ are disjoint.  For otherwise there would exist a
  point $[f]_b = [g]_b$ for some $b \in \Delta$, and thus $f \equiv g$
  in a neighbourhood of $b$.  By the uniqueness of analytic
  continuation, $f \equiv g$ in $\Delta$, contradicting $[f]_a \ne
  [g]_a$.
\end{enumerate}

Now $M$ is a topological space.  Note how much information is
contained in the statement of the validity of the Hausdorff separation
axiom --- namely, this property reflects the uniqueness of analytic
continuation.

\begin{mdframed}[skipabove=1ex,skipbelow=1ex]
\centerline{\bf $M$ is surface}
\end{mdframed}

The charts are almost obvious.  Just use the mapping $\pi$ restricted
to the various neighbourhoods of $U(a,f,\Delta)$.  Suppose we call
$\varphi$ the restriction of $\pi$ to $U(a,f,\Delta)$.  Then
$$
\varphi : U(a,f,\Delta) \to \Delta
$$
is given by $\varphi([f]_b) = b$, and $\varphi^{-1}(b) = [f]_b$.
Thus, $\varphi$ is a bijection.  Also, if $U(b,f,\Delta') \subset
U(a,f, \Delta)$, then clearly
$$
\varphi(U(b,f,\Delta')) = \Delta'.
$$
%%% page 62
Thus, $\varphi$ induces a one-to-one correspondence between a
neighbourhood basis of $[f]_b$ and a neighbourhood basis of
$\varphi([f]_b) = b$.  Thus, $\varphi$ is a homeomorphism.  This
proves that $M$ is a surface.

Moreover, we now have a nice atlas on $M$ and we claim it is an
analytic atlas and thus

\begin{mdframed}[skipabove=1ex,skipbelow=1ex]
\centerline{\bf $M$ is a Riemann surface}
\end{mdframed}

Suppose $\varphi$ is the restriction of $\pi$ to $U(a,f,\Delta)$ and
$\psi$ is the restriction of $\pi$ to $U(b,g,\Delta')$.  If $z \in
\Delta$ and $\varphi^{-1}(z) \in U(a,f,\Delta) \cap U(b,g,\Delta')$,
then $\varphi^{-1}(z) = [f]_z = [g]_z$, and thus
$\psi(\varphi^{-1}(z)) = z$.  Thus, where it is defined we have
$$
\psi \circ \varphi^{-1} = \text{identity}!
$$
The coordinate transition functions are thus trivially holomorphic and
$M$ is a Riemann surface.

The mapping $\pi : M \to \CCC$ is holomorphic.  This really needs no
checking at all, since $\pi$ restricted to any neighbourhood
$U(a,f,\Delta)$ is a chart in the analytic atlas we have constructed,
and such charts are always holomorphic.

\begin{ques}
  \label{problem:2}
  Define $V : M \to \widehat{\CCC}$ by the formula
  $$
  V([f]_a) = f(a).
  $$
  Prove that $V$ is meromorphic.
\end{ques}

%%% page 63
Thus, we have two meromorphic functions $\pi$ and $V$ on $M$ which are
quite natural and simple functions to consider.  We shall in the next
topic define an extension of $M$ which is quite a bit more
complicated, and again will be able to single out two natural
meromorphic functions, which we shall again designate $\pi$ and $V$.
In that context these functions will appear much alike, although on
$M$ the function $\pi$ seems to be somewhat simpler than $V$.

In terms of $M$ we can give a characterisation of analytic
continuation along paths (see Definition~\ref{def:1}).

\begin{propn}
  \label{propn:1}
  Let $[f]_a \in M$ be given, and let $\gamma$ be a path in $\CCC$
  starting at $a$.  A necessary and sufficient condition that there
  exists an analytic continuation along $\gamma$ with $f_1 = f$ in a
  disk $\Delta_1$ containing $a$ (using the notation of
  Definition~\ref{def:1}) is that there exists a path
  $\widetilde{\gamma}$ in $M$ such that
  $$
  \pi \circ \widetilde{\gamma} = \gamma,\quad
  \widetilde{\gamma}(0) = [f]_a.
  $$
\end{propn}
\begin{myproof}
  The necessity is quite clear.  Using the notation of
  Definition~\ref{def:1} we define
  $$
  \widetilde{\gamma}(t) = [f_k]_{\gamma(t)},\quad t_{k-1} \le t \le t_k.
  $$
  %%% pgage 64
  Since $\gamma(t_{k-1}) \in \Delta_{k-1} \cap \Delta_k$ and $f_{k-1}
  \equiv f_k$ in $\Delta_{k-1} \cap \Delta_k$, we have
  $[f_{k-1}]_{\gamma(t_{k-1})} = [f_{k}]_{\gamma(t_{k-1})}$.  Thus,
  $\widetilde{\gamma}$ is unambiguously defined, and clearly
  $\pi(\widetilde{\gamma}(t)) = \gamma(t)$, $\widetilde{\gamma}(0) =
  [f_1]_{\gamma(0)} = [f]_a$.  The continuity of $\widetilde{\gamma}$
  is immediate from the definition of the topology of $M$ and the
  continuity of $\gamma$.

  The proof of sufficiency relies on a compactness argument.  The
  continuity of $\widetilde{\gamma}$ and the definition of the
  topology of $M$ show that for each $t \in [0,1]$ there exists an
  open interval $I_t$ (open relative to $[0,1]$) containing $g$ and a
  meromorphic function $f_t$ defined on a disk $\Delta_t$ centred at
  $\gamma(t)$ such that
  $$
  \widetilde{\gamma}(I_t) \subset U(\gamma(t), f_t, \Delta_t) 
  \equiv U_t \subset M.
  $$
  The compactness result we need is that there exists $\epsilon > 0$
  such that any interval in $[0,1]$ of length not exceeding $\epsilon$
  is contained in one of the intervals $I_t$.  The proof proceeds in
  the following manner.  For each $s \in [0,1]$, there exists $r(s) >
  0$ such that
  $$
  [0,1] \cap (s-r(s), s + r(s)) \subset I_s.
  $$
  As $[0,1]$ is compact, there exist points $s_1, \cdots, s_k$ such
  that 
  $$
  [0,1] \subset \bigcup_{j=1}^k (s_j - \frac{1}{2}r(s_j), s_j +
  \frac{1}{2}r(s_j)).
  $$
  Let $\epsilon = \min\{r(s_j) ~:~ 1\le j\le k\}$.  Then if $x \in
  [0,1]$ choose $j$ such that $|x-s_j| < \frac{1}{2}r(s_j)$.  If
  $|y-x| \le \frac{1}{2}\epsilon$, then
  %%% page 65
  $$
  |y - s_j| \le |y-x| + |x-s_j| 
  < \frac{1}{2}\epsilon + \frac{1}{2}r(s_j) \le r(s_j),
  $$
  so
  $$
  y \in (s_j - r(s_j), s_j + r(s_j)) \subset I_{s_j}.
  $$
  Thus,
  $$
  [0,1] \cap [x-\frac{1}{2}\epsilon, x+\frac{1}{2}\epsilon]
  \subset I_{s_j},
  $$
  as required.

  Now choose points $t_0, \cdots, t_n$ such that $0 = t_0 < t_1 <
  \cdots < t_n = 1$, $t_k-t_{k-1} \le \epsilon$, $1\le k \le n$.  By
  choice of $\epsilon$, the interval $[t_{k-1},t_k]$ is contained in
  some set $I_{\tau_k}$ constructed above.  Thus, we are given a
  collection of disks $\Delta_{\tau_k}$ and meromorphic functions
  $f_{\tau_k}$ on $\Delta_{\tau_k}$, and we have to check that we have
  thereby obtained an analytic continuation along $\gamma$.  Since
  $\widetilde{\gamma}([t_{k-1}, t_k]) \subset
  \widetilde{\gamma}(I_{\tau_k}) \subset U_{\tau_k}$, we obtain
  $$
  \gamma([t_{k-1},t_k]) = \pi(\widetilde{\gamma}([t_{k-1},t_k]))
  \subset \pi(U_{\tau_k}) = \Delta_{\tau_k}.
  $$
  If $z \in \Delta_{\tau_{k-1}} \cap \Delta_{\tau_k}$, then the
  corresponding points in $U_{\tau_{k-1}}$ and $U_{\tau_{k}}$ are
  $[f_{\tau_{k-1}}]_z$ and $[f_{\tau_{k}}]_z$, respectively.  In
  particular for $z = \gamma(t_{k-1})$ we have
  $$
  \widetilde{\gamma}(t_{k-1}) = [f_{\tau_{k-1}}]_z 
  = [f_{\tau_{k}}]_z
  $$
  so that $f_{\tau_{k-1}} \equiv f_{\tau_{k}}$ in a neighbourhood of
  $\gamma(t_{k-1})$, and by analytic continuation $f_{\tau_{k-1}}
  \equiv f_{\tau_k}$ in $\Delta_{\tau_{k-1}} \cap \Delta_{\tau_k}$.
  Finally, it is clear that since $\widetilde{\gamma}(0) = [f]_a$, we
  have $f_{\tau_1} \equiv f$ in $\Delta_1 \cap \Delta_{\tau_0}$, so
  the analytic continuation along $\gamma$ which we have constructed
  begins with the given meromorphic function $f$ in a neighbourhood of
  $a$.
\end{myproof}

\begin{defn}
  \label{def:5}
  If $\gamma$ and $\widetilde{\gamma}$ are paths into $\CCC$ and $M$,
  respectively, such that $\gamma = \pi \circ \widetilde{\gamma}$,
  then $\widetilde{\gamma}$ is said to be a \emph{lifting} of
  $\gamma$. 
\end{defn}

\begin{propn}[Unique Lifting Theorem (and ``Analytic Continuation'')]
  \label{propn:2}
  If $\widetilde{\gamma}_1$ and $\widetilde{\gamma}_2$ are paths in
  $M$ such that $\pi \circ \widetilde{\gamma}_1 = \pi \circ
  \widetilde{\gamma}_2$, then either
  $$
  \widetilde{\gamma}_1(t) = \widetilde{\gamma}_2(t)
  \text{ for every } t \in [0,1],
  $$
  or
  $$
  \widetilde{\gamma}_1(t) = \widetilde{\gamma}_2(t)
  \text{ for no } t \in [0,1].
  $$
\end{propn}

\begin{myproof}
  Let $A = \{t\in [0,1] ~:~ \widetilde{\gamma}_1(t) =
  \widetilde{\gamma}_2(t) \}$.  By continuity of
  $\widetilde{\gamma}_1$ and $\widetilde{\gamma}_2$, $A$ is a closed
  set.  It is also an open set, for consider any $t_0 \in A$.  Let $a
  = \pi \circ \widetilde{\gamma}_i(t_0)$ and
  $\widetilde{\gamma}_i(t_0) = [f]_a$.  Then $f$ is meromorphic on a
  disk $\Delta$ centred at $a$ and a neighbourhood $U(a,f,\Delta)$ of
  $[f]_a$ is defined.  By continuity of $\widetilde{\gamma}_i$,
  $\widetilde{\gamma}_i(t)$ is contained in $U(a,f,\Delta)$ for $t$
  sufficiently near $t_0$ and thus for those values of $t$
  $$
  \widetilde{\gamma}_i(t) = [f]_{\pi \circ \widetilde{\gamma}_i(t)}
  $$
  %%% page 67
  and since $\pi \circ \widetilde{\gamma}_1 = \pi \circ
  \widetilde{\gamma}_2$ we obtain $\widetilde{\gamma}_1(t) =
  \widetilde{\gamma}_2(t)$, or, $t \in A$.

  %%% page 67
  Since $A$ is both open and closed and $[0,1]$ is connected, we have
  either $A = [0,1]$ or $A$ is empty.
\end{myproof}

Thus, although in the sense of Definition~\ref{def:1} analytic
continuation is not a uniquely defined construction (since difference 
choices could be made for the $t_k$'s and the disks $\Delta_k$), yet 
viewed as a lifting problem we do have a strong uniqueness statement.  
Moreover, we see now that the natural choice we have used in the proof 
of necessity of Proposition~\ref{propn:1} was really forced upon us.
There was no other way to choose $\widetilde{\gamma}(t)$.

This discussion definitely does not imply that the unique continuation
property along paths leads to a germ at the end point of the path
which is uniquely determined by the end point.  The discussion in
between Definitions~\ref{def:1} and \ref{def:2} makes this clear.  
In terms of the example of the square root mentioned there, observe 
that if $f_1$ is the
principal determination of $z^{1/2}$ near 1 and if $\gamma$ is the
path $\gamma(t) = e^{2\pi it}$, $0 \le t \le 1$, then $\gamma$ can be
lifted to a path $\widetilde{\gamma}$ such that $\widetilde{\gamma}(0)
= [f_1]_1$, but $\widetilde{\gamma}(1) = [-f_1]_1$.  Thus,
$\widetilde{\gamma}(0) \ne \widetilde{\gamma}(1)$, although
$\gamma(0) = \gamma(1)$.  Another way of stating this is that 
$[f_1]_1$ and $[-f_1]_1$ are ``far apart'' in the
topology of $M$, and yet both lie in $M_1$ and can be connected by a
path in $M$.


\section{Monodromy Theorem}
%%% page 68
Now we begin to prove the famous ``monodromy theorem'', which
essentially states that the phenomenon just discussed cannot occur on
simply connected regions.  A consequence will be the fact that on any
simply connected region in $\CCC$ which does not contain the origin
one can define a (single-valued) analytic determination of $z^{1/m}$,
$\log(z)$, etc.   First we introduce the notation $I = [0,1]$, $I^2 =
[0,1] \times [0,1]$.

\begin{lem}
  \label{lem:1}
  Let $\Gamma : I^2 \to \CCC$ be continuous, and let
  $\widetilde{\Gamma} : I^2 \to M$ satisfy $\pi \circ
  \widetilde{\Gamma} = \Gamma$.  Assume that for each fixed $u \in I$,
  $\widetilde{\Gamma}(t,u)$ is a continuous function of $t$; and also
  that $\widetilde{\Gamma}(0,u)$ is a continuous function of $u$.
  Then $\widetilde{\Gamma}$ is continuous.
\end{lem}

\begin{myproof}
  This follows in a purely topological manner from the unique lifting
  theorem (Proposition~\ref{propn:2}) and the description of lifting
  in terms of analytic continuous given in Proposition~\ref{propn:1}.
  Let $\beta \in I$.  We shall then prove that there exists $\epsilon
  > 0$ such that $\widetilde{\Gamma}$ is continuous on $I \times (I
  \cap (\beta-\epsilon, \beta + \epsilon))$, and the lemma will then
  be proved.  For each fixed $u$ define for $t\in I$
  $$
  \gamma_u(t) = \Gamma(t,u),\quad \widetilde{\gamma}_u(t) =
  \widetilde{\Gamma}(t,u).
  $$
  Then $\gamma_u$ and $\widetilde{\gamma}_u$ are continuous and $\pi
  \circ \widetilde{\gamma}_u = \gamma_u$.  Now we apply
  Proposition~\ref{propn:1}, which guarantees the existence of a
  collection of disks $\Delta_1, \cdots, \Delta_n$, meromorphic $f_k$
  defined in $\Delta_k$, $1\le k\le n$, and points $t_k$ such that $0
  = t_0 < t_1 < \cdots < t_n = 1$ and
  $$
  \gamma_\beta([t_{k-1},t_k]) \subset \Delta_k,\quad
  f_{k-1} = f_k \text{ on }\Delta_{k-1}\cap \Delta_k,
  $$
  and
  $$
  \widetilde{\gamma}_\beta(t) = [f_k]_{\widetilde{\gamma}_\beta(t)},
  \quad t_{k-1} \le t \le t_k,
  $$
  the latter choice being forced as follows from the proof of
  Proposition~\ref{propn:1} and the unique lifting theorem.  Note that
  we have \textbf{fixed} $\beta$ and applied Proposition~\ref{propn:1}
  to the paths $\gamma_\beta$ and $\widetilde{\gamma}_\beta$.

  Since $\Gamma$ is \textbf{uniformly} continuous, there exists
  $\epsilon_1 > 0$ such that
  $$
  \gamma_u([t_{k-1},t_k]) \subset \Delta_k \text{ if }
  |u - \beta| < \epsilon_1, \quad u \in I
  $$
  (recall that $\Delta_k$ is an \textbf{open} disk).  Also, since
  $\widetilde{\Gamma}(0,u)$ is a continuous function of $u$, there
  exists $\epsilon_2 > 0$ such that 
  $$
  \widetilde{\Gamma}(0,u) \in U(a,f_1,\Delta_1)
  \text{ if } |u-\beta| < \epsilon_2,\quad u \in I
  $$
  ($a$ = centre of $\Delta_1$).  Thus, if $\epsilon =
  \min\{\epsilon_1, \epsilon_2\}$,
  $$
  \widetilde{\gamma}_u(0) = [f_1]_{\widetilde{\gamma}_u(0)}
  \text{ if }|u-\beta| < \epsilon,\ u\in I.
  $$
  %%% page 70
  The unique lifting theorem now implies that
  $$
  \widetilde{\gamma}_u(t) = [f_k]_{\gamma_u(t)},\quad
  t_{k-1} \le t \le t_k,\ |u-\beta| < \epsilon,\ u\in I.
  $$
  That is,
  $$
  \widetilde{\Gamma}(t,u) = [f_k]_{\Gamma(t,u)},\quad
  t_{k-1} \le t \le t_k,\ |u-\beta| < \epsilon,\ u\in I.
  $$
  But the continuity of $\Gamma(t,u)$ in the indicated range implies
  that of $\widetilde{\Gamma}(t,u)$ in the same range.
\end{myproof}

\begin{defn}
  \label{def:6}
  If $T$ is a topological space and $\gamma_0 : I \to T$ and $\gamma_1
  : I \to T$ are paths in $T$ having the same end points, then
  $\gamma_0$ and $\gamma_1$ are \emph{homotopic with fixed end points}
  if there exists a continuous
  $$
  \Gamma : I^2 \to T
  $$
  such that
  $$
  \Gamma(t,0) = \gamma_0(t),\ 
  \Gamma(t,1) = \gamma_1(t),\ 
  \Gamma(0,u) = \gamma_0(0) = \gamma_1(0),\ 
  \Gamma(1,u) = \gamma_0(1) = \gamma_1(1).
  $$
  The function $\Gamma$ is called a \emph{homotopy} between $\gamma_0$
  and $\gamma_1$.  If there is possibility of confusion we will say
  $\gamma_0$ and $\gamma_1$ are \emph{$T$-homotopic} with fixed end
  points.
\end{defn}

%%% page 71
\begin{defn}
  \label{def:7}
  A connected topological space $T$ is \emph{simply connected} if each
  pair of paths $\gamma_0$ and $\gamma_1$ in $T$ having the same end
  points are homotopic with fixed end points.
\end{defn}

Now we can state various trivial consequences of Lemma~\ref{lem:1}.

\begin{thm}[Covering Homotopy Theorem]
  Let $\Gamma : I^2 \to \CCC$ be a homotopy in $\CCC$ and let $p \in
  M$ such that $\pi(p) = \Gamma(0,u)$ ($0 \le u \le 1$), and suppose
  for each $u \in I$ the path $t \mapsto \Gamma(t,u)$ can be lifted to
  a path in $M$ starting at $p$, say $\widetilde{\Gamma}(t,u)$ so that
  $$
  \pi \circ \widetilde{\Gamma} = \Gamma.
  $$
  Then $\widetilde{\Gamma}$ is homotopy in $M$.
\end{thm}
\begin{myproof}
  For each $u \in I$ the function $t \mapsto \widetilde{\Gamma}(t,u)$
  is continuous, by hypothesis.  And $\widetilde{\Gamma}(0,u) = p$ is
  constant and thus a continuous function of $u$.  Therefore,
  Lemma~\ref{lem:1} implies $\widetilde{\Gamma}$ is continuous.
  Finally, consider the two paths
  $$
  \widetilde{\gamma}(u) = \widetilde{\Gamma}(1,u),\quad
  \widetilde{\gamma}'(u) = \widetilde{\Gamma}(1,0).
  $$
  We then have
  $$
  \widetilde{\gamma}(0) = \widetilde{\gamma}'(0)
  $$
  and
  $$
  \pi \circ \widetilde{\gamma}(u) 
  = \Gamma(1,u)
  = \Gamma(1,0)
  = \pi \circ \widetilde{\gamma}'(u)
  $$
  Thus Proposition~\ref{propn:2} implies $\widetilde{\gamma} =
  \widetilde{\gamma}'$.  That is, $\widetilde{\Gamma}(1,u) =
  \widetilde{\Gamma}(1,0)$, and thus $\widetilde{\Gamma}$ is a
  homotopy.
\end{myproof}

%%% page 72
\begin{thm}[Monodromy Theorem]
  Let $D$ be a simply connected region in $\CCC$, $a \in D$, and $f$ a
  meromorphic function in a neighbourhood of $a$.  Assume that $f$ has
  an analytic continuation along every path in $D$ which starts at
  $a$.  Then there exists a meromorphic function $F$ on $D$ such that
  $f \equiv F$ in a neighbourhood of $a$.
\end{thm}

\begin{myproof}
  The hypothesis means that for every path $\gamma$ in $D$ such that
  $\gamma(0) = a$, there exists a path $\widetilde{\gamma}$ in $M$
  such that $\pi \circ \widetilde{\gamma} = \gamma$ and
  $\widetilde{\gamma}(0) = [f]_a$.  If $\gamma_0$ and $\gamma_1$ are
  paths in $D$ from $a$ to $z$, then $\gamma_0$ and $\gamma_1$ are
  $D$-homotopic with fixed end points, and by the covering homotopy
  theorem the paths $\widetilde{\gamma}_0$ and
  $\widetilde{\gamma}_1$.  Thus, we can define unambiguously
  $$
  F(z) = V(\widetilde{\gamma}(1)),
  $$
  where $\widetilde{\gamma}$ is a path in $M$ such that
  $\widetilde{\gamma}(0) = [f]_a$ and $\pi \circ \widetilde{\gamma}$
  is a path in $D$ from $a$ to $z$.  Now we must check the properties
  of $F$.
  %%% page 73
  First, suppose $\widetilde{\gamma}(1) = [g]_z$, where $g$ is
  holomorphic in a disk $\Delta$ centred at $z$.  For $w \in \Delta$
  we use the path which goes from $a$ to $z$ along $\gamma$ and then
  from $z$ to $w$ along a line segment.  The lifting from $a$ to $z$
  is $\widetilde{\gamma}$ and the lifting from $z$ to $w$ is just the
  germ of $g$ at pints on the segment from $z$ to $w$.  Since $F$ is
  unambiguously defined, $F(w0 = V([g]_w) = g(w)$.  Thus, $F$ is
  meromorphic in $\Delta$ and this proves $F$ is meromorphic in $D$.
  In particular, if $z = a$ we can take $g = f$ and we obtain $F(w) =
  f(w)$ for $w$ near $a$.
\end{myproof}

One application of the monodromy theorem has already been mentioned.
Namely, on a simply connected region $D \subset \CCC - \{0\}$, there
exists a holomorphic determination of $\log(z)$.  The only hypothesis
which needs to be checked is that $\log(z)$ can be analytically
continued along  all paths in $D$.  This can be verified in a simple
manner, but we omit the proof now since a slightly different version
of the same result will be given in the discussion of algebraic
functions in Topic 5.

We next want to give and example pertinent to the monodromy theorem,
but we shall first give a rather simple but important theorem on
analytic continuation, the so-called ``permanence of functional
relations''.  This is a generalisation of a familiar result on
single-valued functions, an example of which is the fact that the
identity $\sin 2z = 2\sin z \cos z$ follows from its validity for
\textbf{real} $z$ and the analyticity of all the functions involved.
%%% page 74
The theorem we shall give is really a generalisation of usual theorems
on unique analytic continuation because we are not here dealing with
single-valued functions.  Also, a more general theorem could be
stated.

\begin{thm}[Permanence of Functional Relations]
  Let $A(z,w)$ be a holomorphic function for $z$ in a region $D
  \subset \CCC$ and all $w \in \CCC$.  Let $\widetilde{\gamma}$ be a
  path in $M$ such that $\gamma = \pi \circ \widetilde{\gamma}$ is a
  path in $D$ and each $t \in I$ yields $\widetilde{\gamma}(t) =
  [f_t]_{\gamma(t)}$, where $f_t$ is a holomorphic function in a
  neighbourhood of $\gamma(t)$.  If $A(z,f_0(z)) \equiv 0$ in a
  neighbourhood of $\gamma(0)$, then for each $t \in I$, $A(z,f_t(z))
  \equiv 0$ in a neighbourhood of $\gamma(t)$.
\end{thm}

\begin{rem}
  We have not given a definition for a function $A$ to be holomorphic
  in two complex variables.  One definition states that for any $(z_0,
  w_0)$ in the domain of definition of $A$, $A$ has a power series
  expansion
  $$
  A(z,w) = \sum_{j,k=0}^{\infty} a_{jk}(z-z_0)^j (w-w_0)^k
  $$
  converging absolutely for $z$ near $z_0$ and $w$ near $w_0$.  The
  important property we need is that if $f$ is a holomorphic function
  of one variable near $z_0$, then $A(z,f(z))$ is holomorphic for $z$
  near $z_0$.  For most important case we shall consider this is quite
  obvious; namely, the case in which the function $A(z,w)$ is a
  polynomial in which $w$ with coefficients holomorphic functions of
  $z$:
  $$
  A(z,w) = a_0(z) w^n + \cdots + a_{n-1}w + a_n(z).
  $$
\end{rem}

%%% page 75
\begin{myproof}
  Since $f_t$ is holomorphic near $\gamma(t)$, the function
  $A(z,f_t(z))$ is holomorphic near $\gamma(t)$ and thus defines a
  germ at $\gamma(t)$ which we denote
  $$
  \widetilde{\gamma}_1(t) = [A(z,f_t(z))]_{\gamma(t)}.
  $$
  Since $\widetilde{\gamma}(t) = [f_t]_{\gamma(t)}$, it follows that
  $\widetilde{\gamma}_1$ is a path in $M$ (i.e., $\widetilde{\gamma}_1$
  is continuation) and obviously $\pi \circ \widetilde{\gamma}_1 =
  \gamma$.  Also define
  $$
  \widetilde{\gamma}_2(t) = [0]_{\gamma(t)}.
  $$
  Then $\widetilde{\gamma}_2$ is a path in $M$ with $\pi \circ
  \widetilde{\gamma}_2 = \gamma$.  By hypothesis,
  $\widetilde{\gamma}_1(0) = \widetilde{\gamma}_2(0)$.  Thus,
  Proposition~\ref{propn:2} implies $\widetilde{\gamma}_1 =
  \widetilde{\gamma}_2$ and this implies the result.
\end{myproof}

Before giving the example, let us make one important observation about
analytic continuation.  This is the fact that if two germs at a point
$a$ are different, then they remain different under analytic
continuation along any fixed path.  This is another consequence of
Proposition~\ref{propn:2}, which in this case would read that if
$\widetilde{\gamma}_1(1) = \widetilde{\gamma}_2(1)$, then
$\widetilde{\gamma}_1(0) = \widetilde{\gamma}_2(0)$.  Also, if $[f]_a$
is a germ at $a$ and if $f$ can be continued analytically along every
path in a region $D$ and if the continuation of $f$ depends only on
the terminal point of the path and not on the path itself, then there
is a meromorphic $F$ defined in $D$ such that $F = f$ near $a$.  The
proof of this is exactly like the proof of the monodromy theorem was
once we knew that analytically continuation did not depend on the
path.

%%% page 76
The monodromy theorem has of course two critical hypothesis.  We have
already indicated the reason for assuming $D$ is simply connected, and
now we shall examine the other main hypothesis, that $f$ has an
analytic continuation along \textbf{every} path in $D$.  Note
especially that the hypothesis does not state that $f$ can be
continued analytically to each point of $D$ along \textbf{some} path
in $D$.  We shall now give an example to refute such a possibility for
a weakening of the hypothesis of the theorem.

This example will be the Riemann surface for the ``inverse'' of the
function $G(w) = w^3 - 3w$, and the analytic continuation process will
reduce to finding paths on the surface.  As $G'(w) = 3w^2 - 3$, the
inverse function theorem of complex analysis will apply if $w \ne \pm
1$.  Since $G(1) = -2$ and $G(-1) = 2$, we conclude that if $G(w_0) =
z_0 \ne \pm 2$, then there exists a unique holomorphic function $f$ in
a neighbourhood of $z_0$ such that $G(f(z)) = z$ near $z_0$ and
%%% page 77
$f(z_0) = w_0$.  But for each $z_0 \ne \pm 2$ there are three distinct
corresponding values of $w_0$ and thus three distinct solutions $f$ of
$G(f(z)) = z$ defined near $z_0$.  We shall make this multiple-valued
correspondence $z \mapsto w$ into a single-valued function on an
appropriate Riemann surface by the technique of the introduction, even
though we no longer possess and explicit formula for $w$ in terms of
$z$.  Thus, we select three copies of the $z$-plane cut along the real
axis from 2 to $\infty$ and from $-2$ to $\infty$:

\begin{mdframed}
  \begin{minipage}{7cm}
  \begin{tikzpicture}
    \draw 
    (-3,0) node [below] {$\infty$} -- (-1,0) node [below] {$-2$}
    (1,0) node [below] {$2$} -- (3,0) node [below] {$\infty$}
    ;
  \end{tikzpicture}    
  \end{minipage}%
  \begin{minipage}{8cm}
    Each of these slit planes is simply connected, so the monodromy
    theorem applies to show that in each plane we can define a global
    solution $f$ to the equation $G(f(z)) = z$ and $f$ is holomorphic
    in the slit plane.
  \end{minipage}
\end{mdframed}

In order to accomplish the corresponding gluing we must see what
happens to these functions at the slits.  So we wish to examine
carefully the values of $w$ corresponding to real $z$ such that $2 <
|z| < \infty$.  To do this we introduce coordinates $z = x+iy$, $w =
u+iv$ and compute from
$$
(u+iv)^3 - 3(u+iv) = x+iy.
$$
We find
%%% page 78
$$
u^3 - 3uv^2 - 3u = x,\quad
3u^2v-v^3-3v = y.
$$
Along the slits we have $y=0$, or $3v(u^2 - \frac{v^2}{3} - 1) = 0$.
Thus, $v=0$ or $u^2 - \frac{v^2}{3} = 1$.  This locus in the $w$-plane
looks like the real axis and a hyperbola:

\begin{mdframed}
  \vspace{3cm}
  %page 78
\end{mdframed}

For $v = 0$ we have $x = u^3 - 3u$.  Thus, $x>2 \Leftrightarrow u > 2$
and $x < -2 \Leftrightarrow u < -2$, as one easily sees by considering
the graph of $u^3-3u$.  For $u^2 - \frac{v^2}{3} = 1$ we have $x = u^3
- 3u(3u^2 - 3) - 3u = -8u^3 + 6u$.  Again, it is easily seen that $x >
2 \Leftrightarrow u < -1$ and $x < -2 \Leftrightarrow u > 1$.

Now we distinguish three regions in the $w$-plane:
$$
\begin{aligned}
  A &= \{(u,v) ~:~ u^2 - \frac{v^2}{3} > 1,\ u>0\}
  - \{(u,0) ~:~ 2 \le u < \infty\},\\
  B &= \{(u,v) ~:~ u^2 - \frac{v^2}{3} < 1\},\\
  C &= \{(u,v) ~:~ u^2 - \frac{v^2}{3} > 1,\ u<0\}
  - \{(u,0) ~:~ -\infty < u \le 2\}.
\end{aligned}
$$
%%% page 79
Then one easily sees that the function $G$ maps $A$, $B$, and $C$ each
onto a copy of the $z$-plane, cut as described.  Suppose we use three
copies of the $z$-plane, labelled $\CCC_A$, $\CCC_B$, and $\CCC_C$.
In order to see how these should be glued along the cuts, we just need
to check the sign of $y$ near the boundaries of $A$, $B$, and $C$ in
the $w$-plane.  This is indicated in the figure.

\begin{mdframed}
  \vspace{3cm}
  %page 79
\end{mdframed}

Now we can easily indicate the method of gluing the planes $\CCC_A$,
$\CCC_B$, $\CCC_C$:

\begin{mdframed}
  \vspace{3cm}
  %page 79
\end{mdframed}

%%% page 80
Note in particular that the cuts from 2 to $\infty$ in $\CCC_A$ and
from $-2$ to $\infty$ in $\CCC_C$ can now be erased.  \emph{This is
  the basic reason this example has been introduced.} ``Over''
the point $z = 2$ lie \textbf{two} points of our Riemann surface, one
a branch point, the other not.  Likewise for $z = -2$.

Now we have a function $f$ defined on this Riemann surface which
represents all the solutions of $G(w) = z$ for any $z$.  Now suppose
we start at $z = 0$ with the solution $f_0$ of the equation $G(f_0(z))
= z$ near $z = 0$, $f_0(0) = 0$, $f_0$ holomorphic.  Given any complex
number $a$, there is \emph{some} path $\gamma$ from 0 to $a$ along
which $f_0$ has an analytic continuation.  If $a \ne \pm 2$, one can
indeed go along any path from 0 to $a$ which does not pass through
$\pm 2$.  If $a = 2$, use the path:

\begin{mdframed}
  \begin{minipage}{7cm}
    %page 80
  \end{minipage}%
  \begin{minipage}{8cm}
    Here is the reason.  The starting point corresponds to $z= 0$, $w
    =0$ and thus to the origin in $\CCC_B$.  In order to get to the
    point 2 in $\CCC_A$ (where this is not a branch point), we pass
    through the cut joining $\CCC_B$ to $\CCC_A$. 
  \end{minipage}
\end{mdframed}

Likewise, if $a = -2$, use the path

\begin{mdframed}
  \vspace{3cm}
  %page 80
\end{mdframed}

%%% page 81
But the conclusion of the monodromy theorem fails.  Otherwise, by the
permanence of functional relations there would exist a function $F$
holomorphic in all of $\CCC$ such that $G(F(z)) \equiv z$, $z \in
\CCC$. It is rather clear that this cannot happen since by its very
nature the relation $z \mapsto w$ must be multiple valued.  A direct
proof would be this.  Since $F^3 - 3F \equiv z$, we have $F(z) \to
\infty$ as $z \to \infty$.  Thus $F$ has a pole at $\infty$ and so the
Laurent expansion of $F$ at $\infty$ shows that $F(z) = \alpha z^n +
\cdots$ (smaller powers of $z$), where $\alpha \ne 0$ and $n$ is a
positive integer.  But then $F^3 - 3F = \alpha^3 z^{3n} + \cdots$ and
there is no way this can behave like $z$ near $\infty$.

We shall return this example in Topic 5, where algebraic functions
in general are treated.  But it should even be noted here that the
branch point 2 lying in $\CCC_B$ and $\CCC_C$ and the branch point
$-2$ lying in $\CCC_A$ and $\CCC_B$ can be added to the surface in the
way described in Topic 1, and likewise $\infty$ in $\CCC_A$,
$\CCC_B$,  and $\CCC_C$ can be added, all three sheets being joined
there.  The resulting surface is a Riemann surface and the function
$f$ on it corresponding to the mapping $z \mapsto w$ is meromorphic.
Also, $f$ is easily seen to be one-to-one since the inverse mapping $w
\mapsto z$ is single-valued.  Therefore, since $f$ is also onto, $f$
is an analytic equivalence with $\widehat{\CCC}$, so this Riemann
surface is equivalent to $\widehat{\CCC}$.

%%% page 82
\begin{thm}[Poincar\'e and Volterra]
  Let $S$ be a connected open subset of $M$.  Then for any $a \in
  \CCC$ the set
  $$
  \{[f]_a ~:~ [f]_a \in S\} = S \cap \pi^{-1}(a)
  $$
  is countable or finite.  
\end{thm}

\begin{myproof}
  Since $S$ is connected we can consider some fixed $[g]_b \in S$ and
  the note that each element of $S\cap \pi^{-1}(a)$ can be connected
  to $[g]_b$ by a path $\widetilde{\gamma}$ in $S$, by
  Proposition~2.11 of Topic 2.  So, if $\gamma = \pi \circ
  \widetilde{\gamma}$, then analytic continuation of $g$ along
  $\gamma$ results in $f$, if $[f]_a$ is the point we are
  considering.  By Proposition~\ref{propn:1} it follows that if
  $\gamma'$ is another path such that for a sufficiently small
  $\epsilon > 0$
  $$
  | \gamma'(t) - \gamma(t) | < \epsilon, \quad 0 \le t \le 1,
  $$
  then $[g]_{\gamma'(0)}$ can be analytically continued along
  $\gamma'$ and the resulting germ is $[f]_{\gamma'(1)}$.  We have
  again appealed to the unique lifting theorem and the argument used
  in the proof of Lemma~\ref{lem:1}.  Now there is such a $\gamma'$
  with initial point $b$ and terminal point $a$, such that $\gamma'$
  is a polygon with vertices (except for $a$ and $b$) at rational
  complex numbers (i.e., complex numbers whose real and imaginary
  parts are both rational).  Thus, $S \cap \pi^{-1}(a)$ consists of
  germs $[f]_a$ which come from analytic continuation from $[g]_b$
  along paths which are polygons with rational vertices.  There are
  only countable many such paths so the theorem is proved.
\end{myproof}

%%% page 83
Of course, the example which is immediately suggested by this theorem
is the Riemann surface for $\log(z)$ which has countably many sheets.
In the language of germs, we have over a point $a \ne 0, \infty$, the
germs $[\log(z) + 2n\pi i]_a$, where $\log(z)$ represents an arbitrary
determination of the logarithm near $a$, and $n$ is any integer.

\begin{defn}
  \label{def:8}
  Let $f$ be a meromorphic function in a neighbourhood of a point $a
  \in \CCC$.  The \emph{Riemann surface (in $M$) of $f$} is the
  component of $[f]_a$ in $M$.
\end{defn}

Here we have used a topological word ``component'', which by
definition is a maximal connected set --- a connected set contained in
no strictly larger connected set.  Since $M$ is a surface, in this
case the component containing $[f]_a$ (the component of $[f]_a$ (is
the collection of germs which can be joined to $[f]_a$ by a path (in
$M$).

For example, the Riemann surface of any determination of $z^{1/m}$
near a point $a \ne 0$ consists of all germs $[f]_b$ such that $f(z)^m
\equiv z$ near $b$, $b\ne 0$.  By the permanence of functional
relations all the terms in this Riemann surface must satisfy this
identity, and we thus need only verify that any germ satisfying the
identity can be joined to any other such germ.  This can of course be
easily checked directly, but an argument will be given in Topic 5
for a general theorem along these lines.

%%% page 84
There is an obvious deficiency in the Riemann surface for $z^{1/m}$.
Namely, the branch points 0 and $\infty$ are missing.  This situation
is true in general for $M$ --- it has been constructed without branch
points (a phrase which we haven't even yet defined), and also it does
not contain germs of functions meromorphic at $\infty$.  The latter is
not a serious omission and indeed we could have considered from the
start germs of meromorphic functions on any fixed Riemann surface.
But in the next topic we shall construct a Riemann surface which
contains $M$ in a very precise sense and has all the branch points and
also the germs at $\infty$.  Then we shall give a satisfactory
definition of the Riemann surface of a meromorphic function, replacing
Definition~\ref{def:8}. 

\end{document}


%%% Local Variables: 
%%% mode: latex
%%% TeX-master: t
%%% End: 
