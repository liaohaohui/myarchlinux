\documentclass[a4paper,11pt]{article}
\usepackage{amsmath}
\usepackage{amsfonts}
\usepackage{verbatim}
\usepackage{url}
\usepackage{framed}
\usepackage{epsfig}
\usepackage{xypic}
\usepackage{enumerate}
\usepackage{textcomp} %\textquotesingle
\usepackage[
%sorting=nyt,
firstinits=true, % render first and middle names as initials
useprefix=true,
maxcitenames=3,
maxbibnames=99,
style=authoryear,
dashed=false, % re-print recurring author names in bibliography
natbib=true,
url=false
]{biblatex}
%%% http://tex.stackexchange.com/questions/12254/biblatex-how-to-remove-the-parentheses-around-the-year-in-authoryear-style
\usepackage{xpatch}
\addbibresource{complex.bib} % run: biber topic1
\usepackage{color}
\usepackage{listings}
\definecolor{gray}{gray}{0.5} 
\definecolor{key}{rgb}{0,0.5,0} 
\lstset{ 
  language=[90]Fortran,
  basicstyle=\ttfamily\small, 
  keywordstyle=\color{blue}, 
  stringstyle=\color{red}, 
  showstringspaces=false, 
  emphstyle=\color{black}\bfseries, 
  emph={[2]True, False, None, self}, 
  emphstyle=[2]\color{key}, 
  emph={[3]from, import, as},
  emphstyle=[3]\color{blue}, 
  upquote=true, 
  morecomment=[s]{"""}{"""}, 
  commentstyle=\color{gray}\slshape, 
  %framexleftmargin=1mm, framextopmargin=1mm, frame=shadowbox, 
  rulesepcolor=\color{blue},
  numbers=left,
  stepnumber=1,
}
\usepackage{enumerate}
\usepackage{tikz}
\usepackage{todonotes}


%%% Page Layout
\oddsidemargin=0truecm
\evensidemargin=0truecm
\textwidth=160truemm
\textheight=260truemm
\leftmargin=0truemm
\rightmargin=0truemm
\voffset=-23truemm
\topmargin=0truemm

\newif\iflecturer
%\lecturerfalse
\lecturertrue

\iflecturer
\usepackage{marginnote} % \marginpar
%\usepackage[color]{showkeys}
\definecolor{refkey}{rgb}{1,0,0}
\definecolor{labelkey}{rgb}{1,0,0}
\else
\def\marginpar#1{}
\fi

\iflecturer
\newcommand{\Answer}[1]{\dotfill\underline{\mbox{\hspace{1em}\color{blue}#1}}}
\newcommand{\BoxAns}[1]{\fbox{\color{blue}#1}}
\newcommand{\Reason}[1]{{\par\color{blue}\par{}Reason: #1}}
\else
\newcommand{\Answer}[1]{\dotfill\underline{\mbox{\hspace{1em}\color{white}#1}}}
\newcommand{\BoxAns}[1]{\fbox{\color{white}#1}}
\newcommand{\Reason}[1]{{\par\color{white}\par{}Reason: #1}}
\fi

\newif\ifamsstyle
\amsstylefalse
\newcounter{topic}
\input common
\input symbol
\makeatletter
\newcommand{\Arctan}{{\mathop{\operator@font Arctan}\nolimits}}
\makeatother

%\parindent=0pt
\parskip=1pt
\linespread{1}


\setcounter{topic}{6}

\begin{document}

\title{{\sc Rudiments of Riemann Surfaces\\
    Topic \thetopic{}: Existence of Meromorphic Functions}}
\author{Author: B. Frank Jones, Jr. (Rice Univ. 1971)\\
Seminar: Dr Liew How Hui (\url{liewhh@utar.edu.my})}
\date{}

\maketitle

%%% page 157
The thrust of this topic is the proof that there exist nonconstant
meromorphic functions on \textbf{any} Riemann surface.  It will take a
tremendous amount of machinery to achieve this result; in particular,
we will need to give a careful and fairly complete discussion of
harmonic functions on Riemann surfaces.  But before beginning this
topic, we shall exhibit one problem which can be solved using the
existence of meromorphic functions.

First, we introduce a lemma which really logically belongs in Topic~4,
but has not been needed before now.

\begin{lem}
  \label{lem:1}
  Let $m$ be a positive integer and $Q$ a meromorphic function near 0
  having Laurent expansion
  $$
  Q(s) = \sum_{j=-\infty}^{\infty} \alpha_j s^j,
  $$
  and assume that no positive integer except 1 is a common factor of
  all $j$ such that $\alpha_j \ne 0$.  Let $n$ be an integer
  relatively prime to $m$.  Then $(t^m, Q(t^n))$ is a pair.
\end{lem}

\begin{myproof}
  We have to prove that the mapping $t \mapsto (t^m, Q(t^n))$ is
  one-to-one near 0.  If this is not the case, then there exist $s_k
  \to 0$ and $t_k \to 0$ such that $s_k \ne t_k$ and $s_k^m = t_k^m$, 
  %%% page 158
  $Q(s_k^n) = Q(t_k^n)$.  Thus, $(s_k/t_k)^m = 1$, and by taking a
  sub-sequence we can assume that there exists a fixed $w$ such that
  $w \ne 1$, $w^m = 1$, $s_k = wt_k$.  Therefore, $Q(w^n t_k^n) =
  Q(t_k^n)$, so that the two functions $Q(w^n s)$ and $Q(s)$ agree on
  a sequence $s = t_k^n \to 0$.  Since they are both meromorphic, they
  must be identical:
  $$
  \sum_{j=-\infty}^{\infty} \alpha_j w^{nj} s^j 
  \equiv \sum_{j=-\infty}^{\infty} \alpha_j s^j,\quad s \text{ near } 0.
  $$
  Therefore the coefficients must agree: $\alpha_j w^{nj} = \alpha_j$
  for all $j$.  This means that $\alpha_j \ne 0$ implies $w^{nj} =
  1$.  It follows easily that $w^n = 1$.  For the set $\{ j ~:~ w^{nj}
  = 1\}$ is an additive subgroup of the integers and the Euclidean
  algorithm implies that any subgroup equals the integer  multiples of
  a fixed positive integer $j_0$.  Thus, $\alpha_j \ne 0$ implies $j$
  contains $j_0$ as a factor.  By hypothesis, $j_0 = 1$ and therefore
  $w^n = 1$.  Since $n$ and $m$ are relatively prime, the Euclidean
  algorithm again implies there exist integers $p$ and $q$ such that
  $pm + qn = 1$.  Thus,
  $$
  w = w^{pm + qn} = (w^m)^p(w^n)^q = 1,
  $$
  a contradiction.
\end{myproof}

\begin{thm}
  \label{thm:1}
  Let $S$ be any connected Riemann surface.  Let $f$ and $g$ be
  meromorphic functions on $S$ such that $f \not\equiv$ constant.
  Then there exists a unique analytic $\Phi : S \to \overline{M}$ such
  that
  $$
  f = \pi \circ \Phi, \quad g = v \circ \Phi.
  $$
\end{thm}

%%% page 159
It is convenient to draw a diagram to indicate these two equations:
$$
\xymatrix {
  S \ar[rr]^{\Phi} \ar[dr]_{f,g} && \overline{M} \ar[dl]^{\pi, v}\\
  & \widehat{\CCC} \times \widehat{\CCC}
}
$$
The statement of the theorem is then exactly that there exists an
analytic $\Phi$ making this diagram commutative.

\begin{myproof}
  \underline{Uniqueness:}  Suppose $p \in S$ and that $m_f(p) = 1$.
  Since $f = \pi \circ \Phi$, it follows that $m_\Phi(p) = 1$.   Let
  $\psi$ be any compatible chart in a neighbourhood of $p$.  Suppose
  $\Phi(p) = e(P, Q)$ and let $\varphi : U(P,Q,\Delta) \to \Delta$ be
  a canonical chart.  We assume $\psi(p) = 0$.

  \begin{mdframed}
    \vspace{3cm}
  \end{mdframed}

  %%% page 160
  Recall from Topic~5 (Definition 5.10) that
  $$
  P = \pi \circ \varphi^{-1},\quad
  Q = v \circ \varphi^{-1}.
  $$
  Now the mapping $\varphi \circ \Phi \circ \psi^{-1}$ is a
  \emph{parameter change}, since it is one-to-one near 0 and maps 0 to
  0.  Thus, we consider
  $$
  \begin{aligned}
    P \circ (\varphi \circ \Phi \circ \psi^{-1}) 
    &= \pi \circ \Phi \circ \psi^{-1} = f \circ \psi^{-1},\\
    Q \circ (\varphi \circ \Phi \circ \psi^{-1}) 
    &= v \circ \Phi \circ \psi^{-1} = g \circ \psi^{-1},
  \end{aligned}
  $$
  and we thus have
  $$
  (P, Q) \sim (f \circ \psi^{-1}, g\circ \psi^{-1}).
  $$
  Therefore, if $m_f(p) = 1$ we have
  $$
  \Phi(p) = e(f \circ \psi^{-1}, g\circ \psi^{-1}).
  $$
  This proves that $\Phi$ is uniquely determined except on the
  discrete set where the multiplicity of $f$ is greater that 1.  Since
  $\Phi$ is continuous on $S$, then $\Phi$ is also uniquely determined
  everywhere.

  \underline{Existence:}
  We already know how to define $\Phi$ at points where $m_f = 1$.
  Therefore, we so define $\Phi$ at those points, just noting that the
  definition $\Phi(p) = e(f\circ \psi^{-1}, g\circ \psi^{-1})$ really
  is independent of the particular chart $\psi$; a different selection
  of the chart merely gives a parameter change.

  %%% page 161
  Now suppose $p_0 \in S$ and $m_f(p_0) = m$.  Choose a chart $\psi$
  near $p_0$ such that $\psi(p_0) = 0$ and
  $$
  f \circ \psi^{-1}(t) = f(p_0) + t^m
  $$
  if $f(p_0) \ne \infty$.  As usual, if $f(p_0) = \infty$ we have
  instead
  $$
  f \circ \psi^{-1}(t) = t^{-m}.
  $$
  Then consider the Laurent expansion of $g \circ \psi^{-1}$:
  $$
  g \circ \psi^{-1}(t) = \sum_{k=-\infty}^{\infty} a_k t^k.
  $$
  Let $n$ be the largest positive integer which is a factor of all $k$
  such that $a_k \ne 0$; in case $a_k = 0$ for all $k$, let $n=m$.
  Then we can let $k = nj$ in the above series and we obtain
  $$
  g \circ \psi^{-1}(t) = \sum_{j=-\infty}^{\infty} a_{nj} t^{nj} = Q(t^n),
  $$
  where
  $$
  Q(s) = \sum_{j=-\infty}^{\infty} \alpha_j s^j
  $$
  and $\alpha_j = a_{nj}$.  Thus, either $\alpha_j = 0$ for all $j$ or
  there is no common factor of all $j$ with $\alpha_j \ne 0$ except 1
  (and $-1$).  Let $u$ be the positive integer which is the greatest
  common divisor of $m$ and $n$ and define
  $$
  \Phi(p_0) = e(f(p_0) + t^{m/\mu}, Q(t^{n/\mu}))
  $$
  %%% page 162
  (replace $f(p_0) + t^{m/\mu}$ by $t^{-m/\mu}$ if $f(p_0) =
  \infty$).  We have to check that this is really a meromorphic
  element, i.e., that
  $$
  (t^{m/\mu}, Q(t^{m/\mu}))
  $$
  is a pair.  If $\alpha_j = 0$ for all $j$, then $\frac{m}{\mu} = 1$
  and it is obvious.  Otherwise, Lemma \ref{lem:1} applied to the 
  relatively prime integers $\frac{m}{\mu}$ and $\frac{n}{\mu}$ shows 
  that we do have a pair.  Thus, $\Phi(p_0)$ makes sense; we do not 
  need to check that we have defined it independently of the choice of 
  $\psi$ (there are only $m$ choices to make) since we can regard the 
  choice of $\psi$ to be an arbitrary ``function'' of $p_0$.

  We now observe that this definition of $\Phi(p_0)$ works even when
  $m = 1$; then $\mu = 1$ and the definition agrees with the earlier
  definition of $\Phi$ at points where $f$ has multiplicity 1.  Note
  that obviously
  $$
  \pi \circ \Phi(p_0) = f(p_0),\quad
  v \circ \Phi(p_0) = Q(0) = g\circ \psi^{-1}(0) = g(p_0).
  $$
  Thus, the required commutativity of the diagram is proved.  We thus
  need to check the analyticity of $\Phi$ in order to finish the
  proof.  We prove that $\Phi$ is analytic in a neighbourhood of
  $p_0$, using the above notation.

  Let $z$ be near 0, $z \ne 0$, and let $p = \psi^{-1}(z)$.  Then for
  sufficiently small $z$, $p$ is a point where $f$ has multiplicity
  1.
  %%% page 163
  Define the chart $\psi_1 = \psi-z$, so that $\psi_1(p) = 0$ and
  $$
  \psi_1^{-1}(t) = \psi^{-1}(z + t).
  $$
  Therefore, according to our first definition of $\Phi$,
  $$
  \begin{aligned}
    \Phi(\psi^{-1}(z)) 
    &= e(f \circ \psi_1^{-1}, g\circ \psi_1^{-1})\\
    &= e(f \circ \psi^{-1}(z+t), g\circ \psi^{-1}(z+t))\\
    &= e(f(p_0) + (z+t)^m, Q((z+t)^n)).
  \end{aligned}
  $$
  Now we introduce the canonical chart near $\Phi(p_0)$: call it
  $\varphi : U \to \Delta$, where
  $$
  \varphi^{-1}(t_0) = e(f(p_0) + (t_0 + t)^{m/\mu},\ 
  Q((t_0 + t)^{n/\mu})).
  $$
  We introduce next the parameter change $\rho$ defined by
  $$
  \rho(t) = (z+t)^\mu - z^\mu;
  $$
  since $z \ne 0$, this is a parameter change.  And we have
  $$
  (z+t)^m = (z^\mu + \rho(t))^{m/\mu}\quad
  (\text{and a similar formula with }n),
  $$
  showing that 
  $$
  \begin{aligned}
  \Phi(\psi^{-1}(z))
  &= e(f(p_0) + (z^\mu + \rho(t))^{m/\mu}, Q((z^\mu +
  \rho(t))^{n/\mu}))\\
  &= e(f(p_0) + (z^\mu + t)^{m/\mu}, Q((z^\mu + t)^{n/\mu}))
  = \varphi^{-1}(z^\mu).
  \end{aligned}
  $$
  Thus,
  %%% page 164
  $$
  \varphi \circ \Phi \circ \psi^{-1}(z) = z^\mu,
  $$
  so $\varphi \circ \Phi \circ \psi^{-1}$ is holomorphic near 0. This
  proves that $\Phi$ is analytic near $p_0$.
\end{myproof}

\begin{cor}
  Let $S$ be any compact connected Riemann surface and $f,g$
  meromorphic functions on $S$ such that $f \not\equiv$ constant.
  Then there exist a unique compact analytic configuration $T$ and
  analytic function $\Phi$ from $S$ onto $T$ such that the diagram
  commutes:
  $$
  \xymatrix {
    S \ar[rr]^{\Phi} \ar[dr]_{f,g} && T \ar[dl]^{\pi, v}\\
    & \widehat{\CCC} \times \widehat{\CCC}
  }
  $$
\end{cor}

\begin{myproof}
  This is trivial.  We just let $T = \Phi(S)$. Since $S$ is compact
  and connected and $\Phi$ is continuous, $T$ is also compact and
  connected.  Since $\Phi$ is analytic and nonconstant,
  Proposition~2.18 of Topic~2 implies $T$ is an open subset of
  $\overline{M}$.  As $T$ is thus closed and open and connected, it is
  an analytic configuration.
\end{myproof}

\begin{cor}
  Under the hypothesis of the previous corollary, there exists a
  unique irreducible polynomial $A(z,w)$ such that
  %%% page 165
  $$
  A(f(p), g(p)) = 0\text{ for } p \in S.
  $$
\end{cor}

\begin{myproof}
  We first prove uniqueness.  If $A$ has the required properties, then
  $A(\pi(\Phi(p)), v(\Phi(p))) = 0$ for $p \in S$.  Since $\Phi$ is
  onto, this implies $A(\pi(e), v(e)) = 0$ for $e \in T$.  Therefore
  the Riemann surface for $A$ satisfies $S_A \supset T$.  Since $T$ is
  a component of $\overline{M}$ and $S_A$ is connected, $S_A = T$.
  Thus, Theorem~5.17 of Topic~5 shows $A$ is unique.

  Existence is trivial.  Simply let $A$ be chosen by Theorem~5.17
  such that $S_A = T$.  The above argument worked the other direction 
  proves $A(f,g) = 0$.
\end{myproof}

We are most interested in the possibility that the mapping $\Phi$ of
$S$ onto $T$ is also one-to-one.  For then we will have an analytic
equivalence of the compact Riemann surface $S$ with an analytic
configuration.  The next theorem gives some equivalent conditions.

\begin{thm}
  \label{thm:2}
  Let $S$ be a compact connected Riemann surface and $f, g$
  meromorphic functions on $S$ such that $f \not\equiv$ constant.  Let
  $\Phi$, $T$, $A$ be the objects of the two previous corollaries.
  Assume that $f$ takes every value $n$ times.  Then the following
  conditions are equivalent.
  \begin{enumerate}
  \item $\Phi$ is an analytic equivalence of $S$ onto $T$.
  \item There exists a point $z \in \widehat{\CCC}$ such that $\{g(p)
    ~:~ f(p) = z \}$ has $n$ points.
  \item For all except finitely many $z \in \widehat{\CCC}$, $\{ g(p)
    ~:~ f(p) = z\}$ has $n$ points.
  \item The polynomial $A$ has degree $n$ in $w$.
  \end{enumerate}
\end{thm}

%%% page 166
\begin{myproof}
  $1 \Rightarrow 4$:  Since $\Phi$ is an analytic equivalence and $f =
  \pi \circ \Phi$, $\pi$ also takes every value $n$ times.  The
  results of Topic~5, especially Theorem 5.17, imply that $A$ has
  degree $n$ in $w$.

  $4 \Rightarrow 3$: If $z$ is a regular point for $A$ (and this is
  true for all but the finitely many critical points), then $T \cap
  \pi^{-1}(\{z\}) = \{e_1,\cdots, e_n\}$ and the numbers $v(e_1),
  \cdots, v(e_n)$ are distinct, being the solutions of $A(z, w) = 0$.
  Since $\Phi$ is onto, there exist $p_1, \cdots, p_n \in S$ such that
  $\Phi(p_k) = e_k$.  Then $g(p_k) = v(e_k)$ and $f(p_k) = \pi(e_k) =
  z$, so
  $$
  \{g(p) ~:~ f(p) = z\}
  $$
  has at least $n$ points $v(e_1), \cdots, v(e_n)$.  Since $f$ takes
  every value $n$ times, this set can contain no more than $n$
  points. 

  $3 \Rightarrow 2$: Trivial.

  $2 \Rightarrow 1$: Finally we come to an interesting proof.  By
  hypothesis there are $n$ points $p_1, \cdots, p_n$ such that $f(p_k)
  = z$ and the numbers $g(p_1), \cdots, g(p_n)$ are distinct.  In
  particular, since $f$ takes the value $z$ $n$ times, $m_f(p_1) =
  1$.  Since $f = \pi \circ \Phi$, $m_\Phi(p_1) = 1$.  Let $e =
  \Phi(p_1)$.  We shall show that $\Phi$ takes the value $e$ one
  time.  Suppose then that $\Phi(p) = e$.  Then $f(p) = \pi(e) = \pi
  \circ \Phi(p_1) = f(p_1) = z$, so $p = p_k$ for some $k$.  Then
  $g(p_k) = v(e) = v \circ \Phi(p_1) = g(p_1)$, so $p_k = p_1$.  Thus
  $\Phi(p) = e$ if and only if $p = p_1$.  Since moreover $m_\Phi(p_1)
  = 1$, we have now proved that $\Phi$ takes the value $e$ one time.
  But Proposition~2.24 item 1 of Topic~2 implies $\Phi$ takes every
  value one time.  That is, $\Phi$ is one-to-one.  Thus, $\Phi$ is an
  analytic equivalence.
\end{myproof}

%%% page 167
Let us comment on 4.  Suppose that the mapping $\Phi$ takes every
value $k$ times, and that $\pi : T \to \widehat{\CCC}$ takes every
value $m$ times.  Then since $f = \pi \circ \Phi$, it is easy to see
that $f$ takes every value $mk$ times.  In the notation of
Theorem~\ref{thm:2}, this means $n=mk$ and $A$ has degree $m$ in $w$.
Thus, in general the degree of $A$ is a factor of $n$.  For example,
consider the trivial case in which $S = \widehat{\CCC}$, $f(z) = z^4$
and $g(z) = z^2$.  The uniqueness assertion of the previous corollary
implies
$$
A(z, w) = w^2 - z.
$$
For, $A$ is irreducible and $A(f(z), g(z)) = g(z)^2 = f(z) = z^4 - z^4
= 0$.  Thus, $f$ takes every value 4 times, the degree of $A$ is 2, so
we conclude that $\Phi$ takes every value 2 times.  In fact, our
explicit construction shows that for $z \ne 0, \infty$,
$$
\begin{aligned}
  \Phi(z) &= e((z+t)^4, (z+t)^2).\\
  \Phi(-z) &= e((-z+t)^4, (-z+t)^2) = e((z-t)^4, (z-t)^2)
  = e((z+t)^4, (z+t)^2) = \Phi(z)
\end{aligned}
$$
by the parameter change $t \to -t$.

%%% page 168
Now we state the main theorem of this section and show how it can be
used to produce functions $f$ and $g$ which satisfy the criteria of
Theorem~\ref{thm:2}.  Note that we must at least produce a nonconstant
meromorphic function $f$ on $S$; the following theorem allows us to do
even better.

\begin{thm}
  \label{thm:3}
  Let $S$ be any connected Riemann surface and let $p,q \in S$, $p \ne
  q$.  Then there exists a meromorphic function $f$ on $S$ such that
  $f(p) \ne f(q)$.
\end{thm}

We are nowhere near being able to prove this yet.  But assuming its
validity for the moment we prove

\begin{cor}
  Let $S$ be a compact connected Riemann surface.  Then $S$ is
  analytically equivalent to an analytic configuration.
\end{cor}

\begin{myproof}
  First apply Theorem~\ref{thm:3} to find a nonconstant meromorphic
  $f$ on $S$.  Now we show how to construct another meromorphic $g$ on
  $S$ which satisfies criterion 2 of Theorem~\ref{thm:2}.  Assume that
  $f$ takes every value $n$ times.  
  %%% page 169
  If $n=1$, take $g=0$.  Suppose $n
  > 1$.  Since the points of $S$ where the multiplicity of $f$ is
  greater than 1 are isolated, there exists $z \in \widehat{\CCC}$
  such that $f^{-1}(\{z\})$ consists of $n$ distinct points $p_1,
  \cdots, p_n$.  Theorem~\ref{thm:3} implies that if $j\ne 1$, there
  exists a meromorphic $h$ on $S$ such that $h(p_j) \ne h(p_1)$.
  Choose a complex number $\alpha \notin \{h(p_1), \cdots, h(p_n)\}$.
  Then there exists a M\"obius transformation
  $$
  F(w) = \frac{aw+b}{cw+d}
  $$
  such that $F(h(p_1)) = 1$, $F(h(p_j)) = 0$, $F(\alpha) = \infty$.
  Thus, there exists a meromorphic $h_j = F\circ h$ such that
  $$
  h_j(p_1) = 1,\quad
  h_j(p_j) = 0,\quad
  h_j(p_k)\text{ is in }\CCC\text{ for }1\le k\le n.
  $$
  Define $g_1 = \prod_{j=1}^n h_j$.  Then $g_1$ is meromorphic on $S$
  and
  $$
  g_1(p_1) = 1,\quad
  g_1(p_j) = 0, \quad 2\le j\le n.
  $$
  Repeating this construction, there exist meromorphic functions $g_1,
  \cdots, g_n$ on $S$ such that
  $$
  g_k(p_k) = 1,\quad
  g_k(p_j) = 0 \text{ if } j\ne k.
  $$
  Now define
  $$
  g = \sum_{k=1}^{n} kg_k.
  $$
  %%% page 170
  Then $g$ is meromorphic on $S$ and $g(p_k) = k$, $1\le k \le n$.
  Therefore,
  $$
  g(p): f(p) \equiv z = \{1,2,\cdots, n\}
  $$
  has $n$ points.  So criterion 3 of Theorem~\ref{thm:2} is satisfied
  and therefore $S$ is analytically equivalent to an analytically
  configuration (criterion 1 of Theorem~\ref{thm:2}.
\end{myproof}

Now we shall begin to introduce the machinery needed to prove
Theorem~\ref{thm:3}.  The basis is the idea of harmonic functions on
Riemann surfaces.  First, we recall that a function $u$ on an open set
in $\CCC$ is said to be harmonic if $u$ is a class $C^2$ and
$\frac{\p^2 u}{\p x^2} + \frac{\p u^2}{\p y^2} = 0$.  A convenient way
of discussing this is to define the differential operators
$$
\p u \equiv \frac{1}{2}\frac{\p u}{\p x} 
+ \frac{1}{2i}\frac{\p u}{\p y},
\quad
\overline{\p} u \equiv \frac{1}{2}\frac{\p u}{\p x} 
- \frac{1}{2i}\frac{\p u}{\p y}.
$$
Then
$$
\p\overline{\p} u = \overline{\p} \p u
= \frac{1}{4}(\frac{\p^2 u}{\p x^2} + \frac{\p u^2}{\p y^2}).
$$
Now the equation $\overline{\p} f = 0$ is exactly the Cauchy-Riemann
equation.  Thus, $f$ is holomorphic if and only if $\overline{\p} f =
0$; moreover, in this case $\p f = f'$, the ordinary complex
derivative of $f$.  Thus, if $u$ is of class $C^2$, then $u$ is
harmonic if and only if $\p u$ is holomorphic.  In particular,
%%% page 171
if $u$ is harmonic then $\p u$ has derivatives of all orders.
Likewise, $\overline{u}$ is harmonic so that $\overline{\p} u =
\overline{\p \overline{u}}$ has derivatives of all orders.  Thus,
$\frac{\p u}{\p x} = \p u + \overline{\p} u$ has derivatives of all
orders, and the same is true for $\frac{\p u}{\p y}$.  Thus,
\emph{harmonic functions have derivatives of all orders}.

\textbf{Chain rule}.  There is a chain rule for these differential
operators, which we now describe.  Suppose $V$ and $W$ are open sets
in $\CCC$ and $h : V \to W$ is a class $C^1$ mapping of $V$ into $W$.

Let $u : W \to \CCC$ be of class $C^1$.  Then $u \circ h$ is also of
class $C^1$ and if we let $D_1$ denote partial differentiation with
respect to the first argument and $D_2$ with respect to the second
argument, the usual chain rule reads
$$
\begin{aligned}
  D_1(u \circ h) &= (D_1 u) \circ h D_1 (\Re(h)) 
  + (D_2 u) \circ h D_1 (\Im(h)),\\
  D_2(u \circ h) &= (D_1 u) \circ h D_2 (\Re(h)) 
  + (D_2 u) \circ h D_2 (\Im(h)).  
\end{aligned}
$$
Therefore,
$$
\begin{aligned}
  \p(u\circ h) &= (D_1 u) \circ h \p(\Re(h)) + (D_2 u)\circ h
  \p(\Im(h))\\
  &= (D_1 u) \circ h \frac{\p h + \p \overline{h}}{2} 
  + (D_2 u)\circ h \frac{\p h - \p \overline{h}}{2i}\\
  &= (\frac{1}{2}D_1u + \frac{1}{2i}D_2 u)\circ h \p h
  + (\frac{1}{2}D_1u - \frac{1}{2i}D_2 u)\circ h \p \overline{h}\\
  &= (\p u) \circ h \p h + (\overline{\p} u) \circ h \p \overline{h}.
\end{aligned}
$$
The corresponding formula for $\overline{\p}(u\circ h)$ follows the
same way.  We thus obtain
%%% page 171
$$
\begin{aligned}
  \p(u\circ h) &= (\p h) \circ h \p h + (\overline{\p}u) \circ 
  \p\;\overline{h},\\
  \overline{\p}(u\circ h) &= (\p h) \circ h \overline{\p} h 
  + (\overline{\p}u) \circ    \overline{\p}\;\overline{h}.
\end{aligned}
$$
As a special case, suppose $h$ is holomorphic.  Then $\p h = h'$,
$\overline{\p} h = 0$, so we obtain
\begin{equation}
  \label{e:1}
  \p(u\circ h) = (\p u) \circ h h',\quad
  \overline{\p}(u\circ h) = (\overline{\p} u) \circ h 
  \overline{h'}.
\end{equation}

Now define $\Delta = \frac{\p^2}{\p x^2} + \frac{\p^2}{\p y^2}$ (the
\emph{Laplacian}); as we have seen, $\Delta = 4\p \overline{\p} 
= 4 \overline{\p}\p$. Thus, if $h$ is holomorphic, the above chain
rule implies
$$
\Delta(u \circ h) 
= 4\overline{\p}[(\p u)\circ h \times h']
= 4\overline{\p}[(\p u)\circ h] \times h'
+ 4(\p u)\circ h \times \overline{\p} h']
= 4[\overline{\p}\p u \circ h]\overline{h'}h' + 0,
$$
so we obtain
\begin{equation}
  \label{e:2}
  \Delta(u \circ h) = (\Delta u) \circ h |h'|^2.
\end{equation}

We need one more formula involving $\p$.  Suppose $f$ is holomorphic.
Then
$$
\p \Re(f) = \frac{\p f + \p \overline{f}}{2} = \frac{f'+0}{2},
$$
so we have
%%% page 173
\begin{equation}
  \label{e:3}
  \p \Re(f) = \frac{1}{2}f'.
\end{equation}

\begin{defn}
  \label{def:1}
  Let $u$ be a real-valued function defined on a Riemann surface $S$.
  Then $u$ is \emph{harmonic} if for every chart $\varphi : U \to W$
  in the complete analytic atlas for $S$, $u\circ \varphi^{-1}$ is
  harmonic on $W$.
\end{defn}

\begin{propn}
  \label{propn:1}
  Let $S$ be a Riemann surface and $u : S \to \RR$.  Then the
  following conditions are equivalent.
  \begin{enumerate}
  \item $u$ is harmonic.
  \item For each $p \in S$ there exists a chart $\varphi : U \to W$ in
    the complete analytic atlas for $S$ such that $p \in U$ and $u
    \circ \varphi^{-1}$ is harmonic in a neighbourhood of $\varphi(p)$.
  \item In a neighbourhood of each point of $S$ there exists
    holomorphic functions $f$ and $g$ such that $u = f +
    \overline{g}$.
  \item In a neighbourhood of each point of $S$ there exists a
    holomorphic function $F$ such that $u = \Re(F)$.
  \end{enumerate}
\end{propn}

\begin{mdframed}
  \textbf{Proof}:
  \begin{itemize}
  \item $1 \Rightarrow 2$: Trivial.
  \item $2 \Rightarrow 1$: If $u \circ \varphi^{-1}$ is harmonic near
    $\varphi(p)$ as in condition 2., and if $\psi$ is any compatible
    chart near $p$, then
    $$
    u \circ \psi^{-1} = u\circ \varphi^{-1} \circ (\varphi \circ
    \psi^{-1}),
    $$
    so $u \circ \psi^{-1}$ is also harmonic by formula
    \eqref{e:2}. This proves 1.
  %%% page 174
  \item $2\Rightarrow 3$: Since $u \circ \varphi^{-1}$ is harmonic,
    $\p (u \circ \varphi^{-1})$ is holomorphic.  Locally, any
    holomorphic function has a primitive, so there exists a
    holomorphic function $f$ near $p$ such that near $\varphi(p)$
    $$
    \p(u\circ \varphi^{-1}) = (f \circ \varphi^{-1})'.
    $$
    Define $\overline{g} = u - f$.  Then
    $$
    \overline{\p}(g \circ \varphi^{-1})
    = \overline{\p(\overline{g} \circ \varphi^{-1})}
    = \overline{\p(u\circ \varphi^{-1}) - \p(f\circ \varphi^{-1})}
    = \overline{\p(u\circ \varphi^{-1}) - (f\circ \varphi^{-1})'}
    = 0.
    $$
    Thus, $g \circ \varphi^{-1}$ is holomorphic, proving $g$ is
    holomorphic.
  \item $3 \Rightarrow 4$: Using $u = f + \overline{g}$, we have since
    $u$ is real, $u = \Re(u) = \Re(f) + \Re(\overline{g}) = \Re(f) +
    \Re(g)$, so we merely take $F = f+g$.
  \item $4 \Rightarrow 2$: We have $u \circ \varphi^{-1} = \Re(F \circ
    \varphi^{-1})$ is harmonic near $\varphi(p)$.
  \end{itemize}
\end{mdframed}

\begin{propn}
  \label{propn:2}
  Let $S$ and $T$ be Riemann surfaces, $F : S \to T$ an analytic
  mapping.  If $u$ is a harmonic function on $T$, then $u \circ F$ is
  harmonic on $S$.
\end{propn}

%%% page 175
\begin{myproof}
  If $\varphi$ is a chart on $S$ and $\psi$ a chart on $T$, then we
  must investigate $(u\circ F) \circ \varphi^{-1}$.  This is
  $$
  (u \circ F)\circ \varphi^{-1} 
  = u \circ \varphi^{-1} \circ (\varphi \circ F \circ \varphi^{-1})
  $$
  and we know $u \circ \varphi^{-1}$ is harmonic and $\varphi \circ F
  \circ \varphi^{-1}$ is holomorphic. Therefore, formula \eqref{e:2}
  implies $u \circ F$ is harmonic.
\end{myproof}

\begin{propn}
  \label{propn:3}
  Let $S$ be a connected Riemann surface and $u$ a harmonic function
  on $S$.  If $u$ vanishes on a neighbourhood of some point of $S$,
  then $u \equiv 0$.
\end{propn}

\begin{myproof}
  Define $A = \{p \in S ~:~ u \equiv 0$ in a neighbourhood of $p\}$.
  Then $A$ is open by definition and $A \ne \emptyset$ by hypothesis.
  We now demonstrate that $A$ is closed: suppose $p_0$ is a limit
  point of $A$>  By criterion 4 of Proposition~\ref{propn:1}, there
  exists a holomorphic function $F$ near $p_0$ such that $u = \Re(F)$
  near $p_0$.  Thus, $\Re(F)$ vanishes on some open set near $p_0$,
  namely on the intersection of $A$ with any neighbourhood of $p_0$
  where $F$ is defined.  But the $F$ must be construction on this open
  set and by the uniqueness of analytic continuation $F$ is constant.
  Thus, $u$ is constant near $p_0$ and thus $p_0 \in A$.  Since $A$ is
  open and closed and not empty, and since $S$ is connected, $A = S$. 
\end{myproof}

The fundamental Theorem~\ref{thm:3} actually follows from a theorem on
the existence of harmonic functions, which we now state.
%%% page 176
\begin{thm}
  \label{thm:4}
  Let $S$ be any connected Riemann surface and let $p \in S$.  Let
  $\varphi : U \to W$ be a chart in the complete analytic atlas for
  $S$ with $p \in U$ and $\varphi(p) = 0$.  Let $n$ be a positive
  integer.  Then there exists a harmonic function $u$ on $S-\{p\}$
  such that for $z$ near 0
  $$
  u \circ \varphi^{-1}(z) = c\log|z| + \Re(f(z)),
  $$
  where $c$ is some real constant and $f$ is meromorphic in a
  neighbourhood of 0 and has a pole of order $n$ at 0.
\end{thm}

Thus, Theorem~\ref{thm:4} guarantees the existence of a harmonic
function on $S - \{p\}$ with prescribed singularity at $p$.  For
emphasis, we repeat that the order of the pole of $f$ at 0 is
\emph{exactly} $n$: $\p_f(0) = -n$.

Now we shall indicate how the knowledge of Theorem~\ref{thm:4} leads
to a proof of Theorem~\ref{thm:3}.  Let $p_0, q_0$ be the distinct
points on $S$ mentioned in the hypothesis of Theorem~\ref{thm:3}.  Let
$u$ be harmonic on $S - \{p_0\}$ with representation near $p_0$ as
prescribed by Theorem~\ref{thm:4} with (say) $n=1$:
$$
\begin{aligned}
  u \circ \varphi^{-1}(z) &= c\log|z| + \Re(f(z)),\\
  f(z) &= \frac{\alpha}{2} + \text{(Laurent expansion near 0)},\
  \alpha \ne 0.
\end{aligned}
$$
Let $\psi$ be a chart near $q_0$, $\psi(q_0) = 0$, and let $v$ be a
harmonic function on $S - \{q_0\}$ with expansion near 0 of the form
%%% page 177
$$
\begin{aligned}
  v \circ \psi^{-1}(z) &= d\log|z| + \Re(g(z)),\\
  g(z) &= \frac{\beta}{2} + \cdots,\ \beta \ne 0.
\end{aligned}
$$
Using these two harmonic functions we shall construct the meromorphic
functions required in Theorem~\ref{thm:3}.  Here is how it is done:
let $p_1 \in S$ and let $\sigma$ be a chart near $p_1$ (in the
complete analytic atlas for $S$).  Near $p_1$ we define
$$
F(p) = \frac{\p(u\circ \sigma^{-1})(\sigma(p))}{
  \p(v\circ \sigma^{-1})(\sigma(p))}.
$$
First, we show this definition to be independent of $\sigma$.  Let
$\sigma_1$ be another chart near $p_1$ and let $h = \sigma \circ
\sigma_1^{-1}$, so that $h$ is holomorphic and has a holomorphic
inverse.  Then formula \eqref{e:1} implies
$$
\begin{aligned}
  \p(u\circ \sigma_1^{-1})(\sigma_1(p))
  &= \p(u\circ \sigma^{-1} \circ h)(\sigma_1(p))\\
  &= \p(u\circ \sigma^{-1})(h(\sigma_1(p)))h'(\sigma_1(p))\\
  &= \p(u\circ \sigma^{-1})(\sigma(p))h'(\sigma_1(p)).
\end{aligned}
$$
Therefore,
$$
\frac{\p(u\circ\sigma_1^{-1})(\sigma_1(p))}{
  \p(v\circ\sigma_1^{-1})(\sigma_1(p))}
= \frac{\p(u\circ\sigma^{-1})(\sigma(p))}{
  \p(v\circ\sigma^{-1})(\sigma(p))}
$$
since the common nonzero factor $h'(\sigma_1(p))$ cancels after
division.  Thus, the definition of $F$ is independent of the choice of
chart.

%%% page 178
Next, since $u \circ \sigma^{-1}$ and $v \circ \sigma^{-1}$ are
harmonic, the functions $\p(u \circ \sigma^{-1})$ and $\p(v\circ
\sigma^{-1})$ are holomorphic, and not identically zero since
otherwise e.g. $\overline{v \circ \sigma^{-1}}$ would be holomorphic
(Cauchy-Riemann equation) and thus constant (since it is
real-valued).  But then Proposition~\ref{propn:3} would imply that $v$
is constant on $S - \{q_0\}$, which manifestly contradicts its
singular behaviour near $q_0$.  Thus, the zeros of $\p(v\circ
\sigma^{-1})$ are isolated, so the formula for $F$ exhibits $F$ as the
quotient of two holomorphic functions near $p_1$, the denominator not
vanishing identically, and thus $F$ is meromorphic near $p_1$.  Thus,
$F$ is meromorphic on $S - \{p_0\} - \{q_0\}$.

Finally, we must examine the behaviour of $F$ near $p_0$ and $q_0$.
Near $p_0$ we use the chart $\varphi$ and compute according to
\eqref{e:3}
$$
\p(u \circ \varphi^{-1}(z)) = \frac{c}{2z} + \frac{1}{2}f'(z)
= -\frac{\alpha}{2z^2} + \cdots,
$$
so that $F\circ \varphi^{-1}$ has a pole of order at least 2 at 0.
Thus, $F(P_0) = \infty$.  Likewise, near $q_0$ we use the chart $\psi$
and compute
$$
\p(v \circ \psi^{-1}(z)) = -\frac{\beta}{2z^2} + \cdots,
$$
so that $F \circ \psi^{-1}$ has a zero of order at least 2 at 0.

Thus, $F(q_0) = 0$.  This concludes the proof of Theorem~\ref{thm:3}.

%%% page 179
We have therefore finally reduced the problem to that of demonstrating
Theorem~\ref{thm:4}.  It will take a considerable amount of machinery
and technique in the area of harmonic function theory to accomplish
this, so we now begin a discussion of the relevant properties we
need.

\begin{propn}
  \label{propn:4}
  Let $u$ be continuous on $\overline{\Delta}$, the closure of an open
  disk $\Delta \subset \CCC$, and harmonic in $\Delta$.  Suppose
  $\Delta$ has centre $z_0$ and radius $r$.  Then
  $$
  u(z_0) = \frac{1}{2\pi} \int_0^{2\pi} u(z_0 + re^{i\theta})
  d\theta.
  $$
\end{propn}

\begin{myproof}
  Let $0 < \rho < r$.  Then the divergence theorem implies
  $$
  0 = \int_{|z-z_0|< \rho} (\frac{\p^2u}{\p x^2} + \frac{\p^2u}{\p
    y^2}) dx dy = \int_{|z-z_0|<\rho} \frac{\p u}{\p \nu} dS,
  $$
  where $dS$ is the element of arc length on the circle $|z-z_0| =
  \rho$ and $\frac{\p u}{\p \nu}$ is the directional derivative in the
  direction of the outer normal.  Another way of writing this is
  $$
  0 = \int_0^{2\pi} (\frac{\p}{\p \rho} u(z_0 + \rho e^{i\theta}))
  \rho d\theta.
  $$
  %%% page 180
  Dividing by $\rho$ and then moving $\frac{\p}{\p \rho}$ outside the
  sign of integration implies
  $$
  0 = \frac{\p}{\p \rho}\int_0^{2\pi} u(z_0 + \rho e^{i\theta}) d\theta.
  $$
  Therefore, the continuous function of $\rho \in [0,r]$ given by
  $$
  \rho \mapsto \frac{1}{2\pi} \int_0^{2\pi} u(z_0 + \rho e^{i\theta})
  d\theta
  $$
  is constant.  Since its value at $\rho = 0$ is $u(z_0)$, the result
  follows. 
\end{myproof}

Now we show how to apply this simple property of harmonic functions to
obtain a representation of $u$ in all of $\Delta$, not just at the
centre.  First, we take $\Delta$ to be the unit disk for simplicity of 
computations.  Let $a \in \Delta$ and consider the M\"obius 
transformation
$$
T(z) = \frac{z-a}{1-\overline{a}z};
$$
$T$ maps $\Delta$ onto $\Delta$ conformally, $\overline{\Delta}$ onto
$\overline{\Delta}$, and $T(a) = 0$.  Thus $u \circ T^{-1}$ is
harmonic on $\Delta$, continuous on $\overline{\Delta}$, so that
Proposition~\ref{propn:4} implies
$$
u \circ T^{-1}(0) = \frac{1}{2\pi} \int_0^{2\pi} u \circ
T^{-1}(e^{i\varphi}) d\varphi.
$$
Now we introduce the change of variable
%%% page 181
$$
e^{i\theta} = T^{-1}(e^{i\varphi}).
$$
Then $e^{i\varphi} = T(e^{i\theta})$, so that a simple computation
yields
$$
\begin{aligned}
  \frac{d \varphi}{d\theta}
  &= \frac{e^{i\theta}}{e^{i\theta}-a} 
  + \frac{\overline{a} e^{i\theta}}{1-\overline{a} e^{i\theta}}
  = \frac{e^{i\theta}}{e^{i\theta}-a} 
  + \frac{\overline{a}}{e^{-i\theta}-\overline{a}} \\
  &= \frac{1-\overline{a}e^{i\theta} + \overline{a} e^{i\theta} -
    \overline{a} a}{(e^{i\theta}-a)(e^{-i\theta}-\overline{a})}
  = \frac{1-|a|^2}{|e^{i\theta} - a|^2}
\end{aligned}
$$
Therefore,
$$
u(a) = \frac{1}{2\pi} \int_0^{2\pi} \frac{1-|a|^2}{|e^{i\theta} -
  a|^2} u(e^{i\theta}) d\theta.
$$
Define
\begin{equation}
  \label{e:4}
  P(z, e^{i\theta}) = \frac{1}{2\pi} \frac{1-|z|^2}{|e^{i\theta} -
    z|^2}, \quad |z| < 1;
\end{equation}
this is the so-called \emph{Poisson kernel}.  We want to observe
certain things about it:
\begin{enumerate}
\item $P \ge 0$;
\item $\int_0^{2\pi} P(z, e^{i\theta}) d\theta = 1$, $|z| < 1$;
\item $P(z, e^{i\theta})$ is harmonic function of $z$;
\item for any $\delta > 0$,
  %%% page 182
  $$
  \lim_{z \to e^{i\theta_0}} \int_{|e^{i\theta} - e^{i\theta_0}|
    \ge \delta} P(z, e^{i\theta}) d\theta = 0.
  $$
\end{enumerate}
The first property is obvious and the second follows from formula
\eqref{e:4} applied to the harmonic function $u \equiv 1$.  The third
follows from the formula for $\frac{d\varphi}{d\theta}$, which reads
$$
2\pi P(z, e^{i\theta}) = \frac{e^{i\theta}}{e^{i\theta}-z}
+ \frac{\overline{z}e^{i\theta}}{1-\overline{z}e^{i\theta}},
$$
exhibiting $P$ as a sum of two harmonic functions of $z$.

To prove the fourth, assume $|z-e^{i\theta_0}| < \delta/2$.  Then
$$
|e^{i\theta} - z| \ge |e^{i\theta} - e^{i\theta_0}| - |e^{i\theta_0} -
z| > \delta - \delta/2 = \delta/2,
$$
so that
$$
\int_{|e^{i\theta} - e^{i\theta_0}| \ge \delta} P(z, e^{i\theta})
d\theta \le \frac{1}{2\pi} \frac{1-|z|^2}{(\delta/2)^2} \cdot 2\pi
< \frac{8}{\delta^2}(1-|z|),
$$
and this clearly tends to zero as $z \to e^{i\theta_0}$.

These four properties are all we need to establish the following
converse to formula \eqref{e:4}.

\begin{propn}
  \label{propn:5}
  Let $f$ be a continuous function on the circle $|z| = 1$.  Define
  $$
  u(z) =
  \begin{cases}
    \int_0^{2\pi} P(z,e^{i\theta}) f(e^{i\theta}) d\theta, & |z| <
    1,\\
    f(z), & |z| = 1.
  \end{cases}
  $$
  Then $u$ is harmonic for $|z| < 1$ and continuous for $|z| \ge 1$.
\end{propn}

\begin{myproof}
  The fact that $u$ is harmonic for $|z| < 1$ follows from 3. by
  differentiation under the integer sign.  Clearly, we need only prove
  that $\lim u(z) = f(e^{i\theta_0})$ for $|z| < 1$, $z \to
  e^{i\theta_0}$, in order to finish the proof.  Let $\epsilon > 0$.
  By continuity of $f$ at $e^{i\theta_0}$, there exists $\delta > 0$
  such that $|f(e^{i\theta}) - f(e^{i\theta_0})| < \frac{\epsilon}{2}$
  if $|e^{i\theta} - e^{i\theta_0}| < \delta$.  Now 2. implies
  $$
  u(z) - f(e^{i\theta_0}) = \int_0^{2\pi} P(z, e^{i\theta})
  [f(e^{i\theta}) - f(e^{i\theta_0})] d\theta.
  $$
  Choose a constant $C$ such that $|f(e^{i\theta})| \le C$ for all
  $\theta$.  Then
  $$
  |u(z) - f(e^{i\theta_0})|
  \le \frac{\epsilon}{2}\int_{|e^{i\theta}-e^{i\theta_0}|< \delta}
  P(z, e^{i\theta})d\theta 
  + 2C \int_{|e^{i\theta}-e^{i\theta_0}|< \delta}
  P(z, e^{i\theta})d\theta.
  $$
  Since the first integral is bounded by 1, and property 4. implies
  there exists $\delta' > 0$ such that the second integral is bounded
  by $\frac{\epsilon}{4C}$ if $|z - e^{i\theta_0}| < \delta'$, we
  obtain
  $$
  |u(z) - f(e^{i\theta_0})| < \epsilon
  $$
  if $|z - e^{i\theta_0}| < \delta'$.
\end{myproof}

%%% page 184
Of course, it is not necessary to restrict our attention to the unit
disk.  If we consider functions in the disk $\Delta_r(z_0)$, the
formula analogous to that of Proposition~\ref{propn:5} is
$$
u(z) = \frac{1}{2\pi} \int_0^{2\pi} \frac{r^2 - |z-z_0|^2}{
|re^{i\theta} - (z-z_0)|^2} f(z_0 + re^{i\theta}) d\theta.
$$
This can be derived in the same manner, or merely by considering the
change of variable $z - z_0 = rw$ and using Proposition~\ref{propn:5}
as it stands.

The Poisson integral formula we have just derived has several
immediate applications which will be of great importance to us.  For
example, we have 

\begin{propn}
  \label{propn:6}
  Let $D$ be an open set in $\CCC$ and $K$ a compact subset of $D$.
  Then there exists a constant $C$ which depends only on $K$ and $D$
  such that if $u$ is harmonic in $D$ then
  $$
  \sup_K \left|\frac{\p u}{\p x}\right|
  \le C \sup_D |u|.
  $$
  A similar result holds will $\frac{\p}{\p x}$ replaced by any
  derivative of any order.
\end{propn}

\begin{myproof}
  For any $z_0 \in K$ there exists a disk $\Delta_r(z_0)$ such that
  the close of $\Delta$ is contained in $D$.  For $|z-z_0| <
  \frac{1}{2}r$, the Poisson integral formula implies
  $$
  \left|\frac{\p u}{\p x}(z)\right| \le C \sup_D |u|,
  $$
  where
  %%% page 185
  $$
  C = \sup_{\substack{|z-z_0|<\frac{1}{2}r\\ 0\le \theta \le 2\pi}}
  \left|\frac{\p}{\p x} \frac{r^2-|z-z_0|^2}{
      |re^{i\theta} - (z-z_0)|^2}\right|
  $$
  is easily seen to be finite.  Since $K$ can be covered by finitely
  many such disks as $\{z ~:~ |z-z_0| < \frac{1}{2}r\}$, the results
  follows.
\end{myproof}

\begin{propn}
  \label{propn:7}
  Let $D$ be an open set in $\CCC$ and $u_1, u_2, \cdots$ a sequence
  of harmonic functions in $D$ which converge uniformly on compact
  subsets of $D$ to a function $u$.  Then $u$ is harmonic in $D$ and
  the sequence $\frac{\p u_n}{\p x}$ converges to $\frac{\p u}{\p x}$,
  also uniformly on compact sets in $D$.
\end{propn}

\begin{myproof}
  If $\Delta$ is a disk whose closure is contained in $D$, then $u_n$
  has a Poisson integral representation in $\Delta$ of the form
  $$
  u_n(z) = \frac{1}{2\pi} \int_0^{2\pi} \frac{r^2 - |z-z_0|^2}{
  |re^{i\theta} - (z-z_0)|^2} u_n(z_0 + re^{i\theta}) d\theta.
  $$
  For fixed $z \in \Delta$ let $n\to \infty$ in this formula and use
  the uniform convergence to pass the limit under the integral sign to
  obtain
  $$
  u(z) = \frac{1}{2\pi} \int_0^{2\pi} \frac{r^2 - |z-z_0|^2}{
  |re^{i\theta} - (z-z_0)|^2} u(z_0 + re^{i\theta}) d\theta.
  $$
  Therefore, $u$ is harmonic in $\Delta$.  Therefore, $u$ is harmonic
  in $D$.

  Now suppose $K$ is a compact subset of $D$.  Choose an open set
  $D_1$ such that $K \subset D_1$ and the closure of $D_1$ is compact
  subset of $D$.  Let $C$ be the constant of Proposition~\ref{propn:6}
  relative to $K$ and $D_1$.  Then
  %%% page 186
  $$
  \sup_K \left|\frac{\p u_n}{\p x} - \frac{\p u}{\p x}\right|
  \le C \sup_{D_1} |u_1 - u|.
  $$
  By hypothesis, $\sup_{D_1}|u_n-u| \to 0$ as $n \to \infty$, and
  therefore $\frac{\p u_n}{\p x} \to \frac{\p u}{\p x}$ uniformly on
  $K$.  As $K$ is arbitrary, the result follows.
\end{myproof}

\begin{propn}
  \label{propn:8}
  Let $D$ be an open set in $\CCC$ and $u_1, u_2, \cdots$ a sequence
  of harmonic functions in $D$ which are uniformly bound on every
  compact subset of $D$.  Then there exists a subsequence $n_1 < n_2 <
  \cdots$ such that
  $$
  \lim_{k\to \infty} u_{n_k}
  $$
  exists uniformly on compact subsets of $D$.
\end{propn}

\begin{myproof}
  If $\Delta$ is a disk such that its closure is a compact subset of
  $D$, then there exists a constant $C$ depending only on $\Delta$
  such that $|u_n(z)| \le C$ for $z \in \Delta$, $n \ge 1$.
  Therefore, Proposition~\ref{propn:6} implies that if
  $\frac{1}{2}\Delta$ is the concentric disk with half the radius of
  $\Delta$ then for some other constant $C_1$
  $$
  \left|\frac{\p u_n}{\p x}\right| \le C_1,\quad
  \left|\frac{\p u_n}{\p y}\right| \le C_1 \quad\text{on }
  \frac{1}{2}\Delta.
  $$
  Now we apply the mean value theorem on the disk $\frac{1}{2}\Delta$
  (details omitted) to conclude for $z, z' \in \frac{1}{2}\Delta$,
  %%% page 187
  $$
  |u_n(z) - u_n(z')| \le 2C_1 |z-z'|.
  $$
  This proves that the family of functions $u_1, u_2, \cdots$ is
  \emph{equi-continuous} on $\frac{1}{2}\Delta$.  Since $\Delta$ was
  arbitrary, it follows that the family $u_1, u_2, \cdots$ is
  equi-continuous on each compact subset of $D$.  By the Arzela-Ascoli
  theorem, there exists a subsequence with the required property that
  $u_{n_k}$ converges uniformly on compact subsets of $D$.
\end{myproof}

\begin{defn}
  \label{def:2}
  An open subset $D$ of a Riemann surface is an \emph{analytic disk}
  if there exists a chart $\varphi : U \to W$ in the complete analytic
  atlas such that $\varphi(D)$ is a disk whose closure is a (compact)
  subset of $W$.
\end{defn}

\textbf{Notation}.  If $A$ is a subset of a topological space, $A^-$
denotes the closure of $A$ and $\p A$ denotes the boundary of $A$.

\begin{propn}
  \label{propn:9}
  Let $D$ be an analytic disk in a Riemann surface $S$ and let $f : \p
  D \to \RR$ be continuous.  Then there exists a unique function $P_f$
  on $D^-$ such that $P_f$ is continuous on $D^-$, harmonic in $D$,
  and $P_f \equiv f$ on $\p D$.
\end{propn}

\begin{myproof}
  Let $\varphi : U \to W$ be a chart in the complete analytic atlas
  for $S$ satisfying the condition of Definition~\ref{def:2}.  If
  $\varphi(D) = \{z ~:~ |z-z_0| < r\}$, then $P_f \circ \varphi^{-1}$
  must be continuous on $\varphi(D)^-$, harmonic on $\varphi(D)$, and
  $P_f \circ \varphi^{-1} \equiv f \circ \varphi^{-1}$ on $\p
  \varphi(D)$ (= $\varphi(\p D)$).  Thus, if $z \in \varphi(D)$, then
  %%% page 188
  $$
  P_f \circ \varphi^{-1}(z) = \frac{1}{2\pi}\int_0^{2\pi}
  \frac{r^2-|z-z_0|^2}{|re^{i\theta} - (z-z_0)|^2}
  f\circ \varphi^{-1}(z_0 + re^{i\theta}) d\theta.
  $$
  Therefore, $P_f$ is uniquely determined and
  Proposition~\ref{propn:5} implies that $P_f$ as defined by this
  formula satisfies the conditions of Proposition~\ref{propn:9}.
\end{myproof}

\begin{defn}
  \label{def:3}
  If $D$ is an analytic disk in Riemann surface $S$ and if $u : S \to
  \RR$ is continuous, $u_D$ is the unique continuous function on $S$
  which agrees with $u$ on $S - D$ and is harmonic in $D$.  The
  existence and uniqueness of $u_D$ are guaranteed by
  Proposition~\ref{propn:9}.
\end{defn}

\begin{lem}
  \label{lem:2}
  Let $S$ be a Riemann surface and $p_0 \in S$.  Let $\epsilon >
  0$. Then there exists a neighbourhood $U$ of $p_0$ such that for all
  functions $u$ which are harmonic and nonnegative on $S$, and for all
  $p, q \in U$
  $$
  u(p) \le (1+\epsilon) u(q).
  $$
\end{lem}

\begin{myproof}
  There exists a chart $\varphi : U_0 \to W$ in the complete analytic
  atlas for $S$ such that $W$ contains $\{z : |z| < 1\}$ and
  $\varphi(p_0) = 0$.  This can obviously be achieved by composing an
  arbitrary chart with a suitable linear transformation of $\CCC$ onto
  itself.  Let $v = u \circ \varphi^{-1}$.  Then
  %%% page 189
  $$
  v(z) = \int_0^{2\pi} P(z, e^{i\theta} v(e^{i\theta}) d\theta,
  \quad |z| < 1.
  $$
  Now $1 - |z| \le |e^{i\theta} - z| \le 1 + |z|$, so we obtain
  $$
  \frac{1-|z|}{1+|z|}
  = \frac{1-|z|^2}{(1+|z|)^2}
  \le \frac{1-|z|^2}{|e^{i\theta}-z|^2} \le \frac{1-|z|^2}{(1-|z|)^2}
  = \frac{1+|z|}{1-|z|}.
  $$
  Therefore, since $v(e^{i\theta}) \ge 0$,
  $$
  \frac{1-|z|}{1+|z|} \frac{1}{2\pi} \int_0^{2\pi} v(e^{i\theta})
  d\theta
  \le v(z) \le
  \frac{1+|z|}{1-|z|} \frac{1}{2\pi} \int_0^{2\pi} v(e^{i\theta})
  d\theta.
  $$
  By Proposition~\ref{propn:4} this pair of inequalities can be
  written in the form
  $$
  \frac{1-|z|}{1+|z|}v(0) \le v(z) \le \frac{1+|z|}{1-|z|}v(0).
  $$
  If $0 < \delta < 1$ and $|z| \le \delta$, we obtain
  $$
  \frac{1-\delta}{1+\delta} v(0) \le v(z) \le
  \frac{1+\delta}{1-\delta} v(0).
  $$
  Therefore, if $|z| \le \delta$ and $|w| \le \delta$,
  $$
  v(z) \le \frac{1+\delta}{1-\delta}v(0) 
  \le (\frac{1+\delta}{1-\delta})^2 v(w).
  $$
  %%% page 190
  Pick $\delta$ such that
  $$
  (\frac{1+\delta}{1-\delta})^2 \le 1 + \epsilon.
  $$
  Then let $U = \varphi^{-1}(\{z ~:~ |z| < \delta\})$.  This is a
  neighbourhood of $p_0 = \varphi^{-1}(0)$ and for $p,q \in U$,
  $$
  u(p) = v(\varphi(p)) \le (1+\epsilon) v(\varphi(q))
  = (1 + \epsilon) u(q).
  $$
\end{myproof}

\begin{thm}[Harnack's Inequality]
  Let $S$ be a connected Riemann surface and $K$ a compact subset of
  $S$.  Then there exists a constant $C$ depending only on $K$ and $S$
  such that for all nonnegative harmonic functions $u$ on $S$ and all
  $p, q\in K$,
  $$
  u(p) \le Cu(q).
  $$
\end{thm}

\begin{mdframed}
  \textbf{Proof}:
  If obviously suffices to consider the class $H$ of functions which
  are harmonic and positive on $S$; if $u$ is harmonic and $u \ge 0$,
  then for every $\epsilon > 0$, $u + \epsilon \in H$ and if the
  inequality is true for functions in $H$ then $u(p) + \epsilon \le
  C(u(q) + \epsilon)$.  Then let $\epsilon \to 0$.  Of course, we are
  debating a triviality anyway, because Harnack's inequality implies
  that if $u \ge 0$ and $u$ is harmonic, then either $u \equiv 0$ or
  $u> 0$.  Now choose some fixed point $p_0 \in S$ and define
  $$
  F(p) = \sup\ \max\ (\frac{u(p)}{u(p_0)}, \frac{u(p_0)}{u(p)}) ~:~ u
  \in H.
  $$
  %%% page 191
  We are going to prove $F$ is continuous.  Let $p_1 \in S$ and let
  $\epsilon > 0$.  Let $U$ be a neighbourhood of $p_1$ satisfying the
  condition of Lemma~\ref{lem:2}.  Then for $u \in H$ and $p,q\in U$,
  $$
  \begin{aligned}
    \frac{u(p)}{u(p_0)} &\le (1 + \epsilon) \frac{u(q)}{u(p_0)}
    \le (1 + \epsilon) F(q),\\
    \frac{u(p_0)}{u(p)} &\le (1 + \epsilon) \frac{u(p_0)}{u(q)}
    \le (1 + \epsilon) F(q).
  \end{aligned}
  $$
  Therefore,
  \begin{equation}
    \label{e:5}
    F(p) \le (1 + \epsilon) F(q) \quad \text{for }p,q\in U.
  \end{equation}
  In particular, if $F(p_1) < \infty$ we choose $q = p_1$ to conclude
  that $F(p) < \infty$ for all $p \in U$; if $F(p_1) = \infty$ we
  choose $p=p_1$ to conclude that $F(q) = \infty$ for all $q \in U$.
  Therefore, the sets
  $$
  \{p \in S ~:~ F(p) < \infty\}, \quad
  \{p \in S ~:~ F(p) = \infty\}
  $$
  are open.   As they are obviously disjoint and their union is
  obviously $S$< the connectedness of $S$ implies one of these sets is
  $S$, the other empty.  Since $F(p_0) = 1$, $p_0$ belongs to the
  first of the sets, and thus we have proved that $F < \infty$
  everywhere on $S$.

  Now we obtain the continuity.  Taking $q = p_1$ in \eqref{e:5},
  $$
  F(p) - F(p_1) \le \epsilon F(p_1) \text{ if } p \in U;
  $$
  taking $p = p_1$,
  $$
  -\epsilon F(p_1) \le (1 + \epsilon) (F(q) - F(p_1)) \text{ if }
  q \in U.
  $$
  %%% page 192
  Thus, we obtain
  $$
  - \frac{\epsilon}{1+\epsilon} F(p_1) \le F(p) - F(p_1) \le \epsilon
  F(p_1) \text{ if } p \in U,
  $$
  and since $\epsilon$ is arbitrary, this proves that $F$ is
  continuous at $p_1$.

  Since $F$ is continuous on $S$ and $K$ is compact, there exists a
  constant $c$ such that $F(p) \le c$ for $p \in K$.  Therefore, if
  $p, q \in K$
  $$
  u(p) < F(p) u(p_0) \le F(p) F(q) u(q) \le c^2 u(q).
  $$
\end{mdframed}

\begin{thm}[Harnack's Convergence Theorem]
  Let $S$ be a connected Riemann surface and $\calH$ a nonvoid family
  of harmonic functions on $S$ which is directed upwards, i.e., if $u,
  v \in \calH$, there exists $w \in \calH$ such that $w \ge u$, $w \ge
  v$.  Let $U = \sup \calH$, i.e. for $p \in S$
  $$
  U(p) = \sup\{u(p) ~:~ u \in \calH\}.
  $$
  Then there exists a sequence $u_1 \le u_2 \le u_3 \le \cdots$ such
  that $u_n \in \calH$ for all $n$ and $u_n \to U$ uniformly on
  compact subsets of $S$.  Moreover, either $U \equiv \infty$ or $U$
  is harmonic on $S$.
\end{thm}

\begin{mdframed}
  \textbf{Proof}:  Let $u_0$ be an arbitrary function in $\calH$ and
  let
  $$
  \calH' = \{u \in \calH ~:~ u \ge u_0\}.
  $$
  Then $\sup \calH' = \sup \calH$.  Indeed, since $\calH' \subset
  \calH$ the inequality $\sup \calH' \ge \sup \calH$ is obvious, and
  if $u \in \calH$ then there exists $v \in \calH$ such that $v \ge u$
  and $v \ge u_0$; therefore, $v \in \calH'$ and $v \ge u$, proving
  that $\sup \calH' \ge \sup \calH$.

  %%% page 193
  For any compact set $K \subset S$ let $C_K$ be the corresponding
  constant in the conclusion of Harnack's inequality.  The for $u \in
  \calH'$, $u - u_0$ is a nonnegative harmonic function, so for all
  $p, q \in K$ it follows that
  $$
  u(p) - u_0(p) \le C_K(u(q)-u_0(q)) 
  \le C_K(U(q)-u_0(q)).
  $$
  Taking the supremum over all $u \in \calH'$ implies
  \begin{equation}
    \label{e:6}
    U(p) - u_0(p) \le C_K(u(q)-u_0(q)).
  \end{equation}
  It follows that if $U(q) < \infty$, then $U(p) < \infty$, and here
  $p,q$ can be any points in $S$ (just take $K$ to be the compact set
  $[p,q]$).  Therefore, either $U \equiv \infty$ or $U < \infty$.

  Let $p_0 \in S$ be fixed and choose a sequence $u_1', u_2', u_3',
  \cdots$ from $\calH$ such that $u_n'(p_0) \to U(p_0)$.  By
  hypothesis, we can let $u_1 = u_1'$ and then find inductively $u_n
  \in \calH$ such that
  $$
  u_n \ge u_n',\quad u_n \ge u_{n-1}.
  $$
  Then we have a sequence $u_1 \le u_2 \le \cdots$ from $\calH$ such
  that $u_n(p_0) \to U(p_0)$.  Note that \eqref{e:6} holds for an
  arbitrary $u_0 \in \calH$, so we can take $u_0 = u_n$ in
  \eqref{e:6}.  If $U(p_0) = \infty$, then for any compact set $K$
  containing $p_0$ Harnack's inequality implies
  $$
  u_n(p_0) - u_1(p_0) \le C_K(u_n(q) - u_1(q)), \quad q\in K,
  $$
  %%% page 194
  and therefore $u_n(q) \to \infty$ uniformly on $K$.  This proves the
  result in case $U \equiv \infty$.  If $U(p_0) < \infty$, the
  \eqref{e:6} implies
  $$
  U(p) - u_n(p) \le C_K(U(p_0)-u_n(p_0)),\quad p \in K,
  $$
  and therefore $u_n(p) \to U(p)$ uniformly for $p \in K$.  Finally,
  Proposition~\ref{propn:7} shows that $U$ is harmonic in this case.
\end{mdframed}

Now we need to introduce the basic building block other than the
Poisson integral, which is \emph{subharmonic} functions.  The basic
theory is contained in the following proposition.

\begin{propn}
  \label{propn:10}
  Let $u$ be a continuous real-valued function on a connected Riemann
  surface $S$.  Then the following conditions are equivalent.
  \begin{enumerate}
  \item For every analytic disk $D \subset S$, $u \le u_D$.
    (Cf. Definition~\ref{def:3}.)
  \item If $\sO$ is a proper open subset of $S$, if $\sO^-$ is
    compact, if $h$ is continuous on $\sO^-$ and harmonic in $\sO$,
    and if $u \le h$ on $\p \sO$, then $u = h$ in $\sO$.
  \item For each $p \in S$ there exists a chart $\varphi : U \to W$ in
    the complete analytic atlas for $S$ such that $p \in U$ and
    $$
    u(p) \le \frac{1}{2\pi} \int_0^{2\pi} u \circ \varphi^{-1}(
    \varphi(p) + re^{i\theta}) d \theta
    $$
    for small positive $r$.
    %%% page 195
  \item For each $p \in S$ and every chart $\varphi : U \to W$ in the
    complete analytic atlas for $S$ such that $p \in U$ and every $z$
    sufficiently close to $\varphi(p)$,
    $$
    u \circ \varphi^{-1}(z) \le \frac{1}{2\pi} \int_0^{2\pi}
    u \circ \varphi^{-1}(z + re^{i\theta}) d\theta
    $$
    for small positive $r$.
  \end{enumerate}
\end{propn}

\begin{mdframed}
  \textbf{Proof}: We are going to establish four implications, three
  of which are absolutely trivial.

  \begin{itemize}
  \item $2 \Rightarrow 1$: Assume that 2 holds and let $D$ be an
    analytic disk.  Use 2 with $\sO = D$ and $h = u_D$ restricted to
    $\sO^-$.  Then $u = h$ on $\p D$ so we obtain $u \le h$ in $D$,
    i.e., $u \le u_D$ in $D$.  Since $u = u_D$ outside $D$, 1 follows.
  \item $1 \Rightarrow 4$: Assume that 1 holds and consider the
    analytic disk
    $$
    D = \varphi^{-1}(\{w ~:~ |z-w| < r\})
    $$
    for sufficiently small $r$.  Then $u \le u_D$, so in particular
    $$
    u \circ \varphi^{-1}(z) \le u_D \circ \varphi^{-1}(z).
    $$
    Since $u_D \circ \varphi^{-1}$ is continuous on $\{w ~:~ |z-w| \le
    r\}$ and harmonic in the interior, the mean value property of
    Proposition~\ref{propn:4} implies
    $$
    u_D \circ \varphi^{-1}(z)
    = \frac{1}{2\pi} \int_0^{2\pi} u_D \circ
    \varphi^{-1}(z+re^{i\theta}) d \theta
    = \frac{1}{2\pi} \int_0^{2\pi} u \circ
    \varphi^{-1}(z+re^{i\theta}) d \theta
    $$
    since $u_D = u$ on $\p D$.  Thus, 4 follows.
  \item $4 \Rightarrow 3$: Completely trivial: we allow \emph{every}
    chart in 4 and moreover 3 is just 4 at the single point $z =
    \varphi(p)$.
  \item $3 \Rightarrow 2$: Finally here is something which requires
    thought.  Assume that 3 holds and assume we have the hypothesis of
    2.  Define $v = u-h$ in $\sO^-$.  Let $M = \sup_{\sO^-} v$.  Since
    $\sO^-$ is compact and $v$ is continuous, the supremum is
    attained, so the set
    $$
    A = \{p \in \sO ~:~ v(p) = M\}
    $$
    is either nonvoid or $v = M$ somewhere on $\p \sO$.  In the latter
    case, since $v \le 0$ on $\p \sO$ we obtain $M \le 0$ and the
    result follows. So we assume $v < M$ everywhere on $\p \sO$, in
    which case $A$ is not empty.  Since $v$ is continuous, the set $A$
    is \emph{closed} relative to $\sO$.  We use 3 to show that $A$ is
    \emph{open}: suppose $p \in A$.  Pick a chart $\varphi$ according
    to 3 with respect to the point $p$.  Then for small positive $r$
    $$
    u(p) \le \frac{1}{2\pi}\int_0^{2\pi} u \circ
    \varphi^{-1}(\varphi(p) + re^{i\theta}) d \theta.
    $$
    Since $h$ is harmonic, it satisfies the similar relation with
    \emph{equality} instead of inequality (Proposition~\ref{propn:4})
    and thus we obtain by subtraction
    $$
    v(p) \le \frac{1}{2\pi} \int_0^{2\pi} v \circ
    \varphi^{-1}(\varphi(p) + re^{i\theta}) d\theta.
    $$
    But the left side is $v(p) = M$ and the integrand
    %%% page 197
    $v \circ \varphi^{-1}(\varphi(p) + re^{i\theta}) \le M$ by
    definition of $M$.  Thus, we conclude that equality holds
    everywhere, so
    $$
    v \circ \varphi^{-1}(\varphi(p) + re^{i\theta}) \equiv M
    $$
    for $0 \le \theta \le 2\pi$ and small positive $r$.  Thus, $v
    \equiv M$ near $p$, so $A$ is open.

    It is now a simple topological argument to show that $\p A$ and
    $\p \sO$ have a point in common.  To see this, let $p_0 \in A$ and
    $p_1 \in S - \sO$ be chose arbitrarily, and use the connectedness
    of $S$ to conclude that there exists a path $\gamma$ in $S$ from
    $p_0$ to $p_1$.  Since $p_0 \in A$ and $p_1 \notin A$ and the
    image of $\gamma$ is connected, there exists a point $p_2$ in the
    image of $\gamma$ such that $p_2 \in \p A$.  If a neighbourhood of
    $p_2$ were disjoint from $\sO$, it would also be disjoint from
    $A$, contradicting $p_2 \in A^-$.  Therefore, $p_2 \in \sO^-$.  If
    $p_2 \in \sO$, then since $A$ is closed in $\sO$, $p_2 \in A$;
    since $A$ is open, this contradicts the fact that $p_2 \in
    (S-A)^-$.  Thus $p_2 \in \p \sO$.  Since $p_2 \in \p A$, $v(p_2) =
    M$ by continuity of $v$.  Since $p_2 \in \p\sO$, $v(p_2) \le 0$ by
    hypothesis.  Therefore, $M \le 0$, and the conclusion of 2
    follows.
  \end{itemize}
\end{mdframed}

\begin{defn}
  \label{def:4}
  A continuous real-valued function $u$ on a Riemann surface $S$ is
  \emph{subharmonic} if it satisfies condition 4 of
  Proposition~\ref{propn:10}.
\end{defn}

%%% page 198
\begin{thm}[Strong Maximum Principle]
  Let $u$ be a subharmonic function on a connected Riemann surface $S$
  such that $u \le 0$.  Then either $u < 0$ on $S$ or $u \equiv 0$ on
  $S$.
\end{thm}

\begin{myproof}
  This is contained in the proof of $3 \Rightarrow 2$ of
  Proposition~\ref{propn:10}.  For the set $A = \{p \in S ~:~ u(p) =
  0\}$ is closed since $u$ is continuous and is open by Condition~4 of
  Proposition~\ref{propn:10}, and thus either $A = S$ or $A$ is
  empty.
\end{myproof}

\begin{thm}[Weak Maximum Principle]
  Let $S$ be a connected Riemann surface, and $\sO$ a proper open
  subset of $S$ such that $\sO^-$ is compact.  Let $u$ be continuous
  on $\sO^-$ and subharmonic on $\sO$.  Assume $u \le 0$ on $\p
  \sO$. Then $u \le 0$.
\end{thm}

\begin{myproof}
  This is again contained in the proof of $3 \Rightarrow 2$ of
  Proposition~\ref{propn:10}.  If $M = \max_{\sO^-} u$ and if $M > 0$,
  let $A = \{p \in \sO ~:~ u(p) = M\}$.  Then the argument proving $3
  \Rightarrow 2$ shows that $\p A$ and $\p \sO$ have a point in common
  and thus $M \le 0$, a contradiction.
\end{myproof}

\begin{cor}
  If $\sO$ is a proper open subset of a connected Riemann surface $S$
  such that $\sO^-$ is compact and if $u$ is harmonic on $\sO$ and
  continuous on $\sO^-$, then 
  $$
  \sup_{\sO^-} |u| = \sup_{\p\sO} |u|.
  $$
\end{cor}

%%% page 199
\begin{myproof}
  Let $M = \sup_{\p \sO}|u|$.  Then $-M+u$ and $-M-u$ are subharmonic
  in $\sO$ and nonpositive on $\p\sO$, so the weak maximum principle
  implies $-M+u \le 0$ and $-M-u\le 0$ in $\sO$.  That is, $-M \le u
  \le M$.
\end{myproof}

\begin{propn}
  \label{propn:11}
  Let $u$ be a continuous real-valued function on a Riemann surface
  $S$.  Then $u$ is harmonic if and only if $u$ and $-u$ are
  subharmonic.
\end{propn}

\begin{mdframed}
  \textbf{Proof}:
  We only have to prove the ``if'' part of the assertion.  Since the
  proposition deals with local properties, we can assume $S$ is
  connected.  By part 1 of Proposition~\ref{propn:10}, if $D$ is an
  analytic disk, then $u \le u_D$ and $-u \le (-u)_D$.  But clearly
  $(-u)_D = -u_D$, so we have $u \le u_D$ and $-u \le -u_D$.  Thus, $u
  \equiv u_D$.  Therefore, $u$ is harmonic in $D$.  Since every point
  of $S$ is contained in an analytic disk, $u$ is harmonic in $S$.
\end{mdframed}

\begin{propn}
  \label{propn:12}
  Let $u, u_1, \cdots, u_n$ be subharmonic on a Riemann surface
  $S$, and let $a_1, a_2, \cdots, a_n$ be nonnegative real numbers.
  Then the functions
  $$
  a_1 u_1 + \cdots + a_n u_n,\quad \max(u_1,\cdots,u_n)
  $$
  are subharmonic.  Also, if $D$ is an analytic disk, $u_D$ is
  subharmonic.
\end{propn}

%%% page 200
\begin{myproof}
  This follows directly from the definition.  Condition 4 of
  Proposition~\ref{propn:10} asserts in that notation that for small
  positive $r$
  $$
  u_k \circ \varphi^{-1}(z) \le \frac{1}{2\pi} \int_0^{2\pi} u_k \circ
  \varphi^{-1}(z + re^{i\theta}) d\theta.
  $$
  Multiplying by $a_k$ and adding, the function $a_1 u_1 + \cdots +
  a_n u_n$ is seen to satisfy condition 4.  If $u = \max(u_1, \cdots,
  u_n)$, then we have
  $$
  u_k \circ \varphi^{-1}(z) \le \frac{1}{2\pi} \int_0^{2\pi}
  u \circ \varphi^{-1}(z+re^{i\theta}) d \theta,\quad 1 \le k \le n.
  $$
  Therefore,
  $$
  u \circ \varphi^{-1}(z) \le \frac{1}{2\pi} \int_0^{2\pi}
  u \circ \varphi^{-1}(z + re^{i\theta}) d \theta,
  $$
  proving that $u$ is subharmonic.  The last statement will be proved
  later.
\end{myproof}

The basic theorem we need is the following.

\begin{thm}
  \label{thm:5}
  Let $S$ be a connected Riemann surface and $\calI$ a nonempty family
  of subharmonic functions on $S$ such that
  \begin{enumerate}
  \item if $u, v \in \calI$, then $\max(u,v) \in \calI$,
  \item if $u \in \calI$ and $D$ is an analytic disk in $S$, then $u_D
    \in \calI$.
  \end{enumerate}
  Then $\sup \calI$ is either harmonic in $S$ or $\sup \calI \equiv
  \infty$.
\end{thm}

%%% page 201
\begin{myproof}
  Let $U = \sup \calI$.  If $D$ is an analytic disk in $S$, let
  $\calI_D$ be the functions on $D$ defined by
  $$
  \calI_D = \{u_D ~:~ u \in \calI\}.
  $$
  Then $\calI_D$ is a family of harmonic functions on $D$ and
  $$
  \sup \calI_D = U\text{ in }D.
  $$
  For, $u \in \calI$ implies $u_D \in \calI$, so that any function in
  $\calI_D$ is the restriction to $D$ of a function in $\calI$ and
  thus $\sup \calI_D \le U$ in $D$.  On the other hand, $u \in \calI$
  implies $u \le u_D$ by item 1 of Proposition~\ref{propn:10}.
  Therefore, $U \le \sup \calI_D$ in $D$.

  Now we apply Harnack's convergence theorem to the family $\calI_D$
  on the Riemann surface $D$.  We have to check that $\calI_D$ is
  directed upwards.  So suppose $u,v \in \calI$.  Let $w = \max(u,v)$,
  so that $w \in \calI$ by property 1.  Then $u \le w$ implies $u_D
  \le w_D$ and $v \le w$ implies $v_D \le w_D$, so that we have found
  $w_D \in \calI_D$ such that $w_D \ge u_D$, $w_D \ge v_D$.  Thus
  $\calI_D$ is directed upwards.  Thus, Harnack's convergence theorem
  implies that either $\sup \calI_D$ is harmonic or $\sup \calI_D
  \equiv \infty$.  Therefore, either $U$ is harmonic on $D$ or $U
  \equiv \infty$ on $D$.

  Finally, we have the familiar connectivity argument: if $A = \{p \in
  S ~:~ U(p) = \infty\}$ and $B = \{p \in S ~:~ U(p) < \infty\}$, then
  $A$ and $B$ are disjoint open sets with union $S$.  Since $S$ is
  connected, either $A$ is empty or $A = S$.  Thus, either $U <
  \infty$ on $S$ or $U \equiv \infty$ on $S$.  If $U < \infty$ on $S$,
  we have shown that $U$ is harmonic in every analytic disk in $S$.
  Therefore, $U$ is harmonic.
\end{myproof}

%%% page 202
\begin{ques}[The Dirichlet problem for an annulus]
  \label{ques:8}
  \begin{enumerate}
  \item Prove the Weierstrass approximation theorem for a circle.
    That is, if $f$ is a continuous complex-valued function on the
    circle $|z| = 1$ and $\epsilon > 0$, then there exists a
    \emph{finite} sum
    $$
    g(z) = \sum a_n z^n \text{ (positive and negative }n)
    $$
    such that $|f(z) - g(z)| < \epsilon$ for $|z| = 1$.

    \textbf{Hint}: Use Proposition~\ref{propn:5} and
    Proposition~\ref{propn:1}, with the obvious remark that the proof
    of $1 \Rightarrow 3$ for the disk $|z| < 1$ gives two holomorphic
    functions defined on the entire disk.
  \item Consider the annulus $r < |z| < 1$, where $0 < r < 1$ is
    fixed.  Let $n$ be an integer.  Exhibit the (unique) harmonic
    functions which equals $z^n$ for $|z| = 1$ and equals 0 for $|z| =
    r$. 
  %%% page 203
  \item Combine 1 and 2 to conclude that there exist a function
    $u_\epsilon$ which is harmonic for $r < |z| < 1$, continuous for
    $r \le |z| \le 1$ (in fact, it will be harmonic for $0 < |z| <
    \infty$) such that
    $$
    \begin{aligned}
      |u_\epsilon(z) - f(z)| < \epsilon & \quad\text{for}\quad |z|=1,\\
      u_\epsilon(z) = 0 & \quad\text{for}\quad |z|=r.
    \end{aligned}
    $$
  \item By a limiting argument, prove there exists a function $u$
    which is harmonic for $r < |z| < 1$, continuous for $r \le |z| \le
    1$, such that
    $$
    u(z) =
    \begin{cases}
      f(z) &\text{for } |z| = 1,\\
      0 &\text{for } |z| = r.
    \end{cases}
    $$
  \item Use this result and an appropriate conformal mapping to treat
    any continuous boundary values on $|z| = r$ as well.
  \end{enumerate}
\end{ques}

Now we state a corollary, and we use the obvious terminology that a
function $w$ is \emph{superharmonic} if $-w$ is subharmonic.

\begin{cor}
  Let $w$ be a superharmonic function on a connected Riemann surface
  $S$.  Let
  $$
  \calI = \{v ~:~ v \text{ subharmonic on }S,\ v \le w\}.
  $$
  Then $\sup \calI$ is either harmonic in $S$ or $\sup \calI \equiv
  -\infty$. 
\end{cor}

%%% page 204
\begin{myproof}
  We have $\sup \calI \equiv -\infty$ if and only if $\calI$ is empty,
  so we assume from now on that $\calI$ is not empty.  We verify the
  two properties required in Theorem~\ref{thm:5}.  First, if $v_1, v_2
  \in \calI$, then clearly $\max(v_1, v_2) \le w$ and
  Proposition~\ref{propn:12} implies $\max(v_1, v_2)$ is subharmonic;
  thus, $\max(v_1, v_2) \in \calI$.  If $D$ is an analytic disk in
  $S$, then for $v \in \calI$,
  $$
  v_D \le w_D \le w,
  $$
  the latter inequality being a consequence of criterion 1 of
  Proposition~\ref{propn:10} for \emph{superharmonic} functions.  So
  we need only check $v_D$ is subharmonic.  This amounts to checking
  the local criterion 3 of Proposition~\ref{propn:10}.  This mean
  value criterion clearly holds at any point $p \in D$ (since $v_D$ is
  harmonic near $p$) and at any point $p$ in $S - \overline{D}$ (since
  $v_D \equiv v$ is subharmonic near $p$).  So we consider $p \in \p
  D$ and a chart $\varphi$ in the complete analytic atlas for $S$,
  $\varphi$ defined near $p$.  Since $v$ is subharmonic and $v \le
  v_D$, we obtain for small positive $r$
  $$
  \begin{aligned}
    v_D(p) = v(p) \le \frac{1}{2\pi} \int_0^{2\pi} v \circ
    \varphi^{-1}(\varphi(p) + re^{i\theta}) d\theta
    < \frac{1}{2\pi} \int_0^{2\pi} v_D \circ \varphi^{-1}(\varphi(p)
    + re^{i\theta}) d\theta,
  \end{aligned}
  $$
  establishing the criterion in this case as well.  Thus, $v_D$ is
  subharmonic on $S$.  Now Theorem~\ref{thm:5} implies $\sup \calI
  \equiv \infty$ (which is impossible in this case) or $\sup \calI$ is
  harmonic.
\end{myproof}

%%% page 205
\begin{defn}
  \label{def:5}
  Let $w$ be a superharmonic function on a connected Riemann surface.
  The function
  $$
  u = \sup\{ v ~:~ v \text{ subharmonic on } S,\ v \le w\}
  $$
  is called the \emph{greatest harmonic minorant of $w$}.  This
  terminology agrees with the obvious fact that if $v$ is harmonic on
  $S$ and $v \le w$, then $v \le u$.  Moreover, if $w$ has any
  harmonic minorant at all, then $u$ is harmonic (not $\equiv
  -\infty$).  Of course, our corollary shows that actually $u$ is the
  greatest subharmonic minorant of $w$, and is itself harmonic.  We
  shall use the abbreviation
  $$
  u = \text{GHM of }w.
  $$
\end{defn}

As an application of these ideas, we show how to solve a certain kind
of Dirichlet problem.

\begin{propn}
  \label{propn:13}
  Let $D$ be an analytic disk in a connected Riemann surface $S$ and
  $f : \p D \to \RR$ a continuous function.  Then there exists a
  function $u$ which is continuous in $S-D$, harmonic in $S-D^-$, and
  such that $u = f$ on $\p D$.  Moreover, we can assume
  $$
  \sup_{S-D} u = \sup_{\p D} f,\quad
  \inf_{S-D} u = \inf_{\p D} f.
  $$
\end{propn}

\begin{rem}
  Nothing is claimed about the uniqueness of $u$.  As we shall see,
  $u$ is unique for certain $S$ and not unique for other $S$.
\end{rem}

%%% page 206
\begin{myproof}
  By Definition~\ref{def:2} of analytic disk, there exists a chart
  $\varphi : U \to W$ in the complete analytic atlas for $S$ such that
  $\varphi(D)$ is a disk $\Delta$.  Since $\Delta^- \subset W$, there
  exists a concentric disk $\Delta_1$ with $\Delta^- \subset
  \Delta_1$, $\Delta_1^- \subset W$.

  \begin{mdframed}
    \vspace{4cm}
  \end{mdframed}

  Let $D_1 = \varphi^{-1}(\Delta_1)$. If $c \in ???$ then by
  Problem~\ref{ques:8} there exists a unique function $h_c$ which is
  continuous on $\Delta_1^- - \Delta$, harmonic in $\Delta_1 -
  \Delta^-$, and such that
  $$
  h_c \equiv
  \begin{cases}
    c &\text{on }\p \Delta_1,\\
    f\circ \varphi^{-1} &\text{on }\p \Delta.
  \end{cases}
  $$
  Define a function $v_c$ on $S - D$ by the formula
  $$
  v_c =
  \begin{cases}
    h_c \circ \varphi & \text{in } D_1^- - D,\\
    c & \text{in } S-D_1.
  \end{cases}
  $$
  Then $v_c$ is continuous on $S-D$, $v_c \equiv f$ on $\p D$, and
  $v_c$ is harmonic in $S - D_1^-$ and in $D_1 - D^-$.  If $c \le
  \inf_{\p D} f$, then $v_c$ is subharmonic on $S - D^-$; for, the
  only points where we need to check the mean value criterion 3 of
  Proposition~\ref{propn:10} are on $\p D_1$, and there $v_c$ takes
  the value $c$.  But the minimum principle implies $h_c \ge c$ in
  $\Delta_1^- - \Delta$, and thus $v_c \ge c$ in $S-D$.  Therefore,
  criterion 3 of Proposition~\ref{propn:10} is trivially satisfied at
  a point of $\p D_1$.  Thus $v_c$ is subharmonic in $S-D^-$.
  Likewise, if $c \ge \sup_{\p D} f$, then $v_c$ is superharmonic in
  $S - D^-$.

  %%% page 207
  Let $A = \inf_{\p D} f$, $D = \sup_{\p D} f$.   Then $v_A$ is
  subharmonic in $S-D^-$, $v_B$ is superharmonic in $S-D^-$, and $v_A
  \le v_B$ in $S-D$.  This last inequality follows from the maximum
  principle, since $h_B - h_A$ is continuous in $\Delta_1^- - \Delta$,
  harmonic in $\Delta_1 - \Delta^-$, $\equiv B-A$ on $\p \Delta_1$,
  $\equiv 0$ on $\p \Delta$, and thus $h_B - h_A \ge 0$.  Let $u$ be
  GHM pf $v_B$.  Then since $v_A$ is a subharmonic minorant of $v_B$,
  we have
  \begin{equation}
    \label{e:7}
    v_A \le u \le v_B,
  \end{equation}
  and $u$ is defined and harmonic on $S - D^-$.  We have of course
  applied the corollary of Theorem~\ref{thm:5} to the connected
  Riemann surface $S - D^-$, which is why $u$ is defined only on $S -
  D^-$.  But the inequalities \eqref{e:7} imply that $u$ can be
  extended to a continuous function on $S =- D$ in exactly one way,
  namely by taking $u \equiv v_A \equiv v_B \equiv f$ on $\p D$.

  Finally the last assertion of the proposition follows from
  $$
  A \le v_A \le u < v_B \le B.
  $$
\end{myproof}

\begin{rem}
  The above analytic is typical in the sense that even when we wish to
  have boundary values for a certain harmonic function, the corollary
  to Theorem~\ref{thm:5} does not by itself give anything more than a
  harmonic function on an \emph{open} set.  Some other considerable,
  e.g. \eqref{e:7}, is needed to obtain information about the function
  at the boundary.  We shall see more instances of this phenomenon
  later.
\end{rem}

%%% page 208
To complete the preliminary material, we need to obtain a
representation for harmonic functions in an annulus, analogous to the
Laurent expansion of a holomorphic function.

\begin{propn}
  \label{propn:14}
  Let $u$ be a real harmonic function in an annulus $a < |z| < b$,
  where $0 \le a < b \le \infty$.  Then there exist unique complex
  numbers $c$, $\{a_n\}$, such that for $a < |z| < b$
  $$
  u(z) = c\log|z| + \Re(\sum_{n=-\infty}^{\infty} a_n z^n),
  $$
  and $a_0$ is real.  Furthermore, if $a < a' < b' < b$, then there
  exist constants $K$ and $\{K_n\}$ depending only on $a'$ and $b'$
  (and $n$) such that
  $$
  \begin{aligned}
    |c| &\le K \sup\{|u(z)| ~:~ a' \le |z| \le |b'|\},\\
    |a_n| &\le K_n \sup\{|u(z)| ~:~ a' \le |z| \le |b'|\}.
  \end{aligned}
  $$
\end{propn}

\begin{myproof}
  The discussion in the proof of Theorem~\ref{thm:1} implies $\p u$ is 
  holomorphic for $a < |z| < b$.  Therefore, the Laurent expansion of 
  $\p u$ exists, say
  $$
  \p u = \sum_{-\infty}^{\infty} c_n z^n,\quad a < |z| < b.
  $$
\end{myproof}

%%% page 209
By formula \eqref{e:3}, for $n \ne -1$ we have
$$
\begin{aligned}
  c_n z^n &= \frac{d}{dz} \frac{c_n z^{n+1}}{n+1} = 2 \p \Re(\frac{c_n
  z^{n+1}}{n+1}),\\
  c_{-1} z^{-1} &= \frac{d}{dz} (c_{-1} \log(z)) = 2 \p (c_{-1} \log|z|).
\end{aligned}
$$
Now the Laurent expansion for $\p u$ converges uniformly on compact
subsets of the annulus, and therefore the same is true for the
integrated series, so we obtain
$$
\begin{aligned}
  \p u &= 2 \p (c_{-1} \log|z|) + \sum_{n\ne -1} 2\p \Re(\frac{c_n z^{n+1}}{
    n+1})\\
  &= 2 \p (c_{-1} \log|z|) + \sum_{n\ne 0} 2\p \Re(\frac{c_{n-1} z^{n}}{n}) \\
  &= \p (c \log|z| + \Re(\sum_{n\ne0} a_n z^n)),
\end{aligned}
$$
where $c = 2c_{-1}$ and $a_n = \frac{2c_{n-1}}{n}$.  The
Cauchy-Riemann equation
$$
\overline{\p} (u - \overline{c} \log |z| - \Re(
\sum_{n\ne 0} a_n z^n)) = 0
$$
follows and shows that there exists a function $g$ holomorphic in $a <
|z| < b$ such that
$$
u = \overline{c} \log |z| + \Re(\sum_{n\ne 0} a_n z^n) + g(z).
$$
Taking the imaginary parts,
$$
\Im g = (\Im c) \log|z|.
$$
Since $i\log(z) = -\arg(z) + i\log|z|$, this shows that $g = i(\Im
c)\log(z)$, and thus $g$ is defined only if $\Im c = 0$.  Then $g$ is
a \emph{real} holomorphic function, and thus is constant.  Thus, if $g
\equiv a_0$, we have
%%% page 210
$$
u(z) = c \log|z| + \Re(\sum_{-\infty}^{\infty} a_n z^n),\quad
c,\ a_0\ \text{real},
$$
a representation of the desired form.

Now we obtain the uniqueness: if $a < r < b$, then
$$
\begin{aligned}
  u(re^{i\theta}) 
  &= c\log r 
  + \frac{1}{2}\sum_{n=-\infty}^{\infty} a_n r^n e^{in\theta} 
  + \frac{1}{2}\sum_{n=-\infty}^{\infty} \overline{a}_n r^n e^{-in\theta} \\
  &= c\log r 
  + \frac{1}{2}\sum_{n=-\infty}^{\infty} (a_n r^n + \overline{a_{-n}}
  r^{-n}) e^{in\theta}.
\end{aligned}
$$
Since this series converges uniformly for $0 \le \theta \le 2\pi$, we
obtain by integration
\begin{equation}
  \label{e:8}
  \begin{aligned}
    \frac{1}{2\pi} \int_0^{2\pi} u(re^{i\theta}) d\theta
    &= c\log r + a_0,\\
    \frac{1}{\pi} \int_0^{2\pi} u(re^{i\theta}) e^{-im\theta} d\theta
    &= a_m r^m + \overline{a_{-m}} r^{-m},\quad m \ne 0.
  \end{aligned}
\end{equation}
Here we have used the orthogonality relation
$$
\frac{1}{2\pi} \int_0^{2\pi} e^{in\theta} e^{-im\theta} d \theta
=
\begin{cases}
  0 &\text{if } m \ne n,\\
  1 &\text{if } m = n.
\end{cases}
$$
If we use the relations \eqref{e:8} for two different values $r, r'
\in (a,b)$, we can solve for all coefficients:
$$
\begin{aligned}
  c &= \frac{1}{\log r - \log r'} \frac{1}{2\pi}
  \int_0^{2\pi} [u(re^{i\theta}) - u(r'e^{i\theta})] d\theta,\\
  a_0 &= \frac{1}{\log r - \log r'} \frac{1}{2\pi}
  \int_0^{2\pi} [u(re^{i\theta})\log r - u(r'e^{i\theta})
  \log r'] d\theta,\\
  a_m &= \frac{1}{(\frac{r}{r'})^m - (\frac{r'}{r})^m} \frac{1}{\pi}
  \int_0^{2\pi} [u(re^{i\theta})(r')^{-m} - u(r'e^{i\theta})
  r^{-m}] e^{-im\theta} d\theta,\quad m \ne 0.
\end{aligned}
$$
%%% page 211
This proves that the coefficients are uniquely determined by $u$, and
at the same time shows easily how to obtain the estimates stated in
the last half of the proposition.

\begin{cor}[Removable Singularity Theorem]
  Let $u$ be harmonic and bounded in an annulus $0 < |z| < b$.  Then
  there exists a harmonic function in the disk $|z| < b$ which agrees
  with $u$ in the annulus $0 < |z| < b$.
\end{cor}

\begin{myproof}
  By Proposition~\ref{propn:14} we have
  $$
  u(z) = c\log|z| + \Re(\sum_{n=-\infty}^{\infty} a_n z^n),\quad
  0 < |z| < b.
  $$
  In formula \eqref{e:8} we let $r \to 0$ and we read off the
  relations
  $$
  \begin{aligned}
    &c\log r \text{ is bounded},\\
    &a_m r^m + \overline{a_{-m}} r^{-m}\text{ is bounded, }m\ne 0.
  \end{aligned}
  $$
  Therefore, $c = 0$ and $m < 0$ implies $a_m = 0$.  Thus the
  expansion for $u$ reads
  $$
  u(z) = \Re(\sum_{n=0}^{\infty} a_n z^n),\quad 0 < |z| < b.
  $$
  The right side of this expression is harmonic in the disk $|z| <
  b$.
\end{myproof}

\begin{cor}
  \label{cor:p211b}
  If $u$ is harmonic and bounded for $a < |z| < \infty$ and continuous
  for $a \le |z| < \infty$, and if $u(z) \equiv 0$ for $|z| = a$, then
  $u \equiv 0$.
\end{cor}

%%% page 212
\begin{myproof}
  In formula \eqref{e:8} let $r \to a$ to obtain $c \log a + a_0 = 0$,
  $a_m a^m + \overline{a_{-m}} a^{-m} = 0$.  By the reasoning given in
  the previous corollary, $a_m = 0$ for $m > 0$ and $c = 0$.
  Therefore, $a_0 = 0$ and $a_m = 0$ for $m < 0$.  Thus, $u \equiv
  0$.
\end{myproof}

We are now almost ready to prove Theorem~\ref{thm:4}.  But something
rather strange will arise in the proof.  Namely, we shall see that
there is a certain dichotomy of Riemann surfaces which requires that
the proof of Theorem~\ref{thm:4} be quite dependent on this
classification, although the statement of the theorem is the same in
both cases.  We present this phenomenon in the form of a proposition.

\begin{propn}
  \label{propn:15}
  The following conditions on a connected Riemann surface $S$ are
  equivalent.
  \begin{enumerate}
  \item Every bounded subharmonic function on $S$ is constant.
  \item If $D$ is any analytic disk and $u$ is a bounded continuous
    nonnegative function in $S - D$ which is harmonic in $S - D^-$ and
    which vanishes identically on $\p D$, then $u \equiv 0$.
  %%% page 213
  \item If $D$ is any analytic disk and $u$ is a bounded continuous
    function in $S - D$ which is harmonic in $S-D^-$, then
    $$
    \sup_{S-D} u = \sup_{\p D} u.
    $$
  \item Same as 3. with ``harmonic'' replace by ``subharmonic''.
  \item Condition 2. holds for \emph{some} analytic disk $D$.
  \item Condition 3. holds for \emph{some} analytic disk $D$.
  \item Condition 4. holds for \emph{some} analytic disk $D$.
  \end{enumerate}
\end{propn}

\begin{mdframed}
  \textbf{Proof}:
  We shall prove $1 \Rightarrow 2 \Rightarrow 3 \Rightarrow 4$ and $5
  \Rightarrow 6 \Rightarrow 7 \Rightarrow 1$.  Since the assertions $2
  \Rightarrow 5$, $3 \Rightarrow 6$ and $4\Rightarrow 7$ are trivial,
  the proposition will follow.  The proof that $2 \Rightarrow 3$ is
  identical to the proof that $5 \Rightarrow 6$ and likewise for $3
  \Rightarrow 4$ and $6\Rightarrow 7$.

  $\boxed{3 \Rightarrow 4}$: As in the proof of
  Proposition~\ref{propn:13}, we choose a ``concentric'' analytic disk
  $D_1$ with $D^- \subset D_1$.  Suppose $u$ is a bounded continuous
  function in $S - D$ which is subharmonic in $S - D^-$.  Choose a
  constant $C$ such that $\sup_{S-D} u \le C$.  Let $w$ be the unique
  function which is continuous in $S-D$ harmonic in $D_1 - D^-$, such
  that
  %%% page 214
  $$
  w =
  \begin{cases}
    u &\text{on }\p D,\\
    C &\text{in }S-D_1.
  \end{cases}
  $$
  Then $w$ is superharmonic in $S-D^-$ and criterion 2 of
  Proposition~\ref{propn:10} implies $u \le w$ in $D_1^- - D$ and
  therefore $u \le w$ in $S-D$.  Let $v = $GHM of $w$.  Then $v$ is
  harmonic in $S-D^-$ and $u \le v \le w$.  Therefore, we can
  naturally extend $v$ to be continuous in $S-D$ by setting $v = u =
  w$ on $\p D$.  Condition 3 applies to $v$ and thus
  $$
  \sup_{S-D} u \le \sup_{S-D} v = \sup_{\p D} v - \sup_{\p D} u.
  $$

  $\boxed{7\Rightarrow 1}$: Suppose $u$ is a bounded subharmonic
  function on $S$.  Let $D$ be an analytic disk on $S$ for which
  condition 4 holds.  Then
  $$
  \sup_{S-D} u = \sup_{\p D} u.
  $$
  Therefore,
  $$
  \sup_{S} u = \sup_{D^-} u,
  $$
  and since $D^-$ is compact, we see that $u$ assumes its maximum.  By
  the strong maximum principle, $u \equiv$ constant.

  $\boxed{1 \Rightarrow 2}$: Let $u$ be the function in the hypothesis
  of 2.  Define $v$ on $S$ by
  $$
  v =
  \begin{cases}
    u &\text{in }S-D,\\
    0 &\text{in }D.
  \end{cases}
  $$
  %%% page 215
  Then $v$ is subharmonic and bounded on $S$, so 1 implies $v \equiv$
  constant.  Thus, $v \equiv 0$ and it follows that $u \equiv 0$.

  $\boxed{2\Rightarrow 3}$: Let $u$ be the function in the hypothesis
  of 3.  Define
  $$
  A = \inf_{S-D} u,\quad B = \sup_{S-D} u,\quad C = \sup_{\p D} u.
  $$
  Then $A \le C \le B$ and we want to prove $B = C$.  As in the proof
  of Proposition~\ref{propn:13} and also the current proof that $3
  \Rightarrow 4$ we take a disk $D_1$ and define $v_A$ to be
  continuous in $S-D$, harmonic in $D_1 - D^-$, such that
  $$
  v_A =
  \begin{cases}
    u &\text{on }\p D,\\
    A &\text{in }\p S-D_1.
  \end{cases}
  $$
  We define $v_B$ the same way with $A$ replaced by $B$.  Then $v_A$
  is subharmonic, $v_b$ is superharmonic in $S-D^-$, and $v_A \le C$,
  $v_A \le u$, $u \le v_B$ all of which follow from the maximum
  principle.  Let
  $$
  \begin{aligned}
    w_1 &= \text{GHM of min }(u,C),\\
    w_2 &= \text{GHM of }v_B.
  \end{aligned}
  $$
  Note that $\min(u,C)$ is superharmonic by
  Proposition~\ref{propn:12}.  Then the inequalities we have obtained
  for $v_A$ and $v_B$ show
  $$
  v_A \le w_1 \le u \le w_2 \le v_B
  $$
  Thus, $w_1$ and $w_2$ have continuous extensions to $S-D$ with $w_1
  = w_2 = u$ on $\p D$.  Therefore, $w_2 - w_1$ satisfies the
  hypothesis of 2, the required boundedness following from $w_2 - w_2
  \le v_B - v_A \le B-A$.  Thus, condition 2 implies $w_2 - w_1 \equiv
  0$.  Thus
  $$
  B = \sup_{S-D} u = \sup_{S-D} w_1 \le C.
  $$
\end{mdframed}

%%% page 216
\begin{defn}
  A noncompact connected Riemann surface satisfying the conditions of
  Proposition~\ref{propn:15} is a \emph{parabolic}\todo{Why the term
    parabolic???}\ Riemann surface.  A noncompact connected Riemann
  surface not satisfying condition is \emph{hyperbolic}.
\end{defn}

\begin{exmp}
  \label{eg:1}
  If $S$ is compact and connected, $S$ satisfies the conditions of
  Proposition~\ref{propn:15}.  For suppose $u$ is a bounded
  subharmonic function on $S$.  Then $u$ assumes its maximum, so the
  strong maximum principle implies $u$ is constant.
\end{exmp}

\begin{exmp}
  \label{eg:2}
  $\CCC$ is parabolic.  We verify conditions 5 for $D = \{z ~:~ |z| <
  1\}$.  Suppose $u$ is a function satisfying the hypothesis of
  Proposition~\ref{propn:15}, criterion 5.  By the
  corollary~\ref{cor:p211b}, $u \equiv 0$.
\end{exmp}

\begin{exmp}
  \label{eg:3}
  If $S = \{z ~:~ |z| < 1\}$ has its usual complete analytic atlas,
  then $S$ is hyperbolic.  This is obvious, a nonconstant subharmonic
  function which is bounded on $S$ being, for example, $z \mapsto
  \Re(z)$.
\end{exmp}

%%% page 217
Finally, the stage is set for the proof of Theorem~\ref{thm:4}.  In
the statement of the theorem, there is a given chart $\varphi : U \to
W$ in the complete analytic atlas for $S$, where $U$ contains the
given point $p$ and $\varphi(p) = 0$.  By a simple change of variable,
we can assume taht
$$
\{ z ~:~ |z| \le 2 \} \subset W.
$$
Let $D_r = \varphi^{-1}(\Delta_r(0))$ for $0 < r \le 2$.

\textbf{Proof of Theorem~\ref{thm:4} in case $S$ does not satisfy the
  conditions of Proposition~\ref{propn:15}}:  By criterion 5, there
exists a bounded continuous nonnegative function $v$ in $S - D_1$
which is harmonic in $S - D_1^-$ and which is identically zero on $\p
D_1$, and yet $v \ne 0$.  The strong maximum principle implies that $v
> 0$ in $S-D_1^-$.  The function $\varphi$ is holomorphic on $U$, and 
therefore $\Re(\varphi^{-n} - \varphi^n)$ is harmonic on $U-\{p\}$ and
in particular is harmonic on $D_2$.  On $\p D_1$, $|\varphi| = 1$ so
that $\varphi^{-n} = \overline{\varphi}^n$ and thus $\varphi^{-n} -
\varphi^n$ is purely imaginary and thus $\Re(\varphi^{-n} - \varphi^n)
= 0$.  Since $v > 0$ on the compact set $\p D_2$, it is bounded below
by a positive constant there.  Therefore, there exists a constant $C$
such that
%%% pag 218
$$
|\Re(\varphi^{-n} - \varphi^n)| \le C v \quad \text{on}\quad \p D_2.
$$
The same inequality holds trivially on $\p D_1$, both sides vanishing,
and therefore the weak maximum principle implies
\begin{equation}
  \label{e:9}
  |\Re(\varphi^{-n} - \varphi^n)| \le Cv \quad\text{in}\quad D_2^- - D_1.
\end{equation}
Now we define
$$
\begin{aligned}
  w_1 &=
  \begin{cases}
    -Cv &\text{in }S-D_1,\\
    \Re(\varphi^{-n} - \varphi^n) &\text{in }D_1-\{p\},
  \end{cases}\\
  w_2 &=
  \begin{cases}
    Cv &\text{in }S-D_1,\\
    \Re(\varphi^{-n} - \varphi^n) &\text{in }D_1-\{p\},
  \end{cases}  
\end{aligned}
$$
Then $w_1$ and $w_2$ are clearly continuous on $S-\{p\}$ and
\eqref{e:9} implies $w_1$ is superharmonic and $w_2$ is subharmonic in
$S - \{p\}$: it suffices to check the mean value property of
item 3 of Proposition~\ref{propn:10} at points on $\p D_1$ and at such
a point $w_1 = 0 = \Re(\varphi^{-n} - \varphi^n) = $ the mean value of
$\Re(\varphi^{-n} - \varphi^n)$ on small circles centred at the point
$\ge$ the corresponding mean value of $w_1$ (since $\Re(\varphi^{-n} -
\varphi^n) \ge - Cv$ on the part of the circle lying outside $D_1$).
Thus, $w_1$ is superharmonic,and a similar proof $w_2$ is
subharmonic.

Choose a constant $A \ge 2C \sup_{S-D_1} v$.  Let $u = $GHM of $w_1 +
A$.  Note that trivially $w_2 - w_2 \le A$, and therefore
%%% page 219
$$
w_2 \le u \le w_1 + A.
$$
We have of course used here the existence of GHM on the Riemann
surface $S - \{p\}$.  Therefore, $u$ is harmonic on $S - \{p\}$ and
our inequalities show
$$
0 \le u - \Re(\varphi^{-n} - \varphi^n) \le A \quad\text{in}\quad
D_1 - \{p\}.
$$
Therefore, the function $u - \Re(\varphi^{-n} - \varphi^n)$ is
harmonic and \emph{bounded} in $D_1 - \{p\}$, so the removable
singularity theorem shows that there is a harmonic function $h$ in
$D_1$ such that
$$
u = \Re(\varphi^{-n}) + h\quad\text{in}\quad D_1.
$$
Since $h = \Re(F)$ for some holomorphic function $F$ in $D_1$
(Proposition~\ref{propn:1} item 4), we have proved Theorem~\ref{thm:4}
in this case, and we can even assert that no term $\log|\varphi|$
appears in the representation for $u$.

\textbf{Proof of Theorem~\ref{thm:4} in case $S$ does satisfy the
  conditions of Proposition~\ref{propn:15}}:  By
Proposition~\ref{propn:13}, there exists for $0 < r < 1$ a bounded
continuous function $u_r$ in $S-D_r$ such that $u_r =
\Re(\varphi^{-n})$ on $\p D_r$ and $u_r$ is harmonic in $S - D_r^-$.
If $u_r$ is constant on $\p D_1$, then Proposition~\ref{propn:15} item
3 implies $u_r$ is constant in $S-D_1$ (apply the criterion 3 both to
$u_r$ and $-u_r$) and thus $u_r$ is constant in $S-D_r$ by
Proposition~\ref{propn:3}, which is not true.  Thus $u_r$ is not
constant on $\p D_1$, and therefore there exist unique coefficients
$\alpha_r$ and $\beta_r$ such that if $v_r = \alpha_r u_r + \beta_r$,
then 
%%% page 220
$$
\max_{\p D_1} v_r = 1,\quad \min_{\p D_1} v_r = 0.
$$
Proposition~\ref{propn:15} item 3 implies $0 \le v_r \le 1$ in $S -
D_1$.

By Proposition~\ref{propn:8} there exists a sequence $r_k \to 0$ such
that $v_{r_k}$ converges uniformly on compact subsets of $D_2 -
D_1^-$.  Moreover, Proposition~\ref{propn:15} item 3 applied to
$D_{3/2}$ implies that $v_{r_k}$ converges uniformly on $S -
D_{3/2}$.  If $v = \lim_{k\to\infty} v_{r_k}$, then
Proposition~\ref{propn:7} implies $v$ is harmonic in $S-D_1^-$.

We now write down the Laurent expansion of Proposition~\ref{propn:14}
for $v_r$ in the set $D_2 - D_r^-$:
$$
v_r \circ \varphi^{-1}(z) = c(r) \log|z| +
\Re(\sum_{n=-\infty}^{\infty} a_j(r) z^j),\quad
r < |z| < 2,
$$
where $c(r)$ and $a_0(r)$ are real, and formula \eqref{e:8} shows
\begin{equation}
  \label{e:10}
  \begin{aligned}
    c(r)\log s  + a_0(r) &= \frac{1}{2\pi} \int_0^{2\pi} v_r \circ
    \varphi^{-1}(s e^{i\theta}) d\theta,\\
    a_j(r) s^j + \overline{a_{-j}(r)} s^{-j}
    &= \frac{1}{\pi} \int_0^{2\pi} v_r \circ \varphi^{-1}(s
    e^{i\theta}) e^{-ij\theta} d \theta,\quad j\ne 0,
  \end{aligned}
\end{equation}
for $r < s < 2$.

%%% page 221
Now 
$$
v_r \circ \varphi^{-1}(re^{i\theta}) 
= \alpha_r \Re(r^{-n} e^{-in\theta}) + \beta_r 
= \alpha_r r^{-n} \frac{e^{in\theta} + e^{-in\theta}}{2} + \beta_r 
$$
and therefore if we let $s \to r$ in the second part of \eqref{e:10}
we obtain
$$
a_j(r) r^j + \overline{a_{-j}(r)} r^{-j} = 0
\quad\text{if}\quad j \ge 1,\ j \ne n.
$$
Taking $s=1$ in \eqref{e:10},
$$
|a_j(r) + \overline{a_{-j}(r)}| \le 2 \quad\text{if}\quad j\ge 1.
$$
Therefore, for $j \ge 1$ and $j \ne n$,
$$
2 \ge |a_j(r) + \overline{s_{-j}(r)}|
= |a_j(r) - a_j(r)r^{2j}|
= |a_j(r)|(1-r^{2j}) \ge \frac{1}{2}|a_j(r)|
\text{ if } 0 < r < \frac{1}{2};
$$
thus $|a_j(r)| \le 4$ and therefore
$$
|a_{-j}(r)| \le 4r^{2j}.
$$

Now the estimates in Proposition~\ref{propn:14} imply that as $r_k \to
0$, the coefficients in the Laurent expansion for $v_{r_k}$ converge
to the coefficients in the Laurent expansion for $v$.  Therefore,
$$
v\circ \varphi^{-1}(z) 
= c \log|z| + \Re(a_{-n}z^{-n} + \sum_{j=0}^{\infty} a_jz^j),\quad
1 < |z| < 2,
$$
%%% page 222
since $a_{-j}(r_k) \to 0$ for $j \ge 1$, $j \ne n$.  Now we define $u$
on $S-\{p\}$ by
$$
\begin{aligned}
  u &= v \quad\text{in }S-D_1^-,\\
  u \circ \varphi^{-1}(z) 
  &= c\log|z| + \Re(a_{-n}z^{-n} + \sum_{j=0}^{\infty} a_jz^j),\quad
  1 < |z| < 2.
\end{aligned}
$$
It is clear that $u$ is harmonic on $S-\{p\}$.  The theorem will be
proved once we establish that $a_{-n} \ne 0$.  This involves a rather
delicate argument.

Suppose that $a_{-n} = 0$.  Then the formula for $u$ near $p$ shows
that there exists
$$
\ell = \lim_{q\to p} u(q),
$$
where $-\infty \le \ell \le \infty$.  By Proposition~\ref{propn:15}
item 3 applied to $u$ and $-u$ and small analytic disks containing
$p$, we conclude that $u = \ell$ on $S - \{p\}$; therefore, $-\infty <
\ell < \infty$.  Now we shall prove that $v_{r_k} \to u$ uniformly on
$\p D_1$, and therefore that $\max_{\p D_1} u = 1$, $\min_{\p D_1} u =
0$, contradicting the fact that $u$ is constant.  Again,
Proposition~\ref{propn:15} item 3 shows that it is sufficient to prove
that $v_{r_k} \to u$ uniformly on $\p D_{1/2}$.  For $|z| =
\frac{1}{2}$ and $0 < r < \frac{1}{2}$,
$$
\begin{aligned}
  |v_r \circ \varphi^{-1}(z) - u \circ \varphi^{-1}(z)|
  &\le |c(r) - c| \log 2 + |a_{-n}(r) - a_{-n}| 2^n\\
  &\qquad + \sum_{j=0}^{N} |a_j(r) - a_j| 
  + \sum_{j=N+1}^{\infty} (4 + 4) 2^{-j} 
  + \sum_{\substack{j<0\\j\ne -n}} 4r^{-2j} 2^{-j} ???\\
  &\le C_n\left[
    |c(r) - r| + |a_{-n}(r) - a_{-n}| 
    + \sum_{j=0}^{N} |a_j(r) - a_j|
  \right] + 8\cdot 2^{-N} + \frac{8r^2}{1-2r^2}.
\end{aligned}
$$
%%% page 223
Therefore, if $\epsilon > 0$ we can choose a fixed $N$ such that $8
\cdot 2^{-N} < \frac{\epsilon}{3}$ and a $k_0$ such that
$\frac{8r_k^2}{1-2r_k^2} < \frac{\epsilon}{3}$ and $r_k < \frac{1}{2}$
for $k \ge k_0$.  Then we choose $k_1 \ge k_1$ such that
$$
C_n\left[
  |c(r_k) - r| + |a_{-n}(r_k) - a_{-n}| 
  + \sum_{j=0}^{N} |a_j(r_k) - a_j|
\right] < \frac{\epsilon}{3}
$$
if $k \ge k_1$.  Therefore, if $k \ge k_1$, 
$$
|v_r - u| < \epsilon \quad\text{on}\quad \p D_{1/2}.
$$

Thus, we have completed the proof of Theorem~\ref{thm:4}.  We have
already indicated the use of this theorem in establishing the
existence of meromorphic functions, in the proof of
Theorem~\ref{thm:3}.  In the next topic, we shall give further
applications.

\begin{rem}
  In the above proof in the second case, the second part of
  \eqref{e:10} in the case $j=n$ was not used.  But we get additional
  information by using this formula:
  %%% page 224
  $$
  a_n(r) s^n + \overline{a_{-n}(r)} s^{-n}
  = \frac{1}{\pi} \int_0^{2\pi} v_r \circ \varphi^{-1}(s e^{i\theta})
  e^{-in\theta} d\theta.
  $$
  Letting $s = r$, we obtain
  $$
  a_n(r) r^n + \overline{a_{-n}(r)} r^{-n}
  = \alpha_r r^{-n} \frac{1}{2\pi} \int_0^{2\pi} (e^{in\theta} 
  + e^{-in\theta}) e^{-in\theta} d\theta 
  + \beta_r \frac{1}{\pi} \int_0^{2\pi} e^{-in\theta} d \theta
  = \alpha_r r^{-n}.
  $$
  We already know from $s = 1$ that
  $$
  |a_n(r) + \overline{a_{-n}(r)}| \le 2.
  $$
  Therefore,
  $$
  |\alpha_r - \overline{a_{-n}(r)} 
  + \overline{a_{-n}(r)}r^{2n}| \le 2r^{2n}.
  $$
  Taking imaginary parts, we conclude
  $$
  |\Im(a_{-n}(r)| (1 - r^{2n}) \le 2r^{2n},
  $$
  showing that $\Im(a_{-n}(r)) \to 0$ as $r \to 0$.  Therefore, letting
  $r = r_k$, we have
  $$
  a_{-n} = \lim_{k\to \infty} a_{-n}(r_k) \text{ is \emph{real}}.
  $$
\end{rem}

We obtain from this remark the following result:

\begin{cor}[Corollary to Theorem~\ref{thm:4}]
  Let $S$ be any connected Riemann surface and let $p \in S$.  Let
  $\varphi: U \to W$ be a chart in the complete analytic atlas for $S$
  with $p \in U$ and $\varphi(p) = 0$.
  %%% page 225
  Let $n$ be a positive integer.  Then there exists a harmonic
  function $u$ on $S - \{p\}$ such that for $z$ near 0
  $$
  u \circ \varphi^{-1}(z)
  = \Re(\frac{\alpha}{z^n}) + \Re(f(z)),
  $$
  where $\alpha$ is a nonzero complex number and $f$ is holomorphic in
  a neighbourhood of 0.  Moreover, $u$ can be taken to be bounded
  outside a neighbourhood of $p$.
\end{cor}

\begin{mdframed}
  \textbf{Proof}: If $S$ is hyperbolic, the proof of
  Theorem~\ref{thm:4} already contained this result with $\alpha =
  1$.  The boundedness of $u$ away from $p$ has also been shown in
  this case.

  If $S$ is compact or parabolic, the assertion about the boundedness
  of $u$ is automatic.  What we must do is eliminate the term
  involving $\log|z|$.  By the previous remark, we have obtained a
  harmonic function $v$ on $S - \{p\}$ such that near 0
  $$
  v \circ \varphi^{-1}(z) 
  = c \log|z| + \Re(\frac{a}{z^n}) + \Re(g(z)),
  $$
  where $g$ is holomorphic near 0, $c$ is real, and $a \ne 0$ is
  real.  If $c=0$, we are through.  Otherwise, we replace $\varphi$ by
  the chart $w\varphi$, where $w$ is a fixed complex number with $w^n
  = i$.  Applying our result in this case, we obtain a harmonic
  function $w$ on $S - \{p\}$ such that near 0
  $$
  w \circ \varphi^{-1}(z) = d \log|z| + \Re(\frac{b}{iz^n})
  + \Re(h(z))
  $$
  where $h$ is holomorphic near 0, $d$ is real, and $b \ne 0$ is
  real.  We have left out a trivial intermediate calculation here.
  %%% page 226
  Now define
  $$
  u = w - \frac{d}{c}v.
  $$
  Then $u$ is harmonic on $S - \{p\}$, and near 0
  $$
  u \circ \varphi^{-1}(z) = \Re(\frac{\alpha}{z^n} + \Re(h(z) -
  \frac{d}{c}g(z)),
  $$
  where
  $$
  \alpha = \frac{b}{i} - \frac{da}{c} \ne 0.
  $$
\end{mdframed}

Also in the next section we shall require the existence of a Green's
function on a parabolic Riemann surface.

\begin{defn}
  \label{def:7}
  Let $S$ be a connected Riemann surface and $p \in S$.  A function
  $g$ defined on $S-\{p\}$ is a \emph{Green's function} if
  \begin{enumerate}
  \item $g$ is positive and harmonic in $S - \{p\}$;
  \item if $\varphi$ is an analytic chart near $p$ with $\varphi(p) =
    0$, then $g + \log|\varphi|$ is harmonic in a neighbourhood of
    $p$;
  \item if $h$ has properties 1 and 2, then $g \le h$.
  \end{enumerate}
\end{defn}

We first remark that condition 2 is independent of the particular
chart $\varphi$, since any other analytic chart $\psi$ can be
expressed as
$$
\psi = a\varphi(1 + \sum_{k=1}^{\infty} \alpha_k \varphi^k)
$$
and so
$$
\log|\psi| = \log|\varphi|
+ \log|a| + \log\left| 1 + \sum_{k=1}^{\infty} \alpha_k \varphi^k \right|
$$
and we see that $\log|\psi| - \log|\varphi|$ is harmonic in a
neighbourhood of $p$.

\begin{propn}
  \label{propn:16}
  Let $S$ be a connected hyperbolic Riemann surface and $p \in S$.
  Then there exists a unique Green's function on $S - \{p\}$.
\end{propn}

\begin{mdframed}
  \textbf{Proof}:  Uniqueness is clear by property 3 of a Green's
  function.  The proof of existence is like the proof of
  Theorem~\ref{thm:4} in the hyperbolic case.  Thus, $v$ is bounded
  continuous function on $S - D_1$, $v > 0$ and $v$ is harmonic in
  $S-D_1^-$, and $v \equiv 0$ on $\p D_1$.  As before, there exists a
  constant $C > 0$ such that
  $$
  \log|\varphi| \le C v \text{ in } D_2^- - D_1.
  $$
  As before, the function
  $$
  w_1 =
  \begin{cases}
    -Cv \text{ in } S-D_1,\\
    -\log|\varphi| \text{ in }D_1,
  \end{cases}
  $$
  is superharmonic in $S - \{p\}$.  Much more trivially, the function
  %%% page 228
  $$
  w_2 =
  \begin{cases}
    0 & \text{in } S-D-1,\\
    -\log|\varphi| & \text{in } D-1,
  \end{cases}
  $$
  is subharmonic in $S-\{p\}$.  Let $g$ be the \emph{least harmonic
    majorant} of $w_2$.  As in Definition~\ref{def:5}, $g$ is harmonic
  on $S - \{p\}$ and if $A \ge C \sup_{S-D_1} v$, then $w_2 \le w_1 +
  A$, so that
  $$
  w_2 \le g \le w_1 + A.
  $$
  By the removable singularity theorem, $g + \log|\varphi|$ is
  harmonic in $D_1$, so property 2 follows.  Also since $w_2 \ge 0$,
  also $g \ge 0$ and since $g$ is not constant, the strong maximum
  principle implies property 1.  To check property 3, suppose $h$ has
  properties 1 and 2.  Then $h + \log|\varphi|$ is harmonic in $D_1$
  and is positive on $\p D_1$, so the minimum principle for harmonic
  functions implies $h + \log|\varphi| > 0$ in $D_1$.  Therefore, $h >
  w_2$ on $S - \{p\}$, and the definition of $g$ therefore implies $g
  \le h$.
\end{mdframed}

\begin{rem}
  We can prove even more.  Namely, if $h$ is positive and
  superharmonic on $S - \{p\}$ and  $h + \log|\varphi|$ is
  superharmonic near $p$, then $g \le h$.  It's exact the same proof.
\end{rem}

\begin{ques}
  \label{ques:9}
  Find the Green's function for the unit disk $\{z ~:~ |z| < 1\}$.
\end{ques}

\end{document}


%%% Local Variables: 
%%% mode: latex
%%% TeX-master: t
%%% End: 
