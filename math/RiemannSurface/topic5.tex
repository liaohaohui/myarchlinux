\documentclass[a4paper,11pt]{article}
\usepackage{amsmath}
\usepackage{amsfonts}
\usepackage{verbatim}
\usepackage{url}
\usepackage{framed}
\usepackage{epsfig}
\usepackage{enumerate}
\usepackage{textcomp} %\textquotesingle
\usepackage[
%sorting=nyt,
firstinits=true, % render first and middle names as initials
useprefix=true,
maxcitenames=3,
maxbibnames=99,
style=authoryear,
dashed=false, % re-print recurring author names in bibliography
natbib=true,
url=false
]{biblatex}
%%% http://tex.stackexchange.com/questions/12254/biblatex-how-to-remove-the-parentheses-around-the-year-in-authoryear-style
\usepackage{xpatch}
\addbibresource{complex.bib} % run: biber topic1
\usepackage{color}
\usepackage{listings}
\definecolor{gray}{gray}{0.5} 
\definecolor{key}{rgb}{0,0.5,0} 
\lstset{ 
  language=[90]Fortran,
  basicstyle=\ttfamily\small, 
  keywordstyle=\color{blue}, 
  stringstyle=\color{red}, 
  showstringspaces=false, 
  emphstyle=\color{black}\bfseries, 
  emph={[2]True, False, None, self}, 
  emphstyle=[2]\color{key}, 
  emph={[3]from, import, as},
  emphstyle=[3]\color{blue}, 
  upquote=true, 
  morecomment=[s]{"""}{"""}, 
  commentstyle=\color{gray}\slshape, 
  %framexleftmargin=1mm, framextopmargin=1mm, frame=shadowbox, 
  rulesepcolor=\color{blue},
  numbers=left,
  stepnumber=1,
}
\usepackage{enumerate}
\usepackage{tikz}
\usetikzlibrary{lindenmayersystems}
\usetikzlibrary[shadings]


%%% Page Layout
\oddsidemargin=0truecm
\evensidemargin=0truecm
\textwidth=160truemm
\textheight=260truemm
\leftmargin=0truemm
\rightmargin=0truemm
\voffset=-23truemm
\topmargin=0truemm

\newif\iflecturer
%\lecturerfalse
\lecturertrue

\iflecturer
\usepackage{marginnote} % \marginpar
%\usepackage[color]{showkeys}
\definecolor{refkey}{rgb}{1,0,0}
\definecolor{labelkey}{rgb}{1,0,0}
\else
\def\marginpar#1{}
\fi

\iflecturer
\newcommand{\Answer}[1]{\dotfill\underline{\mbox{\hspace{1em}\color{blue}#1}}}
\newcommand{\BoxAns}[1]{\fbox{\color{blue}#1}}
\newcommand{\Reason}[1]{{\par\color{blue}\par{}Reason: #1}}
\else
\newcommand{\Answer}[1]{\dotfill\underline{\mbox{\hspace{1em}\color{white}#1}}}
\newcommand{\BoxAns}[1]{\fbox{\color{white}#1}}
\newcommand{\Reason}[1]{{\par\color{white}\par{}Reason: #1}}
\fi

\newif\ifamsstyle
\amsstylefalse
\newcounter{topic}
\input common
\input symbol
\makeatletter
\newcommand{\Arctan}{{\mathop{\operator@font Arctan}\nolimits}}
\makeatother

%\parindent=0pt
\parskip=1pt
\linespread{1}


\setcounter{topic}{5}

\begin{document}

\title{{\sc Rudiments of Riemann Surfaces\\
    Topic \thetopic{}: Algebraic Functions}}
\author{Author: B. Frank Jones, Jr. (Rice Univ. 1971)\\
Seminar: Dr Liew How Hui (\url{liewhh@utar.edu.my})}
\date{}

\maketitle

%%% E.page 118
What we are going to study is solutions of an algebraic equation in
two complex variables, i.e., equations of the form
$$
A(z,w) = 0
$$
where $A$ is a polynomial in $z$ and $w$. The viewpoint is that we
want to regard $w$ as a function of $z$ satisfying $A(z,w(z)) = 0$.
Of course, we expect $w$ to be multiple-valued and then we construct a
Riemann surface on which a function like $w$ can be defined.  Examples
of this procedure were given in Topic 1.  There we treated the
following examples of $A$:
$$
\begin{aligned}
  &w^m - z,\\
  &w^2 - (z-a)(z-b)\\
  &(z-b)w^2 - (z-a),\\
  &w^2 - (z-a_1)(z-a_2)\cdots (z-a_m),
\end{aligned}
$$
and in Topic 3, we discussed $w^3-3w-z$.  All the Riemann surfaces
associated with these examples can be easily visualised as subsets of
$\overline{M}$ and as such enjoy the topological property of
\emph{compactness}.  The main fact 
%%% page 119
to come out of this topic is that algebraic equations always lead to
compact surfaces and that, conversely, every compact analytic
configuration has a unique algebraic equation associated with it.

It follows from general topological considerations that every compact
orientable surface (as Riemann surfaces are) is homeomorphic to a
sphere with a certain number $g$ handles and $g$ is called the
\emph{genus} of the surface.  Before analysing algebraic equations, we
shall discuss heuristically a remarkable formula involving the genus,
the number of sheets, and the branching of a compact Riemann surface.

\begin{mdframed}
\centering
\bf Riemann-Hurwitz Formula
\end{mdframed}

Consider a compact analytic configuration $S$.  We first discuss its
\emph{Euler characteristic}.  This can be defined in terms of a
``triangulation'' of $S$.  We do not wish to pause to define
triangulation, but if $f$ is the number of triangles (faces), $e$ the
number of edges, and $v$ the number of vertices, then the Euler
characteristic is $v-e+f$.  A theorem of topology is that this number
is a topological invariant of the surface and equals $2-2g$:
$$
v - e + f = 2-2g.
$$
Now $S$ has certain branch points $e_1, \cdots, e_\ell$ of orders
$b_1, \cdots, b_\ell$, respectively, $b_j \ge 1$.  Define
%%% page 120
$$
V = \sum_{j=1}^{\ell} b_j.
$$
The number $V$ is called the \emph{ramification index} or \emph{total
  branching order} of $S$.  Also $S$ has a certain number $n$ of
\emph{sheets} when viewed as spread over $\widehat{\CCC}$; this is the
number such that $\pi$ takes every value in $\widehat{\CCC}$ $n$ times
... see the proof of Proposition 2.24 in Topic 2.  The Riemann-Hurwitz 
formula is
$$
\boxed{
\frac{V}{2} = n + g - 1.
}
$$
To prove this formula consider a triangulation of the sphere
$\widehat{\CCC}$ such that every point $\pi(e_j)$ is a vertex.  Let
$f$, $e$, and $v$ be the number of faces, edges, and vertices.  Since
$\widehat{\CCC}$ has genus 0, we have the Euler formula
$$
v - e + f = 2.
$$
Now consider the pre-image by $\pi$ of these triangles.  By lifting
the triangulation of $\widehat{\CCC}$ to $S$ we obtain $nf$ faces and
$ne$ edges in the triangulation of $S$, since $S$ has $n$ sheets.  And
each vertex which is not a $\pi(e_j)$ is lifted to $n$ new
vertices.  But each vertex $\pi(e_j)$ does not get lifted to $n$ new
vertices.  Rather, if $z_0$ is one of these values, then
$\pi^{-1}(\{z_0\})$ consists of exactly
%%% page 121
$$
n = \sum_{\pi(e_j)=z_0} b_j
$$
distinct points.  Thus, the number of vertices in the triangulation of
$S$ is 
$$
nv = V.
$$
Therefore,
$$
(nv-V) - ne + nf = 2-2g.
$$
Since $v-e+f = 2$ we can write this relation as
$$
2n - V = 2-2g,
$$
and the assertion is proved.

Let us test this formula on some of the cases we have considered.  For
example, we have treated
$$
w^2 = (z-a_1) \cdots (z-a_m),
$$
$a_1, \cdots, a_m$ distinct in Topic 1.  If $m$ is even there are
branch points of order 1 at each $a_j$ and nowhere else, so $V = m$
and thus
$$
\frac{m}{2} = n + g - 1 = 2 + g - 1
\Rightarrow g = \frac{m-2}{2}.
$$
If $m$ is odd the also $\infty$ is a branch point of order 1 and so $V
= m+1$ and $\ds g = \frac{m-1}{2}$.

%%% page 122
Next, consider the example $w^2 - 3w - z$ discussed in Topic 3.
There the points 2 and $-2$ are branch points of order 2, so $V = 4$.
Since $n = 3$, we find $g = 0$ and the Riemann surface is homeomorphic
to a sphere.  This again agrees with our earlier findings in Topic
3, we discussed an analytic equivalence of the surface with
$\widehat{\CCC}$.

Of course, we have not rigorously derived the formula, but we have
given a sketch of a rigorous proof.  But the formula should prove
useful as a check in working out other examples.  Every time one sees
a compact Riemann surface, he should try out this formula on it.  Two
things in the formula deserve special attention.  One is that $V$ is
always an even integer.  The other is that a purely topological number
$g$ is equal to the number $\frac{V}{2} - n + 1$ which depends very
much on features of the surface which are not purely topological.

\medskip
\begin{mdframed}
Now we proceed to the analytical discussion of algebraic equations.
\end{mdframed}

\begin{ques}
  \label{prob:5}
  In the spirit of Topic 3, discuss the algebraic equation $(w^2 -
  1)^2 - z = 0$.
\end{ques}

\begin{lem}
  \label{lem:1}
  Let $a_1, \cdots, a_n$ be holomorphic on an open set $D \subset
  \widehat{\CCC}$ and $A(z,w) = w^n + a_1(z) w^{n-1} + \cdots +
  a_{n-1}(z) w + a_n(z)$.
  %%% page 123
  Suppose $z_0 \in D$ and
  $$
  A(z_0, w_0) = 0,\quad \frac{\p A}{\p w}(z_0, w_0) \ne 0.
  $$
  Then there exists a function $f$ holomorphic in a neighbourhood
  $z_0$ such that  
  $$
  \begin{aligned}
    A(z,f(z)) &= 0,\quad z\text{ near }z_0,\\
    f(z_0) &= w_0,\\
    A(z,w) &= 0,\ z\text{ near }z_0,\ w\text{ near }w_0 = w = f(z).
  \end{aligned}
  $$
\end{lem}

\begin{myproof}
  This is merely an implicit function theorem and could be derived
  from the general implicit function theorem of differential calculus
  --- we would just have to check the validity of the Cauchy-Riemann
  equation.  However, the proof is much simpler in the present case
  than the proof of the general theorem and is even almost elegant, so
  we present it.

  Since $A$ is not constant in $w$, the zeros of $A(z_0, w)$ are
  isolated.  Thus, there exists, $\epsilon > 0$ such that $A(z_0,w)
  \ne 0$ for $0 < |w-w_0| \le \epsilon$.  Let $\gamma$ be the path
  $\gamma(t) = w_0 + \epsilon e^{2\pi i t}$, $0 \le t \le 1$.  Since
  the image of $\gamma$ is compact and $A(z_0, w) \ne 0$ there, there
  exists $\delta > 0$ such that
  $$
  A(z,w) \ne 0\text{ for } |z-z_0| < \delta,\ |w-w_0| = \epsilon.
  $$
  %%% page 124
  Therefore, the residue theorem implies that for each fixed $z$,
  $|z-z_0| < \delta$,
  $$
  \frac{1}{2\pi i} \int_\gamma \frac{\frac{\p A}{\p w}(z,w)}{A(z,w)} dw
  $$
  is equal to the number of zeros of $A(z,w)$ (minus the number of
  poles of $A(z,w)$) for $|w-w_0| < \epsilon$.  And it is clear that
  this number is a continuous function of $z$ for $|z-z_0| < \delta$,
  and it therefore constant.  For $z=z_0$ we are counting the number
  of zeros of $A(z_0, w)$ in $|w-w_0| < \epsilon$.  Since $A(z_0, w) =
  0$ only at $w = w_0$ and since $w_0$ is a \emph{first order} zero
  ($\frac{\p A}{\p w}(z_0, w_0) \ne 0$), we have proved that the above
  integral is equal to 1 for $|z-z_0| < \delta$.  Thus, $|z-z_0| <
  \delta$ implies there exists a unique $f(z)$ such that $|f(z) - w_0|
  < \epsilon$ and $A(z, f(z)) = 0$.  Again, the residue theorem
  implies
  $$
  f(z) = \frac{1}{2\pi i} \int_\gamma w \frac{\frac{\p A}{\p
      w}(z,w)}{A(z,w)} dw,\quad
  |z-z_0| < \delta.
  $$
  From this formula it follows immediately that $f$ is holomorphic.
  Of course, $f(z_0) = w_0$.

  To prove uniqueness, suppose $g$ is holomorphic near $z_0$ and $A(z,
  g(z)) = 0$, $g(z_0) = w_0$.  Then by continuity of $g$ it follows
  that there exists $0 < \delta_1 \le \delta$ such that for $|z - z_0|
  < \delta_1$, $|g(z) - w_0| < \epsilon$.  Therefore, $g(z) = f(z)$
  for $|z-z_0| < \delta_1$.
\end{myproof}

%%% page 125
\begin{cor}
  Suppose $z_0 \in D$ and that there exists no $w$ satisfying
  $$
  A(z_0, w) = 0, \quad \frac{\p A}{\p w}(z_0, w) = 0.
  $$
  Then there exist unique holomorphic functions $f_1, \cdots, f_n$ in
  a neighbourhood of $z_0$ such that
  \begin{quote}
    $A(z, f_k(z)) = 0$ near $z_0$, $1 \le k \le n$, for each $z$ near
    $z_0$, the numbers $f_1(z), \cdots, f_n(z)$ are distinct.
  \end{quote}
\end{cor}

\begin{myproof}
  Since $A(z_0, w)$ is a polynomial in $w$ of degree $n$, it has $n$
  zeros.  By hypothesis these zeros are distinct, say $A(z_0, w_k) =
  0$, $1 \le k \le n$, $w_1, \cdots, w_n$ distinct.  Apply
  Lemma~\ref{lem:1} to $w_0 = w_k$ to obtain the holomorphic solutions
  $f_k$.  Since $f_1(z_0), \cdots, f_n(z_0)$ are distinct, it follows
  by continuity that for $z$ sufficiently near $z_0$, $f_1(z), \cdots,
  f_n(z)$ are distinct.
\end{myproof}

\begin{lem}
  \label{lem:2}
  Let $A$ be defined as in Lemma~\ref{lem:1}.  Assume that for every
  $z \in D$ there exists no $w$ satisfying
  %%% page 126
  $$
  A(z,w) = 0,\quad \frac{\p A}{\p w}(z,w) = 0.
  $$
  Let $f$ be a holomorphic function in a neighbourhood of $z_0 \in D$
  satisfying
  $$
  A(z,f(z)) \equiv 0 \text{ near } z_0.
  $$
  Then $f$ can be analytically continued along any path in $D$
  starting at $z_0$.
\end{lem}

\begin{myproof}
  Let $\gamma : [0,1] \to D$ be a path with $\gamma(0) = z_0$.  We are
  trying to prove the existence of a path $\widetilde{\gamma} : [0,1]
  \to M$ such that $\widetilde{\gamma}(0) = [f]_{z_0}$ and $\pi \circ
  \widetilde{\gamma} = \gamma$.  By the general discussion of analytic
  continuation we know that $\widetilde{\gamma}$ exists on some
  interval $[0,t_0]$, $t_0 > 0$, and that $\widetilde{\gamma}$ is
  uniquely determine (Proposition 3.8, Unique Lifting Theorem, of
  Topic 3).  Let $s_0$ be the supremum of such $t_0$.  Then
  $\widetilde{\gamma}$ exists on the interval $[0,s_0)$.  Now we apply
  the above corollary to the point $\gamma(s_0)$.  Obtaining
  holomorphic functions $f_1, \cdots, f_n$ in the neighbourhood of
  $\gamma(s_0)$ satisfying the conclusion of the corollary on a disk
  $\Delta$ centred at $\gamma(s_0)$.  Choose any $s_1 < s_0$ such that
  $\gamma(s_1) \in \Delta$.  Then $\widetilde{\gamma}(s_1) =
  [g]_{\gamma(s_1)}$, where $g$ is holomorphic in a neighbourhood of
  $\gamma(s_1)$ and by the permanence of function relations
  $$
  A(z,g(z)) \equiv 0 \text{ near }\gamma(s_1).
  $$

  \begin{mdframed}
    \vspace{3cm}
  \end{mdframed}

  %%% page 127
  \noindent
  Thus, $g(z)$ is one of the $n$ zeros of the polynomial $A(z,w)$ and
  must therefore the equal to one of the $f_k(z)$.  Thus, by
  Lemma~\ref{lem:1} and its corollary we find that for a unique $k$,
  $g(z) \equiv f_k(z)$, $z$ near $\gamma(s_1)$.  By the uniqueness of
  analytic continuation,
  $$
  \widetilde{\gamma}(s) = [f_k]_{\gamma(s)},\quad
  s_1 \le s < s_0.
  $$
  This formula serves to define $\widetilde{\gamma}$ for $s = s_0$ as
  well and eve for $s>s_0$, $s-s_0$ sufficiently small, if $s_s < 1$.
  Thus we conclude that $s_0 = 1$ and that $\widetilde{\gamma}$ exists
  on $[0,1]$.
\end{myproof}

\begin{cor}
  In addition to the hypothesis of Lemma~\ref{lem:2}, assume that $D$
  is a simply connected region.  Then there exist holomorphic
  functions $f_1, \cdots, f_n$ on $D$ such that
  \begin{quote}
    $A(z, f_k(z)) \equiv 0$ for $z \in D$, $1\le k \le n$, for each $z
    \in D$, the numbers $f_1(z), \cdots, f_n(z)$ are distinct.
  \end{quote}
\end{cor}

\begin{myproof}
  Use the corollary of Lemma~\ref{lem:1} to obtain $f_1, \cdots, f_n$
  near some point in $D$, say $z$.  Use Lemma~\ref{lem:2} and the
  Monodromy Theorem (Topic 3) to obtain holomorphic extensions on all
  of $D$, noting that $A(z,f_k(z)) \equiv 0$ on $D$ follows from the
  permanence of functional relations.  If for some $z \in D$, $f_j(z)
  = f_k(z)$, Lemma~\ref{lem:1} implies $f_j \equiv f_k$ near $z$ and
  then $f_j \equiv f_k$ in $D$ by analytic continuation, contradicting
  $f_j(z_0) \ne f_k(z_0)$ if $j\ne k$.  Thus, $j=k$.
\end{myproof}

%%% page 128
The above corollary is about as far as we can go without really
analysing what happens near points $z$ such that $A(z,w)$ has a
multiple zero.  To carry out such an analysis will require a little
algebraic background, which we now begin.

First of all, what we shall be considering is functions $A$ which are
polynomials in $z$ and $w$.  it is always possible and frequently
useful to arrange $A$ according to powers of $w$ or according to
powers of $z$.  Thus, we write
$$
A(z,w) = a_0(z) w^n + a_1(z) w^{n-1} + \cdots + a_{n-1}(z) w + a_n(z),
$$
where $a_0, a_1, \cdots, a_n$ are polynomials in $z$, and we assume
$a_0 \not\equiv 0$.  We then say that $A$ has degree $n$ with respect
to $w$.  We say that a polynomial $B$ is a \emph{factor} of $A$ if
there exists another polynomial $C$ such that $A = BC$.  If $A$ has no
factors other than constants or constant multiples of $A$, we say that
$A$ is \emph{irreducible}.  It will also frequently be useful to
factor $a_0$ from $A$, writing
$$
A(z,w) = a_0(z)[w^n + \alpha_1(z) w^{n-1} + \cdots + \alpha_{n-1}(z) w
+ \alpha_n(z)],
$$
%%% page 129
where $\alpha_k = \frac{a_k}{a_0}$ is a \emph{rational} function of
$z$.  Conversely, given rational functions of $z$, $\alpha_1, \cdots,
\alpha_n$, we can let $a_0$ be the least common multiple of the
denominators of $\alpha_1, \cdots, \alpha_n$, and use the above
formula to define a polynomial $A$.  This innocent statement will
prove to be extremely useful in constructing polynomials.  We shall
frequently be able to construct holomorphic functions $\alpha_k$ on
$\widehat{\CCC}$ minus a finite set, and by some argument show that
$\alpha_k$ has no essential singularities in $\widehat{\CCC}$.  Then
we use the fact that a meromorphic function $\alpha_k$ on
$\widehat{\CCC}$ must be rational.

\begin{lem}
  \label{lem:3}
  Let $A$ and $B$ be polynomials in $z$ and $w$ which have no common
  nontrivial factor, and assume $A, B \not\equiv 0$.  Then there are
  at most finitely many $z$ such that there exists $w$ such that
  $$
  A(z,w) = 0,\quad B(z,w) = 0.
  $$
\end{lem}

\begin{myproof}
  We shall use the Euclidean algorithm.  To do this it is most
  convenient to regard $A$ and $B$ as polynomials in $w$.  Then we
  employ the factorisation mentioned above to write
  $$
  A = a_0(z) A',\quad B = b_0(z) B'
  $$
  where
  %%% page 122
  $$
  \begin{aligned}
    A'(z,w) &= w^n + \alpha_1(z) w^{n-1} + \cdots + \alpha_n(z),\\
    B'(z,w) &= w^m + \beta_1(z) w^{m-1} + \cdots + \beta_m(z),
  \end{aligned}
  $$
  and $\alpha_1, \cdots, \alpha_n$, $\beta_1, \cdots, \beta_m$ are
  rational functions of $z$.  We rely heavily on the fact that the
  rational functions of $z$ form a field.  Also, we write for short
  $\deg A' = n$ and $\deg B' = m$.  By long division we have uniquely
  $$
  A' = B' Q_1 + R_1, \quad \deg R_1 < \deg B'.
  $$
  Here $Q_1$ and $R_1$ are polynomials in $w$ with coefficients in the
  field of rational functions of $z$, and if $R_1 \equiv 0$ we set
  $\deg R_1 = -\infty$.  If $R_1 \not\equiv 0$, we apply this again to
  obtain
  $$
  B' = R_1 Q_2 + R_2, \quad \deg R_2 < \deg R_1.
  $$
  Continue this division process:
  $$
  \begin{aligned}
    R_1 &= R_2 Q_3 + R_3, \quad R_3 < \deg R_2,\\
    \vdots &\\
    R_{k-2} &= R_{k-1} Q_k + R_k, \quad R_k < \deg R_{k-1},\\
    R_{k-1} &= R_k Q_{k+1}.
  \end{aligned}
  $$
  As indicated in this scheme, the process eventually terminates
  ($R_{k+1} \equiv 0$) since the degrees of the $R_j$'s keep
  decreasing.  We assume of course that $R_k \not\equiv 0$.  Note that
  if $R_1 \equiv 0$, then $B'$ is a factor of both $A'$ and $B'$, thus
  $B$ is a polynomial in $z$ alone and the conclusion of the lemma is
  trivial.  Thus, we can assume $R_1 \not\equiv 0$.  
  %%% page 131
  Working up through the above scheme, we see successively that $R_k$
  is a factor of $R_{k-1}$, thus $R_{k-2}, \cdots$, and finally $R_k$
  is a factor of $B'$, and thus of $A'$.  By hypothesis, $R_k$ must
  have degree 0 in $w$, so $R_k$ is just a rational function of $z$.
  Now we eliminate finitely many $z$ by requiring that $a_0(z) \ne 0$,
  $b_0(z) \ne 0$, and $z$ is not a pole of any of the coefficients of
  any of the polynomials $Q_1, \cdots, Q_k$ and $R_k(z) \ne 0$.  Then
  we claim that there does not exist $w$ such that $A(z,w) = 0$,
  $B(z,w) = 0$.  For suppose such $w$ exists.  Then also $A'(z,w) =
  0$, $B'(z,w) = 0$, since $a_0(z) \ne 0$, $b_0(z) \ne 0$.  Since
  $Q_1(z,w) \ne \infty$, the first equation in our division scheme
  implies $R_1(z,w) = 0$.  Likewise, $R_2(z,w) = 0$ and on down the
  line until we reach the contradiction $R_k(z) = 0$.
\end{myproof}

\begin{rem}
  Perhaps a cleaner way of giving this argument is to work up through
  the above equations to write
  $$
  R_k = CA' + DB',
  $$
  where $C$ and $D$ are polynomials in $w$ with coefficients which are
  rational function of $z$.  By clearing all the fractions out of this
  expression, we obtain
  $$
  R = EA + FB,
  $$
  where $R$ is a not identically vanishing polynomial in $z$ alone,
  and $E$ and $F$ are polynomials in $z$ and $w$.  Then if $A(z,w) =
  0$ and $B(z,w) = 0$, it follows that $R(z) = 0$.  Since $R$ has only
  finitely many zeros, this proves the lemma.
\end{rem}

%%% page 132
For our purposes the most important applications of this lemma occur
when the polynomial $A$ is \emph{irreducible}.  Then $A$ and $B$ have
no common nontrivial factor except perhaps $A$ itself.  Thus, if $A$
is not a factor of $B$, Lemma~\ref{lem:3} is in force.  The most
important example is the case in which the degree of $B$ with respect
to $w$ is lower than that of $A$.

\begin{defn}
  \label{def:1}
  Let $A$ be a polynomial,
  $$
  A(z,w) = a_0(z) w^n + a_1(z) w^{n-1} + \cdots + a_n(z),\quad
  a_0 \not\equiv 0.
  $$
  Then a point $z \in \widehat{\CCC}$ is a \emph{critical point} for
  $A$ if one of the following conditions holds:
  \begin{enumerate}
  \item $z = \infty$;
  \item $a_0(z) = 0$;
  \item there exists $w \in \CCC$ such that $A(z,w) =0$, $\frac{\p
      A}{\p w}(z,w) =0$.
  \end{enumerate}
  If $z$ is not critical, then $z$ is a \emph{regular point} for $A$.
\end{defn}

\begin{propn}
  \label{propn:1}
  If $A$ is irreducible, then there are only finitely many critical
  points for $A$.
\end{propn}

\begin{myproof}
  Since $a_0$ has only finitely many zeros, there are only finitely
  many $z$ satisfying 1 or 2.  Since the degree of $\frac{\p A}{\p w}$
  with respect to $w$ is less than $n$, $A$ and $\frac{\p A}{\p w}$
  have no nontrivial factor in common, and Lemma~\ref{lem:3} implies
  that at most finitely many $z$ satisfy condition 3.
\end{myproof}

Of course, what we are aiming for is an analytic description of the
solutions of $A(z,w) = 0$.  If we wish to do this in a neighbourhood
of \emph{regular} point $z_0$, the corollary to Lemma~\ref{lem:1}
contains all the information we need, namely that there are $n$
distinct holomorphic solutions $f_1, \cdots, f_n$ near $z_0$:
$A(z,f_k(z)) \equiv 0$.  Viewed as points in $\overline{M}$, we have
found
$$
e_k = e(z_0 + t, f_k(z_0 + t))
$$
such that
$$
A(z_0 + t, f_k(z_0 + t)) \equiv 0, \quad t\text{ near }0.
$$
Another way of expressing this relation is that
$$
A(\pi(e), v(e)) = 0\text{ for }e\text{ near }e_k.
$$
Likewise, for the simplest example of critical point we have
$$
A(z,w) = w^n - z
$$
and the element
$$
e_0 = e(t^n,t)
$$
satisfies $A(t^n, t) \equiv 0$ near 0, or
$$
A(\pi(e), v(e)) \equiv 0 \text{ for } e \text{ near } e_0.
$$
%%% page 134
Thus, we make the following definition.

\begin{defn}
  \label{def:2}
  The \emph{Riemann surface} of the polynomial $A(z,w)$ is the largest
  open subset of $\overline{M}$ on which $A(\pi,v) = 0$.  Thus, a
  meromorphic element $e(P,Q)$ belongs to the Riemann surface of $A$
  if and only if $A(P(t),Q(t)) \equiv 0$ for $t$ near 0.
\end{defn}

This latter assertion follows form the fact that if $\varphi$ is a
chart defined on $U(P,Q,\Delta)$ in the canonical way indicated in
Topic 4, then
$$
P(t_0) = \pi \circ \varphi^{-1}(t_0),\quad
Q(t_0) = v \circ \varphi^{-1}(t_0),
$$
so that
$$
e(P,Q) = e(\pi \circ \varphi^{-1}, v \circ \varphi^{-1}).
$$

\textbf{Notation}.  $S_A$ is the Riemann surface of $A$.

The first main result we shall obtain is that if $A$ is irreducible
and has degree $n$ in $w$, then $S_A$ is compact, connected, and $\pi$
restricted to $S_A$ takes every value in $\widehat{\CCC}$ $n$ times.
First, we need a lemma on polynomials and their zeros.

\begin{lem}
  \label{lem:4}
  If $w, \alpha_1, \cdots, \alpha_n$ are complex numbers such that
  %%% page 135
  $$
  w^n + \alpha_1 w^{n-1} + \cdots + \alpha_{n-1} w + \alpha_n = 0,
  $$
  then
  $$
  |w| < |\alpha_1| + \cdots + |\alpha_n| + 1.
  $$
\end{lem}

\begin{myproof}
  If $|w| < 1$, the result holds.  If $|w| \ge 1$, then
  $$
  |w|^n \le |\alpha_1||w|^{n-1} + \cdots + |\alpha_{n-1}| |w|
  + |\alpha_n| \le (|\alpha_1| + \cdots + |\alpha_n|)|w|^{n-1}
  $$
  so that
  $$
  |w| \le |\alpha_1| + \cdots + |\alpha_n|.
  $$
\end{myproof}

\begin{thm}
  \label{thm:1}
  If $A$ is irreducible, then $S_A$ is an analytic configuration.
\end{thm}

\begin{myproof}
  By Proposition~\ref{propn:1}, if $D$ is the set of regular points
  for $A$, then $\widehat{\CCC}-D$ is finite.  We shall first prove
  that $S_A \cap \pi^{-1}(D)$ is connected; this assertion forms the
  main point of the proof.  Note that
  $$
  S_A \cap \pi^{-1}(D) \subset M.
  $$
  For suppose $e(P,Q) \in S_A \cap \pi^{-1}(D)$, and let $z_0 = P(0)$,
  $w_0 = Q(0)$.  Then $z_0 \in D$ and $A(z_0, w_0) = 0$.  Since $z_0$
  is a regular point, $\frac{\p A}{\p w}(z_0, w_0) \ne 0$.  Thus,
  Lemma~\ref{lem:1}, implies there is a unique holomorphic $f$ near
  $z_0$ such that
  %%% page 136
  $A(z,f(z)) \equiv 0$, $f(z_0) = w_0$.  Since $A(P(t), Q(t)) \equiv
  0$ and $P(t)$ is near $z_0$, $Q(t)$ near $w_0$ for small $t$, we
  then have $Q(t) = f(P(t))$.  If the mapping $t \mapsto (P(t), Q(t))$
  is to be one-to-one (as it must), then the mapping $t \mapsto P(t)$
  must be one-to-one, showing that $P$ has multiplicity 1 at 0.  Thus,
  $e(P,Q) \in M$.

  So we must now prove that if $z_0$ and $z_1 \in D$ and $[f]_{z_0}$
  and $[g]_{z_1} \in S_A$, then there is a path in $S_A \cap
  \pi^{-1}(D)$ connecting these two germs.  Since $\widehat{\CCC} - D$
  is finite, $D$ is connected, and thus there is a path $\gamma$ in
  $D$ with initial point $z_0$ and terminal point $z_0$.  By
  Lemma~\ref{lem:2} there exists a (unique) path $\widetilde{\gamma}$
  in $M$ such that $\pi \circ \widetilde{\gamma} = \gamma$ and
  $\widetilde{\gamma}(0) = [g]_{z_1}$.  by the permanence of
  functional relations, $\gamma(t)$ is in $S_A$ for every $t$.  In
  particular, $\widetilde{\gamma}(1) \in S_A$ and thus is represented
  by a holomorphic function near $z_0$ which forms zeros of $A$.  By
  the corollary to Lemma~\ref{lem:1}, there are unique holomorphic
  functions $f_1, \cdots, f_n$ in a neighbourhood of $z_0$ such that
  $[f_k]_{z_0} \in S_A$ and $f_1(z), \cdots, f_n(z)$ are the distinct
  zeros of the function $A(z,w)$, if $z$ is near $z_0$.  Thus,
  $[f]_{z_0} = [f_j]_{z_0}$ and $\widetilde{\gamma}(1) = [f_k]_{z_0}$
  for some $j$ and $k$.  To finish the proof that $S_A \cap
  \pi^{-1}(D)$ is connected, it suffices to prove that for any $j$ and
  $k$ there exists a path $\gamma$ in $D$ from $z_0$ to $z_0$ such
  that analytic continuation of $f_j$ along $\gamma$ leads to $f_k$.
  %%% page 137
  Let us suppose that in all such analytic continuations $f_1$ can be
  analytically continued to $f_1, f_2, \cdots, f_m$, but \textbf{not}
  to $f_{m+1}, \cdots, f_n$ (where we have renumbered the $f_j$'s).
  Here $1 \le m \le n$, and we want to prove $m=n$.

  Now consider the function
  $$
  B(z,w) = \prod_{k=1}^m (w-f_k(z))
  $$
  defined for all $w \in \CCC$ and for $z$ in a neighbourhood of
  $z_0$.  For each fixed $w$ the function $B(z,w)$ can be analytically
  continued along all paths in $D$ with initial point $z_0$
  (Lemma~\ref{lem:2}), and analytic continuation along a closed path
  of this nature must simply lead to a \emph{permutation} of $f_1,
  \cdots, f_m$: such a continuation could not lead to any of the
  $f_{m+1}, \cdots, f_n$, and two different $f_j$'s could not be
  continued to the same $f_k$, by the unique lifting theorem.
  Therefore, $B(z,w)$ is analytically continued \emph{into itself}
  along any closed path in $D$ from $z_0$ to $z_0$, since $B$ is a
  \emph{symmetric} function of $f_1, \cdots, f_m$.  Another way of
  looking at this is to perform the indicated multiplication in $B$
  and write near $z_0$
  $$
  B(z,w) = w^m + \alpha_1(z) w^{m-1} + \cdots + \alpha_m(z),
  $$
  where
  $$
  \alpha_k(z) = (-1)^k \sum_{i_1 < i_2 < \cdots < i_k} f_{i_1} f_{i_2}
  \cdots f_{i_k}.
  $$
  By the same reasoning, each $\alpha_k$ is symmetric in $f_1, \cdots,
  f_m$, so $\alpha_k$ has the property that it can be analytically
  %%% page 138
  continued along all paths in $D$ and analytic continuation along
  closed path leads back to $\alpha_k$.  Thus each $\alpha_k$ can be
  extended to a \emph{single-valued} holomorphic function in $D$.

  Now for a trick that will be used over and over.  The function
  $\alpha_k$ is holomorphic in $\widehat{\CCC}$ except at finitely
  many points.  We shall now estimate the growth of $\alpha_k$ at
  these point to conclude $\alpha_k$ does not possess any essential
  singularity.  Suppose now that $a$ is one of the critical points
  (one of the points in $\widehat{\CCC}-D$).  Then for some
  sufficiently large integer $N$ we have near $a$
  $$
  |a_0(z)| \ge |z-a|^N, \quad |a_k(z)| \le C \quad (1\le k\le n)
  $$
  ($C$ is some constant) if $a\ne \infty$; if $a=\infty$ we have near
  $a$
  $$
  |a_0(z)| \ge c, \quad |a_k(z)| \le |z|^N \quad (1\le k\le n)
  $$
  ($c$ is some positive constant).  Thus, for $z$ near $a$ and $A(z,w)
  = 0$, Lemma~\ref{lem:4} implies:
  \begin{itemize}
  \item if $a \ne \infty$, $|w| < nC|z-a|^{-N} + 1$,
  \item if $a = \infty$, $|w| < \frac{n}{c}|z|^{N} + 1$.
  \end{itemize}
  Since $A(z, f_k(z)) = 0$, we thus obtain for $z$ near $a$,
  $$
  |f_k(z)| \le \text{const}|z-a|^{-N}\text{ or const }|z|^N
  $$
  if $a \ne \infty$ or $a = \infty$, respectively.  Thus, the formula
  for $\alpha_k$ shows that for $z$ near $a$,
  %%% page 139
  $$
  |\alpha_k(z)| \le \text{const}|z-a|^{-Nk}\text{ or const }|z|^{Nk}
  $$
  in the two cases.  Thus, $\alpha_k$ has either a pole or a removable
  singularity at $a$.  Since this is true at each critical point,
  $\alpha_k$ is meromorphic in $\widehat{\CCC}$ and is thus a
  \emph{rational} function.

  Let $b_0$ be the least common multiple of all the denominators of
  the $\alpha_k$'s expressed as fractions without common factors, and
  let
  $$
  \widetilde{B}(z,w) = b_0(z)B(z,w) = b_0(z)w^m + b_1(z)w^{m-1} +
  \cdots + b_m(z),
  $$
  a polynomial in $z,w$ of degree $m$ in $w$.  Since for $z$ near
  $z_0$
  $$
  \widetilde{B}(z,f_1(z)) = A(z,f_1(z)) = 0,
  $$
  the conclusion of Lemma~\ref{lem:3} does not hold for the
  polynomials $A$ and $\widetilde{B}$.  Thus, $A$ and $\widetilde{B}$
  possess a common nontrivial factor.  Since $A$ is irreducible, this
  factor must be $A$ itself.  Thus, the degree of $\widetilde{B}$ must
  be at least the degree of $A$, so $m=n$.

  We have now completed the proof that $S_A \cap \pi^{-1}(D)$ is
  connected.  The rest is easy.  Suppose $e(P,Q) \in S_A$.  Then for a
  sufficiently small disk $\Delta$ centred at 0, $U(P,Q,\Delta)$
  consists only of points in $S_A \cap \pi^{-1}(D)$ with the possible
  exception of $e(P,Q)$, since $\widehat{\CCC}-D$ is finite.  Thus,
  $e(P, Q)$ can be joined to a point in $S_A \cap \pi^{-1}(D)$ by a
  path in $\overline{M}$.  Thus, $S_A$ is connected.

  %%% page 140
  To prove $S_A$ is a component we show it is both open and closed in
  $\overline{M}$.  It is trivially open by Definition~\ref{def:2}.
  Suppose $e(P,Q)$ is in the closure of $S_A$ and let $\varphi :
  U(P,Q,\Delta) \to \Delta$ be a canonical chart.  Then there exists
  $t_0 \in \Delta$ such that $\varphi^{-1}(t_0) \in S_A$.  Thus, since
  $\varphi^{-1}(t_0) = e(P(t_0 + t), Q(t_0 + t))$, we have
  $$
  A(P(t_0 + t), Q(t_0 + t)) \equiv 0\text{ for }t\text{ small}.
  $$
  Thus, since $A(P(t), Q(t))$ is a meromorphic function for $t \in
  \Delta$ which vanishes for $t$ near $t_0$, $A(P(t), Q(t)) \equiv 0$
  in $\Delta$.  That is, $e(P,Q) \in S_A$, proving $S_A$ is closed.
\end{myproof}

\textbf{WARNING} It is tempting to think that if $z_0$ is a critical
point of the type 3, that is, if the equation $A(z_0, w) = 0$ has a
double root; and if $e(P,Q) \in S_A$, $P(0) = z_0$, and $Q(0)$ is a
double zero, then $e(P,Q)$ is a branch point of order at least 1.
This is not true in general.  For example, let
$$
A(z,w) = w^2 - z^2 - z^3.
$$
Then $A$ is irreducible and $z=0$ is a critical point, the zeros of
$A(0, w) = w^2$ both vanishing.  There are two points in $S_A$ lying
near $z=0$, and these are given by
%%% page 141
$$
e(t, t\sqrt{1+t}), \quad e(t, -t\sqrt{1+t}),
$$
where $\sqrt{1+t}$ is the principal determination of the square root
for $t$ small.  Clearly, neither of those meromorphic elements is a
branch point of order $\ge 1$.

\begin{thm}
  \label{thm:2}
  $S_A$ is compact.
\end{thm}

\begin{myproof}
  We again write
  $$
  A(z,w) = a_0(z) w^n + \cdots + a_n(z).
  $$
  Consider the function $\pi : S_A \to \widehat{\CCC}$.  By
  Proposition~2.24 item 2 of Topic 2, it suffices to prove that the
  restriction of $\pi$ to $S_A$ takes every value in $\widehat{\CCC}$
  $n$ times.  Of course, it suffices to consider the case in which $A$
  is irreducible.  Let $D$ be the set of regular points for $A$; by
  Proposition~\ref{propn:1} the set $\widehat{\CCC}-D$ is finite.  Let
  $a \in \widehat{\CCC}$ and choose $\epsilon>0$ sufficiently small
  that
  $$
  \Delta = \{ z ~:~ |z-a|<\epsilon \}\quad
  (\Delta = \{z ~:~ |z| > \epsilon^{-1}\} \text{ if }a=\infty)
  $$
  contains only points of $D$ except possibly for $a$ itself.  Let
  $\Delta'$ be the set $\Delta$ with a line from $a$ to the
  circumference removed; for definiteness, let
  $$
  \Delta' = \{ z ~:~ z \in \Delta,\ z-a \text{ not a negative real
    number} \}.
  $$
  Since $\Delta'$ is simply connected and contains no critical points
  for $A$, the corollary to Lemma~\ref{lem:2} implies that there are
  functions $f_1, \cdots, f_n$ holomorphic in $\Delta'$ such that for
  each $z \in \Delta'$, $f_1(z), \cdots, f_n(z)$ are the distinct
  solutions of $A(z,w) = 0$.
  %%% page 142
  Likewise, there are functions $g_1, \cdots, g_n$ holomorphic in the
  region $\Delta''$ as illustrated:
  \begin{center}
    \begin{tikzpicture}[scale=1]
      \filldraw[fill=black] (0,0) circle [radius=2pt];
      \draw (0,0) node [below] {$a$}
      (0,0) circle [radius=1.5cm]
      -- (-1.5,0)
      (1.4,1.4) node {$\Delta''$};
    \end{tikzpicture}
  \end{center}
  Now just \emph{below} the slit in $\Delta'$ the function $f_k$ must
  coincide with a unique $g_j$.  In turn, $g_j$ must coincide with a
  unique $f_\ell$ just \emph{above} the slit in $\Delta'$.  Let us
  denote $\ell = \sigma(k)$.  Thus, $f_{\sigma(k)}$ is the result of
  analytically continuing $f_k$ in a counterclockwise manner around
  $\Delta'$.  By the unique lifting theorem, the function $\sigma$ is
  a permutation of the integers $1, 2, \cdots, n$.  This permutation
  has a unique decomposition into cycles.  Let us consider a cycle of
  length $m$ and let us renumber the functions $f_k$ so that this
  cycle is represented by $\sigma(1) = 2$, $\sigma(2) = 3$, $\cdots$,
  $\sigma(m-1) = m$, $\sigma(m) = 1$.  Define for small $t$
  $$
  Q(t) =
  \begin{cases}
    f_1(a+t^m), & 0 < \arg t < \frac{2\pi}{m},\\
    f_2(a+t^m), & \frac{2\pi}{m} < \arg t < \frac{4\pi}{m},\\
    & \vdots \\
    f_m(a+t^m), & \frac{2(m-1)\pi}{m} < \arg t < 2\pi.
  \end{cases}
  $$
  (If $a = \infty$ replace $a+t^m$ by $t^{-m}$ throughout.)  By the
  definition of $\sigma$ and the particular enumeration of this
  %%% page 143
  cycle, $Q$ has an obvious extension to a holomorphic function defined
  for $0 < |t| < \epsilon^{1/m}$.  Also, since each $f_k(a+t^m)$ is a
  solution of $A(a+t^m, w) = 0$, Lemma~\ref{lem:4} can be applied
  to show that $|f_k(a+t^m)| \le $const $|t|^{-N}$ as $t \to 0$, for
  some positive integer $N$.  Therefore, $Q$ cannot have an essential
  singularity at 0, and thus $Q$ is \emph{meromorphic} for $|t| <
  \epsilon^{1/m}$.

  Now we prove that $(a+t^m, Q(t))$ is a \emph{pair}.  Suppose that
  for small $s$ and $t$, $a+t^m = a + s^m$, $Q(t) = Q(s)$.  Then $t^m
  = s^m$.  If $\frac{2(k-1)\pi}{m} \le \arg t < \frac{2k\pi}{m}$ and 
  $\frac{2(j-1)\pi}{m} \le \arg t < \frac{2j\pi}{m}$, then
  $$
  Q(t) = f_k(a+t^m) = \lim_{\delta\to 0^+} f_k(a+t^m
  e^{i\delta}),\quad
  Q(s) = f_j(a+t^m).
  $$
  Since $t^m = s^m$, $f_k(a+t^m) = f_j(a+t^m)$.  Since the functions
  $f_1, \cdots, f_n$ represent \emph{distinct} solutions (and likewise
  $g_1, \cdots, g_n$), we must have $j=k$.  By the inequalities for
  $\arg t$ and $\arg s$, the equation $t^m = s^m$ now implies $t =
  s$.  Thus, the mapping $t \mapsto (a+t^m, Q(t))$ is one-to-one.

  Since $A(a+t^m, Q(t)) \equiv 0$, this argument finds an element
  $$
  e(a+t^m, Q(t))
  $$
  belonging to $S_A$.

  %%% page 144
  If the permutation $\sigma$ is decomposed into cycles of lengths
  $m_1, \cdot, m_\ell$, then by the same argument we produce element
  in $S_A$ of the forms
  $$
  \begin{array}{c}
    e(a + t^{m_1}, Q_\ell(t)),\\
    \vdots\\
    e(a + t^{m_\ell}, Q_\ell(t)).
  \end{array}
  $$
  Since the multiplicity of $\pi$ at each point $e(a+t^{m_i}, Q_i(t))$
  is $m_i$ (Proposition~4.15 of Topic 4), it follows that $\pi$
  takes the value $a$ \emph{at least} $m_1 + \cdots + m_\ell = n$
  times.  The same is true if $a = \infty$, though we of coure need to
  use slightly different notation.

  An obvious remark shows that $\pi$ takes each value \emph{at most}
  $n$ times... of course, we speak of the restriction of $\pi$ to
  $S_A$.  In fact, if $a$ is a regular point for $A$, then the points
  in $S_A \cap \pi^{-1}(\{a\})$ are in $M$ and these elements are
  exactly the germs of the $n$ holomorphic solutions near $a$ by the
  corollary to Lemma~\ref{lem:1}.  Thus, $\pi$ takes on the value $a$
  exactly $n$ times in $S_A$.  If $\pi$ takes some value (a critical
  value) more than $n$ times in $S_A$, then $\pi$ takes every
  neighbouring value more than $n$ times in $S_A$, which implies $\pi$
  takes some regular value more than $n$ times in $S_A$, a
  contradiction.
\end{myproof}

%%% page 145
\begin{rem}
  \begin{enumerate}
  \item One sees finally the reason for discussing $\overline{M}$ ---
    it contains precisely enough points to discuss branch points in
    general, and in particular to discuss all the solutions of
    algebraic equations.  If $e(P,Q) \in S_A$ and $\varphi : U(P, Q,
    \Delta) \to \Delta$ is the canonical chart, the function
    $\varphi^{-1}$ is called a \emph{uniformizer} for $A$ near the
    point $P(0)$.  It replaces the multiple-valued solutions of $A =
    0$ by two single-valued meromorphic functions.  It is of course
    only defined locally.

  \item The elements $e(a+t^{m_i}, Q_i(t))$ produced in the above
    proof are obviously different.  The only possibility for two of
    them to coincide is for two of the multiplicities $m_i$ and $m_j$
    to coincide and for $Q_i(t) \equiv Q_j(wt)$ for some root of unity
    $w$.  By this would force the corresponding cycles to overlap, as
    can be easily checked.

  \item The function $Q$ in Definition~\ref{def:2} is meromorphic and
    thus has a Laurent expansion:
    $$
    Q(t) = \sum_{k=N}^{\infty} a_k t^k.
    $$
    Substituting formally $z = a + t^m$, or $t = (z-a)^{1/m}$, gives a
    series
    $$
    \sum_{k=N}^{\infty} a_k(z-a)^{k/m},
    $$
    with a similar series
    %%% page 146
    $$
    \sum_{k=N}^{\infty} a_k z^{-k/m}
    $$
    in case $a = \infty$.  These are called \emph{Puiseaux} series,
    and have the property that for any determination of $z^{1/m}$ the
    sum of the series gives a solution of $A(z,w) = 0$, and different
    determinations of $z^{1/m}$ yield different solutions.  Of course,
    all this information is contained in the idea of the corresponding
    meromorphic element.

  \item It is almost amazing how easy it was to find the elements $e(a
    + t^{m_i}, Q_i(t))$ in $S_A$.  However, when one observes what had
    to be known, it is quite obvious that it should be easy.  Namely,
    we had to have completely solved the equation $A(z,w) = 0$ away
    from the critical points, and then it was a simple matter of
    checking what $S_A$ looks like above these finitely many critical
    points.  But this sort of procedure can almost never be carried
    out in practice for rather obvious reasons.  We can't even usually
    hope to solve the equation near a critical point and observe how
    the zeros behave under analytic continuation around the critical
    point.

  \item Even without knowing Proposition~2.24 item 2 of Topic 2, it is
    almost obvious why $S_A$ is compact.  For $S_A$ consists
    essentially of $n$ copies of the (compact) sphere $\widehat{\CCC}$
    branched above certain finitely many points.  The only way $S_A$
    could fail to be compact would be for certain of these branch
    points not to be included in $S_A$.  Essentially the proof shows
    %%% page 147
    they are indeed all included and this statement is phrased in the
    perhaps deceptive statement that the restriction of $\pi$ to $S_A$
    takes every value $n$ times.
  \end{enumerate}
\end{rem}

\begin{ques}
  \label{ques:6}
  Let $A(z,w) = w^3 - 3zw + z^3$.  Prove that $A$ is irreducible.
  Find its critical points and discover the type of meromorphic
  elements which belong to $S_A$.  Compute the genus of $S_A$ by the
  Riemann-Hurwitz formula.
\end{ques}

\begin{ques}
  \label{ques:7}
  Same for $A(z,w) = zw^3 - 3w + 2z^a$, where $a$ is any integer
  (positive or negative).  Of course, if $a < 0$ then this is
  interpreted to be the problem for the \emph{polynomial}
  $$
  z^{1-a} w^3 - 3z^{-a} w + 2.
  $$
\end{ques}

Now we pass to the converse of Theorem~\ref{thm:2}.  This states that
every compact analytic configuration is the Riemann surface of a
unique (to within a constant factor) irreducible algebraic function.
In Topic 6 this statement will be improved considerably and will state
that \emph{any} compact connected Riemann surface is analytically
equivalent to a compact analytic configuration (and thus has an
associated irreducible polynomial).

Before stating this converse of Theorem~\ref{thm:2}, we make a useful
observation about $S_A$. First, divide out the leading coefficient
$a_0(z)$ to write $A(z,w) = a_0(z) A'(z,w)$ where
%%% page 148
$$
A'(z,w) = w^n + \alpha_1(z) w^{n-1} + \cdots + \alpha_n(z)
$$
and $\alpha_1, \cdots, \alpha_n$ are rational functions of $z$.  We
assume $A$ (and thus $A')$ to be irreducible.  If $a$ is a regular
point for $A$, then the corollary to Lemma~\ref{lem:2} shows the
existence of the holomorphic zeros $f_1, \cdots, f_n$ as usual.  Thus,
$$
e_k  = e(a+t, f_k(a+t))
$$
is a point in $S_A$, $1\le k \le n$, and the elements $e_k$ are the
only ones in $S_A \cap \pi^{-1}(\{a\})$.  Also, $v(e_) = f_k(a)$, so
the numbers $v(e_k)$ are the $n$ solutions of $A'(a,w) = 0$.  Thus, we
obtain a factorisation
\begin{equation}
  \label{e:A'}
A'(a,w) = \prod_{k=1}^n (w - v(e_k)) = \prod_{\substack{e\in S_A\\
    \pi(e) = a}}(w - v(e)).
\end{equation}

\begin{thm}
  \label{thm:3}
  Let $S$ be a compact analytic configuration.  Then there exists a
  unique (up to constant factor) irreducible polynomial $A$ such that
  $S = S_A$.
\end{thm}

\begin{myproof}
  Since $S$ is compact and $\pi : S \to \widehat{\CCC}$ is analytic,
  item 1 of Proposition~2.24 (Topic 2) shows that the restriction of $\pi$
  to $S$ takes every value the same number $n$ of times.  Let $D$ be
  the subset of $\widehat{\CCC}$ defined by
  $$
  \widehat{\CCC} - D
  = \{\pi(e) ~:~ e\in S,\ m_\pi(e) > 1\text{ or }v(e) = \infty\}.
  $$
  %%% page 149
  Thus, if $e \in S$ and $\pi(e) \in D$, then $m_\pi(e) = 1$ and $v(e)
  \ne \infty$.  Since $S$ is compact, the set of elements $e$ such
  that $e \in S$ and $m_\pi(e) > 1$ or $e(e) = \infty$ is
  \emph{finite}.  \emph{A fortiori}, $\widehat{\CCC} - D$ is finite.
  We now take our clue from the discussion above and define for 
  $z \in D$, $w \in \CCC$
  $$
  A'(z,w) = \prod_{\substack{e\in S\\\pi(e) = z}} (w-v(e)).
  $$
  That discussion implies that \emph{if} $S = S_A$ for some $A$, then
  this must be the formula for $A'(z,w)$ for regular points $z$, since
  all the regular points must be contained in $D$.  Thus, the
  uniqueness assertion of the theorem is established.  Moreover, we
  have an \emph{explicit} formula for $A'$ and we now just have to
  check various details.

  First, if $z_0 \in D$ then there are exactly $n$ elements $e_1,
  \cdots, e_n \in S$ with $\pi(e_k) = z_0$, since $\pi$ takes the
  value $z_0$ $n$ times and $\pi$ must have multiplicity 1 at each
  $e_k$.  Suppose $e_k = e(P(t), Q_k(t))$, where $P(t) = z_0 + t$ if
  $z_0 \ne \infty$ and $P(t) = t^{-1}$ if $z_0 = \infty$.  Let
  $\varphi_k : U(P,Q_k,\Delta) \to \Delta$ be a canonical chart.  For
  small $t_0$,
  $$
  S \cap \pi^{-1}(\{P(t_0)\}) = \{\varphi_1^{-1}(t_0), \cdots, 
  \varphi_n^{-1}(t_0)\},
  $$
  so that
  $$
  A'(P(t_0), w) = \prod_{k=1}^n (w - Q_k(t_0))
  $$
  since $v(\varphi_k^{-1}(t_0)) = Q_k(t_0)$.  This equation shows that
  if we expand
  %%% page 150
  $$
  A'(z,w) = w^n + \alpha_1(z) w^{n-1} + \cdots + \alpha_n(z),
  $$
  then $\alpha_1, \cdots, \alpha_n$ are holomorphic on $D$.

  Now we examine the behaviour of $\alpha_k$ at the (isolated) points
  of $\widehat{\CCC} - D$.  Suppose $a \in \widehat{\CCC} - D$.  Let
  $e(a+t^m, \widetilde{Q}(t))$ be one of the points in $S \cap
  \pi^{-1}(\{a\})$; in case $a = \infty$ this must be replaced by
  $e(t^{-m}, \widetilde{Q}(t))$.  Then for $z$ near $a$ but not equal
  to $a$, there are $m$ points in $S \cap \pi^{-1}(\{z\})$ determined
  by this one element, namely,
  $$
  e(a + (t_k+t)^m, \widetilde{Q}(t_k + t)),\quad\text{where }t_k^m = z-a.
  $$
  (We now discuss the case $a\ne \infty$; the case $a = \infty$ is
  handled entirely similarly.)  The corresponding values of $v(e)$ are
  $\widetilde{Q}(t_k)$, $1\le k \le m$.  Thus, for some $N$ we have
  $$
  |v(e)| \le |t_k|^{-N} = |z-a|^{-N/m}
  $$
  for these $m$ points $e\in S\cap \pi^{-1}(z)$.  Treating the other
  points in $S \cap \pi^{-1}(z)$ in a similar fashion, we obtain for
  some integer $M$
  $$
  |v(e)| \le |z-a|^{-M} \text{ if } e \in S \cap \pi^{-1}(z), \ z
  \text{ near }a.
  $$
  Thus
  $$
  |\alpha_k(z)| < \text{const }|z-a|^{-Mk} \text{ if }z\text{ is near
  }a;
  $$
  if $a = \infty$ this estimate should read
  $$
  |\alpha_k(z)| \le \text{const }|z|^{Mk}.
  $$
  Therefore, $\alpha_k$ is meromorphic on $\widehat{\CCC}$ and thus
  $\alpha_k$ is \emph{rational}.

  %%% page 151
  Now that we have produced a polynomial $A$ we must show that $A$ is
  irreducible and that its Riemann surface is $S$.  This will
  essentially be done all at once.  Suppose that there exists a
  factorisation of $A$ in the form $A = BC$, where $B$ and $C$ are
  polynomials and $B$ is irreducible ... in fact, there is always such
  a fact with a polynomial $B$ of degree at least one in $w$ (perhaps
  $C$ is constant).  Then $B$ has a Riemann surface $S_B$ which is a
  compact analytic configuration.  Let $e$ be an element in $S_B$ such
  that $m_{\pi}(e) = 1$ and $v(e) \ne \infty$; this includes all but
  finitely many points in $S_B$.  We also assume $\pi(e) \in D$,
  eliminating again at most finitely many points.  Then if $\pi(e) =
  z_0$ we let $P(t) = z_0 + t$ if $z_0 \ne \infty$ and $P(t) = t^{-1}$
  if $z_0 = \infty$.  Then $e = e(P,Q)$, and $B(P(t), Q(t)) = 0$ for
  $t$ near 0.  Thus, since $A = BC$ we have
  $$
  A'(P(t), Q(t)) = 0 \text{ for } t \text{ near } 0.
  $$
  The formula for $A'$ in \eqref{e:A'} implies
  $$
  \prod_{k=1}^n (Q(t) - Q_k(t)) = 0
  \text{ for }t\text{ near }0.
  $$
  Since each factor $Q-Q_k$ which is not identically zero can have
  only isolated zeros, it follows that for some $k$,
  $$
  Q = Q_k.
  $$
  Thus, $e = e(P,Q_k) = e_k \in S$.  Thus, except for finitely many
  %% page 152
  points $S_B \subset S$.  Since $S$ is compact in the Hausdorff space
  $\overline{M}$, $S$ is closed and thus
  $$
  S_B \subset S.
  $$
  Since $S_B$ is a component of $\overline{M}$ and since $S$ is
  connected it follows that $\boxed{S_B = S}$.

  Now it is all done.  For, $\pi$ assumes (when restricted to $S_B$)
  every value the same number of times, this number being the degree
  of $B$ as a polynomial in $w$.  But $\pi$ assumes (when restricted
  to $S$) every value $n$ times.  Thus, $B$ has degree $n$ in $w$.
  Thus, $C$ has degree 0 in $w$ and thus is just a polynomial in $z$.
  Therefore, if we discard all the common polynomial factors in $z$
  from the polynomial $A(z,w)$, we must have $C \equiv$ const.  This
  shows that $A$ is irreducible, its only possible notrivial factor
  turning out to be itself.  And $S_A = S_B = S$.
\end{myproof}

We thus see that on any compact analytic configuration $S$ the two
meromorphic functions $\pi$ and $v$ are related by an algebraic
equation.  These two functions of course allow us to construct other
meromorphic functions on $S$; namely any rational function of $\pi$
and $v$ is meromorphic on $S$.  The amazing fact is that there are no
other meromorphic functions on $S$.  In fact, we have

%%% page 153
\begin{thm}
  \label{thm:4}
  Let $S$ be a compact analytic configuration on which $\pi$ assumes
  every value $n$ times.  Let $f$ be any meromorphic function on $S$.
  Then there exist unique rational functions $\alpha_0, \cdots,
  \alpha_{n-1}$ such that
  $$
  f = \sum_{j=0}^{n-1} \alpha_j \circ \pi \cdot v^j.
  $$
\end{thm}

\begin{myproof}
  Suppose $z$ is a regular point for $A$, the polynomial is such that
  $A$ is irreducible, and $S = S_A$.  If the formula for $f$ is to
  hold, then we must have
  $$
  f(e) = \sum_{j=0}^{n-1} \alpha_j(z) v(e)^j \text{ if }e \in S,\ 
  \pi(e) = z.
  $$
  Now $S \cap \pi^{-1}(\{z\}) = \{e_1, \cdots, e_n\}$ has exactly $n$
  points and the numbers $v(e_k)$ are distinct.  The above equations
  read
  $$
  f(e_k) = \sum_{j=0}^{n-1} \alpha_j(z) v(e_k)^j,\quad 1\le k \le n.
  $$
  These are $n$ equations in $n$ ``unknowns'', $\alpha_0(z), \cdots,
  \alpha_{n-1}(z)$, and the determinant of the system is
  $$
  \det
  \begin{pmatrix}
    1 & v(e_1) & v(e_1)^2 & \cdots & v(e_1)^{n-1} \\
    1 & v(e_2) & v(e_2)^2 & \cdots & v(e_2)^{n-1} \\
    \vdots & \vdots & \vdots & \ddots & \vdots \\
    1 & v(e_n) & v(e_n)^2 & \cdots & v(e_n)^{n-1} \\
  \end{pmatrix}.
  $$
  %%% page 154
  This is a so-called Vandermonde determinant and its value is well
  known and easily seen to be
  $$
  \prod_{1\le \ell < k \le n} (v(e_k) - v(e_\ell)),
  $$
  which is not zero.  Thus, $\alpha_0(z), \cdots, \alpha_{n-1}(z)$ are
  uniquely determined.  It is also clear that these numbers
  $\alpha_j(z)$ really depend only on $z$ and not on a particular
  ordering $e_1, \cdots, e_n$ of the points in $\pi^{-1}(\{z\})$.
  Thus, $\alpha_0, \cdots, \alpha_{n-1}$ are uniquely determined at
  the regular points for $A$, and thus are unique since they are to
  be rational functions.

  Knowing what $\alpha_j$ must be, we now prove that they exist.  By
  Cramer's rule, we can write down a formula for $\alpha_j(z)$ in
  terms of a determinant involving $f(e_k)$ and $v(e_k)$, divided by
  the Vandermonde determinant.  Near a fixed regular point we can
  choose the $e_k$ in terms of charts to be analytic functions and
  thus $f(e_k)$ and $v(e_k)$ become analytic functions, proving
  $\alpha_j$ is holomorphic on the set of regular points.  As usual,
  we now prove that $\alpha_j$ cannot have any essential
  singularities.  Since we obtain upper bounds for $f(e_k)$ and
  $v(e_k)$ in the standard manner we are used to by now, it remains to
  obtain a lower bound for the Vandermonde.

  Suppose then that $a$ is a critical point.  We assume in the
  following that $a \ne \infty$; the case $a = \infty$ is treated by
  mere formal changes in the analysis.  The points in $S \cap
  \pi^{-1}(\{a\})$ have the forms
  %%% page 155
  $$
  e(a + t^{m_j}, Q_j(t)), \quad 1 \le j \le J, \quad 
  \sum_{j=1}^{J} m_j = n,
  $$
  where $Q_j$ is meromorphic near 0.  Let the positive integer $m$ be
  the least common multiple of the integers $m_j$.  If $z$ is a number
  sufficiently near $a$ but not equal to $a$, choose an arbitrary $s
  \in \CCC$ such that
  $$
  z - a = s^m.
  $$
  Let
  $$
  w_j = e^{2\pi i/m_j}.
  $$
  Then for $0 \le \ell \le m_j-1$,
  $$
  z -a = (w_j^\ell s^{m/m_j})^{m_j},
  $$
  and the numbers $w_j^\ell s^{m/m_j}$ are different for $0 \le \ell
  \le m_j - 1$.  Thus, $S \cap \pi^{-1}(\{z\})$ consists of the points
  $$
  e_{j\ell} = e(a + (w_j^\ell s^{m/m_j} + t)^{m_j}, 
  Q(w_j^\ell s^{m/m_j} + t))
  $$
  for $0 \le \ell \le m_j -1$, $1 \le j \le J$.  Thus, $v(e_{j\ell}) =
  Q_j(w_j^\ell s^{m/m_j})$.  The Vandermonde contains terms of the
  form
  $$
  v(e_{j\ell}) - v(e_{j'\ell'}) 
  = Q_j(w_j^{\ell} s^{m/m_j}) - Q_{j'}(w_{j'}^{\ell'} s^{m/m_{j'}}),
  $$
  which is a \emph{meromorphic function} of $s$, not vanishing for
  small $s \ne 0$ since $z$ is regular for $a$.  Thus, there exists
  an integer $N$ such that
  %%% page 156
  $$
  |v(e_{j\ell}) - v(e_{j'\ell'})| \ge |s|^N = |z-a|^{N/m},
  $$
  so the Vandermonde has modulus bounded below by
  $$
  |z-a|^{\frac{N}{m}\cdot \frac{n(n-1)}{2}}.
  $$

  Thus, we have proved that each $\alpha_j$ is rational and by
  definition
  $$
  f(e) = \sum_{j=0}^{n-1} \alpha_j(\pi(e)) v(e)^j
  $$
  for all but finitely many $e \in S$ (those such that $\pi(e)$ is a
  critical point for $A$). Thus, these two meromorphic functions on
  $S$ coincide.
\end{myproof}


\end{document}


%%% Local Variables: 
%%% mode: latex
%%% TeX-master: t
%%% End: 
