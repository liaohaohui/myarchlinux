\documentclass[a4paper,11pt]{article}
\usepackage{amsmath}
\usepackage{amsfonts}
\usepackage{verbatim}
\usepackage{url}
\usepackage{framed}
\usepackage{epsfig}
\usepackage{enumerate}
\usepackage{textcomp} %\textquotesingle
\usepackage[
%sorting=nyt,
firstinits=true, % render first and middle names as initials
useprefix=true,
maxcitenames=3,
maxbibnames=99,
style=authoryear,
dashed=false, % re-print recurring author names in bibliography
natbib=true,
url=false
]{biblatex}
%%% http://tex.stackexchange.com/questions/12254/biblatex-how-to-remove-the-parentheses-around-the-year-in-authoryear-style
\usepackage{xpatch}
\addbibresource{complex.bib} % run: biber topic1
\usepackage{color}
\usepackage{listings}
\definecolor{gray}{gray}{0.5} 
\definecolor{key}{rgb}{0,0.5,0} 
\lstset{ 
  language=[90]Fortran,
  basicstyle=\ttfamily\small, 
  keywordstyle=\color{blue}, 
  stringstyle=\color{red}, 
  showstringspaces=false, 
  emphstyle=\color{black}\bfseries, 
  emph={[2]True, False, None, self}, 
  emphstyle=[2]\color{key}, 
  emph={[3]from, import, as},
  emphstyle=[3]\color{blue}, 
  upquote=true, 
  morecomment=[s]{"""}{"""}, 
  commentstyle=\color{gray}\slshape, 
  %framexleftmargin=1mm, framextopmargin=1mm, frame=shadowbox, 
  rulesepcolor=\color{blue},
  numbers=left,
  stepnumber=1,
}
\usepackage{enumerate}
\usepackage{tikz}
\usetikzlibrary{lindenmayersystems}
\usetikzlibrary[shadings]


%%% Page Layout
\oddsidemargin=0truecm
\evensidemargin=0truecm
\textwidth=160truemm
\textheight=260truemm
\leftmargin=0truemm
\rightmargin=0truemm
\voffset=-23truemm
\topmargin=0truemm

\newif\iflecturer
%\lecturerfalse
\lecturertrue

\iflecturer
\usepackage{marginnote} % \marginpar
%\usepackage[color]{showkeys}
\definecolor{refkey}{rgb}{1,0,0}
\definecolor{labelkey}{rgb}{1,0,0}
\else
\def\marginpar#1{}
\fi

\iflecturer
\newcommand{\Answer}[1]{\dotfill\underline{\mbox{\hspace{1em}\color{blue}#1}}}
\newcommand{\BoxAns}[1]{\fbox{\color{blue}#1}}
\newcommand{\Reason}[1]{{\par\color{blue}\par{}Reason: #1}}
\else
\newcommand{\Answer}[1]{\dotfill\underline{\mbox{\hspace{1em}\color{white}#1}}}
\newcommand{\BoxAns}[1]{\fbox{\color{white}#1}}
\newcommand{\Reason}[1]{{\par\color{white}\par{}Reason: #1}}
\fi

\newif\ifamsstyle
\amsstylefalse
\newcounter{topic}
\setcounter{topic}{1}
\input common
\input symbol
\makeatletter
\newcommand{\Arctan}{{\mathop{\operator@font Arctan}\nolimits}}
\makeatother

%\parindent=0pt
\parskip=1pt
\linespread{1}


\begin{document}

\title{{\sc Rudiments of Riemann Surfaces\\
    Topic \thetopic{}: Introduction}}
\author{Author: B. Frank Jones, Jr. (Rice Univ. 1971)\\
Seminar: Dr Liew How Hui (\url{liewhh@utar.edu.my})}
\date{}

\maketitle

\begin{abstract}
  First, we will study the need for Riemann surfaces to made some
  ``multi-valued functions'' in complex analysis into ``single-valued
  functions''.  Then we will define the notion of abstract Riemann
  surface, where it is the right place to learn the notion of
  ``basic algebraic topology''.  After these, we will investigate the
  generalisation of ``analyticity'' from complex analysis to Riemann
  surface.  We will investigate algebraic functions, meromorphic
  functions and ``uniformisation theorem'' and hopefully a little bit
  about the classification of compact Riemann surfaces.
\end{abstract}

\tableofcontents

\section{Prerequisites for Riemann Surface}

Complex Analysis is probably the only prerequisite to Riemann
Surface.  A little bit of point-set topology is important but not
compulsory.  The following are some complex analysis books:
\begin{itemize}
\item Complex Analysis: The Geometric Viewpoint by Steven G. Krantz. A
  nice little book where you can learn more about Picard's theorems,
  for example.
\item Hayman, W.K.: Multivalent functions, 1958.
\item Jameson, G.J.O.: A first course on complex functions, 1970.
\item Kodaira, Kunihiko: Introduction to complex analysis, 1984.
\item Littlewood, J.E.: Lectures on the theory of functions, 1944.
\item Narasimhan, Raghavan: Complex analysis in one variable, 1985.
\item Remmert, Reinhold: Classical topics in complex function theory,
  1998.
\item Stewart, Ian; Tall, David: Complex Analysis, 1983.
\item Visual Complex Functions: An Introduction with Phase Portraits
  by Elias Wegert.
\end{itemize}



\section{Recommended Books for Riemann Surface}

This whole seminar will be based on the book ``Rudiments of Riemann
Surfaces'' by B. Frank Jones, Jr.  The electronic copy of the book
is available for download on the Internet.  It is neither famous nor
comprehensive but it is relatively short and should be fine for a
short semester discussion.

According to Kerry (see
\url{http://math.stackexchange.com/questions/135047/perspectives-on-riemann-surfaces/135108}),
randomly picking a book and read is not so bad.  There is no need to
worry about one book is better or worse than the other because of
``algebraic'', ``analytic'' --- we will find a synthesis!  We learn
something by reading other's treatment of a subject we do not
understand well enough.

Kerry recommended
\begin{itemize}
\item Curtis Mcmullen's notes and course outline.
  (Harvard U. Math 275 (Spring 1998): Riemann surfaces, dynamics and
  geometry)
\item R. Miranda, Algebraic Curves and Riemann Surfaces, AMS 1995
  % DA08
\item Simon Donaldson, Riemann Surfaces, OUP 2011 % DA115
\item R. Gunning, Lectures on Riemann Surfaces, Princeton Univ. Press
  1966 (The classical reference for researchers should be this)
  % DA10
\end{itemize}

Riemann Surfaces is a huge subject, with on going research in multiple
fronts in different levels. Hoping to achieve a deep understanding of
it in one semester is difficult --- because we have to do a lot of
problems and usually they are not easy.

There are many recommendations from Internet
(e.g. \url{math.stackexchange.com}, etc.) regarding good complex
analysis textbooks:
\begin{itemize}
\parskip=1pt
\item 
  It depends partly what you are more interested in, geometry or
  analysis. There are two relevant categories: compact complex
  manifolds of dimension one (and holomorphic maps), and algebraic
  complex curves (and rational maps). The approach in the wonderful
  book of Miranda is to consider the functor from algebraic curves to
  compact complex one manifolds, although he never fully proves it is
  well defined. The more analytic approach is to begin with compact
  complex one manifolds and prove they are all representable as
  algebraic curves. This is probably the approach of Forster. Another
  excellent analytic monograph from this point of view is the
  Princeton lecture notes on Riemann surfaces by Robert Gunning, which
  is also a good place to learn sheaf theory. His main result is that
  all compact complex one manifolds occur as the Riemann surface of an
  algebraic curve. Miranda's book contains more study of the geometry
  of algebraic curves.

  Riemann himself, as I recall, took an intermediate view, showing the
  equivalence of the categories of (irreducible) algebraic curves with
  that of (connected) compact complex manifolds equipped with a finite
  holomorphic map to $\PP^1$. Another extremely nice book, a little more
  advanced than Miranda, is the China notes on algebraic curves by
  Phillip Griffiths. Mumford's book Complex projective varieties I,
  also has a terrific chapter on curves from the complex analytic
  point of view.

  After you learn the basics, the book ``Geometry of Algebraic Curves
  Volumes I and II'' of Arbarello, Cornalba, Griffiths, Harris, is
  just amazing. Of course Riemann's thesis and followup paper on
  theory of abelian functions is rather incredible as well.

\item \url{http://www-home.math.uwo.ca/~masoud/riemannsurfaces.html}

  I shall try to closely follow the newly published Riemann Surfaces,
  by Simon Donaldson, Oxford University Press, 2011. Among all modern
  expositions of the theory, this one is particularly close to 
  Riemann's original analytic approach based on potential theory,
  PDE's and ideas of mathematical physics. It is also remarkable in
  its breath and scope and the clarity of exposition while keeping the
  required prerequisites to a bare minimum.

  Here are some other good texts on the subject (the list is by no
  means exhaustive):
  \begin{enumerate}
    %%%   zStudy/physics/inDataArchive83++/WaldschmidtM-FromNumberTheoryToPhysics-1992.djvu
  \item J.-B. Bost, Introduction to compact Riemann surfaces,
    Jacobians, and abelian varieties, in From number theory to physics
    (Les Houches, 1989), Springer, Berlin.1992, pp. 64--211. This is a
    particularly inspiring text full of historical background and
    references to original works (check Abel's amazing 2-page proof of
    his addition theorem, or his discovery of what later came to be
    known as the genus of a curve). 

  \item Otto Forster (1981), Lectures on Riemann surfaces. A very
    clear and lucid presentation.
  \item H. M. Farkas, I. Kra, Riemann Surfaces.
  \item J. Jost, Compact Riemann Surfaces.
  \item F. Kirwan, Complex Algebraic Curves.
    This one is more on the algebraic geometry side and at a more
    elementary level. Very well written and suitable for an advanced
    undergraduate course.
  \item H. McKean, V. Moll, Elliptic curves: function theory,
    geometry, arithmetic.

    A good place to start learning about connections between Riemann
    surfaces and arithmetic. Covers Kronecker-Weber and Mordel-Weil
    theorems. Very close to the style of originals. 
  \end{enumerate}
\end{itemize}


%There are many lecture notes on complex analysis from the Internet as
%well such as \citet{neubrander:_lectur_notes_compl_analy}.



\section{Why is Riemann Surface Interesting?}

Because this subject provides many examples for the more abstract
mathematical subjects:
\begin{itemize}
\item Algebraic Geometry
\item Algebraic Topology
\item Complex Dynamical System (Teichmuller Theory)
\item Hyperbolic Geometry and Metric Geometry?
\item Conformal Field Theory???
\end{itemize}

For me, it is interesting because can explain why a general polynomial
of degree $\ge 5$ cannot have radical solutions.  A rather detail
historical account of the solution of polynomials by radicals was
given in Jim Brown's article ``Abel and the Insolvability of the
Quintic''.



\section{Frank Jones' Preface}

...

The purpose of the course was to introduce the various ideas of
surfaces, sheaves, algebraic functions and potential theory in a
rather concrete setting, and to show the usefulness of the concepts
the students had learned abstractly in previous courses.  As a result,
I discussed the material carefully and leisurely, and for example did
not even attempt to discuss the notions of covering surface,
differential forms, Fuchsian groups, etc.  Therefore, these notes are
quite incomplete.  For comprehensive treatments of the subject, please
consult the bibligraphy.

I gratefully acknowledge some of the standard books which I consulted,
especially M. H. Hein's Complex Function Theory, G. Springer's
Introduction to Riemann Surfaces, and H. Weyl's The Concept of a
Riemann Surface.  Also, I relied heavily on L. Ber's lecture notes,
Riemann Surfaces, and especially on his Lectures 15--18.

...

Houston, 12 July, 1971

\section{Introduction}

The basic reason for the consideration and invention of Riemann
surfaces is the desire to consider holomorphic functions on a region
which in some sense is the ``largest'' possible. What makes this at
all feasible is the principle of the \emph{uniqueness of analytic
  continuation}, which states that if $f$ is holomorphic on a region
(open, connected set) in $\CCC$ and vanishes in a non-void open subset
of the region, then $f\equiv 0$.  This principle implies that if $f$
is holomorphic in a region $D$ and if it is possible to extend $f$ to
a larger region $\widetilde{D}$ in such a way that the extension is
analytic, i.e. if there is an analytic function $g$ on $\widetilde{D}$
such that $g=f$ on $D$, then $g$ is unique.  For if $h$ is another
function with this same property of extension, then $h=f$ on $D$, so
that $g-h$ on $D$.  Thus $g-h=0$ on $\widetilde{D}$.

Of course, it is not always possible to extend analytic functions at
all.  The easiest examples are the ``gap'' (lacunae)  power series, of
which one is
$$
f(z) = \sum_{n=0}^{\infty} z^{2^n}.
$$
This function is holomorphic in the unit disk $|z|<1$ and there is no
holomorphic function in a larger region which is an extension of $f$.
But this is not the kind of difficulty that we wish to consider.

Rather, the basic problem is that of multiple-valued ``functions''.
Phrased in terms of continuations, there is not always a largest
region to which a holomorphic function can be extended.  As an example
let $D = \{z ~:~ |z-1| < 1\}$ and 
$$
f(z) = \text{principle determination of }\log z
= \sum_{n=1}^{\infty} \frac{(-1)^{n-1}}{n} (z-1)^n
$$
defined on $D$.  Of course, we also have $f(z) = \log|z| + i\arg(z)$,
where $\arg(z)$ is between $-\frac{\pi}{2}$ and $\frac{\pi}{2}$.  Now
$f$ can be extended to a holomorphic function on the (complex) plane
$\CCC$ with the negative real axis removed, the extension being
$\log|z| + i\arg(z)$ where $-\pi < \arg(z) < \pi$.  But there are
other regions which can be considered as largest regions of extension;
e.g. the plane $\CCC$ with the positive imaginary axis removed and the
extension being $\log|z| + i\arg(z)$, where $-\frac{3\pi}{2} < \arg(z)
< \frac{\pi}{2}$.

It is admittedly frequently useful to ``cut'' the plane $\CCC$ along a
line from 0 to $\infty$ as visualised in the above cases, and to
consider there a single-value ``branch'' or ``determination'' of $\log
z$, and such a technique is exploited e.g. in contour integrals.

But from the point of view of this course, the cutting of $\CCC$ really
enables one to evade the issue, which is namely how can one speak of
$\log z$ and face up to its multiple-valuedness in a fearless way.
And the same question for other functions.  The answer given by
Riemann is that the plane $\CCC$ is too deficient to admit such
functions, so we consider other \emph{surfaces} where functions can be
defined which are single-valued and still exhibit the essential
behaviour of (in our example) $\log z$.

Let us now consider an explicit method for building such a surface for
$\log z$.  Take an infinite sequence of planes minus the origin, which
are to be considered as \emph{distinct}; call them $\CCC_n'$, where $n
\in \ZZ$.  On $\CCC_n'$ define a function $f_n$ by
$$
f_n(z) = \log|z| + i\arg(z) + 2n\pi i,
$$
where $-\pi < \arg(z) \le \pi$.  Now we ``glue'' the planes $\CCC_n'$
in a reasonable way.  This ``gluing'' is tantamount to defining a
topology on the union of the (disjoint) sets $\CCC_n'$.  To define
this topology we shall describe a neighbourhood basis of each point.
For a point $z \in \CCC_n'$ which does not lie on the negative real
axis a neighbourhood basis shall consist of all open disks in
$\CCC_n'$ with centre at $z$.  If $z\in \CCC_n'$ and $z$ is a negative
real number, a neighbourhood basis shall consist of all sets
$$
\{w \in \CCC_n' : |w-z|<\epsilon,\ \Im(w)\ge0\} \cup
\{w \in \CCC_{n+1}' : |w-z|<\epsilon,\ \Im(w)<0\},
$$
where $0 < \epsilon < |z|$.

\begin{mdframed}
  \vspace{1in}  

  Figure 1
\end{mdframed}



It is then easily checked that the set $S = \bigcup_{n=-\infty}^\infty
\CCC_n'$ becomes a topological space with a neighbourhood basis for
each point of $S$ being described as above.  Also, if $f$ is the
function from $S$ to $\CCC$ which equals $f_n$ on $\CCC_n'$ for each
$n$, then $f$ becomes a \emph{continuous} function on $S$.  Indeed, it
suffices to check continuity at points $z \in \CCC_n'$ which are
negative real numbers.  In the semidisk in $\CCC_n'$ depicted above $f
= f_n$ takes values close to $\log|z| + i\pi + 2n\pi i$, and in the
semidisk in $\CCC_{n+1}'$ $f=f_{n+1}$ takes values close to $\log|z| -
i\pi + 2(n+1)\pi i$, so in the whole neighbourhood of $z$, $f$ is
close to $\log|z| + i\pi + 2n\pi i = f(z)$.  Thus $f$ is continuous.

Thus, we have succeeded in defining a set $S$ which caries a
single-valued function $f$ which obviously is closely related to $\log
z$.  We shall later point out the essential feature of $S$ which
allows us to call it a Riemann surface (see Topic 2).

We remark that it is easy to visualise $S$ as a collection of planes
glued together as indicated and forming in $\RR^3$ an infinite
spiral.  The next surface we construct will not have so simple a
form.

%%% p13
For this construction consider the function $z^{1/m}$ where $m$ is an
integer $\ge 2$.  Since each $z\ne 0$ has $m$ distinct
$m$\textsuperscript{th} roots, this is a multiple-valued
``function''.  In order to treat this function consider distinct
copies of the plane minus the origin, $\CCC_1'$, $\CCC_2'$, $\cdots$,
$\CCC_m'$.  Define a function $f_n$ on $\CCC_n'$ by the formula:
$$
f_n(re^{i\theta}) = r^{1/m}e^{i\theta/m} e^{2i\pi(n-1)/m},\quad r>0,\ 
-\pi < \theta \le \pi.
$$
Let $T = \bigcup_{n=1}^m \CCC_n'$ and define a topology on $T$ exactly
as before, except that a neighbourhood basis of a negative real number
$z \in \CCC_m'$ is treated a little differently.  Note that an attempt
to visualise $T$ as a spiral in $\RR^3$ is doomed, since the ``top''
level $\CCC_m'$ has to be glued to the ``bottom'' level $\CCC_1'$
along their negative real axes, and this without crossing any of the
intermediate levels $\CCC_2', \cdots, \CCC_{m-1}'$ and also without
crossing the seam where $\CCC_1'$ is joined to $\CCC_2'$ (in case
$m=2$).  As before, define a function $f$ on $T$ by the formula $f =
f_n$ on $\CCC_n'$.  If $z \in \CCC_n'$ is a negative real number, then
in the semidisk in $\CCC_n'$ $f$ takes values close to $|z|^{1/m}
e^{i\pi/m} e^{2i\pi(n-1)/m}$, and in the semidisk in $\CCC_{n+1}'$ $f$
takes values close to $|z|^{1/m} e^{-i\pi/m} e^{2i\pi n/m}$, so $f$
stays close to $|z|^{1/m} e^{i\pi(2n-1)/m} = f(z)$ in a neighbourhood
of $z$.  And this holds even if $n=m$, in which case $\CCC_{n+1}'$ is
replaced by $\CCC_1'$.  Thus, $f$ is continuous on $T$ and gives a
reasonable representation of $z^{1/m}$.

%%% Page 14
Now a very interesting addition can be made to $T$.  Namely, consider
each $\CCC_n'$ to have its origin replaced, but with the origins in
each $\CCC_n$ representing a single point to be added to $T$.  Thus
consider $T\cup \{0\}$ (the original set with one point added) and let
a neighbourhood basis of 0 consist of sets of the form
$$
\{0\} \cup \bigcup_{n=1}^m \{z\in \CCC_n' ~:~ |z| < \epsilon\}
$$
for $0 < \epsilon < \infty$.  Extend $f$ by $f(0) = 0$.  Then $f$ is
again continuous on $T\cup \{0\}$.  In the very same way, the point
$\infty$ can be added.  Let
$$
\widetilde{T} = T \cup \{0\} \cup \{\infty\},
$$
let a neighbourhood basis of $\infty$ consist of sets of the form
$$
\{\infty\} \cup \bigcup_{n=1}^m \{z \in \CCC_n' ~:~
|z|>\frac{1}{\epsilon}\},
$$
and let $f(\infty) = \infty$.  Then $f$ is a continuous function from
$\widetilde{T}$ to the extended complex plane $\widehat{\CCC}$.
Obviously the points 0 and $\infty$ are in some sense different from
the other points in $\widetilde{T}$.  They are called \emph{branch
  points}, and are said to have \emph{order} $m-1$.

Although $\widetilde{T}$ is somewhat difficult to visualise as
situated in $\RR^3$, we shall now easily see that it is homeomorphic
to the sphere $\widehat{\CCC}$!  In fact, the mapping $f :
\widetilde{T} \to \widehat{\CCC}$ is a homeomorphism.  We have shown
that it is continuous; it is onto since every complex number is an 
$m$\textsuperscript{th} root;  it is 1-1 since different complex
numbers definitely have different $m$\textsuperscript{th} roots and
also the same complex number $z \ne 0$ has $m$ distinct
$m$\textsuperscript{th} roots.  General topology then shows $f^{-1}$
is continuous since $\widetilde{T}$ is compact and $\widehat{\CCC}$ is
Hausdorff; but it is quite easy to see directly that $f^{-1}$ is
continuous.  Indeed, $f^{-1}(z)$ is essentially $z^m$ (positioned on
the correct $\CCC_n'$).
%%% https://jmanton.wordpress.com/2013/12/11/why-does-compactness-ensure-a-bijective-map-is-a-homeomorphism/

%%% page 15
The nature of this homeomorphism and the geometry involved in the
construction of $\widetilde{T}$ are perhaps better seen when one
considers the Riemann sphere $\widehat{\CCC}$ instead of $\CCC$ as the
basic region from which $f$ is to be built.  If one regards
$\widehat{\CCC}$ as the Euclidean sphere $\{(x,y,z) ~:~ x^2+y^2+z^2 =
1\}$ in $\RR^3$ by means of stereographic projection and uses $m$
distinct copies $\widehat{\CCC}_1, \cdots, \widehat{\CCC}_m$ with the
gluing described above to be done along the meridians corresponding to
the negative real axis, then an essentially equivalent surface
$\widetilde{T}$ is obtained.  Now consider the action of the function
$f$.  On $\widehat{\CCC}_n$ it is given by the determination $f_n$ of
the $m$\textsuperscript{th} root and maps $\widehat{\CCC}_n$ onto a
portion of $\widehat{\CCC}$ cut off by two meridians which correspond
to rays in the plane with an included angle of $2\pi/m$.  In other
words, it ``spreads open'' the cut in $\widehat{\CCC}_n$ from a hole
with 0 opening to a hole with $(1-\frac{1}{m})2\pi$ opening.  Here is
a picture, a ``top'' view looking ``down'' on the north pole,
$\infty$:

%%% page 16
\begin{mdframed}
  \vspace{3.5cm}

  Figure 2
\end{mdframed}

Thus, the image of $\widetilde{T}$ under $f$ consists of $m$
``slices'' of $\widehat{\CCC}$, and the gluing in $\widetilde{T}$
shows that these slices of $\widehat{\CCC}$ are pieced together in
such a way that $\widetilde{T}$ is mapped homeomorphically onto
$\widehat{\CCC}$.

If one is interested only in the topological properties of
$\widetilde{T}$, then the procedure discussed in the above paragraph
can be considerably shortened by ignoring the specific nature of the
cuts and of the function $f$.  We illustrate with the case
$m=2$. Since we shall only discuss topological properties, we replace
the cut along a meridian by any old cut on the sphere which looks
reasonable, and take two copies of the sphere:

\begin{mdframed}
  \vspace{3.5cm}

  Figure 3
\end{mdframed}

The gluing is to be done in such a way that the shaded areas are to be
attached, as are the unshaded areas.  The action of the function $f$
is now replaced by a continuous opening of the two holes:

%%% page 17
\begin{mdframed}
  \vspace{3.5cm}

  Figure 4
\end{mdframed}

It is then obvious how to attach these two spheres with holes;  the
resulting figure looks like a figure which is obviously homeomorphic
to a sphere.

\begin{mdframed}
  \vspace{3cm}

  Figure 5

\end{mdframed}

Now we shall briefly indicate the construction of some other Riemann
surfaces.  For example, suppose $a$ and $b$ are distinct complex
numbers and consider the multiple-valued ``function'' $z \mapsto
\sqrt{(z-a)(z-b)}$.  The same procedure which works for $\sqrt{z}$ can
be applied here if $\CCC$ or $\widehat{\CCC}$ is cut between $a$ and
$b$. In trying to define this function one finds that the sign changes
when a circuit is made around either $a$ or $b$, so two copies of
$\widehat{\CCC}$ can be joined along the cut as before to provide a
surface on which a function which is single-valued and has the
properties of $\sqrt{(z-a)(z-b)}$ can be defined; the figure is
exactly that which appears in Figure 3, where the two slits go
from $a$ to $b$ on each sphere.  Here it should be remarked that
either branch of $\sqrt{(z-a)(z-b)}$ is meromorphic at $\infty$, since
one branch is approximately $z$ at $\infty$ and the other branch
approximately $-z$.  The branch points on the surface we have
constructed are $a$ and $b$, and the surface is again homeomorphic to
$\widehat{\CCC}$.  However note that the function $\sqrt{(z-a)(z-b)}$
is not the homeomorphism in this case.  Indeed, this function assumes
every value in $\widehat{\CCC}$ exactly twice.  A natural
homeomorphism in this case is the function on this surface
corresponding to $\sqrt{\frac{z-a}{z-b}}$.  Note in particular that if
we begin with this function and two copies of $\widehat{\CCC}$ cut
from $a$ to $b$, we obtain the same surface.

%%% page 18
Using the same process, we shall now construct a Riemann surface which
is note homeomorphic to a sphere.  For this consider the expression
$\sqrt{(z-a)(z-b)(z-c)}$, where $a,b,c$ are distinct complex numbers.
In order to attempt to define a single-valued function from this
formula, consider two copies of the sphere each having two cuts, say
from $a$ to $b$ and from $c$ to $\infty$; these cuts should not
intersect:

\begin{mdframed}
  \vspace{3.5cm}

  Figure 6
\end{mdframed}

%%% page 19
In defining continuously the square root in this case, a change of
sign results in going around $a$, or $b$, or $c$, or $\infty$.  The
cuts we have provided prohibit this, and we also see just how to glue
in order to obtain a continuous function: the shaded areas along the
cuts from $a$ to $b$ are to be attached, and likewise along $c$ to
$\infty$.  Now let $\widetilde{S}$ denote the resulting surface with
the four branch points $a, b, c, \infty$ included, the topology being
defined in the by now usual manner.  This surface is \emph{not}
homeomorphic to a sphere.  To see this we will exhibit a closed curve
on $\widetilde{S}$ which does not separate $\widetilde{S}$ into two
components.  This is the curve shown on the left sphere

\begin{mdframed}
  \vspace{3cm}

  Figure 7
\end{mdframed}

\noindent
which encircles the cut from $a$ to $b$.  To see that this curve does
not disconnect $\widetilde{S}$ consider the typical example of the
curve (shown by a dotted line) which connects two points which at
first glance might be separated by the given closed curve.

Probably the best way to see this topological property is to apply the
method of ``enlarging'' the seam.  After the first step we obtain the
following spaces to be glued:

\begin{mdframed}
  \vspace{3.5cm}

  Figure 8
\end{mdframed}

%%% page 20
After the gluing, the resulting figure appears as shown:

\begin{mdframed}
  \vspace{3.5cm}

  Figure 9
\end{mdframed}

This figure is clearly homeomorphic to a torus or a sphere with ``one
handle''.  The same topological type of surface arises from the
function $\sqrt{(z-a)(z-b)(z-c)(z-d)}$, where $a,b,c,d$ are distinct.
The only difference is that the cuts on $\widehat{\CCC}$ go from $a$
to $b$ and from $c$ to $d$.

This same argument allows the treatment of the function
$\sqrt{(z-a_1)(z-a_2)\cdots(z-a_m)}$, where $a_1, \cdots, a_m$ are
distinct.  Two copies of $\widehat{\CCC}$ are used with cuts from
$a_1$ to $a_2$, $a_3$ to $a_4$, etc.  If $m$ is even , the last cut is
from $a_{m-1}$ to $a_m$, and if $m$ is odd, from $a_m$ to $\infty$.
The same gluing procedure gives a topological type as illustrated:

\begin{mdframed}
  \vspace{4cm}

  Figure 10
\end{mdframed}

%%% page 21
\noindent
there are $\lfloor \frac{m+1}{2}\rfloor$ connecting tubes.  This is
homeomorphic to a sphere with ``handles'': there are $\lfloor
\frac{m-1}{2}\rfloor$ handles.  This is said to be a surface having
\emph{genus} equal to the number of handles.

\section{Local Coordinates}

In preparation for the definition of abstract Riemann surfaces to be
given in the next topic, we shall now examine a common property of
all the surfaces we have constructed.  Namely, each point on the
surface has a neighbourhood homeomorphic to an open subset of $\CCC$
--- the essential defining property for a surface.  This assertion is
of course completely trivial except where we have made cuts and where
we have inserted branch points, for outside these exceptional points
the neighbourhoods can just be taken to be disks on the various copies
of $\CCC$ and the homeomorphism essentially the identity mapping  on
the same disk, now regarded as lying in some other fixed copy of
$\CCC$.  The situation for points on the cuts which are not branch
points is not much more involved.  Refer to the neighbourhoods
related to Figure 1 which are defined and depicted there; call this
neighbourhood $U(z)$ and let $\Delta_\epsilon(z)$ be the disk $\{w\in
\CCC : |w-z|<\epsilon\}$.  Then define
$$
\varphi ~:~ U(z) \to \Delta_\epsilon(z)
$$
%%% page 22
by the obvious relation $\varphi(w) = w$.

The effect of $\varphi$ is obviously to attach the two semi-disks used
to make up $U(z)$.  It is now trivial to check that each point which
is not a branch point has a neighbourhood homeomorphic to an open set
(a disk) in $\CCC$, and this is true for all the surfaces we have
constructed.  If $\infty$ is not a branch point and does not lie on a
cut, a neighbourhood can be taken to be the complement of a large
closed disk in the appropriate copy of $\widehat{\CCC}$ and the
mapping into $\CCC$ the function $\varphi(z) = z^{-1}$.

Now for the branch points.  It should be no surprise that the branch
points can be treated, for we have pointed out how the surface with
branch points added is homeomorphic to a sphere or a sphere with
handles (in the cased we have considered), making the neighbourhoods
of the branch points look not very special at all.  Now we write down
this homeomorphism explicitly in the case of the Riemann surface for
$z^{1/m}$, since all the other branch points we have considered have
the same behaviour as is exhibited in this case (for $m=2$).  In fact,
the homeomorphism is exactly the ``function'' $z^{1/m}$ (which has
been made single-valued).  A similar construction works when the
branch points at $\infty$ are considered.

%%% page 23
Finally, consider how these various homeomorphisms are related.  That
is, suppose given two overlapping neighbourhoods $U_1$ and $U_2$ on
the surface with corresponding homeomorphisms $\varphi_1$ and
$\varphi_2$.  Then the function $\varphi_2 \circ \varphi_1^{-1}$ is
defined on an open subset of $\CCC$ and has values in another open
subset of $\CCC$, and is clearly a homeomorphism.  The thing to be
noted is that it is \emph{holomorphic}.  Except where $U_1$ or $U_2$
involved a branch point this is trivial, as the map $\varphi_2 \circ
\varphi_1^{-1}$ is the identity where it is defined.  If $U_1$
involves a branch point with $m$ sheets, then $\varphi_1^{-1}$ is
essentially the $m$\textsuperscript{th} power, and $\varphi_2 \circ
\varphi_1^{-1}(z) = z^m$, which is holomorphic.  If $U_2$ involves a
branch point, the $\varphi_2 \circ \varphi_1^{-1}$ is a holomorphic
determination of the $m$\textsuperscript{th} root.

The observation made above will be used to give a definition of
Riemann surface in the next topic.

\newpage
\section{Why does Compactness ensure a Bijective Map is a Homeomorphism?}

Source: \url{https://jmanton.wordpress.com/2013/12/11/why-does-compactness-ensure-a-bijective-map-is-a-homeomorphism/}

The motivation for this exposition is threefold:
\begin{itemize}
\item It helps illustrate the mantra that intuition is as equally
  important as algebraic manipulation.
\item It relates to a fundamental concept in mathematics, namely
  morphisms (from category theory).
\item It helps explain why compact spaces and Hausdorff spaces are
  important.
\end{itemize}

The following facts are readily found in textbooks; the tenet of this
exposition is that it is impossible, without further thought, to
appreciate any of these facts.

A topological space is compact if every open cover has a finite
subcover.

A topological space is Hausdorff if every two points can be separated
by neighbourhoods.

A map (i.e., a continuous function) $f \colon X \rightarrow Y$ between
topological spaces $X$ and $Y$ is a homeomorphism if it is bijective
and its inverse $f^{-1}$ is continuous.

If $X$ is compact and $Y$ is Hausdorff then a bijective map $f \colon
X \rightarrow Y$ is automatically a homeomorphism.

Questions:
\begin{itemize}
\item How can a bijective map not be a homeomorphism?
\item How can compactness possibly have any relevance to whether a map
  is a homeomorphism?
\item Why the Hausdorff assumption on $Y$?
\item How does this relate to ``morphisms'', as mentioned at the start
  of this exposition?
\end{itemize}

Before addressing these questions, a standard proof of the above
result is given, the purpose being to emphasise that the proof, on its
own, does not directly answer any of these questions. Memorising the
proof is essentially worthless. (Important are being able to derive
the proof from scratch and being able to understand why the result
should indeed be true.)

Showing $f^{-1}$ is continuous means showing the preimage of every
open set is open, or equivalently, showing the preimage of every
closed set is closed. Since $f$ is bijective, the preimage under
$f^{-1}$ of a set $A$ is simply $f(A)$. Hence it suffices to prove
that $f$ is closed (the image of every closed set is closed). Let $A
\subset X$ be closed. Since $X$ is compact, $A$ must be compact. The
image of a compact set under a continuous function is itself compact,
that is, $f(A)$ is compact. A compact subset of a Hausdorff space must
be closed, that is, $f(A)$ is a closed subset of $Y$, proving $f$ is
closed and therefore that $f^{-1}$ is continuous, as required.

While basing topology on open sets works very well from the
perspective of being able to give concise proofs such as the one
above, it is perhaps the worst way of introducing topology. It is
noteworthy that when topology was being developed, it took decades for
the importance of ``open sets'' to be recognised.

Let's start from scratch and try to understand things intuitively. One
approach (but certainly not the only approach) to understanding a
topological space is understanding what converges to what. In special
cases, such as when the topology comes from a metric, it suffices to
consider sequences (the curious reader may wish to read about
sequential spaces). In general, nets must be considered instead of
sequences, nevertheless, for the purposes of answering the questions
above, it suffices to restrict attention to sequences. (It is often
enough to understand intuitively why a result is true in special
cases, leaving the proof to justify the result in full generality.)

\textbf{Why a Bijective Map need not be a Homeomorphism}

By restricting attention to sequences (and therefore to nice
topological spaces), the definition of continuity becomes: $f$ is
continuous if, whenever $x_n \rightarrow x$, it holds that $f(x_n)
\rightarrow f(x)$. Therefore, a bijective map $f$ is a homeomorphism
if and only if $f(x_n) \rightarrow f(y) implies x_n \rightarrow y$.

Whereas bijectivity just tests whether distinct points are mapped to
distinct points, being a homeomorphism means nearby points must be
mapped to nearby points, in both directions, from $X$ to $Y$ and from
$Y$ to $X$. (That is, a homeomorphism preserves the topology.)

This suggests the following example. Let $X$ be the interval $[0,2\pi)
\subset \mathbb{R}$ and let $Y$ be the unit circle $Y = \{ (x_1,x_2)
\in \mathbb{R}^2 \mid x_1^2 + x_2^2 = 1\}$. Let $f\colon X \rightarrow
Y$ be the function sending $\theta \in [0,2\pi)$ to
$(\cos\theta,\sin\theta)$. Then $f$ is continuous because it sends
nearby points in $X$ to nearby points in $Y$. But $f^{-1}$ is not
continuous because the point $(1,0) \in Y$ is sent to the point $0 \in
X$ but a point just below $(1,0) \in Y$, say $(\cos\theta,\sin\theta)$
for $\theta$ just less than $2\pi$, is sent to a point far away from
$0 \in X$. (The reader should pause to understand this fully; drawing
a picture may help.)

The above example is quintessential; an injective map $f$ is a
homeomorphism only if it is not possible to map a ``line
$\longrightarrow$'' to a ``loop $\circlearrowleft$``. In detail, let
$x_n$ be a sequence of points that either does not have a limit point,
or that converges to a point $x$ that is different from $x_1$. Such a
sequence can be obtained by considering the location at regularly
spaced intervals of an ant walking along a path in $X$ that does not
eventually return to where it started. It is therefore referred to
loosely as a ``line''. Let it loosely be said that this ``line'' gets
mapped to a ``loop'' if $\lim f(x_n) = f(x_1)$. If it is possible for
a line to be mapped to a loop, then f cannot be a homeomorphism. (Draw
some pictures!)

To summarise, injectivity of $f$ means that if $f(x) = f(y)$ then  $x
= y$. For $f$ to be a homeomorphism requires more: if $\lim f(x_n) =
f(y)$ then it must be that $\lim x_n = y$. Whereas injectivity just
means the preimages of distinct points are distinct, a homeomorphism
means the preimages of  convergent sequences with distinct limits are
convergent sequences with distinct limits. Perhaps the best way of
visualising this though is asking whether a ``line'' can be mapped to
a ``loop''.

\textbf{Relevance of the Domain being Compact and the Codomain being Hausdorff}

Let $x_n$ be a ``line'', as defined above. There are two cases to
consider: either $x_n$ does not have a limit point, in which case it
will be seen that compactness of $X$ is relevant, or $x_n$ has a
limit, in which case it will be seen that $Y$ being Hausdorff is
significant. 

When dealing with sequences, the role of compactness is guaranteeing
every sequence has a limit point (or equivalently, that a convergent
subsequence exists). Therefore, if $X$ is compact, it is not possible
for $x_n$ to lack a limit point, and this option is immediately off
the table.

If it is possible for a ``line'' to be mapped to a ``loop'' despite
$X$ being compact then it is possible to achieve this assuming $x_n$
has a limit (by considering a convergent subsequence if
necessary). Henceforth, assume $x_n \rightarrow x \neq x_1$ yet
$f(x_n) \rightarrow f(x_1)$. Continuity of $f$ implies $f(x_n)
\rightarrow f(x)$. When dealing with sequences, the role of the
Hausdorff requirement is to ensure limits are unique. (Prove this for
yourself, or look up a proof that limits are unique in Hausdorff
spaces.) Therefore, $Y$ being Hausdorff means $f(x) = f(x_1)$ since
$f(x_n)$ converges to both $f(x)$ and $f(x_1)$. Since $f$ is
injective, this means $x = x_1$, contradicting $x_n$ being a
``line''. Therefore, it is impossible for a ``line'' to be mapped to a
``loop''; every injective map f from a compact space to a Hausdorff
space is automatically a homeomorphism.

The above has hopefully taken away a bit of the mystery of why
compactness and the Hausdorff condition enter into the picture.
\begin{itemize}
\item An injective map $f$ (on a nice topological space where it
  suffices to reason with sequences) is a homeomorphism if and only if
  it is not possible for a ``line $\longrightarrow$'' to be mapped to
  a ``loop $\circlearrowleft$''. (A line and a loop are topologically
  distinct whereas the purpose of a homeomorphism is to preserve
  topological structure.)
\item If $y_n$ is a loop in $Y$, that is, $y_n \rightarrow y_1$, then
  consider its preimage $f^{-1}(y_n)$. Unless there is some reason for
  $f^{-1}(y_n)$ to converge, the fact that $f$ is injective is
  powerless to prevent $f^{-1}(y_n)$ from being a ``line'', i.e., it
  may well be the case that $f$ is not a homeomorphism.
\item If $X$ is compact then $f^{-1}(y_n)$ has a limit point. By
  considering a subsequence if necessary, assume without loss of
  generality that $f^{-1}(y_n) \rightarrow x$. Continuity of $f$
  implies $y_n \rightarrow f(x)$.
\item If $Y$ is Hausdorff then limits are unique, therefore, $f(x) =
  y_1$. Therefore, the preimage of a loop is a loop, not a line. The
  topology has been preserved, hence $f$ is a homeomorphism.
\item The above is ``intuitively'' why the result holds, yet the
  formal proof based on open sets is shorter, more general but devoid
  of intuition (at least for beginners); this is the joy of
  mathematics.
\end{itemize}

\textbf{Morphisms}

Very briefly, it is remarked that structure-preserving functions are
called morphisms (in category theory). Two sets are equivalent if
there exists a bijective function between them. As more structure is
given to these sets, requirements stronger than just bijectivity are
necessary if the structure is to be preserved.

In linear algebra, two vector spaces are the same if there exists a
bijective function $f$ between them that is linear and whose inverse
is linear; the linearity of $f$ and $f^{-1}$ is what preserves the
structure. Interestingly, if $f$ is linear and bijective then its
inverse $f^{-1}$ is automatically linear.

In topology, a continuous function preserves structure in one
direction, therefore, to preserve structure in both directions
requires a bijective f that is continuous and whose inverse $f^{-1}$
is also continuous; as introduced earlier, such a function is called a
homeomorphism. Unlike the linear case, $f$ being bijective and
continuous is not enough to imply $f^{-1}$ is continuous.


%%% References
%\bibliography{complex}
%\bibliographystyle{alpha}
%\bibliographystyle{amsalpha}

%%% BibLaTeX
\xpatchbibmacro{date+extrayear}{%
  \printtext[parens]%
}{%
  \setunit{\addperiod\space}%
  \printtext%
}{}{}
\printbibliography


\end{document}

